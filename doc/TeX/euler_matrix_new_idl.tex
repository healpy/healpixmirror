% -*- LaTeX -*-


\renewcommand{\facname}{{euler\_matrix\_new}}
\renewcommand{\FACNAME}{{EULER\_MATRIX\_NEW}}

\sloppy



\title{\healpix IDL Facility User Guidelines}
\docid{\facname} \section[\facname]{ }
\label{idl:euler_matrix_new}
\docrv{Version 1.0}
\author{Eric Hivon}
\abstract{This document describes the \healpix IDL facility \facname.}




\begin{facility}
{This IDL facility provides a means to generate a 3D rotation Euler
matrix parametrized by three angles and three axes of rotation.
}
{src/idl/misc/\facname.pro}
\end{facility}

\begin{IDLformat}
{%
\mylink{idl:euler_matrix_new:matrix}{matrix} = \FACNAME(%
\mylink{idl:euler_matrix_new:a1}{a1}, %
\mylink{idl:euler_matrix_new:a2}{a2}, %
\mylink{idl:euler_matrix_new:a3}{a3} [,% 
\mylink{idl:euler_matrix_new:deg}{DEG=}, % 
\mylink{idl:euler_matrix_new:help}{HELP=}, % 
\mylink{idl:euler_matrix_new:x}{X=}, %
\mylink{idl:euler_matrix_new:y}{Y=}, %
\mylink{idl:euler_matrix_new:zyx}{ZYX=}%
]) }
\end{IDLformat}

\begin{qualifiers}
  \begin{qulist}{} %%%% NOTE the ``extra'' brace here %%%%
    \item[matrix] \mytarget{idl:euler_matrix_new:matrix} 
      a 3x3 array containing the Euler matrix
    \item[a1] \mytarget{idl:euler_matrix_new:a1} 
      input, float scalar, 
    angle of the first rotation, expressed in radians,
    unless DEG (see below) is set
    \item[a2] \mytarget{idl:euler_matrix_new:a2} 
      angle of the second rotation, same units as a1
    \item[a3] \mytarget{idl:euler_matrix_new:a3} 
      angle of the third rotation, same units as a1
  \end{qulist}
\end{qualifiers}
    
\begin{keywords}
  \begin{kwlist}{} %%% extra brace
    \item[DEG=] \mytarget{idl:euler_matrix_new:deg} 
      if set, the angles are in degrees instead of radians

    \item[HELP=] \mytarget{idl:euler_matrix_new:help} 
      if set, the routine prints its documentation header and exits

    \item[X=] \mytarget{idl:euler_matrix_new:x} 
      if set, uses the classical mechanics convention (ZXZ):\\
	rotation a1 around original Z axis, \\
	rotation a2 around intermediate X axis, \\
        rotation a3 around final Z axis \\
        (see \htmladdnormallink{Goldstein (1951)}{https://en.wikipedia.org/wiki/Classical_Mechanics_(Goldstein_book)} for more details). \\
	{\em Equivalent to:} \\
	rotation a3 around Z axis, \\
	rotation a2 around initial (unrotated) X axis, \\
	rotation a1 around initial (unrotated) Z axis.\\
	\default{this convention is used}

    \item[Y=] \mytarget{idl:euler_matrix_new:y} 
      if set, uses the quantum mechanics convention (ZYZ):\\
	rotation a1 around original Z axis, \\
	rotation a2 around intermediate Y axis, \\
        rotation a3 around final Z axis.\\
	{\em Equivalent to:} \\
	rotation a3 around Z axis, \\
	rotation a2 around initial (unrotated) Y axis, \\
	rotation a1 around initial (unrotated) Z axis.

    \item[ZYX=] \mytarget{idl:euler_matrix_new:zyx} 
      if set, uses the aeronautics convention (ZYX):\\
	rotation a1 around original Z axis, \\
	rotation a2 around intermediate Y axis, \\
        rotation a3 around final X axis.\\
	{\em Equivalent to:} \\
	rotation a3 around X axis, \\
	rotation a2 around initial (unrotated) Y axis, \\
	rotation a1 around initial (unrotated) Z axis.
  \end{kwlist}
\end{keywords}  

\begin{codedescription}
{\parbox[t]{\hsize}{\facname\ ~\ allows the generation of a rotation Euler matrix. The user
can choose the three Euler angles, and the three axes of rotation.

If \texttt{vec} is an N$\times$3 array containing N 3D vectors, \\
    \texttt{vecr = vec  \# euler\_matrix\_new(a1,a2,a3,/Y)}\\
will be the rotated vectors. 
Alternatively, \htmlref{rotate\_coord}{idl:rotate_coord} can also be used to rotate 
\texttt{vec} into \texttt{vecr}.\\
 
\small{
This routine supersedes euler\_matrix, which had inconsistent angle
definitions. The relation between the two routines is as follows  :
\\[.1cm]
%
euler\_matrix\_new(a,b,c,/X)  =  euler\_matrix($-$a,$-$b,$-$c,/X) \\
= Transpose(euler\_matrix(c, b, a,/X)) \\[.1cm]
%
euler\_matrix\_new(a,b,c,/Y)  =  euler\_matrix($-$a, b,$-$c,/Y) \\
= Transpose(euler\_matrix(c,$-$b, a,/Y)) \\[.1cm]
%
euler\_matrix\_new(a,b,c,/Z)  =  euler\_matrix($-$a, b,$-$c,/Z)
}
}}
\end{codedescription}

\begin{related}
  \begin{sulist}{} %%%% NOTE the ``extra'' brace here %%%%
    \item[idl] version \idlversion or more is necessary to run \thedocid.
    \item[\htmlref{rotate\_coord}{idl:rotate_coord}] apply a rotation to a set of position vectors and
    polarization Stokes parameters.
  \end{sulist}
\end{related}

% \begin{example}
% {
% \begin{tabular}{ll} %%%% use this tabular format %%%%
% \end{tabular}
% }
% {
% }
% \end{example}

