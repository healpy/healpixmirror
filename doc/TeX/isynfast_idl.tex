% -*- LaTeX -*-

\sloppy

\title{\healpix IDL Facility User Guidelines}
\docid{isynfast} \section[isynfast]{ }
\label{idl:isynfast}
\docrv{Version 1.0}
\author{Eric Hivon}
\abstract{This document describes the \healpix IDL facility \thedocid.}

\begin{facility}
{This IDL facility provides an interface to F90 '\htmlref{synfast}{fac:synfast}' facility. It can be
used to generate sky maps and/or $a_{\ell m}$ from power spectra ($C(\ell)$), synthesize maps from
$a_{\ell m}$ or simulate maps from $C(\ell)$ and constraining $a_{\ell m}$.}
{src/idl/interfaces/isynfast.pro}
\end{facility}

\begin{IDLformat}
{ISYNFAST%
, \mylink{idl:isynfast:cl_in}{cl\_in}%
 [, \mylink{idl:isynfast:map_out}{map\_out}%
,
      \mylink{idl:isynfast:alm_in}{alm\_in=}%
,  \mylink{idl:isynfast:alm_out}{alm\_out=}%
, \mylink{idl:isynfast:apply_windows}{apply\_windows=}%
, \mylink{idl:isynfast:beam_file}{beam\_file=}%
, \mylink{idl:isynfast:binpath}{binpath=}%
, \mylink{idl:isynfast:double}{double=}%
, \mylink{idl:isynfast:fwhm_arcmin}{fwhm\_arcmin=}%
, \mylink{idl:isynfast:help}{help=}%
,
      \mylink{idl:isynfast:iseed}{iseed=}%
, \mylink{idl:isynfast:keep_tmp_files}{keep\_tmp\_files=}%
, \mylink{idl:isynfast:lmax}{lmax=}%
, \mylink{idl:isynfast:nlmax}{nlmax=}%
, \mylink{idl:isynfast:nside}{nside=}%
, \mylink{idl:isynfast:nsmax}{nsmax=}%
, \mylink{idl:isynfast:plmfile}{plmfile=}%
,
      \mylink{idl:isynfast:simul_type}{simul\_type=}%
, \mylink{idl:isynfast:silent}{silent=}%
, \mylink{idl:isynfast:tmpdir}{tmpdir=}%
, \mylink{idl:isynfast:windowfile}{windowfile=}%
, \mylink{idl:isynfast:winfiledir}{winfiledir=}%
]}
\end{IDLformat}

\begin{qualifiers}
  \begin{qulist}{} %%%% NOTE the ``extra'' brace here %%%%
   \item[cl\_in\mytarget{idl:isynfast:cl_in}%
] input power spectrum, can be a FITS file, or a memory array containing the
        $C(\ell)$, used to generate a map or a set of gaussian alm \\
   If empty quotes ('') or a zero (0) are provided, it will be interpreted as "No input C(l)", in
   which case some input alm's (alm\_in) are required.
    \item[map\_out\mytarget{idl:isynfast:map_out}%
] optional output: {\em RING ordered} map synthetised from the power spectrum or from constraining alm
  \end{qulist}
\end{qualifiers}

\begin{keywords}
  \begin{kwlist}{} %%% extra brace
 \item[alm\_in\mytarget{idl:isynfast:alm_in}%
=]    optional input (constraining) alm (must be a FITS file)           \default {no alm}

 \item[alm\_out\mytarget{idl:isynfast:alm_out}%
=]   contains on output the effective alm (must be a FITS file)

 \item[/apply\_windows\mytarget{idl:isynfast:apply_windows}%
] if set, beam and pixel windows are applied to input alm\_in
(if any)

 \item[beam\_file\mytarget{idl:isynfast:beam_file}%
=] beam window function, either a FITS file or an array %
(see \htmlref{''Beam window function files''}{fac:subsec:beamfiles} section
in the \linklatexhtml{\healpix Fortran Facilities document}{facilities.pdf}{facilities.htm})

 \item[binpath\mytarget{idl:isynfast:binpath}%
=] full path to back-end routine \default {\$HEXE/synfast, then \$HEALPIX/bin/synfast}\\
              -- a binpath starting with / (or $\backslash$), $~$ or \$ is interpreted as absolute\\
              -- a binpath starting with ./ is interpreted as relative to current directory\\
              -- all other binpathes are relative to \$HEALPIX

 \item[/double\mytarget{idl:isynfast:double}%
]    if set, I/O is done in double precision \default {single precision I/O}

 \item[fwhm\_arcmin\mytarget{idl:isynfast:fwhm_arcmin}%
=] gaussian beam FWHM in arcmin \default {0}

 \item[/help\mytarget{idl:isynfast:help}%
]      if set, prints extended help

 \item[iseed\mytarget{idl:isynfast:iseed}%
=] integer seed of radom sequence \default {0}

 \item[/keep\_tmp\_files\mytarget{idl:isynfast:keep_tmp_files}%
] if set, temporary files are not discarded at the end of the
                   run

 \item[lmax\mytarget{idl:isynfast:lmax}=, %
       nlmax\mytarget{idl:isynfast:nlmax}=]   maximum multipole simulation \default {2*$\nside$}

 \item[nside\mytarget{idl:isynfast:nside}=, %
       nsmax\mytarget{idl:isynfast:nsmax}=]  Healpix resolution parameter $\nside$

 \item[plmfile\mytarget{idl:isynfast:plmfile}%
=] FITS file containing precomputed Spherical Harmonics (deprecated) \default {no file}

 \item[simul\_type\mytarget{idl:isynfast:simul_type}%
=] 
        1) Temperature only \\
        2) Temperature + polarisation \\
        3) Temperature + 1st derivatives \\
        4) Temperature + 1st \& 2nd derivatives \\
        5) T+P + 1st derivatives \\
        6) T+P + 1st \& 2nd derivates
	\default {2: T+P}

 \item[/silent\mytarget{idl:isynfast:silent}%
]    if set, works silently

 \item[tmpdir\mytarget{idl:isynfast:tmpdir}%
=]      directory in which are written temporary files 
 \default {IDL\_TMPDIR (see IDL documentation)}

 \item[windowfile\mytarget{idl:isynfast:windowfile}%
=]    FITS file containing pixel window 
        \default {determined automatically by back-end routine}.
      Do not set this keyword unless you really know what you are doing

  \item[winfiledir\mytarget{idl:isynfast:winfiledir}%
=]     directory where the pixel windows are to be found 
        \default {determined automatically by back-end routine}.
      Do not set this keyword unless you really know what you are doing

  \end{kwlist}
\end{keywords}  

\begin{codedescription}
{\thedocid\ is an interface to F90 '\htmlref{synfast}{fac:synfast}' F90 facility. It
requires some disk space on which to write the parameter file and the other
temporary files. Most data can be provided/generated as an external FITS
file, or as a memory array.}
\end{codedescription}



\begin{related}
  \begin{sulist}{} %%%% NOTE the ``extra'' brace here %%%%
    \item[idl] version \idlversion or more is necessary to run \thedocid.
    \item[synfast] F90 facility called by \thedocid.
    \item[\htmlref{ialteralm}{idl:ialteralm}] IDL Interface to F90 \htmlref{alteralm}{fac:alteralm}
    \item[\htmlref{ianafast}{idl:ianafast}] IDL Interface to F90 \htmlref{anafast}{fac:anafast} and C++ anafast\_cxx
    \item[\htmlref{iprocess\_mask}{idl:iprocess_mask}] IDL Interface to F90 \htmlref{process\_mask}{fac:process_mask}
    \item[\htmlref{ismoothing}{idl:ismoothing}] IDL Interface to F90 \htmlref{smoothing}{fac:smoothing}
%    \item[\htmlref{isynfast}{idl:isynfast}] IDL Interface to F90 \htmlref{synfast}{fac:synfast}
 \end{sulist}
\end{related}

\begin{example}
{
\begin{tabular}{l} %%%% use this tabular format %%%%
\thedocid, '\$HEALPIX/test/cl.fits', map, fwhm=30, nside=256, /silent  \\
\htmlref{mollview}{idl:mollview}, map, 1, title='I'  \\
\htmlref{mollview}{idl:mollview}, map, 2, title='Q'  \\
\end{tabular}
}
{
will synthetize and plot I and Q  maps consistent with WMAP-1yr best fit power
spectrum and observed with a circular gaussian 30 arcmin beam.
}
\end{example}


