% -*- LaTeX -*-

% PLEASE USE THIS FILE AS A TEMPLATE FOR THE DOCUMENTATION OF YOUR OWN
% FACILITIES: IN PARTICULAR, IT IS IMPORTANT TO NOTE COMMENTS MADE IN
% THE TEXT AND TO FOLLOW THIS ORDERING. THE FORMAT FOLLOWS ONE USED BY
% THE COBE-DMR PROJECT.	
% A.J. Banday, April 1999.




\sloppy



\title{\healpix IDL Facility User Guidelines}
\docid{gnomview} \section[gnomview]{ }
\label{idl:\thedocid}
\docrv{Version 1.2}
\author{Eric Hivon}
\abstract{This document describes the \healpix facility gnomview.}



\begin{facility}
{This IDL facility provides a means to visualise a Gnomonic projection 
(radial projection onto a tangent plane) of
\healpix and COBE Quad-Cube maps in an IDL environment. 
It also offers the possibility to
generate GIF, JPEG, PDF, PNG and Postscript color-coded images of the projected map.
The projected (but not color-coded) data can also be output in FITS files and
IDL arrays.}
{src/idl/visu/gnomview.pro}
\end{facility}

\begin{IDLformat}
{GNOMVIEW, 
\normalsize{
\mylink{idl:mollview:file}{File}, 
[ \mylink{idl:mollview:select}{Select}, ]
[ \mylink{idl:mollview:asinh}{ASINH=}, 
\mylink{idl:mollview:bad_color}{BAD\_COLOR=}, 
\mylink{idl:mollview:bg_color}{BG\_COLOR=}, 
\mylink{idl:mollview:charsize}{CHARSIZE=}, 
\mylink{idl:mollview:charsize}{CHARTHICK=}, 
\mylink{idl:mollview:colt}{COLT=}, 
\mylink{idl:mollview:coord}{COORD=}, 
\mylink{idl:mollview:crop}{/CROP}, 
\mylink{idl:mollview:execute}{EXECUTE=}, 
\mylink{idl:mollview:factor}{FACTOR=}, 
\mylink{idl:mollview:fg_color}{FG\_COLOR=}, 
\mylink{idl:mollview:fits}{FITS=}, 
\mylink{idl:mollview:flip}{/FLIP}, 
\mylink{idl:mollview:gal_cut}{GAL\_CUT=}, 
\mylink{idl:mollview:gif}{GIF=}, 
\mylink{idl:mollview:glsize}{GLSIZE=}, 
\mylink{idl:mollview:graticule}{GRATICULE=}, 
\mylink{idl:mollview:half_sky}{/HALF\_SKY}, 
\mylink{idl:mollview:hbound}{HBOUND=}, 
\mylink{idl:mollview:help}{/HELP}, 
\mylink{idl:mollview:hist_equal}{/HIST\_EQUAL}, 
\mylink{idl:mollview:hxsize}{HXSIZE=}, 
\mylink{idl:mollview:iglsize}{IGLSIZE=}, 
\mylink{idl:mollview:igraticule}{IGRATICULE=}, 
\mylink{idl:mollview:jpeg}{JPEG=}, 
\mylink{idl:mollview:latex}{LATEX=}, 
\mylink{idl:mollview:log}{/LOG}, 
\mylink{idl:mollview:map_out}{MAP\_OUT=}, 
\mylink{idl:mollview:max}{MAX=}, 
\mylink{idl:mollview:min}{MIN=}, 
\mylink{idl:mollview:nested}{/NESTED}, 
\mylink{idl:mollview:no_dipole}{/NO\_DIPOLE}, 
\mylink{idl:mollview:no_monopole}{/NO\_MONOPOLE}, 
\mylink{idl:mollview:nobar}{/NOBAR}, 
\mylink{idl:mollview:nolabels}{/NOLABELS}, 
\mylink{idl:mollview:noposition}{/NOPOSITION}, 
\mylink{idl:mollview:offset}{OFFSET=}, 
\mylink{idl:mollview:outline}{OUTLINE=}, 
\mylink{idl:mollview:pdf}{PDF=}, 
\mylink{idl:mollview:pfonts}{PFONTS=}, 
\mylink{idl:mollview:png}{PNG=}, 
\mylink{idl:mollview:polarization}{POLARIZATION=}, 
\mylink{idl:mollview:preview}{/PREVIEW}, 
\mylink{idl:mollview:ps}{PS=}, 
\mylink{idl:mollview:pxsize}{PXSIZE=}, 
\mylink{idl:mollview:pysize}{PYSIZE=}, 
\mylink{idl:mollview:reso_arcmin}{RESO\_ARCMIN=}, 
\mylink{idl:mollview:retain}{RETAIN=}, 
\mylink{idl:mollview:rot}{ROT=}, 
\mylink{idl:mollview:save}{/SAVE}, 
\mylink{idl:mollview:shaded}{/SHADED}, 
\mylink{idl:mollview:silent}{/SILENT}, 
\mylink{idl:mollview:stagger}{STAGGER=}, 
\mylink{idl:mollview:subtitle}{SUBTITLE=}, 
\mylink{idl:mollview:titleplot}{TITLEPLOT=}, 
\mylink{idl:mollview:transparent}{TRANSPARENT=}, 
\mylink{idl:mollview:truecolors}{TRUECOLORS=}, 
\mylink{idl:mollview:units}{UNITS=}, 
\mylink{idl:mollview:window}{WINDOW=}, 
\mylink{idl:mollview:xpos}{XPOS=}, 
\mylink{idl:mollview:ypos}{YPOS=}]
}
}
\end{IDLformat}

%\ mylink: to avoid automatic processing by make_internal_links.sh

\begin{qualifiers}
  \begin{qulist}{} %%%% NOTE the ``extra'' brace here %%%%
\item [{\  }] For a full list of qualifiers see \htmlref{mollview}{idl:mollview}
  \end{qulist}
\end{qualifiers}

\begin{keywords}
  \begin{kwlist}{} %%%% NOTE the ``extra'' brace here %%%%
\item [{\  }] For a full list of keywords see \htmlref{mollview}{idl:mollview}
  \end{kwlist}
\end{keywords}

% \begin{qualifiers}
%   \begin{qulist}{} %%%% NOTE the ``extra'' brace here %%%%
%  	\item[{File}] 
%           by default :          name of a FITS file containing 
%                the healpix map in an extension or in the image field \\
%           if Online is set :    name of a variable containing
%                the healpix map \\
%           if Save is set   :    name of an IDL saveset file containing
%                the healpix map stored under the variable  {\tt data}
%  	\\
% 	\nodefault

%        \item[{Select}]
% 		  column of the BIN FITS table to be plotted  \\
%                can be either -- a name : value given in TTYPEi of the FITS file \\
%                         NOT case sensitive and can be truncated, \\
%                         (only letters, digits and underscore are valid) \\
%                or -- an integer        : number i of the column
%                             containing the data, starting with 1
%                    \default 1
%                (see the Examples below)

%  	\item[{Colt=}]  color table number, in [0,40]
%         	\default {33 (Blue-Red)}

%       	\item[{Coord=}] vector with 1 or 2 elements describing the
%           coordinate system of the map \hfill\newline
%                 either 'C' or 'Q' : Celestial2000 = eQuatorial, \\
%                        'E'        : Ecliptic, \\
%                        'G'        : Galactic  \\
%                if coord = ['x','y'] the map is rotated from system 'x' to system 'y' \\
%                if coord = ['y'] the map is rotated to coordinate system 'y' (with the
%                original system assumed to be Galactic unless indicated otherwise in the
%                 file) \\
%                   \seealso Rot

%        \item[{Fits=}] string containing the name of an output fits file with
%         the rectangular map in the primary image \\
%  	      if set to 1            : output the plot in
%         plot\_gnomonic.fits \\
%  	      if set to a file name  : output the plot in that file 
%  	\default {0: no .FITS done}

% 	\item[{Gif=}] string containing the name of a .GIF output \\
% 	      if set to 1            : output the plot in plot\_mollweide.gif \\
% 	      if set to a file name  : output the plot in that file \\
% 	\default {0: no .GIF done} \\              \seealso Crop, Ps and Preview

%  	\item[{Graticule=}] if set, puts a graticule with delta\_long
%          = delta\_lat = 5 degrees \\
%          if set to a scalar real $>$ 0 delta\_long = delta\_lat = graticule \\
%          if set to [x,y] with x,y $>$ 0 then delta\_long = x and delta\_lat = y
% 	\default {0 [no graticule]}

%  	\item[{Hxsize=}] horizontal dimension (in cm) of the Postscript printout
%     		\default {26 cm ~ 10 in} \\               \seealso Pxsize

%  	\item[{Max=}] Set the maximum value for the plotted signal \default{is to use the actual signal maximum}.
%  	\item[{Min=}] Set the minimum value for the plotted signal \default{is to use the actual signal minimum}.

% 	\item[{Ps=}] 
% 	      if set to 1            : output the plot in plot\_mollweide.ps \\
% 	      if set to a file name  : output the plot in that file \\
% 		\default 0 \\
%                \seealso Preview

%  	\item[{Pxsize=}] set the number of horizontal screen\_pixels or postscript\_color\_dots of the plot
%     		\default {800} 
%     		(useful for high definition color printer)

%  	\item[{Pysize=}] set the number of vertical screen\_pixels or postscript\_color\_dots of the plot
%     		\default {Pxsize} 

%        	\item[{Reso\_arcmin=}] resolution of the gnomonic map in arcmin
%        \default{1.5}

%  	\item[{Rot=}] vector with 1, 2 or 3 elements specifing the rotation angles in DEGREE
%                to apply to the map in the 'output' coordinate system (see Coord)
%              = ( lon0, [lat0, rat0])  \\
%                lon0 : longitude of the point to be put at the center of the plot
% 		       the longitude increases Eastward, ie to the left of the plot
%  		      \default 0 \\
%                lat0 : latitude of the point to be put at the center of the plot
%  		      \default 0 \\
%                rot0 : anti clockwise rotation to apply to the sky around the
%                      center (lon0, lat0) before projecting
%                      \default 0 

%  	\item[{Subtitle=}] String containing the subtitle to the plot\\ \seealso Titleplot

%  	\item[{Titleplot=}] String containing the title of the plot, 
%      		if not set the title will be File\\ \seealso Subtitle

% 	\item[{Units=}] String containing the units, to be put on the right
% 		hand side of the color bar, overrides the value read from the input file,
% 		if any\\ \seealso Nobar

% 	\item[{Xpos=}] The X position on the screen of the lower left corner
% 	        of the window, in device coordinate

% 	\item[{Ypos=}] The Y position on the screen of the lower left corner 
%                of the window, in device coordinate

%   \end{qulist}
% \end{qualifiers}

% \begin{keywords}
%   \begin{kwlist}{} %%% extra brace
%        \item[{Crop}] if set the GIF file only contains the mollweide map and
%                not the title, color bar, ... \\
%                 \seealso Gif

% 	\item[{Hist\_Equal}]  if set,     uses a histogram equalized color mapping
% 			(useful for non gaussian data field)
% 		\default {uses linear color mapping and 
%                      		puts the level 0 in the middle
%                      		of the color scale (ie, green for Blue-Red)
% 				unless Min and Max are not symmetric}\\
%                      \seealso Log

%  	\item[{Log}] display the log of map. This is intended for
%  	application to positive definite maps only, eg. Galactic foreground emission templates \\	\seealso Hist

% 	\item[{Nested}] specify that the online file is ordered in the nested scheme

%  	\item[{Nobar}] if set, no color bar is present

% 	\item[{Nolabels}] if set, no color bar label (min and max) is present, \default
% 	{labels are present}

%  	\item[{Online}] if set, you can put directly a \healpix vector or an IDL
%     		structure on File (useful when the data to be plotted are already
%     		available on line)
% 		N.B. : the content of File is NOT altered in the
% 		process, \\ **  can not be used with /SAVE  **

% 	\item[{Preview}] if set, there is a 'ghostview' preview of the
% 	        postscript file or a 'xv' preview of the gif file\\
% 	 \seealso Gif and Ps

%  	\item[{Save}] if set, tells that File is in IDL saveset format, 
%     		the variable saved should be DATA \\
%                  ** can not be used with /ONLINE **

%   \end{kwlist}
% \end{keywords}

\renewcommand{\projfullname}{a gnomic (or gnomonic)}
\begin{codedescription}
{
% to be included in codedesciption part of 
% cartview_idl.tex, mollview_idl.tex
% with
%\renewcommand{\projfullname}{a Mollweide}
%\begin{codedescription}
%{
% to be included in codedesciption part of 
% cartview_idl.tex, mollview_idl.tex
% with
%\renewcommand{\projfullname}{a Mollweide}
%\begin{codedescription}
%{
% to be included in codedesciption part of 
% cartview_idl.tex, mollview_idl.tex
% with
%\renewcommand{\projfullname}{a Mollweide}
%\begin{codedescription}
%{\input{mollview_description_idl}}
%\end{codedescription}
{\thedocid{} reads in a \healpix sky map in FITS format and generates
\projfullname \ projection of it, that can be visualized on the screen or
exported in a GIF, JPEG, PNG, PDF or Postscript file. \thedocid{}  allows the selection of
the coordinate system, map size, color table, color bar inclusion,
linear, log, hybrid or histogram equalised color scaling, 
maximum and 
minimum range for the plot, plot-title {\it etc}. It also allows the representation of the
polarization field.}
}
%\end{codedescription}
{\thedocid{} reads in a \healpix sky map in FITS format and generates
\projfullname \ projection of it, that can be visualized on the screen or
exported in a GIF, JPEG, PNG, PDF or Postscript file. \thedocid{}  allows the selection of
the coordinate system, map size, color table, color bar inclusion,
linear, log, hybrid or histogram equalised color scaling, 
maximum and 
minimum range for the plot, plot-title {\it etc}. It also allows the representation of the
polarization field.}
}
%\end{codedescription}
{\thedocid{} reads in a \healpix sky map in FITS format and generates
\projfullname \ projection of it, that can be visualized on the screen or
exported in a GIF, JPEG, PNG, PDF or Postscript file. \thedocid{}  allows the selection of
the coordinate system, map size, color table, color bar inclusion,
linear, log, hybrid or histogram equalised color scaling, 
maximum and 
minimum range for the plot, plot-title {\it etc}. It also allows the representation of the
polarization field.}
}
\end{codedescription}

%\include{mollview_idl_related}
\begin{related}
  \begin{sulist}{} %%%% NOTE the ``extra'' brace here %%%%
  \item[{\ }] see \htmlref{mollview}{idl:mollview}
  \end{sulist}
\end{related}

\begin{examples}{1}
{
\begin{tabular}{l} %%%% use this tabular format %%%%
gnomview,  'planck100GHZ-LFI.fits', rot=[160,-30], reso\_arcmin=2., \$ \\
\hspace{1em}           pxsize = 500., \$ \\
\hspace{1em}           title='Simulated Planck LFI Sky Map at 100GHz',  \$ \\
\hspace{1em}           min=-100,max=100\\
\end{tabular}
}
{gnomview reads in the map 'planck100GHZ-LFI.fits' and generates
an output image of the size of 500$\times$500 screen pixels, 
with a resolution of 2 arcmin/screen pixel at the center.
The temperature scale has been set to lie between $\pm$ 100, and the units will
show as $\mu$K.
The title 'Simulated Planck
LFI Sky Map at 100GHz' has been appended to the image. 
The map is centered at ($l=$160, $b=$-30) }
\end{examples}

\begin{examples}{2}
{
\begin{tabular}{l} %%%% use this tabular format %%%%

map  = findgen(48) \\
triangle= create\_struct('coord','G','ra',[0,80,0],'dec',[40,45,65]) \\
gnomview,map,/online,res=25,graticule=[45,30],rot=[10,20,30],\$ \\
\hspace{1em}	   title='Gnomic projection',subtitle='gnomview', \$ \\
\hspace{1em}           outline=triangle \\
\end{tabular}
}
{makes a gnomic projection of map (see Figure~\ref{fig:plot_visu}b on
page~\pageref{page:plot_visu}) after an arbitrary rotation, with a graticule grid
(with a 45$^o$ step in longitude and 30$^o$ in latitude) and an arbitrary triangular outline}
\end{examples}
