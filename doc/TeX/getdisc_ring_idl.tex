% -*- LaTeX -*-


\renewcommand{\facname}{{getdisc\_ring }}
\renewcommand{\FACNAME}{{GETDISC\_RING }}
\sloppy



\title{\healpix IDL Facility User Guidelines}
\docid{\facname} \section[\facname]{ }
\label{idl:getdisc_ring}
\docrv{Version 1.0}
\author{Eric Hivon}
\abstract{This document describes the \healpix IDL facility \facname.}




\begin{facility}
% {This IDL facility provides a means find the pixel indexs of all pixels within
% an angular distance radius from a defined center.
% }
{This routine is obsolete. Use \htmlref{query\_disc}{idl:query_disc} instead.
}
{src/idl/toolkit/getdisc\_ring.pro}
\end{facility}

% \begin{IDLformat}
% {\facname, Nside, Vector0, Radius, Listpix, [Nlist, DEG=]}
% \end{IDLformat}

% \begin{qualifiers}
%   \begin{qulist}{} %%%% NOTE the ``extra'' brace here %%%%
%     \item[Nside] \healpix resolution parameter used to index the pixel list (scalar integer)
%     \item[Vector0] position vector of the disc center (3 elements vector)
%           NB : the norm of Vector0 does not have to be one, what is
%           consider is the intersection of the sphere with the line of
%           direction Vector0.
%     \item[Radius] radius of the disc (in radians, unless DEG is set), (scalar
%     real)
%     \item[Listpix] on output: list of RING-ordered index for the pixels found 
%     within a radius Radius of the position defined by vector0.
%      (=-1 if the radius is too small and no pixel is found)
%     \item[Nlist] on output: number of pixels in Listpix (=0 if no pixel is found).
%   \end{qulist}
% \end{qualifiers}

% \begin{keywords}
%   \begin{kwlist}{} %%% extra brace
%     \item[DEG =] if set Radius is in degrees instead of radians
%   \end{kwlist}
% \end{keywords}  

% \begin{codedescription}
% {\facname finds the pixels within the given disc in a selective way WITHOUT
% scanning all the sky pixels.}
% \end{codedescription}



% \begin{related}
%   \begin{sulist}{} %%%% NOTE the ``extra'' brace here %%%%
%     \item[idl] version \idlversion or more is necessary to run \facname.
%     \item[ang2pix, pix2ang] conversion between angles and pixel index
%     \item[vec2pix, pix2vec] conversion between vector and pixel index
%   \end{sulist}
% \end{related}

% \begin{example}
% {
% \begin{tabular}{ll} %%%% use this tabular format %%%%
% \facname, & 256L, [.5,.5,0.], 10., listpix, nlist, /Deg
% \end{tabular}
% }
% {
% On return listpix contains the index of the (6096) pixels within 10 deg from
% the point on the sphere having the direction [.5,.5,0.].
% The pixel indices correspond to the RING scheme with resolution 256.
% }
% \end{example}


