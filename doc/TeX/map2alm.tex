
\sloppy


\title{\healpix Fortran Subroutines Overview}
\docid{map2alm*} \section[map2alm*]{ }
\label{sub:map2alm}
\docrv{Version 2.1}
\author{Frode K.~Hansen, Eric Hivon}
\abstract{This document describes the \healpix Fortran90 subroutine MAP2ALM*.}

\begin{facility}
{This routine is a wrapper to 5 internal routines:map2alm\_sc,
map2alm\_sc\_pre, map2alm\_pol, map2alm\_pol\_pre1,
map2alm\_pol\_pre2. These routines analyse a \healpix {\em RING ordered} map and return
$a_{\ell m}^T$ (and if specified $a_{\ell m}^E$ and $a_{\ell m}^B$) values up to
the desired order in $\ell$ (maximum 3*$\nside$). The different
routines are called depending on what parameters are passed. Some
routines analyse with or without precomputed harmonics and some with
or without polarisation. }
{\modAlmTools}
\end{facility}

\begin{f90format}
{\mylink{sub:map2alm:nsmax}{nsmax}%
, \mylink{sub:map2alm:nlmax}{nlmax}%
, \mylink{sub:map2alm:nmmax}{nmmax}%
, \mylink{sub:map2alm:map_TQU}{map\_TQU}%
, \mylink{sub:map2alm:alm_TGC}{alm\_TGC}%
, \mylink{sub:map2alm:zbounds}{zbounds}%
, \mylink{sub:map2alm:w8ring_TQU}{w8ring\_TQU}%
 [, \mylink{sub:map2alm:plm}{plm}%
]}
\end{f90format}

\begin{arguments}
{
\begin{tabular}{p{0.4\hsize} p{0.05\hsize} p{0.05\hsize} p{0.40\hsize}} \hline  
\textbf{name~\&~dimensionality} & \textbf{kind} & \textbf{in/out} & \textbf{description} \\ \hline
                   &   &   &                           \\ %%% for presentation
nsmax\mytarget{sub:map2alm:nsmax} & I4B & IN & the $\nside$ value of the map to analyse. \\
nlmax\mytarget{sub:map2alm:nlmax} & I4B & IN & the maximum $\ell$ value for the analysis. \\
nmmax\mytarget{sub:map2alm:nmmax} & I4B & IN & the maximum $m$ value for the analysis. \\
map\_TQU\mytarget{sub:map2alm:map_TQU}(0:12*nsmax**2-1) & SP/ DP & IN & if only the temperature map is to be analyse, the map-array should be passed with this rank. \\ 
map\_TQU(0:12*nsmax**2-1, 1:3) & SP/ DP & IN & if both temperature an polarisation maps are to be analysed, the map array should have this rank, where the second index is (1,2,3) corresponding to (T,Q,U). \\ 
\end{tabular}
\begin{tabular}{p{0.4\hsize} p{0.05\hsize} p{0.05\hsize} p{0.40\hsize}}   \hline  
alm\_TGC\mytarget{sub:map2alm:alm_TGC}(1:p, 0:nlmax, 0:nmmax) & SPC/ DPC & OUT & The $a_{\ell m}$ values output from the analysis. p is 1 or 3 dependent on wether polarisation is included or not. In the former case, the first index is (1,2,3) corresponding to (T,E,B). \\
%% cos\_theta\_cut & DP & IN & the cosine of the cutting angle for a cut sky
%%    analysis. {\bf Note}: in order to have no cut at all cos\_theta\_cut needs to be
%%    set to a {\em negative} value, as cos\_theta\_cut=$\cos(90^\circ)=0$ still
%%    removes the equatorial ring.\\
zbounds\mytarget{sub:map2alm:zbounds}(1:2) & DP & IN & section of the map on which to perform the $a_{\ell m}$
                   analysis, expressed in terms of $z=\sin(\mathrm{latitude}) =
                   \cos(\theta).$ %zbounds_sub.tex:  for alm2map, map2alm, remove_dipole: describe processed area
%zbounds2_sub.tex: for apply_mask: describe pixels set to 0 (=unprocessed area)
If zbounds(1)$<$zbounds(2), it is
performed {\em on} the strip zbounds(1)$<z<$zbounds(2); if not,
it is performed {\em outside} the strip
%  zbounds(2)$<z<$zbounds(1). % OLD
zbounds(2)$\le z \le$zbounds(1). % NEW ??
% The whole sphere is treated if \texttt{zbounds=(/-1.0\_dp, 1.0\_dp/)} or if 
% it is absent.
If absent, the whole map is processed.
\\
w8ring\_TQU\mytarget{sub:map2alm:w8ring_TQU}(1:2*nsmax, 1:p) & DP & IN & ring weights for quadrature corrections. If ring weights are not used, this array should be 1 everywhere. p is 1 for a temperature analysis and 3 for (T,Q,U). \\
{\small{plm(0:(nlmax+1)(nlmax+2)nsmax-1)}}\mytarget{sub:map2alm:plm}, \hskip 6cm OPTIONAL & DP & IN & If this optional matrix is passed with this rank, precomputed $P_{\ell m}(\theta)$ are used instead of recursion. Note that since version 2.20 this feature has become obsolete
because of algorithm optimizations.\\ 
{\small{plm(0:(nlmax+1)(nlmax+2)nsmax-1,1:3)}}, \hskip 6cm OPTIONAL & DP & IN & If this optional matrix is passed with this rank, precomputed $P_{\ell m}(\theta)$ AND precomputed tensor harmonics are used instead of recursion. \\
\end{tabular}
}
\end{arguments}
%%\newpage

\begin{example}
{
use \htmlref{healpix\_types}{sub:healpix_types}\\
use alm\_tools\\
use pix\_tools\\
integer(\htmlref{i4b}{sub:healpix_types}) :: nside, lmax \\
real(\htmlref{dp}{sub:healpix_types}), allocatable, dimension(:,:) :: dw8 \\
real(dp), dimension(2) :: z \\
real(\htmlref{sp}{sub:healpix_types}), allocatable, dimension(:,:) :: map \\
complex(\htmlref{spc}{sub:healpix_types}), allocatable, dimension(:,:,:) :: alm \\
nside = 256 \\
lmax = 512 \\
allocate(dw8(1:2*nside, 1:3)) \\
allocate(map(0:nside2npix(nside)-1,1:3)) \\
allocate(alm(1:3, 0:lmax, 0:lmax)\\
dw8 = 1.0\_dp \\
z = sin(10.0\_dp * \htmlref{DEG2RAD}{sub:healpix_types}) \\
call map2alm(nside, lmax, lmax, map, alm, ($\backslash$ z, -z $\backslash$) , dw8, plm(0:(lmax+1)*(lmax+2)*nside-1))  \\
}
{
Analyses temperature and polarisation maps passed in map. The map has
an $\nside$ of 256, and the analysis is performed up
to 512 in $\ell$ and $m$. The resulting $a_{\ell m}$ coefficients for
temperature and polarisation are returned in alm. A $10^\circ$ cut on
each side of the equator is applied. Uniform weights are used. Since
the optional plm array is provided with rank one, precomputed scalar $P_{\ell m}(\theta)$ are
used while tensor harmonics are computed with a recursion.
}
\end{example}

\begin{modules}
  \begin{sulist}{} %%%% NOTE the ``extra'' brace here %%%%
  \item[ring\_analysis] Performs FFT for the ring analysis.
  \item[\textbf{misc\_util}] module, containing:
  \item[\htmlref{assert\_alloc}{sub:assert}] routine to print error message when an array is not
  properly allocated		
  \end{sulist}
%Note: Starting with \htmlref{version 2.20}{sub:new2p20}, {\tt libpsht} routines will be called when precomputed $P_{\ell m}$ are not provided.
Note: Starting with \htmlref{version 3.10}{sub:new3p10}, {\tt libsharp} routines will be called when precomputed $P_{\ell m}$ are not provided.
\end{modules}

\begin{related}
  \begin{sulist}{} %%%% NOTE the ``extra'' brace here %%%%
  \item[anafast] executable using \thedocid{} to analyse maps.
  \item[\htmlref{alm2map}{sub:alm2map}] routine performing the inverse transform
of \thedocid.
  \item[\htmlref{dump\_alms}{sub:dump_alms}] write $a_{\ell m}$ coefficients
computed by \thedocid{} into a FITS file
  \item[\htmlref{map2alm\_iterative}{sub:map2alm_iterative}] similar to
\thedocid{} with iterative scheme.
  \end{sulist}
\end{related}

\rule{\hsize}{2mm}

\newpage
