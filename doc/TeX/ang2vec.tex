

\sloppy


%%%\title{\healpix Fortran Subroutines Overview}
\docid{ang2vec} \section[ang2vec]{ }
\label{sub:ang2vec}
\docrv{Version 1.1}
\author{E. Hivon}
\abstract{This document describes the \healpix Fortran90 subroutine ANG2VEC.}

\begin{facility}
{Routine to convert the position angles $(\theta,\phi)\myhtmlimage{}$ of a point on the sphere 
into its 3D position vector $(x,y,z)$ with
$x = \sin\theta\cos\phi\myhtmlimage{}$, $y=\sin\theta\sin\phi\myhtmlimage{}$, $z=\cos\theta\myhtmlimage{}$. 
}
{\modPixTools}
\end{facility}

\begin{f90format}
{\mylink{sub:ang2vec:theta}{theta}%
, \mylink{sub:ang2vec:phi}{phi}%
, \mylink{sub:ang2vec:vector}{vector}%
}
\end{f90format}


\begin{arguments}
{
\begin{tabular}{p{0.3\hsize} p{0.05\hsize} p{0.1\hsize} p{0.45\hsize}} \hline  
\textbf{name~\&~dimensionality} & \textbf{kind} & \textbf{in/out} & \textbf{description} \\ \hline
                   &   &   &                           \\ %%% for presentation
theta\mytarget{sub:ang2vec:theta} & DP & IN & colatitude in radians measured southward from north pole (in
    $[0,\ \pi]\myhtmlimage{}$). \\
phi\mytarget{sub:ang2vec:phi}   & DP & IN & longitude in radians measured eastward (in $[0,\ 2\pi]\myhtmlimage{}$).\\
vector\mytarget{sub:ang2vec:vector}(3) & DP & OUT & three dimensional cartesian position vector
                   $(x,y,z)$ normalised to unity. The north pole is $(0,0,1)$
\end{tabular}
}
\end{arguments}

% \begin{example}
% {
% call ang2vec(theta,phi,vector) \\
% }
% {
% }
% \end{example}

\begin{related}
  \begin{sulist}{} %%%% NOTE the ``extra'' brace here %%%%
  %\item[\htmlref{ang2vec}{sub:ang2vec}] converts the position angles of a point on the sphere 
  %into its 3D position vector.
  \item[\htmlref{angdist}{sub:angdist}] computes the angular distance between 2 vectors
  \item[\htmlref{vec2ang}{sub:vec2ang}] converts the 3D position vector of point into its position
  angles on the sphere.
  \item[\htmlref{vect\_prod}{sub:vect_prod}] computes the vector product between two 3D vectors
  \end{sulist}
\end{related}

\rule{\hsize}{2mm}

