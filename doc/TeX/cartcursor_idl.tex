% -*- LaTeX -*-

% PLEASE USE THIS FILE AS A TEMPLATE FOR THE DOCUMENTATION OF YOUR OWN
% FACILITIES: IN PARTICULAR, IT IS IMPORTANT TO NOTE COMMENTS MADE IN
% THE TEXT AND TO FOLLOW THIS ORDERING. THE FORMAT FOLLOWS ONE USED BY
% THE COBE-DMR PROJECT.	
% A.J. Banday, April 1999.




\sloppy



\title{\healpix IDL Facility User Guidelines}
\docid{cartcursor} \section[cartcursor]{ }
\label{idl:\thedocid}
\docrv{Version 1.2}
\author{Eric Hivon}
\abstract{This document describes the \healpix facility cartcursor.}


\begin{facility}
{This IDL facility provides a point-and-click interface for finding
the astronomical location, value and pixel index of the pixels nearest 
to the pointed position on a cartesian projection of a \healpix map.}
{src/idl/visu/cartcursor.pro}
\end{facility}

\begin{IDLformat}
{CARTCURSOR, [cursor\_type=, file\_out=]}
\end{IDLformat}

\begin{qualifiers}
\hbox{\hspace{5cm}		see \htmlref{mollcursor}{idl:mollcursor}}
\end{qualifiers}

\begin{codedescription}
{cartcursor should be called immediately after cartview. It gives the longitude,
latitude, map value and pixel number
corresponding to the cursor position in the window containing the map generated
by orthview. For more details, or in case
of problems under {\bf Mac OS X}, see \htmlref{mollcursor}{idl:mollcursor}.}
\end{codedescription}



\begin{related}
\hbox{\hspace{5cm}	see \htmlref{mollcursor}{idl:mollcursor}}
\end{related}


\begin{example}
{
\begin{tabular}{ll} %%%% use this tabular format %%%%
cartcursor & \ 
\end{tabular}
}
{After cartview has read in a map and generated
its cartesian projection, cartcursor is run to determine the
position and flux of bright synchrotron sources, for example.}
\end{example}


