
\sloppy


%%%\title{\healpix Fortran Subroutines Overview}
\docid{ring\_synthesis} \section[ring\_synthesis]{ }
\label{sub:ring_synthesis}
\docrv{Version 1.1}
\author{Frode K.~Hansen}
\abstract{This document describes the \healpix Fortran90 subroutine RING\_SYNTHESIS.}

\begin{facility}
{}
{\modAlmTools}
\end{facility}

\begin{f90format}
{\mylink{sub:ring_synthesis:nsmax}{nsmax}%
, \mylink{sub:ring_synthesis:nlmax}{nlmax}%
, \mylink{sub:ring_synthesis:nmmax}{nmmax}%
, \mylink{sub:ring_synthesis:datain}{datain}%
, \mylink{sub:ring_synthesis:nph}{nph}%
, \mylink{sub:ring_synthesis:dataout}{dataout}%
, \mylink{sub:ring_synthesis:kphi0}{kphi0}%
}
\end{f90format}

\begin{arguments}
{
\begin{tabular}{p{0.4\hsize} p{0.05\hsize} p{0.1\hsize} p{0.35\hsize}} \hline  
\textbf{name~\&~dimensionality} & \textbf{kind} & \textbf{in/out} & \textbf{description} \\ \hline
                   &   &   &                           \\ %%% for presentation
nsmax\mytarget{sub:ring_synthesis:nsmax} & I4B & IN & $\nside$ of the map.  \\
nlmax\mytarget{sub:ring_synthesis:nlmax} & I4B & IN & Maximum $\ell$ of the analysis. \\
nmmax\mytarget{sub:ring_synthesis:nmmax} & I4B & IN & Maximum $m$ of the analysis. \\
nph\mytarget{sub:ring_synthesis:nph} & I4B & IN & The number of points on the ring. \\ 
datain\mytarget{sub:ring_synthesis:datain}(0:nmmax) & DPC & IN & Fourier components as computed from the $a_{lm}$. \\
dataout\mytarget{sub:ring_synthesis:dataout}(0:nph-1) & DP & OUT & Synthesized function values on the ring. \\
kphi0\mytarget{sub:ring_synthesis:kphi0} & I4B & IN &  0 if the first pixel on the ring is  at
                   $\phi=0$; 1 otherwise. \\
\end{tabular}
}
\end{arguments}

\begin{example}
{
call ring\_synthesis(64,128,128,datain,8,dataout,1)   \\
}
{
This computes the inverse (complex-to-real) Fast Fourier Transform for
the second ring from the pole, containing $8$ pixels, for a map
resolution of $\nside=64$. $128$ complex Fourier
compoments contribute to these 8 pixels. The value $kphi0=1$ specifies
that a phase factor needed to be applied to correctly
rotate the ring into position on the \healpix grid.
}
\end{example}

\begin{modules}
  \begin{sulist}{} %%%% NOTE the ``extra'' brace here %%%%
  \item[\textbf{healpix\_fft}] module.
  \end{sulist}
\end{modules}

\begin{related}
  \begin{sulist}{} %%%% NOTE the ``extra'' brace here %%%%
  \item[\htmlref{ring\_analysis}{sub:ring_analysis}] Forward transform, used in
  \htmlref{map2alm}{sub:map2alm} and anafast 
  \end{sulist}
\end{related}

\rule{\hsize}{2mm}

\newpage
