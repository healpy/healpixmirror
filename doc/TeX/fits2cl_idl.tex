% -*- LaTeX -*-


\sloppy

\title{\healpix IDL Facility User Guidelines}
\docid{fits2cl} \section[fits2cl]{ }
\label{idl:fits2cl}
\docrv{Version 1.2}
\author{Eric Hivon, Anthony J.~Banday}
\abstract{This document describes the \healpix IDL facility fits2cl.}

\begin{facility}
{This IDL facility provides a means to
read from a FITS file an ascii or binary table extension containing power 
spectrum ($C(l)$) or spherical harmonics ($a_{l m}$) coefficients, and returns
the corresponding power spectrum ($C(l) = \sum_m a_{lm}a^*_{lm} / (2l+1)$). Reads primary and extension headers if
required. The facility is intended to enable the user to read the
output from the \healpix facility \textbf{anafast}.
}
{src/idl/fits/fits2cl.pro}
\end{facility}

\begin{IDLformat}
{\thedocid, 
\mylink{idl:fits2cl:cl_array}{cl\_array}, 
[%
\mylink{idl:fits2cl:fitsfile}{fitsfile}, %
\mylink{idl:fits2cl:extension}{EXTENSION=} ,
\mylink{idl:fits2cl:hdr}{HDR=} ,
\mylink{idl:fits2cl:help}{/HELP}, 
\mylink{idl:fits2cl:interactive}{/INTERACTIVE}, 
\mylink{idl:fits2cl:llfactor}{LLFACTOR=}, 
\mylink{idl:fits2cl:multipoles}{MULTIPOLES=},  
\mylink{idl:fits2cl:planck1}{/PLANCK1=}, 
%\mylink{idl:fits2cl:planck2}{/PLANCK2=}, 
\mylink{idl:fits2cl:rshow}{/RSHOW}, 
\mylink{idl:fits2cl:show}{/SHOW}, 
\mylink{idl:fits2cl:silent}{/SILENT=}, 
\mylink{idl:fits2cl:wmap1}{/WMAP1=}, 
\mylink{idl:fits2cl:wmap5}{/WMAP5=}, 
\mylink{idl:fits2cl:wmap7}{/WMAP7=}, 
\mylink{idl:fits2cl:xhdr}{XHDR=}%
]}
\end{IDLformat}

\begin{qualifiers}
  \begin{qulist}{} %%%% NOTE the ``extra'' brace here %%%%
    \item[cl\_array] \mytarget{idl:fits2cl:cl_array}%
      real array of $C_\ell$ coefficients read or computed from the
      file. The output dimension depends on the contents of the file. 
	This has dimension either (lmax+1,9) given in the sequence T E B
      TxE TxB ExB ExT BxT BxE {\bf or}
       (lmax+1,6) given in the sequence T E B
      TxE TxB ExB {\bf or} (lmax+1,4) for T E B TxE {\bf or} (lmax+1) for T
    alone. \\
     The convention for the power spectrum is that it is not
      normalised by the Harrison-Zeldovich (flat) spectrum.
%
    \item[fitsfile] \mytarget{idl:fits2cl:fitsfile}%
    String containing the name of the FITS file to be read. The
    file contains either $C(l)$ power spectra or $a_{l m}$ coefficients. In either
    cases, $C(l)$ is returned. If {\tt fitsfile} is not set, then
\mylink{idl:fits2cl:planck1}{\tt /PLANCK1},
%\mylink{idl:fits2cl:planck2}{\tt /PLANCK2},
\mylink{idl:fits2cl:wmap1}{\tt /WMAP1},
\mylink{idl:fits2cl:wmap5}{\tt /WMAP5} or
\mylink{idl:fits2cl:wmap7}{\tt /WMAP7} 
must be set.
  \end{qulist}
\end{qualifiers}

\begin{keywords}
  \begin{kwlist}{} %%% extra brace
       \item[EXTENSION=]\mytarget{idl:fits2cl:extension}%
	extension unit to be read from FITS file: 
 either its 0-based ID number (ie, 0 for first extension {\em after} primary array) 
 or the case-insensitive value of its EXTNAME keyword.
    \item[HDR =] \mytarget{idl:fits2cl:hdr}%
	String array containing on output the primary header
      read from the FITS file. 
    \item[/HELP] \mytarget{idl:fits2cl:help}%
	If set, produces an extended help message (using the doc\_library
    IDL command). 
    \item[/INTERACTIVE] \mytarget{idl:fits2cl:interactive}%
	If set, the plots generated by \mylink{idl:fits2cl:show}{\tt /SHOW} and \mylink{idl:fits2cl:rshow}{\tt
/RSHOW} options are produced using iPlot routine, allowing 
           for interactive cropping, zooming and annotation of the plots. This
           requires IDL 6.4 or newer to work properly.
    \item[LLFACTOR =] \mytarget{idl:fits2cl:llfactor}%
	vector containing on output the factor $l(l+1)/2\pi$ which is often
          applied to $C(l)$ to flatten it for plotting purposes
    \item[MULTIPOLES =] \mytarget{idl:fits2cl:multipoles}%
	vector containing on output the multipoles
    $\ell$ for which the power spectra are provided. They are either\\
           - read from the file (1st column in the Planck format),\\
          - or generated by the routine (assuming that all
               multipoles from 0 to lmax included are provided).
%
    \item[/PLANCK1] \mytarget{idl:fits2cl:planck1}%
           If set, and \mylink{idl:fits2cl:fitsfile}{\tt fitsfile} 
           is not provided, then a Planck 2013+external data best fit
          model
%
%     \item[/PLANCK2] \mytarget{idl:fits2cl:planck2}%
%            If set, and \mylink{idl:fits2cl:fitsfile}{\tt fitsfile} 
%            is not provided, then a Planck 2015+external data best fit
%           model
% (\htmlref{!healpix}{idl:init_healpix}.path.test+\-'??') 
%           defined up to lmax=??, is read.\\
%           See !healpix.path.test+'README' for details
%
    \item[/RSHOW] \mytarget{idl:fits2cl:rshow}%
	If set, the raw power spectra $C(l)$ read from the file are plotted
    \item[/SHOW] \mytarget{idl:fits2cl:show}%
	If set, the rescaled power spectra $l(l+1)C(l)/2\pi$ are plotted
    \item[/SILENT] \mytarget{idl:fits2cl:silent}%
	If set, no message is issued during normal execution
%
    \item[/WMAP1] \mytarget{idl:fits2cl:wmap1}%
           If set, and \mylink{idl:fits2cl:fitsfile}{\tt fitsfile} 
           is not provided, then one WMAP-1yr best fit
          model
(\htmlref{!healpix}{idl:init_healpix}.path.test+\-'wmap\_lcdm\_pl\_model\_yr1\_v1.fits'
which currently matches !healpix.path.test+'cl.fits') 
          defined up to lmax=3000, is read.\\
          See !healpix.path.test+'README' for details
%
    \item[/WMAP5] \mytarget{idl:fits2cl:wmap5}%
           If set, and \mylink{idl:fits2cl:fitsfile}{\tt fitsfile}
           is not provided, then one WMAP-5yr best fit
          model (\htmlref{!healpix}{idl:init_healpix}.path.test+\-'wmap\_lcdm\_sz\_lens\_wmap5\_cl\_v3.fits') 
          defined up to lmax=2000, is read.\\
          See !healpix.path.test+'README' for details
%
    \item[/WMAP7] \mytarget{idl:fits2cl:wmap7}%
           If set, and \mylink{idl:fits2cl:fitsfile}{\tt fitsfile}
           is not provided, then one WMAP-7yr best fit
          model (\htmlref{!healpix}{idl:init_healpix}.path.test+\-'wmap\_lcdm\_sz\_lens\_wmap7\_cl\_v4.fits') 
          defined up to lmax=3726, is read.\\
          {\bf Note:} As opposed to the other WMAP spectra mentionned above, it includes
             a non-vanishing B (or CURL) power spectrum 
             induced by lensing of E (or GRAD) polarization.\\
          See !healpix.path.test+'README' for details
%
    \item[XHDR =] \mytarget{idl:fits2cl:xhdr}%
	String array containing on output the extension header
      read from the FITS file. 
  \end{kwlist}
\end{keywords}  

\begin{codedescription}
{\thedocid\ reads the power spectrum coefficients from a FITS
file containing an ascii table extension. Descriptive headers conforming
to the FITS convention can also be read from the input file.
}
\end{codedescription}



\begin{related}
  \begin{sulist}{} %%%% NOTE the ``extra'' brace here %%%%
    \item[idl] version \idlversion or more is necessary to run \thedocid.
    \item[\htmlref{bin\_llcl}{idl:bin_llcl}] facility to bin a spectrum read
with \thedocid.
    \item[\htmlref{bl2fits}{idl:bl2fits}] facility to write a window function into a FITS file.
    \item[\htmlref{cl2fits}{idl:cl2fits}] provides the complimentary routine to write a
      power spectrum to a FITS file.
    \item[\htmlref{fits2alm}{idl:fits2alm}, \htmlref{alm2fits}{idl:alm2fits}] routines to read and write $a_{lm}$ coefficients
    \item[\htmlref{ianafast}{idl:ianafast}] IDL routine computing $C(l)$ files
that can be read by \thedocid.
    \item[anafast] F90 facility computing $C(l)$ files that can be read by \thedocid.
  \end{sulist}
\end{related}

\begin{example}
{
\begin{tabular}{l} %%%% use this tabular format %%%%
\thedocid, pwrsp, '\$HEALPIX/test/cl.fits', \$ \\
\phantom{blankblank}	HDR=hdr, XHDR=xhdr, MULTI=l, LLFACT=fll \\
plot, l, powrsp[*,0]*fll
\end{tabular}
}
{
\thedocid\ reads a power spectrum $C(l)$ from the input FITS file 
{\tt \$HEALPIX/test/cl.fits}
into the variable {\tt pwrsp},  with optional headers
passed by the string variables {\tt hdr} and {\tt xhdr}. The multipoles $l$ and
factors $l(l+1)/2\pi$ are read into {\tt l} and {\tt fll} respectively.
$l(l+1) C(l)/2\pi$ vs $l$ is then plotted.
}
\end{example}



