
\sloppy


\docid{alm2cl*}\section[alm2cl*]{ }
\label{sub:alm2cl}
\docrv{Version 2.0}
\author{Eric Hivon}
\abstract{This document describes the \healpix Fortran90 subroutine ALM2CL*.}

\begin{facility}
{This routine computes the auto (or cross) power spectra of a one (or two) sets of spherical harmonics
  coefficients $a_{\ell m}$.
$C_{12}(\ell) = \sum_{m=-\ell}^{\ell} a_{1,\ell m}
  a_{2,\ell m}^* / (2 \ell +1) \myhtmlimage{}$ }
{\modAlmTools}
\end{facility}

\begin{f90format}
{\mylink{sub:alm2cl:nlmax}{nlmax}%
, \mylink{sub:alm2cl:nmmax}{nmmax}%
, \mylink{sub:alm2cl:alm1}{alm1}%
, \optional{[\mylink{sub:alm2cl:alm2}{alm2}%
,]} \mylink{sub:alm2cl:cl}{cl}%
}
\end{f90format}
\aboutoptional

\begin{arguments}
{
\begin{tabular}{p{0.35\hsize} p{0.05\hsize} p{0.1\hsize} p{0.40\hsize}} \hline  
\textbf{name~\&~dimensionality} & \textbf{kind} & \textbf{in/out} & \textbf{description} \\ \hline
                   &   &   &                           \\ %%% for presentation
nlmax\mytarget{sub:alm2cl:nlmax} & I4B & IN & the maximum $\ell$ value used for the $a_{\ell m}$. \\
nmmax\mytarget{sub:alm2cl:nmmax} & I4B & IN & the maximum $m$ value used for the $a_{\ell m}$. \\
alm1\mytarget{sub:alm2cl:alm1}(1:p, 0:nlmax, 0:nmmax) & SPC/ DPC & IN & First set of $a_{\ell m}$ values. $p$
                   is 3 or 1 depending on wether polarisation is included or
                   not. In the former case, the first index runs from 1 to 3 corresponding to (T,E,B). \\
\optional{alm2\mytarget{sub:alm2cl:alm2}}(1:p, 0:nlmax, 0:nmmax) & SPC/ DPC & IN & Second set of $a_{\ell m}$
                   values.  \\
%% \end{tabular}
%% \begin{tabular}{p{0.4\hsize} p{0.05\hsize} p{0.1\hsize} p{0.35\hsize}} \hline  
cl\mytarget{sub:alm2cl:cl}(0:nlmax,1:d) & SP/ DP & OUT & resulting auto or cross power spectra. 
                   If both {\tt alm1} and {\tt alm2} are present, {\tt cl} will
                   be their cross power spectrum. If only {\tt alm1} is present,
                   {\tt cl} will be its power spectrum. 
		   If $d=1$, only the temperature spectrum
                   $C_{\ell}^T$ will
                   be output. If $d=4$ and $p=3$, the output will be $C_{\ell}^T$, $C_{\ell}^E$,
                   $C_{\ell}^B$, $C_{\ell}^{T\times E}$, and if $d\geq 6$ and $p=3$, $C_{\ell}^{T\times
                   B}$  $C_{\ell}^{E\times B}$ will also be output.
\end{tabular}
}
\end{arguments}

\begin{example}
{
lmax = 128 ; mmax = lmax \\
call alm2cl(lmax, mmax, alm1, cl\_auto)  \\
call alm2cl(lmax, mmax, alm1, alm2, cl\_cross)  \\
}
{
{\tt cl\_auto} will contain the (auto) power spectrum of the $a_{\ell m}$ coefficients {\tt alm1} up to $\ell = 128$,
while {\tt cl\_cross} will be the cross power spectra of the two sets of $a_{\ell m}$ coefficients {\tt
  alm1} and {\tt alm2}.
}
\end{example}

\begin{modules}
  \begin{sulist}{} %%%% NOTE the ``extra'' brace here %%%%
  \item[none]
  \end{sulist}
\end{modules}

\begin{related}
  \begin{sulist}{} %%%% NOTE the ``extra'' brace here %%%%
  \item[\htmlref{map2alm}{sub:map2alm}] routine extracting the $a_{\ell m}$
  coefficients from a \healpix map
  \item[\htmlref{create\_alm}{sub:create_alm}] routine to generate randomly
  distributed $a_{\ell m}$ coefficients according to a given power spectrum
  \end{sulist}
\end{related}

\rule{\hsize}{2mm}

\newpage
