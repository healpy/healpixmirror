

\sloppy


%%%\title{\healpix Fortran Subroutines Overview}
\docid{concatnl} \section[concatnl]{ }
\label{sub:concatnl}
\docrv{Version 1.1}
\author{E. Hivon}
\abstract{This document describes the \healpix Fortran90 subroutine CONCATNL.}

\begin{facility}
{Function to concatenate up to 10 subtrings interspaced with LineFeed
character. Upon printing each subtring will be on a different line.
}
{\modParamfileIo}
\end{facility}

\begin{f90function}
{string1[, string2, string3, ...]}
\end{f90function}

%\ mylink: to avoid automatic processing by make_internal_links.sh

\begin{arguments}
{
\begin{tabular}{p{0.3\hsize} p{0.05\hsize} p{0.1\hsize} p{0.45\hsize}} \hline  
\textbf{name~\&~dimensionality} & \textbf{kind} & \textbf{in/out} & \textbf{description} \\ \hline
                   &   &   &                           \\ %%% for presentation
string1 & CHR & IN & the first substring to be concatenated. \\
string2 & CHR & IN \hskip 1cm optional& the second substring (if any) to be concatenated. \\
string3 & CHR & IN \hskip 1cm optional& ... up to 10 substrings can be concatenated. \\
var & CHR & OUT & concatenation of the substrings interspaced with LineFeed character.\\
\end{tabular}
}
\end{arguments}

\begin{example}
{
use paramfile\_io \\
print*,concatnl('a','bbbbbbbb','C 10 3') 
}
{\parbox[t]{2.2cm}{
Will~return:
\parbox[t]{2cm}{\tt{a\\ bbbbbbbb\\ C 10 3}}}
}
\end{example}
\begin{related}
  \begin{sulist}{} %%%% NOTE the ``extra'' brace here %%%%
  \item[\htmlref{parse\_xxx}{sub:parse_xxx}] parse an ASCII file for parameters definition
  \end{sulist}
\end{related}

\rule{\hsize}{2mm}

