
\sloppy


%%%\title{\healpix Fortran Subroutines Overview}
\docid{write\_fits\_cut4} \section[write\_fits\_cut4]{ }
\label{sub:write_fits_cut4}
\docrv{Version 1.3}
\author{Eric Hivon \& Frode K.~Hansen}
\abstract{This document describes the \healpix Fortran90 subroutine WRITE\_FITS\_CUT4.}

\begin{facility}
{This routine writes a cut sky \healpix map into a FITS file. The format used for the
FITS file follows the one used for Boomerang98 and is adapted from COBE/DMR. 
This routine can be used to store polarized maps, where the
information relative to the Stokes parameters I, Q and U are placed in extension
0, 1 and 2 respectively by successive invocation of the routine.}
{\modFitstools}
\end{facility}

\begin{f90format}
{\mylink{sub:write_fits_cut4:filename}{filename}%
, \mylink{sub:write_fits_cut4:np}{np}%
, \mylink{sub:write_fits_cut4:pixel}{pixel}%
, \mylink{sub:write_fits_cut4:signal}{signal}%
, \mylink{sub:write_fits_cut4:n_obs}{n\_obs}%
, \mylink{sub:write_fits_cut4:serror}{serror}%
 \optional{[, \mylink{sub:write_fits_cut4:header}{header}%
, \mylink{sub:write_fits_cut4:coord}{coord}%
, \mylink{sub:write_fits_cut4:nside}{nside}%
, \mylink{sub:write_fits_cut4:order}{order}%
,
\mylink{sub:write_fits_cut4:units}{units}%
, \mylink{sub:write_fits_cut4:extno}{extno}%
, \mylink{sub:write_fits_cut4:polarisation}{polarisation}%
]}}
\end{f90format}
\aboutoptional

\begin{arguments}
{
\begin{tabular}{p{0.3\hsize} p{0.05\hsize} p{0.05\hsize} p{0.5\hsize}} \hline  
\textbf{name\&dimensionality} & \textbf{kind} & \textbf{in/out} & \textbf{description} \\ \hline
                   &   &   &                           \\ %%% for presentation
filename\mytarget{sub:write_fits_cut4:filename}(LEN=\filenamelen) & CHR & IN & FITS file into which the cut sky map will be written\\
np\mytarget{sub:write_fits_cut4:np}           & I4B & IN & number of pixels to be written in the file \\
pixel\mytarget{sub:write_fits_cut4:pixel}(0:np-1)    & I4B & IN & index of observed (or valid) pixels \\
signal\mytarget{sub:write_fits_cut4:signal}(0:np-1)    & SP & IN & value of signal in each observed pixel\\
n\_obs\mytarget{sub:write_fits_cut4:n_obs}(0:np-1)   & I4B & IN & number of observation per pixel \\
serror\mytarget{sub:write_fits_cut4:serror}(0:np-1)   & SP  & IN & {\em rms} of signal in pixel, for white noise,
                   this is $\propto 1/\sqrt{{\rm n\_obs}}$. \\
%%%%%If Serror is present N\_Obs should also be present \\
\optional{header\mytarget{sub:write_fits_cut4:header}}(LEN=80)(1:) \ (OPTIONAL)    & CHR & IN &   FITS extension header to be included in the FITS file\\
\optional{coord\mytarget{sub:write_fits_cut4:coord}}(LEN=1)       & CHR & IN &   astrophysical coordinates ('C' or 'Q'
                   Celestial/eQuatorial, 'G' for Galactic, 'E' for Ecliptic)\\
\optional{nside\mytarget{sub:write_fits_cut4:nside}}    & I4B & IN &   \healpix resolution parameter of data set \\
\optional{order\mytarget{sub:write_fits_cut4:order}}     & I4B & IN &   \healpix ordering scheme, 1: RING, 2: NESTED \\
%\optional{header}(LEN=80)    & CHR & IN &   FITS header to be included in the FITS file\\
\optional{units\mytarget{sub:write_fits_cut4:units}}(LEN=20) & CHR & IN &  maps units (applies only to Signal and
                   Serror)\\
\optional{extno}\mytarget{sub:write_fits_cut4:extno}     & I4B & IN & (0 based) extension number in which to write data. \default{0}.
	  If set to 0 (or not set) {\em a new file is written from scratch}.
	  If set to a value
		  larger than 1, the corresponding extension is added or
		  updated, as long as all previous extensions already exist.
		  All extensions of the same file should use the same Nside,
Order and Coord \\
\optional{polarisaton}\mytarget{sub:write_fits_cut4:polarisation} & I4B & IN & if set to a non zero value, specifies that file will contain the I, Q and U polarisation
           Stokes parameter in extensions 0, 1 and 2 respectively, and sets the
FITS header keywords accordingly. If not set, the keywords found in \mylink{sub:write_fits_cut4:header}{\tt
header} will prevail.\\
\  & \ & \ & Note: the information relative to Nside, Order and Coord {\em has} to be
                   given, either thru these keyword or via the FITS Header. \\
\end{tabular}
}
\end{arguments}

% \begin{example}
% {
% npix= write\_fits\_cut4('map.fits', nmaps=nmaps, ordering=ordering,obs\_npix=obs\_npix, nside=nside, mlpol=mlpol, type=type, polarisation=polarisation)  \\
% }
% {
% Returns 1 or 3 in nmaps, dependent on wether 'map.fits' contain only
% temperature or both temperature and polarisation maps. The pixel ordering number is found by reading the keyword ORDERING in the FITS file. If this keyword does not exist, 0 is returned.
% }
% \end{example}
\newpage
\begin{modules}
  \begin{sulist}{} %%%% NOTE the ``extra'' brace here %%%%
  \item[\textbf{fitstools}] module, containing:
  \item[printerror] routine for printing FITS error messages.
  \item[\textbf{cfitsio}] library for FITS file handling.		
  \end{sulist}
\end{modules}

\begin{related}
  \begin{sulist}{} %%%% NOTE the ``extra'' brace here %%%%
  \item[anafast] executable that reads a \healpix map and analyses it. 
  \item[synfast] executable that generate full sky \healpix maps
  \item[\htmlref{getsize\_fits}{sub:getsize_fits}] routine to know the size of a FITS file and its type (eg, full sky vs cut sky)
  \item[\htmlref{input\_map}{sub:input_map}] all purpose routine to input a map of any kind from a FITS file
  \item[\htmlref{output\_map}{sub:output_map}] subroutine to write a FITS file from a \healpix map
  \item[\htmlref{read\_fits\_cut4}{sub:read_fits_cut4}] subroutine to read a \healpix cut sky map from a FITS file
  \item[\htmlref{write\_minimal\_header}{sub:write_minimal_header}] routine to write minimal FITS header
  \end{sulist}
\end{related}

\rule{\hsize}{2mm}

\newpage
