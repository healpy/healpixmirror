% -*- LaTeX -*-


\renewcommand{\facname}{{bin\_llcl}}
\renewcommand{\FACNAME}{{BIN\_LLCL}}

\sloppy



\title{\healpix IDL Facility User Guidelines}
\docid{\facname} \section[\facname]{ }
\label{idl:bin_llcl}
\docrv{Version 1.0}
\author{Eric Hivon}
\abstract{This document describes the \healpix IDL facility \facname.}




\begin{facility}
{This IDL facility provides a means to bin an angular power spectrum into
arbitrary bins.
}
{src/idl/misc/\facname.pro}
\end{facility}

\begin{IDLformat}
{\FACNAME, \mylink{idl:bin_llcl:Llcl_in}{Llcl\_in}%
, \mylink{idl:bin_llcl:Bin}{Bin}%
, \mylink{idl:bin_llcl:L_out}{L\_out}%
, \mylink{idl:bin_llcl:Llcl_out}{Llcl\_out}%
, [\mylink{idl:bin_llcl:Dllcl}{Dllcl}%
, \mylink{idl:bin_llcl:DELTAL}{DELTAL=}%
, \mylink{idl:bin_llcl:FLATTEN}{/FLATTEN}%
, \mylink{idl:bin_llcl:HELP}{/HELP}%
, \mylink{idl:bin_llcl:UNIFORM}{/UNIFORM}%
]
}
\end{IDLformat}

\begin{qualifiers}
  \begin{qulist}{} %%%% NOTE the ``extra'' brace here %%%%
    \item[Llcl\_in\mytarget{idl:bin_llcl:Llcl_in}%
] 1D vector: {\bf input} power spectrum (given for each $l$ starting at 0).
    \item[Bin\mytarget{idl:bin_llcl:Bin}%
] {\bf input}: binning in $l$ to be applied, \hfill \\
--either a scalar interpreted as the step size of a regular binning, the first
bins are then \{0, {\tt bin} - 1\},\{{\tt bin}, 2{\tt bin}-1\}, $\ldots$\hfill
\\ --or a 1D vector, interpreted as the lower bound of 
each bin, ie the first bins are \{bin[0],bin[1]-1\}, \{bin[1], bin[2]-1\}, $\ldots$\\
	\item[L\_out\mytarget{idl:bin_llcl:L_out}%
] contains on {\bf output} the center of each bin $l_b$.
	\item[Llcl\_out\mytarget{idl:bin_llcl:Llcl_out}%
] contains on {\bf output} the binned power spectrum
$C(b)$, ie the (weighted) average of the input $C(l)$ over each bin.
	\item[Dllcl\mytarget{idl:bin_llcl:Dllcl}%
] {\bf optional}, contains on {\bf output} a rough estimate of the rms of the binned C(l) for a full
sky observation $C(b) \sqrt{ 2 / ((2l_b+1) \Delta l_b)}$
	\item[DELTAL\mytarget{idl:bin_llcl:DELTAL}%
=] {\bf optional}, contains on {\bf output} the size of each bin $\Delta l(b)$
  \end{qulist}
\end{qualifiers}

\begin{keywords}
  \begin{kwlist}{} %%% extra brace
	\item[/FLATTEN\mytarget{idl:bin_llcl:FLATTEN}%
] if set, the $C(l)$ is internally multiplied by
$l(l+1)/2\pi$ before being binned. \\
By default, the input {\tt Llcl\_in} is binned as is.
	\item[/HELP\mytarget{idl:bin_llcl:HELP}%
] if set, an extended help is printed and the code exits.
    \item[/UNIFORM\mytarget{idl:bin_llcl:UNIFORM}%
] if set, the $C(l)$ in each bin is given the same weight.\\
By default a weight $\propto 2l+1$ is used (inverse cosmic variance
weighting). Note that this weighting affects {\tt Llcl\_out} but not {\tt
L\_out\mytarget{idl:bin_llcl:L_out}}.
  \end{kwlist}
\end{keywords}  

\begin{codedescription}
{\facname\ bins the input power spectrum (as is, or after flattening by a
$\l(\l+1)/2\pi$ factor) according to an arbitrary binning scheme defined by the
user. Different weighting scheme (uniform or inverse variance) can be applied inside the bins.}
\end{codedescription}



\begin{related}
  \begin{sulist}{} %%%% NOTE the ``extra'' brace here %%%%
    \item[idl] version \idlversion or more is necessary to run \thedocid.
    \item[\htmlref{fits2cl}{idl:fits2cl}] facility to read a power spectrum from
a FITS file.
  \end{sulist}
\end{related}

\begin{example}
 {
 \begin{tabular}{l} %%%% use this tabular format %%%%
\htmlref{init\_healpix}{idl:init_healpix}\\
\htmlref{fits2cl}{idl:fits2cl}, cl, \htmlref{!healpix.directory}{idl:init_healpix}+'/test/cl.fits', multipoles=l \\
fl =  l*(l+1) / (2. * !pi) \\
\thedocid, fl*cl[*,0], 10, lb, bbcb, /uniform \\
plot, l, fl*cl[*,0] \\
oplot, lb, bbcb, psym = 4
 \end{tabular}
 }
 {
Read a power spectrum, bin it with a binsize of 10 and a uniform weighting, and overplot the input
spectrum and its binned version.
 }
 \end{example}

