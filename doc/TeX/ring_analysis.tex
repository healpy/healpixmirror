
\sloppy


%%%\title{\healpix Fortran Subroutines Overview}
\docid{ring\_analysis} \section[ring\_analysis]{ }
\label{sub:ring_analysis}
\docrv{Version 1.1}
\author{Benjamin D.~Wandelt, Frode K.~Hansen}
\abstract{This document describes the \healpix Fortran90 subroutine RING\_ANALYSIS.}

\begin{facility}
{This subroutine computes the Fast Fourier Transform of a single ring
 of pixels
 and extends the computed coefficients up to the maximum
 $m$ of the transform.}
{\modAlmTools}
\end{facility}

\begin{f90format}
{\mylink{sub:ring_analysis:nsmax}{nsmax}%
, \mylink{sub:ring_analysis:nlmax}{nlmax}%
, \mylink{sub:ring_analysis:nmmax}{nmmax}%
, \mylink{sub:ring_analysis:datain}{datain}%
, \mylink{sub:ring_analysis:nph}{nph}%
, \mylink{sub:ring_analysis:dataout}{dataout}%
, \mylink{sub:ring_analysis:kphi0}{kphi0}%
}
\end{f90format}

\begin{arguments}
{
\begin{tabular}{p{0.4\hsize} p{0.05\hsize} p{0.1\hsize} p{0.35\hsize}} \hline  
\textbf{name~\&~dimensionality} & \textbf{kind} & \textbf{in/out} & \textbf{description} \\ \hline
                   &   &   &                           \\ %%% for presentation
nsmax\mytarget{sub:ring_analysis:nsmax} & I4B & IN & $\nside$ of the map. \\
nlmax\mytarget{sub:ring_analysis:nlmax} & I4B & IN & Maximum $\ell$ of the analysis.\\
nmmax\mytarget{sub:ring_analysis:nmmax} & I4B & IN & Maximum $m$ of the analysis.\\
nph\mytarget{sub:ring_analysis:nph} & I4B & IN & The number of points on the ring. \\ 
datain\mytarget{sub:ring_analysis:datain}(0:nph-1) & DP & IN & Function values on the ring. \\
dataout\mytarget{sub:ring_analysis:dataout}(0:nmmax) & DPC & OUT & Fourier components, replicated to $Nmmax$.\\
kphi0\mytarget{sub:ring_analysis:kphi0} & I4B & IN & 0 if the first pixel on the ring is  at
                   $\phi=0$; 1 otherwise. \\
\end{tabular}
}
\end{arguments}

\begin{example}
{
call ring\_analysis(64,128,128,datain,8,dataout,0)  \\
}
{
Returns the periodically extended complex 
Fourier Transform of datain in
dataout. They are returned in the following order: 0 1 2 3 4 5 6 7
6 5 4 3 2 1 $0\dots$. The value $kphi0=0$ specifies that no phase
factor needed to be applied, because the ring starts at $\phi=0$.
}
\end{example}

\begin{modules}
  \begin{sulist}{} %%%% NOTE the ``extra'' brace here %%%%
  \item[\textbf{healpix\_fft}] module.
  \end{sulist}
\end{modules}

\begin{related}
  \begin{sulist}{} %%%% NOTE the ``extra'' brace here %%%%
  \item[\htmlref{ring\_synthesis}{sub:ring_synthesis}] Inverse transform (complex-to-real), used in
  \htmlref{alm2map}{sub:alm2map},
  \htmlref{alm2map\_der}{sub:alm2map_der} and synfast
  \end{sulist}
\end{related}

\rule{\hsize}{2mm}

\newpage
