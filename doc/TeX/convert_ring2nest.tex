
\sloppy


\title{\healpix Fortran Subroutines Overview}
\docid{convert\_ring2nest*} \section[convert\_ring2nest*]{ }
\label{sub:convert_ring2nest}
\docrv{Version 2.0}
\author{Eric Hivon, Frode K.~Hansen}
\abstract{This document describes the \healpix Fortran90 subroutine CONVERT\_RING2NEST.}


\begin{facility}
{Routine to convert a \healpix map from RING to NESTED scheme. \newline
The routine is a
  wrapper for 6 different routines and can threfore process
  integer, single precision and double precision maps as well as mono or bi
  dimensional arrays. \newline This routine is fast, and is parallelized for shared memory
architecture, but requires extra memory to store a temporary map in. }
{\modPixTools}
\end{facility}

\begin{f90format}
{\mylink{sub:convert_ring2nest:nside}{nside}%
, \mylink{sub:convert_ring2nest:map}{map}%
}
\end{f90format}

\begin{arguments}
{
\begin{tabular}{p{0.4\hsize} p{0.05\hsize} p{0.1\hsize} p{0.35\hsize}} \hline  
\textbf{name~\&~dimensionality} & \textbf{kind} & \textbf{in/out} & \textbf{description} \\ \hline
                   &   &   &                           \\ %%% for presentation
nside\mytarget{sub:convert_ring2nest:nside} & I4B & IN & the $\nside$ parameter of the map to be converted. \\
map\mytarget{sub:convert_ring2nest:map}(0:12*nside**2-1) & I4B/ SP/ DP & INOUT & mono-dimensional full sky map to be converted to RING scheme. \\
map\mytarget{sub:convert_ring2nest:map}(0:12*nside**2-1,1:nd) & I4B/ SP/ DP & INOUT & bi-dimensional full sky map to
                   be converted to RING scheme. The routine finds the second
                   dimension (nd) by itself. Processing a bidimensional map with
{\tt nd}$>1$ should be
                   faster than each of the {\tt nd} 1D-maps consecutively.
\end{tabular}
}
\end{arguments}

\begin{example}
{
call convert\_ring2nest(256,map)  \\
}
{
Converts an $\nside=256$ map given in array {\tt map} from RING to NESTED scheme.
}
\end{example}

\begin{modules}
  \begin{sulist}{} %%%% NOTE the ``extra'' brace here %%%%
 \item[\htmlref{ring2nest}{sub:pix_tools}] routine to convert a RING pixel index to NESTED pixel number.		
  \end{sulist}
\end{modules}
%%%%\newpage
\begin{related}
  \begin{sulist}{} %%%% NOTE the ``extra'' brace here %%%%
  \item[\htmlref{convert\_nest2ring}{sub:convert_ring2nest}] convert between
  NESTED and RING schemes.
  \item[\htmlref{convert\_inplace}{sub:convert_inplace}] convert between 
    RING and NESTED schemes inplace. This routine is slower than \thedocid, but doesn't require as much memory.
  \end{sulist}
\end{related}

\rule{\hsize}{2mm}

\newpage
