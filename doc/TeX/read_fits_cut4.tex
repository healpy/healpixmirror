
\sloppy


\title{\healpix Fortran Subroutines Overview}
\docid{read\_fits\_cut4} \section[read\_fits\_cut4]{ }
\label{sub:read_fits_cut4}
\docrv{Version 1.3}
\author{Eric Hivon \& Frode K.~Hansen}
\abstract{This document describes the \healpix Fortran90 subroutine READ\_FITS\_CUT4.}

\begin{facility}
{This routine reads a cut sky \healpix map from a FITS file. The format used for the
FITS file follows the one used for Boomerang98 and is adapted from COBE/DMR}
{src/f90/mod/fitstools.F90}
\end{facility}

\begin{f90format}
{filename, np, pixel, \optional{[signal, n\_obs, serror, header, units, extno]}}
\end{f90format}
\aboutoptional

\begin{arguments}
{
\begin{tabular}{p{0.3\hsize} p{0.05\hsize} p{0.05\hsize} p{0.5\hsize}} \hline  
\textbf{name\&dimensionality} & \textbf{kind} & \textbf{in/out} & \textbf{description} \\ \hline
                   &   &   &                           \\ %%% for presentation
filename(LEN=\filenamelen) & CHR & IN & FITS file to be read from,
                   containing a cut sky map \\
np               & I4B & IN & number of pixels to be read from the file \\
pixel(0:np-1)    & I4B & OUT & index of observed (or valid) pixels \\
\optional{signal}(0:np-1)\hskip 2cm  (OPTIONAL)     & SP & OUT & value of signal in each observed pixel\\
\optional{n\_obs}(0:np-1)     & I4B & OUT & number of observation per pixel \\
\optional{serror}(0:np-1)     & SP  & OUT & {\em rms} of signal in pixel. (For white noise,
                   this would be $\myhtmlimage{}\propto 1/\sqrt{{\rm n\_obs}}$) \\
\optional{header}(LEN=80)(1:)    & CHR & OUT &   FITS extension header \\
\optional{units}(LEN=20)       & CHR & OUT &  maps units (applies only to
                   Signal and Serror, which are assumed to have the same units) \\
\optional{extno}  & I4B & IN & extension number (0 based) for which map
             is read. Default = 0 (first extension). 
\end{tabular}
}
\end{arguments}

% \begin{example}
% {
% npix= read\_fits\_cut4('map.fits', nmaps=nmaps, ordering=ordering,obs\_npix=obs\_npix, nside=nside, mlpol=mlpol, type=type, polarisation=polarisation)  \\
% }
% {
% Returns 1 or 3 in nmaps, dependent on wether 'map.fits' contain only
% temperature or both temperature and polarisation maps. The pixel ordering number is found by reading the keyword ORDERING in the FITS file. If this keyword does not exist, 0 is returned.
% }
% \end{example}
\newpage
\begin{modules}
  \begin{sulist}{} %%%% NOTE the ``extra'' brace here %%%%
  \item[\textbf{fitstools}] module, containing:
  \item[printerror] routine for printing FITS error messages.
  \item[\textbf{cfitsio}] library for FITS file handling.		
  \end{sulist}
\end{modules}

\begin{related}
  \begin{sulist}{} %%%% NOTE the ``extra'' brace here %%%%
  \item[anafast] executable that reads a \healpix map and analyses it. 
  \item[synfast] executable that generate full sky \healpix maps
  \item[\htmlref{getsize\_fits}{sub:getsize_fits}] routine to know the size of a FITS file and its type (eg, full sky vs cut sky)
  \item[\htmlref{input\_map}{sub:input_map}] all purpose routine to input a map of any kind from a FITS file
  \item[\htmlref{output\_map}{sub:output_map}] subroutine to write a FITS file from a \healpix map
  \item[\htmlref{write\_fits\_cut4}{sub:write_fits_cut4}] subroutine to write a cut sky map into a FITS file
  \end{sulist}
\end{related}

\rule{\hsize}{2mm}

\newpage
