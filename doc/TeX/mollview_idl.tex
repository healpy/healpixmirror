% -*- LaTeX -*-

% PLEASE USE THIS FILE AS A TEMPLATE FOR THE DOCUMENTATION OF YOUR OWN
% FACILITIES: IN PARTICULAR, IT IS IMPORTANT TO NOTE COMMENTS MADE IN
% THE TEXT AND TO FOLLOW THIS ORDERING. THE FORMAT FOLLOWS ONE USED BY
% THE COBE-DMR PROJECT.	
% A.J. Banday, April 1999.

\sloppy

\title{\healpix IDL Facility User Guidelines}
\docid{mollview} \section[mollview]{ }
\label{idl:\thedocid}
\docrv{Version 1.2}
\author{Eric Hivon}
\abstract{This document describes the \healpix facility mollview.}



% q customized for mollview IDL routines
  \newenvironment{qualifiers_mollview}
    {\rule{\hsize}{0.7mm}
     \textsc{\Large{\textbf{QUALIFIERS}}}\hfill\newline%
	\renewcommand{\arraystretch}{1.5}%
	}

% k customized for mollview IDL routines
  \newenvironment{keywords_mollview}
    {\rule{\hsize}{0.7mm}
     \textsc{\Large{\textbf{KEYWORDS}}}\hfill\newline%
	\renewcommand{\arraystretch}{1.5}%
	}

% \usepackage{supertab}

% \newcommand{\layout}{{{p{0.18\hsize} p{0.09\hsize} p{0.70\hsize}}}}
\newcommand{\sizeone}{0.19\hsize}
\newcommand{\sizetwo}{0.08\hsize}
\newcommand{\sizethr}{0.70\hsize}

\begin{facility}
{This IDL facility provides a means to visualise a full sky Mollweide projection of
\healpix and COBE Quad-Cube maps in an IDL environment. 
It also offers the possibility to
generate GIF, PNG and Postscript color-coded images of the projected map. 
The projected (but not color-coded) data can also be output in FITS files and
IDL arrays.}
{src/idl/visu/mollview.pro}
\end{facility}

\newcommand{\mollbacktotop}{\latexhtml{%
%%%%%%\mylink{idl:mollview:TOP}{Back to Format}% for PDF
}{
\mylink{idl:mollview:TOP}{Back to Format}% for HTML
}}
\mytarget{idl:mollview:TOP}{}


\begin{IDLformat}
{MOLLVIEW, %  File, [Select, ] [/ASINH, CHARSIZE=, COLT=, ... 
%            ... WINDOW=, XPOS=, YPOS= ]
%grep '&' mollview_idl.tex | grep -v hline | grep -v presentation | grep '^{' | awk '{print $1}'
%
\normalsize{
\mylink{idl:mollview:file}{File}, 
[ \mylink{idl:mollview:select}{Select}, ]
[ \mylink{idl:mollview:asinh}{/ASINH}, 
\mylink{idl:mollview:charsize}{CHARSIZE=}, 
\mylink{idl:mollview:charsize}{CHARTHICK=}, 
\mylink{idl:mollview:colt}{COLT=}, 
\mylink{idl:mollview:coord}{COORD=}, 
\mylink{idl:mollview:crop}{/CROP}, 
\mylink{idl:mollview:execute}{EXECUTE=}, 
\mylink{idl:mollview:factor}{FACTOR=}, 
\mylink{idl:mollview:fits}{FITS=}, 
\mylink{idl:mollview:flip}{/FLIP}, 
\mylink{idl:mollview:gal_cut}{GAL\_CUT=}, 
\mylink{idl:mollview:gif}{GIF=}, 
\mylink{idl:mollview:glsize}{GLSIZE=}, 
\mylink{idl:mollview:graticule}{GRATICULE=}, 
\mylink{idl:mollview:half_sky}{/HALF\_SKY}, 
\mylink{idl:mollview:hbound}{HBOUND=}, 
\mylink{idl:mollview:help}{/HELP}, 
\mylink{idl:mollview:hist_equal}{/HIST\_EQUAL}, 
\mylink{idl:mollview:hxsize}{HXSIZE=}, 
\mylink{idl:mollview:iglsize}{IGLSIZE=}, 
\mylink{idl:mollview:igraticule}{IGRATICULE=}, 
\mylink{idl:mollview:log}{/LOG}, 
\mylink{idl:mollview:map_out}{MAP\_OUT=}, 
\mylink{idl:mollview:max}{MAX=}, 
\mylink{idl:mollview:min}{MIN=}, 
\mylink{idl:mollview:nested}{/NESTED}, 
\mylink{idl:mollview:no_dipole}{/NO\_DIPOLE}, 
\mylink{idl:mollview:no_monopole}{/NO\_MONOPOLE}, 
\mylink{idl:mollview:nobar}{/NOBAR}, 
\mylink{idl:mollview:nolabels}{/NOLABELS}, 
\mylink{idl:mollview:noposition}{/NOPOSITION}, 
\mylink{idl:mollview:offset}{OFFSET=}, 
\mylink{idl:mollview:outline}{OUTLINE=}, 
\mylink{idl:mollview:png}{PNG=}, 
\mylink{idl:mollview:polarization}{POLARIZATION=}, 
\mylink{idl:mollview:preview}{/PREVIEW}, 
\mylink{idl:mollview:ps}{PS=}, 
\mylink{idl:mollview:pxsize}{PXSIZE=}, 
\mylink{idl:mollview:pysize}{PYSIZE=}, 
\mylink{idl:mollview:reso_arcmin}{RESO\_ARCMIN=}, 
\mylink{idl:mollview:retain}{RETAIN=}, 
\mylink{idl:mollview:rot}{ROT=}, 
\mylink{idl:mollview:save}{/SAVE}, 
\mylink{idl:mollview:shaded}{/SHADED}, 
\mylink{idl:mollview:silent}{/SILENT}, 
\mylink{idl:mollview:subtitle}{SUBTITLE=}, 
\mylink{idl:mollview:titleplot}{TITLEPLOT=}, 
\mylink{idl:mollview:transparent}{TRANSPARENT=}, 
\mylink{idl:mollview:truecolors}{TRUECOLORS=}, 
\mylink{idl:mollview:units}{UNITS=}, 
\mylink{idl:mollview:window}{WINDOW=}, 
\mylink{idl:mollview:xpos}{XPOS=}, 
\mylink{idl:mollview:ypos}{YPOS=}]
}
}
\end{IDLformat}

% \newcommand{head_column}{{\textbf{name} & \textbf{routine} &
%            \textbf{description}}}

%\vskip 1cm
Several visualization routines have a similar interface. Their {\bf qualifiers} and
{\bf keywords} are all listed here, and the routines to which they apply are coded
in the 'routine' column as: \\
C: \htmlref{cartview}{idl:cartview},
G: \htmlref{gnomview}{idl:gnomview}, 
M: \htmlref{mollview}{idl:mollview},
O: \htmlref{orthview}{idl:orthview} and all: all of them
%\vskip 2cm 
\vskip 0.4cm

{\bf Qualifiers} should appear in the order indicated. They can take a range of values, and some of them are optional.\\
{\bf Keywords} are optional, and can appear in any order. They take the form {\tt keyword=value} and can be abbreviated to a non ambiguous form
(ie, {\tt factor=10.0} can be replaced by {\tt fac = 10.0}). They generally can take a range of values, but
some of them (noted as {\tt /keyword} below) are boolean switches: they are either present (or set to 1) or absent (or set
to 0).

\vfill
\newpage
\begin{qualifiers_mollview}
\begin{tabular}{p{\sizeone} p{\sizetwo} p{\sizethr}}
%% \begin{tabular}{p{0.14\hsize} p{0.08\hsize} p{0.75\hsize}}
\hline  
\textbf{name} & \textbf{routines} & \textbf{\ description} \\ \hline
%                    &   &                            \\ %%% for presentation
\mytargetb{idl:mollview:file}{File}  & all   & \parbox[t]{0.95\hsize}{%
		Required\\
                    name of a (possibly gzip compressed) FITS file containing 
               the \healpix\ map in an extension or in the image field, \\
          {\em or}\  \  name of an {\em online} variable (either array or
structure) containing the (RING or \mylink{idl:mollview:polarization}{NESTED} ordered) \healpix\ map (See note below);\\
          if Save is set   :    name of an IDL saveset file containing
               the \healpix\ map stored under the variable  {\tt data} \\
	\nodefault\\
\underline{Note on online data}: in order to preserve the integrity of the input data, 
		the content of the array or structure {\tt File} is replicated
before being possibly altered by the map making process. 
Therefore plotting online data will require more memory than reading the data from disc directly, and is not recommended
		to visualize data sets of size comparable to that of the
computer memory.\\ 
\underline{Note on high resolution cut sky data}: cut sky data (in which less
than 50\% of the sky is observed), can be processed with a minimal memory
foot-print, by not allocating fake full map. In the current release, two
restrictions apply: the input data set must be read from a FITS file in 'cut4'
format, and the \mylink{idl:mollview:polarization}{\tt POLARIZATION} IDL keyword (described below) must be 0 (default value). See the
Examples \#4 below (on page~\pageref{page:example_hires_cutsky}). \\
\seealso \mylink{idl:mollview:truecolors}{\tt TrueColors}.}\\

\mytargetb{idl:mollview:select}{Select}  & all   & \parbox[t]{0.95\hsize}{Optional\\
		  column of the BIN FITS table to be plotted, can be either  \\
                -- a name : value given in TTYPEi of the FITS file \\
                        NOT case sensitive and can be truncated, \\
                        (only letters, digits and underscore are valid) \\
               -- an integer        : number i of the column
                            containing the data, starting with 1 (also valid if
		  \mylink{idl:mollview:file}{{\tt File}} is an online array) \\
                   \default {1 for full sky maps, 'SIGNAL' column for FITS files
		  containing cut sky maps}
               (see the Examples below)}\\

\end{tabular}
\end{qualifiers_mollview}
\vskip 1cm
\mollbacktotop
%======================================================
\newpage
\begin{keywords_mollview}
\begin{tabular}{p{\sizeone} p{\sizetwo} p{\sizethr}}
\hline  
\textbf{name} & \textbf{routines} & \textbf{\ description} \\ \hline
%                    &   &                            \\ %%% for presentation
{/ASINH}\mytarget{idl:mollview:asinh}   & all & \parbox[t]{\hsize}{  if set, the color table is altered to emulate the effect of replacing
            the data by $\sinh^{-1}$(data) in order to enhance the low contrast regions.
            Can be used in conjonction with Factor and Offset, but can {\em not} be
            used with /LOG nor /HIST\_EQUAL.
		\seealso \mylink{idl:mollview:factor}{Factor}, \mylink{idl:mollview:hist_equal}{Hist\_Equal}, \mylink{idl:mollview:log}{Log}, \mylink{idl:mollview:offset}{Offset}
	}\\

{CHARSIZE=}\mytarget{idl:mollview:charsize}  & all   &  
	overall multiplicative factor applied to the size of
        all characters appearing on the plot
        	\default {1.0} \\ 

{CHARTHICK=}\mytarget{idl:mollview:charthick} & all & character thickness (in
\mylink{idl:mollview:titleplot}{TITLE}, \mylink{idl:mollview:subtitle}{SUBTITLE} and
color bar labeling). Other characters thickness (such as
\mylink{idl:mollview:glsize}{graticule labels}), can be controlled with !P.CHARTHICK.
                \default {1} \\

{COLT=}\mytarget{idl:mollview:colt}  & all   & \parbox[t]{\hsize}{ color table number, in [-40,40]. If colt$<0$, the IDL color
        	table abs(colt) is used, but the scale is
               reversed (ie a red to blue scale becomes a blue to red
        	scale). Note: -0.1 can be used as negative 0.\\
        	\default {33 (Blue-Red)} \\
\seealso \mylink{idl:mollview:truecolors}{\tt TrueColors}}\\ 

{COORD=}\mytarget{idl:mollview:coord}  & all   & \parbox[t]{\hsize}{
		vector with 1 or 2 elements describing the coordinate system of the map;
                either\\
		       -- 'C' or 'Q' : Celestial2000 = eQuatorial, \\
                       -- 'E'        : Ecliptic, \\
                       -- 'G'        : Galactic  \\
               if coord = ['x','y'] the map is rotated from system 'x' to system 'y' \\
               if coord = ['y'] the map is rotated to coordinate system 'y' (with the
               original system assumed to be Galactic unless indicated otherwise in the
                input file) \\
                  \seealso \mylink{idl:mollview:rot}{Rot}}\\


{/CROP}\mytarget{idl:mollview:crop}  & all   & \parbox[t]{\hsize}{
		if set the GIF/PNG file only contains the map and
               no title, color bar, ... \\
                \seealso \mylink{idl:mollview:gif}{Gif}, \mylink{idl:mollview:png}{Png}}\\

%
\end{tabular}
\mollbacktotop
%-----------------------------------------------------------
\begin{tabular}{p{\sizeone} p{\sizetwo} p{\sizethr}}
\hline  
\textbf{name} & \textbf{routines} & \textbf{\ description} \\ \hline
%                   &   &                            \\ %%% for presentation

{EXECUTE=}\mytarget{idl:mollview:execute}  & all   & \parbox[t]{\hsize}{
		character string containing IDL command(s) to be executed in the
plotting window. See 
Figure~\ref{fig:plot_example_execute} on page~\pageref{page:plot_example_execute}}\\

{FACTOR=}\mytarget{idl:mollview:factor} & all & \parbox[t]{\hsize}{scalar multiplicative factor to be applied to the
	valid data \\
	the data plotted is of the form Factor*(data + Offset) \\
	This does not affect the flagged pixels \\
	Can be used together with ASINH or LOG \\
	When used with TRUECOLORS, FACTOR can be a 3-element vector. \\
        \seealso \mylink{idl:mollview:asinh}{ASINH}, \mylink{idl:mollview:offset}{Offset}, \mylink{idl:mollview:log}{LOG}, \mylink{idl:mollview:truecolors}{Truecolors} \\
	\default {1.0} } \\

{FITS=}\mytarget{idl:mollview:fits} %\hyperdef{cross}{idl:mollview:fits} %
		& all & \parbox[t]{\hsize}{
	string containing the name of an output FITS file with
        the projected map in the primary image \\
 	      -- if set to 1            : output the plot in
        plot\_{\em proj}.fits, where {\em proj} is either {\em cartesian, gnomic,
mollweide}, or  {\em orthographic} depending on the projection in use;\\
 	      -- if set to a file name  : output the plot in that file.  \\
 	\default {0: no .FITS done}\\
In the case of Orthographic projection,
\mylink{idl:mollview:half_sky}{HALF\_SKY} must be set.\\
%
Except for the color mapping, all the keywords and options apply to the
projected map, ie: its size is determined by
\mylink{idl:mollview:pxsize}{PXSIZE} (and \mylink{idl:mollview:pysize}{PYSIZE}
when applicable), its angular resolution by \mylink{idl:mollview:reso_arcmin}{RESO\_ARCMIN}
when applicable, its orientation and coordinates by 
\mylink{idl:mollview:rot}{ROT} and 
\mylink{idl:mollview:coord}{COORD} respectively, ... \\
%
	For compatibility with standard FITS viewers (including 
\htmladdnormallink{STIFF}{http://www.astromatic.net/category/softwareposts/stiff})%
, unobserved pixels, and
pixels outside the sphere, take the value {\tt NaN} (ie {\tt !values.f\_nan} in IDL).
The resulting FITS file can be read in IDL with eg. map={\tt readfits}({\em filename}). 
\\
	\seealso \mylink{idl:mollview:map_out}{Map\_out}}\\

{/FLIP}\mytarget{idl:mollview:flip}  & all   & if set the longitude increases to the right, whereas by
                default (astronomical convention) it increases towards the left \\
 
%
\end{tabular}
\mollbacktotop
%-----------------------------------------------------------
\begin{tabular}{p{\sizeone} p{\sizetwo} p{\sizethr}}
\hline  
\textbf{name} & \textbf{routines} & \textbf{\ description} \\ \hline
%                   &   &                            \\ %%% for presentation


{GAL\_CUT=}\mytarget{idl:mollview:gal_cut}  & --MO   & (positive float) specifies the symmetric galactic cut in degrees
              outside of which the monopole and/or dipole fitting is done
	\default{0: monopole and dipole fit done on the whole sky}
              (\seealso \mylink{idl:mollview:no_dipole}{No\_dipole}, \mylink{idl:mollview:no_monopole}{No\_monopole}) \\

{GIF=}\mytarget{idl:mollview:gif}  & all   & \parbox[t]{\hsize}{
		string containing the name of a .GIF output \\
	      if set to 1            : output the plot in plot\_[projection].gif \\
	      if set to a file name  : output the plot in that file \\
	Please note that the resulting GIF image might not always look
	      as expected. The reason for this is a problem with
	      'backing store' in the IDL-routine TVRD. Please read the IDL
	      documentation for more information. \\
	\default {no .GIF done} \\              \seealso \mylink{idl:mollview:crop}{Crop}, \mylink{idl:mollview:png}{Png}, \mylink{idl:mollview:ps}{Ps}, \mylink{idl:mollview:preview}{Preview}
and \mylink{idl:mollview:retain}{Retain}}\\

{GLSIZE=}\mytarget{idl:mollview:glsize} & all & \parbox[t]{\hsize}{
	character size of the graticule labels in units of {\tt Charsize}. \\
	\default{0: no labeling of graticules}. \\
	\seealso \mylink{idl:mollview:charsize}{Charsize}, \mylink{idl:mollview:graticule}{Graticule}
	}\\

{GRATICULE=}\mytarget{idl:mollview:graticule}  & all   & \parbox[t]{\hsize}{
	if set, puts a graticule (ie, longitude and latitude grid) 
	in the {\em output} astrophysical coordinates
	with delta\_long = delta\_lat = gdef
         degrees \\
         if set to a scalar $x>$ gmin then delta\_long = delta\_lat = $x$ \\
         if set to [x,y] with $x,y >$ gmin then delta\_long = $x$ and delta\_lat = $y$ \\
          cartview : gdef = 45, gmin =  0 \\
          gnomview : gdef =  5, gmin =  0 \\
          mollview : gdef = 45, gmin = 10 \\
          orthview : gdef = 45, gmin = 10 \\
	Note that the graticule will rotate with the sphere is Rot is set.
	To outline only the equator set graticule=[360,90].
	The automatic labeling of the graticule is controlled by {\tt Glsize}\\
	\default {0 [no graticule]} \\
	\seealso \mylink{idl:mollview:igraticule}{Igraticule}, \mylink{idl:mollview:rot}{Rot}, \mylink{idl:mollview:coord}{Coord}, \mylink{idl:mollview:glsize}{Glsize}
	}\\

\end{tabular}
\mollbacktotop
%-----------------------------------------------------------
\begin{tabular}{p{\sizeone} p{\sizetwo} p{\sizethr}}
\hline  
\textbf{name} & \textbf{routines} & \textbf{\ description} \\ \hline
%                   &   &                            \\ %%% for presentation
{/HALF\_SKY}\mytarget{idl:mollview:half_sky} & ---O & if set, only shows only one half of the sky 
   (centered on (0,0) or on the location parametrized by \mylink{idl:mollview:rot}{Rot}) instead of the full sky \\
 
{HBOUND=}\mytarget{idl:mollview:hbound} & all & if set to a valid $\nside$, will overplot the \healpix\ pixel
boundaries corresponding to that $\nside$
on top of the map. \\

{/HELP}\mytarget{idl:mollview:help} & all & if set, the routine header is printed (by doc\_library)
      and nothing else is done \\

{/HIST\_EQUAL}\mytarget{idl:mollview:hist_equal}  & all   &  \parbox[t]{\hsize}{
		if set,     uses a histogram equalized color mapping
			(useful for non gaussian data field)
		\default {uses linear color mapping and 
                     		puts the level 0 in the middle
                     		of the color scale (ie, green for Blue-Red)
				unless \mylink{idl:mollview:min}{Min} and 
	                               \mylink{idl:mollview:max}{Max} are not symmetric}\\
                     \seealso \mylink{idl:mollview:asinh}{Asinh}, \mylink{idl:mollview:log}{Log}}\\

{HXSIZE=}\mytarget{idl:mollview:hxsize}  & all   & \parbox[t]{\hsize}{
		horizontal dimension (in cm) of the \mylink{idl:mollview:ps}{Postscript} printout \\
    		\default {26 cm $\simeq$ 10 in} \\               \seealso \mylink{idl:mollview:pxsize}{Pxsize}}\\

{IGLSIZE=}\mytarget{idl:mollview:iglsize} & all & \parbox[t]{\hsize}{
	character size of the input coordinates graticule labels in units of {\tt Charsize}. \\
	\default{0: no labeling of graticules}. \\
	\seealso \mylink{idl:mollview:charsize}{Charsize}, \mylink{idl:mollview:igraticule}{Igraticule}
	}\\

{IGRATICULE=}\mytarget{idl:mollview:igraticule}  & all   & \parbox[t]{\hsize}{
	if set, puts a graticule (ie, longitude and latitude grid) 
	in the {\em input} astrophysical coordinates.
	See Graticule for conventions and details.
	If both Graticule and Igraticule are set, the latter will
	be represented with dashes.\\
	The automatic labeling of the graticule is controlled by {\tt Iglsize}\\
	\default {0 [no graticule]} \\
	\seealso \mylink{idl:mollview:graticule}{Graticule}, 
\mylink{idl:mollview:rot}{Rot}, \mylink{idl:mollview:coord}{Coord}, 
\mylink{idl:mollview:iglsize}{Iglsize}
	}\\

{/LOG}\mytarget{idl:mollview:log}  & all   & \parbox[t]{\hsize}{
		display the log of map. This is intended for
 	application to positive definite maps only, eg. Galactic foreground
	emission templates; for arbitrary maps, use /ASINH instead.	 \\
	\seealso \mylink{idl:mollview:asinh}{Asinh}, \mylink{idl:mollview:factor}{Factor}, \mylink{idl:mollview:hist_equal}{Hist\_Equal}, \mylink{idl:mollview:offset}{Offset}}\\

{MAP\_OUT=}\mytarget{idl:mollview:map_out} & all & \parbox[t]{\hsize}{
	variable that will contain the projected map on output.\\
%
Except for the color mapping, all the keywords and options apply to the
projected map, ie: its size is determined by
\mylink{idl:mollview:pxsize}{PXSIZE} (and \mylink{idl:mollview:pysize}{PYSIZE}
when applicable), its angular resolution by \mylink{idl:mollview:reso_arcmin}{RESO\_ARCMIN}
when applicable, its orientation and coordinates by 
\mylink{idl:mollview:rot}{ROT} and 
\mylink{idl:mollview:coord}{COORD} respectively, ... \\
%
Unobserved pixels, and pixels outside the sphere, take value \htmlref{{\tt
!healpix.bad\_value}}{idl:init_healpix} ($=-1.6375\,10^{30}$).\\
	\seealso \mylink{idl:mollview:fits}{Fits}}\\

\end{tabular}
\mollbacktotop
%-----------------------------------------------------------
\begin{tabular}{p{\sizeone} p{\sizetwo} p{\sizethr}}
\hline  
\textbf{name} & \textbf{routines} & \textbf{\ description} \\ \hline
%                   &   &                            \\ %%% for presentation
{MAX=}\mytarget{idl:mollview:max}  & all   & \parbox[t]{\hsize}{
	Set the maximum value for the plotted signal \\ \default{is to use the actual signal
	maximum}.}\\

{MIN=}\mytarget{idl:mollview:min}  & all   & \parbox[t]{\hsize}{
	Set the minimum value for the plotted signal \\\default{is to use the actual signal minimum}.}\\	


{/NESTED}\mytarget{idl:mollview:nested}  & all   & \parbox[t]{\hsize}{
	specify that the online data is ordered in the nested scheme}\\

{/NO\_DIPOLE}\mytarget{idl:mollview:no_dipole}  & --MO   & \parbox[t]{\hsize}{
		if set  (and Gal\_cut is not set) 
                the best fit monopole *and* dipole over all valid pixels are
                removed; \\
                if Gal\_cut is set to $b>0$, the best monopole and dipole fit is performed on all valid
                pixels with $|$galactic latitude$|>b$ (in deg) and is removed from
                all valid pixels \\
	     \default {0 (no monopole or dipole removal)}\\
              can NOT be used together with No\_monopole \\ \seealso {\mylink{idl:mollview:gal_cut}{Gal\_cut}, \mylink{idl:mollview:no_monopole}{No\_monopole}}}\\

{/NO\_MONOPOLE}\mytarget{idl:mollview:no_monopole}  & --MO   & \parbox[t]{\hsize}{
		if set (and Gal\_cut is not set) 
                the best fit monopole over all valid pixels is
                removed; \\
                if Gal\_cut is set to $b>0$, the best monopole fit is performed on all valid
                pixels with $|$galactic latitude$|>b$ (in deg) and is removed from
                all valid pixels \\
	     \default {0 (no monopole removal)}\\
              can NOT be used together with No\_dipole \\ \seealso {\mylink{idl:mollview:gal_cut}{Gal\_cut}, \mylink{idl:mollview:no_dipole}{No\_dipole}}}\\


{/NOBAR}\mytarget{idl:mollview:nobar}  & all   & if set, color bar is not present \\

{/NOLABELS}\mytarget{idl:mollview:nolabels}  & all   & if set, color bar labels (min and max) are not present, \default
	{labels are present} \\

{/NOPOSITION}\mytarget{idl:mollview:noposition} & -G-- & if set, the astronomical location of the map
        central point is not indicated \\
 
{OFFSET=}\mytarget{idl:mollview:offset} & all & \parbox[t]{\hsize}{scalar additive factor to be applied to the valid data \\
	the data plotted is of the form Factor*(data + Offset) \\
	This does not affect the flagged pixels \\
	can be used together with ASINH or LOG \\
	When used with TRUECOLORS, OFFSET can be a 3-element vector. \\
        \seealso : \mylink{idl:mollview:asinh}{ASINH}, \mylink{idl:mollview:factor}{Factor}, \mylink{idl:mollview:log}{LOG}, \mylink{idl:mollview:truecolors}{TRUECOLORS} \\
	\default {0.0} } \\
	
% {/ONLINE}  & all   & \parbox[t]{\hsize}{
% 		if set, the argument File can be either an array containing a \healpix map,
% 		or an IDL structure describing a \healpix map, instead of an external filename 
% 		(useful when the data to be plotted are already
%     		available in memory)
% 		Note: in order to preserve the integrity of the input data, 
% 		the content of File is replicated before being possibly altered by the map making process, 
% 		therefore this option will requires more memory than reading the data from directly disc, and is not recommended
% 		to visualize data sets of size comparable to that of the computer memory. \\ can not be used with /SAVE}\\

\end{tabular}
\mollbacktotop
%-----------------------------------------------------------
\begin{tabular}{p{\sizeone} p{\sizetwo} p{\sizethr}}
\hline  
\textbf{name} & \textbf{routines} & \textbf{\ description} \\ \hline
%                   &   &                            \\ %%% for presentation

{OUTLINE=}\mytarget{idl:mollview:outline} & all &  \parbox[t]{\hsize}{
	IDL (meta-)structure, or array of structures, 
       containing the description of one (or several) outline(s) to
      be overplotted on the final map.\\
           For each contour or point list, the corresponding (sub)structure should
	contain the following fields :  \\
            -- 'COORD' coordinate system (either, 'C', 'G', or 'E') of the contour\\
            -- 'RA' RA/longitude coordinates of the contour vertices (array or scalar) \\
            -- 'DEC' Dec/latitude coordinates of the contour vertices (array or scalar)\\
	    -- 'LINE[STYLE]' (optional, scalar) {\bf +2}: black dashes, {\bf
+1}: black dots, {\bf 0}: black solid (default), {\bf -1}: black dots on white background, {\bf -2}: black dashes on
	white background\\
            -- 'PSY[M]' (optional, scalar) symbol used to represent vertices (same meaning as
	    standard PSYM in IDL, if $9\leq|\rm{psym}|\leq 46$, D. Fanning's SYMCAT symbols 
                     definition will be used, for example psym=9 is an open circle). If $\leq 0$, the vertices are represented with the chosen symbols, and
                        connected, by arcs of geodesics;
                    if $>0$, only the vertices are shown
                    \default {0} \\
            -- 'SYM[SIZE]' (optional, scalar) vertice symbol size (same meaning as SYMSIZE in IDL)

	Notes: when applicable, the vertices are connected by segments of geodesics. To
	obtain a better looking outline, increase the number of vertices
	provided. The outline does not have to be closed. The procedure will NOT
	attempt to close the outline.\\Several outlines can be overplotted at
once by gathering the respective structures into one meta-structure or an array.\\ \seealso{\mylink{idl:mollview:coord}{Coord}, \mylink{idl:mollview:graticule}{Graticule}}} \\
 
{PNG=}\mytarget{idl:mollview:png}  & all   & \parbox[t]{\hsize}{
		string containing the name of a .PNG output \\
	      if set to 1            : output the plot in plot\_[projection].png \\
	      if set to a file name  : output the plot in that file \\
	Please note that the resulting PNG image might not always look
	      as expected. The reason for this is problems with
	      'backing store' in the IDL-routine TVRD. Please read the IDL
	      documentation for more information. \\
	\default {no .PNG done} \\              
\seealso 
\mylink{idl:mollview:crop}{Crop}, 
\mylink{idl:mollview:fits}{Fits}, 
\mylink{idl:mollview:gif}{Gif}, 
\mylink{idl:mollview:map_out}{Map\_out}, 
\mylink{idl:mollview:preview}{Preview}
\mylink{idl:mollview:ps}{Ps}, 
and \mylink{idl:mollview:retain}{Retain}}\\


\end{tabular}
\mollbacktotop
%-----------------------------------------------------------
\begin{tabular}{p{\sizeone} p{\sizetwo} p{\sizethr}}
\hline  
\textbf{name} & \textbf{routines} & \textbf{\ description} \\ \hline
%                   &   &                            \\ %%% for presentation

{POLARIZATION=}\mytarget{idl:mollview:polarization} & all & \parbox[t]{\hsize}{
	if set to
	\begin{itemize}
	 \setlength{\itemsep}{0pt}
	 \setlength{\leftmargin}{.1cm}
	 \setlength{\labelwidth}{.1cm}
         \item[0] no polarization information is plotted. 
%
         \item[1]
             the AMPLITUDE $P = \sqrt{\left(U^2 + Q^2\right)}$ of the polarisation is plotted 
	(as long as the input data contains polarisation information
             (ie, Stokes parameter Q and U for each pixel)) 
%
         \item[2]
             the ANGLE $\phi = \tan^{-1}(U/Q) /2$ of the polarisation is plotted \\
             Note: the angles are color coded with a fixed color table (independent of Colt)
%
         \item[3]
             --the temperature is color coded (with a color table defined by Colt)\\
             --and the polarisation is overplotted as headless VECTORS
	\end{itemize}
	\default{0} \\
	\textbf{Note:} The representation of the polarization direction (options 2 and 3 above),
        include the effects of the rotations and/or changes or astronomical coordinates
      (controlled by ROT and COORD respectively) but do not include the effects
      of the distortions induced by the projection from the sphere to the plan.
      Because the polarization usually has more power at small scales, it must
      generally be represented on maps of small patches of the sky to remain
      legible, in which case the projection-induced distortions are small.
	}   \\

{/PREVIEW}\mytarget{idl:mollview:preview}  & all   & \parbox[t]{\hsize}{
		if set, there is a 'ghostview' preview of the
	        postscript file or a 'xv' preview of the gif file\\
	 \seealso \mylink{idl:mollview:gif}{Gif}, \mylink{idl:mollview:png}{Png} and \mylink{idl:mollview:ps}{Ps}}\\


{PS=}\mytarget{idl:mollview:ps}  & all   &  \parbox[t]{\hsize}{
	      if set to 0            : no postscript output \\
	      if set to 1            : output the plot in plot\_cartesian.ps, plot\_gnomic.ps,
	      plot\_mollweide.ps or plot\_orthographic.ps respectively \\
	      if set to a file name  : output the plot in that file \\
		\default 0 \\
               \seealso 
\mylink{idl:mollview:preview}{Preview}, 
\mylink{idl:mollview:gif}{Gif}, 
\mylink{idl:mollview:png}{Png}}\\

{PXSIZE=}\mytarget{idl:mollview:pxsize}  & all   & \parbox[t]{\hsize}{
		set the number of horizontal screen\_pixels or postscript\_color\_dots of the plot
    		(useful for high definition color printer) or elements of the
output map\\
    		\default {800 (Mollview and full sky Orthview), 600 (half sky
		Orthview), 500 (Cartview and Gnomonic)} \\
	\seealso
\mylink{idl:mollview:fits}{FITS}, 
\mylink{idl:mollview:gif}{GIF}, 
\mylink{idl:mollview:map_out}{MAP\_OUT}, 
\mylink{idl:mollview:png}{PNG}, 
\mylink{idl:mollview:ps}{PS}.}\\

\end{tabular}
\mollbacktotop
%-----------------------------------------------------------
\begin{tabular}{p{\sizeone} p{\sizetwo} p{\sizethr}}
\hline  
\textbf{name} & \textbf{routines} & \textbf{\ description} \\ \hline
%                   &   &                            \\ %%% for presentation
{PYSIZE=}\mytarget{idl:mollview:pysize}  & CG--   & \parbox[t]{\hsize}{
		set the number of vertical screen\_pixels or postscript\_color\_dots of the plot \\
    		\default {\mylink{idl:mollview:pxsize}{Pxsize}}.}\\

{RESO\_ARCMIN=}\mytarget{idl:mollview:reso_arcmin} & CG-- &  \parbox[t]{\hsize}{
		size of screen\_pixels or postscript\_color\_dots in arcmin \\
        		\default{1.5}\\
	\seealso
\mylink{idl:mollview:fits}{FITS}, 
\mylink{idl:mollview:gif}{GIF}, 
\mylink{idl:mollview:map_out}{MAP\_OUT}, 
\mylink{idl:mollview:png}{PNG}, 
\mylink{idl:mollview:ps}{PS}.}\\

{RETAIN=}\mytarget{idl:mollview:retain} & all & \parbox[t]{\hsize}{specifies the type of
backing store to use for direct graphics windows in \{0,1,2\}.
\default{2}. See IDL documentation for details.} \\

{ROT=}\mytarget{idl:mollview:rot}  & all & \parbox[t]{\hsize}{
		vector with 1, 2 or 3 elements specifing the rotation angles in DEGREES
               to apply to the map in the 'output' coordinate system (see \mylink{idl:mollview:coord}{Coord})
             = ( lon0, [lat0, rat0])  \\
               lon0 : longitude of the point to be put at the center of the plot
		       the longitude increases Eastward, ie to the left of the plot
 		      \default 0 \\
               lat0 : latitude of the point to be put at the center of the plot
 		      \default 0 \\
               rot0 : anti clockwise rotation to apply to the sky around the
                     center (lon0, lat0) before projecting
                     \default 0 %\\
			%%%\seealso \mylink{idl:mollview:fits}{FITS}%
}\\


{/SAVE}\mytarget{idl:mollview:save}  & all   & \parbox[t]{\hsize}{
		if set, assumes that \mylink{idl:mollview:file}{File} is in IDL saveset format, 
    		the variable saved should be DATA} \\
                % can not be used with /ONLINE}\\

{/SHADED}\mytarget{idl:mollview:shaded} & ---O & \parbox[t]{\hsize}{
		if set, the orthographic sphere is shaded, using a Phong model, to emulate 3D viewing.
              The sphere is illuminated by isotropic ambiant light plus a single light source.
                 Can NOT be used with \mylink{idl:mollview:gif}{GIF}.} \\

{/SILENT}\mytarget{idl:mollview:silent}  & all   & \parbox[t]{\hsize}{
		if set, the program runs silently} \\

{SUBTITLE=}\mytarget{idl:mollview:subtitle}  & all   & \parbox[t]{\hsize}{
		String containing the subtitle to the plot\\ \seealso \mylink{idl:mollview:titleplot}{Titleplot}}\\

{TITLEPLOT=}\mytarget{idl:mollview:titleplot}  & all   & \parbox[t]{\hsize}{
		String containing the title of the plot, 
     		if not set the title will be File\\ \seealso \mylink{idl:mollview:subtitle}{Subtitle}}\\

{TRANSPARENT=}\mytarget{idl:mollview:transparent}  & all   & \parbox[t]{\hsize}{
		If set to 1, the input data pixels with value \htmlref{{\tt
!healpix.bad\_value}}{idl:init_healpix} ($=-1.6375\,10^{30}$) 
will appear totally transparent on the output PNG file (instead of the usual
grey).\\ 
If set to 2, the background pixels will be transparent (instead of the usual white)\\
If set to 3, both the grey and white pixels will look transparent.\\
Active only in conjunction with \mylink{idl:mollview:png}{PNG}}\\

\end{tabular}
\mollbacktotop
%-----------------------------------------------------------
\begin{tabular}{p{\sizeone} p{\sizetwo} p{\sizethr}}
\hline  
\textbf{name} & \textbf{routines} & \textbf{\ description} \\ \hline
%                   &   &                            \\ %%% for presentation

{TRUECOLORS=}\mytarget{idl:mollview:truecolors} & all & \parbox[t]{\hsize}{
if the input data is of the form [Npix,3], then the 3 fields
            are respectively understood as {Red, Green, Blue} True-Color
channels, and the \mylink{idl:mollview:colt}{color table} is ignored.\\
-- If set to 1, the mapping field-intensity to color is done for the 3 channels at once. (\seealso \mylink{idl:mollview:factor}{Factor}, \mylink{idl:mollview:offset}{Offset})\\
-- If set to 2, that mapping is done for each channel separately (in that case, \mylink{idl:mollview:min}{MIN}
and \mylink{idl:mollview:max}{MAX} keywords are ignored). 
}\\

{UNITS=}\mytarget{idl:mollview:units}  & all   & \parbox[t]{\hsize}{
		String containing the units, to be put on the right
		hand side of the color bar, overrides the value read from the input file,
		if any\\ \seealso \mylink{idl:mollview:nobar}{Nobar}, \mylink{idl:mollview:nolabels}{Nolabels}}\\

{WINDOW=}\mytarget{idl:mollview:window}  & all   & \parbox[t]{\hsize}{
                 IDL window index (integer)\\
               -- if WINDOW $< 0$: virtual window: no visible window opened. Can
                 be used with \mylink{idl:mollview:png}{PNG} or \mylink{idl:mollview:gif}{GIF}, in particular is those files are
larger than the screen. {\bf Note:} The Z buffer will be used instead of the X server,
allowing much faster production of the image over a slow network\\
               -- if WINDOW in $[0,31]$: the specified IDL window with index WINDOW is used
               (or reused). Can be used to have a sequence of images appear
in the same window \\
               -- if WINDOW $> 31$: a free ($=$unused) window with a random index $> 31$ will be
               created and used. \\
	       \default{32}		 
                  }\\

{XPOS=}\mytarget{idl:mollview:xpos}  & all   & \parbox[t]{\hsize}{
		The X position on the screen of the lower left corner
	        of the window, in device coordinate}\\

{YPOS=}\mytarget{idl:mollview:ypos}  & all   & The Y position on the screen of the lower left corner 
               of the window, in device coordinate

%% %%%%%%%%%%%%%%%%%%%%%%%%%%%%%%%%%%
%% \end{tabular}
%% %--------------------------------------------------------
%% \begin{tabular}{p{\sizeone} p{\sizetwo} p{\sizethr}}
%% \hline  
%% \textbf{name} & \textbf{routines} & \textbf{\ description} \\ \hline



%% \end{tabular}
%% %--------------------------------------------------------
%% \begin{tabular}{p{\sizeone} p{\sizetwo} p{\sizethr}}
%% \hline  
%% \textbf{name} & \textbf{routines} & \textbf{\ description} \\ \hline

\end{tabular}
\mollbacktotop
\end{keywords_mollview}
%***************************************************************

\begin{codedescription}
{\thedocid \ reads in a \healpix sky map in FITS format and generates a
Mollweide projection of it, that can be visualized on the screen or
exported in a PNG or Postscript file.\thedocid \  allows the selection of
the coordinate system, map size, color table, color bar inclusion,
linear or log scaling, histogram equalised
color scaling, maximum and 
minimum range for the plot, plot-title {\it etc}. It also allows the representation of the
polarization field. }
\end{codedescription}


%
% defines the field 'RELATED' for mollview, gnomview, orthview,
% cartview

\begin{related}
  \begin{sulist}{} %%%% NOTE the ``extra'' brace here %%%%
  \item[idl] version \idlversion or more is necessary to run \thedocid
  \item[ghostview] ghostview or a similar facility is required to view
	  the Postscript image generated by \thedocid.
  \item[xv] xv or a similar facility is required to view the
            GIF/PNG image generated by \thedocid  (a browser can also 
            be used).
  \item[synfast] This \healpix facility will generate the FITS format 
            sky map to be input to \thedocid.
  \item[{\htmlref{cartview}{idl:cartview}}] 
	IDL facility to generate a Cartesian projection of
  	a \healpix map.
  \item[{\htmlref{cartcursor}{idl:cartcursor}}] 
	interactive cursor to be used with cartview
  \item[{\htmlref{gnomview}{idl:gnomview}}] 
	IDL facility to generate a gnomonic projection of
  	a \healpix map.
  \item[{\htmlref{gnomcursor}{idl:gnomcursor}}] 
	interactive cursor to be used with gnomview
  \item[{\htmlref{mollview}{idl:mollview}}] 
	IDL facility to generate a Mollweide projection of
  	a \healpix map.
  \item[{\htmlref{mollcursor}{idl:mollcursor}}] interactive cursor to be used with mollview
  \item[{\htmlref{orthview}{idl:orthview}}] 
	IDL facility to generate an orthographic projection of
  	a \healpix map.
  \item[{\htmlref{orthcursor}{idl:orthcursor}}] 
	interactive cursor to be used with orthview
  \end{sulist}
\end{related}

% defines the field 'RELATED' for mollview, gnomview, orthview,
% cartview

\begin{related}
  \begin{sulist}{} %%%% NOTE the ``extra'' brace here %%%%
  \item[idl] version \idlversion or more is necessary to run \thedocid
  \item[ghostview] ghostview or a similar facility is required to view
	  the Postscript image generated by \thedocid.
  \item[xv] xv or a similar facility is required to view the
            GIF/PNG image generated by \thedocid \ (a browser can also 
            be used).
  \item[synfast, smoothing] These F90 \healpix facilities will generate the FITS format 
            sky maps to be input to \thedocid.
  \item[\htmlref{isynfast}{idl:isynfast}, \htmlref{ismoothing}{idl:ismoothing}] These IDL routines will generate the FITS format 
            sky maps to be input to \thedocid.
  \item[{\htmlref{cartview} {idl:cartview}}] 
	IDL facility to generate a Cartesian projection of
  	a \healpix map.
  \item[{\htmlref{cartcursor} {idl:cartcursor}}] 
	interactive cursor to be used with cartview
  \item[{\htmlref{gnomview} {idl:gnomview}}] 
	IDL facility to generate a gnomonic projection of
  	a \healpix map.
  \item[{\htmlref{gnomcursor}{idl:gnomcursor}}] 
	interactive cursor to be used with gnomview
  \item[{\htmlref{mollview}{idl:mollview}}] 
	IDL facility to generate a Mollweide projection of
  	a \healpix map.
  \item[{\htmlref{mollcursor}{idl:mollcursor}}] interactive cursor to be used with mollview
  \item[{\htmlref{orthview}{idl:orthview}}] 
	IDL facility to generate an orthographic projection of
  	a \healpix map.
  \item[{\htmlref{orthcursor}{idl:orthcursor}}] 
	interactive cursor to be used with orthview
  \end{sulist}
\end{related}



%-------------------------------------------
\begin{examples}
{2}
{
\begin{tabular}{l} %%%% use this tabular format %%%%

map  = findgen(48) \\
triangle= create\_struct('coord','G','ra',[0,80,0],'dec',[40,45,65]) \\
\htmlref{\thedocid}{idl:mollview},\mylink{idl:mollview:file}{map}, \mylink{idl:mollview:graticule}{graticule}=[45,30],\mylink{idl:mollview:rot}{rot}=[10,20,30],\$ \\
$\quad\quad$	  \mylink{idl:mollview:titleplot}{title}='Mollweide projection',\mylink{idl:mollview:subtitle}{subtitle}='mollview', \$ \\
$\quad\quad$          \mylink{idl:mollview:outline}{outline}=triangle \\
\end{tabular}
}
{makes a Mollweide projection of a pixel index map (see Figure~\ref{fig:plot_visu}c on
page~\pageref{page:plot_visu}) after an arbitrary \mylink{idl:mollview:graticule}{rotation}, with a \mylink{idl:mollview:graticule}{graticule} grid
(with a 45$^o$ step in longitude and 30$^o$ in latitude) and an arbitrary
(triangular) \mylink{idl:mollview:outline}{outline}}
\end{examples}
%--------
\begin{figure}[h!]
\latexhtml{%for latex
\centerline{\includegraphics[bb=0pt 0pt 550pt 374pt, width=\textwidth,clip]{fig/merge_visu}}
}{%for html
%\centerline{\includegraphics[bb=1pt 1pt 600pt 800pt, width=5in]{fig/merge_visu}\myhtmlimage{}}
%%%\centerline{\includegraphics[width=0.5\textwidth]{fig/merge_visu_large}{}}
\centerline{\includegraphics[width=520pt]{fig/merge_visu_large}{}}%rescaled for JPL web site -> ~720
}
\caption{%
\label{page:plot_visu}%
\label{fig:plot_visu}%
Figures produced by \htmlref{cartview}{idl:cartview},
\htmlref{gnomview}{idl:gnomview}, \htmlref{mollview}{idl:mollview} and \htmlref{orthview}{idl:orthview}, see respective
routine documentation for details.}
\end{figure}
%--------

%-------------------------------------------
\begin{examples}
{3}
{
\begin{tabular}{l} %%%% use this tabular format %%%%

map  = findgen(48) \\
mycommand = 'x=findgen(64)/10. \& ' + \$ \\
$\quad\quad$	'plot,x,sin(x),pos=[0.8,0.8,0.99,0.99],/noerase \&' +\$ \\
$\quad\quad$	'xyouts,0.5,0.5,''Hello World !'',/normal,charsize=2,align=0.5'  \\
\htmlref{\thedocid}{idl:mollview},map, \mylink{idl:mollview:execute}{execute}=mycommand, \mylink{idl:mollview:png}{png}='plot\_example\_execute.png',\mylink{idl:mollview:preview}{/preview},\$ \\
$\quad\quad$	\mylink{idl:mollview:graticule}{/graticule},\mylink{idl:mollview:glsize}{/glsize} \\
\end{tabular}
}
{produces a PNG file containing a Mollweide projection of a pixel index map
with labeled graticules, a simple sine wave in the
upper right corner, and some greetings, as shown on Figure~\ref{fig:plot_example_execute} on page~\pageref{page:plot_example_execute}
}
\end{examples}
%-------
\begin{figure}[h!]
\latexhtml{%for latex
\centerline{\includegraphics[bb=0pt 20pt 800pt 500pt, width=0.7\textwidth,clip]{fig/plot_example_execute}}
}{%for html
%%%\centerline{\includegraphics[width=0.5\textwidth]{fig/plot_example_execute}{}}
\centerline{\includegraphics[width=520pt]{fig/plot_example_execute}{}}%rescaled for JPL web site -> ~720
}
\caption{%
\label{page:plot_example_execute}%
\label{fig:plot_example_execute}%
Figure produced by example \#3.}
\end{figure}
%-------



%-------------------------------------------
\label{page:example_hires_cutsky}
\begin{examples}
{4}
{
\begin{tabular}{l} %%%% use this tabular format %%%%

pixel  = l64indgen(400000) \\
signal = pixel * 10.0 \\
file = 'cutsky.fits' \\
\htmlref{write\_fits\_cut4}{idl:write_fits_cut4}, file, pixel+100000, signal, nside=32768, /ring \\
\htmlref{gnomview}{idl:gnomview}, \mylink{idl:mollview:file}{file}, \mylink{idl:mollview:graticule}{rot}=[0,90], \mylink{idl:mollview:graticule}{grat}=30, \mylink{idl:mollview:titleplot}{title}='high res. cut-sky map' \\
\end{tabular}
}
{produces and plots a high resolution map (6.4 arcsec/pixel), in which only a very small subset of
pixels is observed}
\end{examples}
%-------------------------------------------
\begin{examples}
{5}
{
\begin{tabular}{l} %%%% use this tabular format %%%%

file = 'wmap\_band\_iqumap\_r9\_5yr\_K\_v3.fits' \\
\htmlref{\thedocid}{idl:mollview}, \mylink{idl:mollview:file}{file}, \mylink{idl:mollview:titleplot}{title}='Linear Color Scale', \mylink{idl:mollview:silent}{/silent} \\
\thedocid, file,\mylink{idl:mollview:asinh}{/asinh},title='Sinh!u-1!n Color Scale' , /silent \\
\thedocid, file,\mylink{idl:mollview:hist_equal}{/hist}, title='Histogram Equalized Color Scale', /silent \\
\thedocid, file,\mylink{idl:mollview:log}{/log},  title='Log Scale', /silent \\
\end{tabular}
}
{produces Mollweide projections of the same map (here the WMAP-5yr K band) with
various color scales: linear, Inverse
Hyperbolic Sine, Histogram Equalized, and Log. See Figure~\ref{fig:merge_wmapKband} on page~\pageref{page:merge_wmapKband}
}
\end{examples}
%
\begin{figure}[h!]
\latexhtml{%for latex
%\centerline{\includegraphics[bb=0pt 0pt 842pt 595pt, width=0.99\textwidth,clip]{fig/merge_wmapKband}}
\centerline{\includegraphics[bb=30pt 40pt 812pt 510pt, width=0.99\textwidth,clip]{fig/merge_wmapKband}}
}{%for html
%%%\centerline{\includegraphics[width=0.5\textwidth]{fig/merge_wmapKband}{}}
\centerline{\includegraphics[width=520pt]{fig/merge_wmapKband}{}}%rescaled for JPL web site -> ~720
}
\caption{%
\label{page:merge_wmapKband}%
\label{fig:merge_wmapKband}%
Illustration (generated by example~\#5) of the various color scales available.}
\end{figure}

%-------------------------------------------
\begin{examples}
{1}
{
\begin{tabular}{l} %%%% use this tabular format %%%%
\htmlref{\thedocid}{idl:mollview},  \mylink{idl:mollview:file}{'planck100GHZ-LFI.fits'}, \mylink{idl:mollview:min}{min}=-100, \mylink{idl:mollview:max}{max}=100, \mylink{idl:mollview:graticule}{/graticule}, \$ \\
$\quad$	 \mylink{idl:mollview:titleplot}{title}='Simulated Planck LFI Sky Map at 100GHz'\\
\end{tabular}
}
{\thedocid \ reads in the map 'planck100GHZ-LFI.fits' and generates
an output image in which
the temperature scale has been set to lie between $\pm$ 100 ($\mu$K),
a \mylink{idl:mollview:graticule}{graticule} with a 45 degree step in longitude and latitude is drawn,
and the \mylink{idl:mollview:titleplot}{title} 'Simulated Planck LFI Sky Map at 100GHz' appended to the image.
}
\end{examples}

