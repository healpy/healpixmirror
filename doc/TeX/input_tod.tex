
\sloppy


\title{\healpix Fortran Subroutines Overview}
\docid{input\_tod*} \section[input\_tod*]{ }
\label{sub:input_tod}
\docrv{Version 2.0}
\author{Eric Hivon \& Frode K.~Hansen}
\abstract{This document describes the \healpix Fortran90 subroutine INPUT\_TOD*.}

\begin{facility}
{This routine reads a large binary table (for instance a Time Ordered Data
 set) from a FITS file. The user can choose to read only a section of the table,
 starting from an arbitrary position. 
The data can be read into a single or double precision array.}
{\modFitstools}
\end{facility}

\begin{f90format}
{\mylink{sub:input_tod:filename}{filename}%
, \mylink{sub:input_tod:tod}{tod}%
, \mylink{sub:input_tod:npix}{npix}%
, \mylink{sub:input_tod:ntods}{ntods}%
 \optional{[, \mylink{sub:input_tod:header}{header}%
, \mylink{sub:input_tod:firstpix}{firstpix}%
, \mylink{sub:input_tod:fmissval}{fmissval}%
]}}
\end{f90format}

\begin{arguments}
{
\begin{tabular}{p{0.3\hsize} p{0.05\hsize} p{0.05\hsize} p{0.5\hsize}} \hline  
\textbf{name~\&~dimensionality} & \textbf{kind} & \textbf{in/out} & \textbf{description} \\ \hline
                   &   &   &                           \\ %%% for presentation
filename\mytarget{sub:input_tod:filename} (LEN = \filenamelen) & CHR & IN & FITS file to be read from \\
tod\mytarget{sub:input_tod:tod}(0:npix-1,1:ntods)    & SP/ DP & OUT & array constructed
                   from the data present in the file (from the sample {\tt
                   firstpix} to {\tt firstpix + npix} - 1. Missing pixels or time
                   samples are filled with {\tt fmissval}. \\
npix\mytarget{sub:input_tod:npix}      & I8B & IN & number of pixels or samples to be read. See Note below. \\
ntods\mytarget{sub:input_tod:ntods}     & I4B & IN &  number of columns to read  \\
\optional{header\mytarget{sub:input_tod:header}}(LEN=80)(1:) \ (OPTIONAL)    & CHR & OUT &   FITS extension header \\
\optional{firstpix\mytarget{sub:input_tod:firstpix}}  & I8B & IN & first pixel (or time sample) to read from
                   (0 based). \default 0. See Note below. \\
\optional{fmissval\mytarget{sub:input_tod:fmissval}}  & SP/ DP & IN &  value to be given to missing pixels, its default
                   value is 0. Should be of the same type as {\tt tod}.
\end{tabular}
{\bf Note :} Indices and number of data elements larger than
                   $2^{31}$ are only accessible in FITS files on computers with 64 bit
                   enabled compilers and with some specific compilation options of
                   cfitsio (see cfitsio documentation).
}
\end{arguments}

\begin{modules}
  \begin{sulist}{} %%%% NOTE the ``extra'' brace here %%%%
  \item[\textbf{fitstools}] module, containing:
  \item[printerror] routine for printing FITS error messages.
  \item[\textbf{cfitsio}] library for FITS file handling.		
  \end{sulist}
\end{modules}

\begin{related}
  \begin{sulist}{} %%%% NOTE the ``extra'' brace here %%%%
  \item[anafast] executable that reads a \healpix map and analyses it. 
  \item[synfast] executable that generate full sky \healpix maps
  \item[\htmlref{getsize\_fits}{sub:getsize_fits}] subroutine to know the size of a FITS file.
  \item[\htmlref{write\_bintabh}{sub:write_bintabh}] subroutine to write large arrays into FITS files
  \item[\htmlref{output\_map}{sub:output_map}] subroutine to write a FITS file from a \healpix map
  \item[\htmlref{input\_map}{sub:input_map}] subroutine to read a \healpix map
  (either full sky of cut sky) from a FITS file
  \end{sulist}
\end{related}

\rule{\hsize}{2mm}

\newpage
