
\sloppy


\title{\healpix Fortran Subroutines Overview}
\docid{write\_asctab*} \section[write\_asctab*]{ }
\label{sub:write_asctab}
\docrv{Version 2.0}
\author{Eric Hivon, Frode K.~Hansen}
\abstract{This document describes the \healpix Fortran90 subroutine WRITE\_ASCTAB.}

\begin{facility}
{This routine stores a power spectrum in an ascii FITS-file. The routine can store temperature coeffecients $C_l^T$ or both temperature and polarisation coeffecients $C_l^T$, $C_l^E$, $C_l^B$, $C_l^{T\times E}$.}
{\modFitstools}
\end{facility}

\begin{f90format}
{\mylink{sub:write_asctab:clout}{clout}%
, \mylink{sub:write_asctab:lmax}{lmax}%
, \mylink{sub:write_asctab:ncl}{ncl}%
, \mylink{sub:write_asctab:header}{header}%
, \mylink{sub:write_asctab:nlheader}{nlheader}%
, \mylink{sub:write_asctab:filename}{filename}%
}
\end{f90format}

\begin{arguments}
{
\begin{tabular}{p{0.4\hsize} p{0.05\hsize} p{0.1\hsize} p{0.35\hsize}} \hline  
\textbf{name~\&~dimensionality} & \textbf{kind} & \textbf{in/out} & \textbf{description} \\ \hline
                   &   &   &                           \\ %%% for presentation
filename\mytarget{sub:write_asctab:filename}(LEN=\filenamelen) & CHR & IN & the FITS file to which the power spectrum is written. \\
lmax\mytarget{sub:write_asctab:lmax} & I4B & IN & Maximum $\ell$ value to be written. \\
ncl\mytarget{sub:write_asctab:ncl} & I4B & IN & 1 for temperature coeffecients only, 4 for polarisation. \\
clout\mytarget{sub:write_asctab:clout}(0:lmax,1:ncl) & SP/ DP & IN & the powerspectrum to be saved in the file.\\
nlheader\mytarget{sub:write_asctab:nlheader} & I4B & IN & number of header lines to write to the file. \\
header\mytarget{sub:write_asctab:header}(LEN=80) (1:nlheader) & CHR & IN & the header to the FITS-file. \\ 
\end{tabular}
}
\end{arguments}

\begin{example}
{
call write\_asctab (cl,64,1,header,80,'cl.fits')  \\
}
{
Writes a powerspectrum in the array cl(0:64,1:1) to a FITS-file called `cl.fits'. The cl array contains the temperature powerspectrum $C_l^T$ up to an $\ell$ value of 64. 80 header lines are written to the file from the array header(1:80).
}
\end{example}
\newpage
\begin{modules}
  \begin{sulist}{} %%%% NOTE the ``extra'' brace here %%%%
  \item[\textbf{fitstools}] module, containing:
  \item[printerror] routine for printing FITS error messages.
  \item[\textbf{cfitsio}] library for FITS file handling.		
  \end{sulist}
\end{modules}

\begin{related}
  \begin{sulist}{} %%%% NOTE the ``extra'' brace here %%%%
  \item[\htmlref{alm2cl}{sub:alm2cl}] Routine computing the power spectrum from
  spherical harmonics coefficients $a_{\ell m}$
  \item[\htmlref{fits2cl}{sub:fits2cl}] Routine to read a FITS file created by write\_asctab.
  \item[\htmlref{write\_minimal\_header}{sub:write_minimal_header}] routine to write minimal FITS header
  \end{sulist}
\end{related}

\rule{\hsize}{2mm}

\newpage
