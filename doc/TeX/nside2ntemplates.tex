
\sloppy


\title{\healpix Fortran Subroutines Overview}
\docid{nside2ntemplates} \section[nside2ntemplates]{ }
\label{sub:nside2ntemplates}
\docrv{Version 1.1}
\author{E. Hivon}
\abstract{This document describes the \healpix Fortran90 subroutine NSIDE2NTEMPLATES.}

\begin{facility}
{Function to provide the number of template pixels $$\ntemplate=\frac{1+\nside(\nside+6)}{4}$$ corresponding
to resolution parameter $\nside$. Each template pixel has a different shape that
{\em can not} be matched (by rotation or reflexion) to that of any of the other templates.
}
{\modPixTools}
\end{facility}

\begin{f90function}
{nside}
\end{f90function}

\begin{arguments}
{
\begin{tabular}{p{0.3\hsize} p{0.05\hsize} p{0.1\hsize} p{0.45\hsize}} \hline  
\textbf{name~\&~dimensionality} & \textbf{kind} & \textbf{in/out} & \textbf{description} \\ \hline
                   &   &   &                           \\ %%% for presentation
nside & I4B & IN & the $\nside$ parameter. \\
var & I8B & OUT & the number of template pixels $\ntemplate$.\\
\end{tabular}
}
\end{arguments}

\begin{example}
{
use \htmlref{healpix\_modules}{sub:healpix_modules} \\
integer(\htmlref{I8B}{sub:healpix_types}) :: ntpl \\
ntpl= nside2ntemplates(256)  \\
}
{
Returns in {\tt ntpl} the number of \healpix template pixels (16768) for the resolution
parameter 256.
}
\end{example}
\begin{related}
  \begin{sulist}{} %%%% NOTE the ``extra'' brace here %%%%
  \item[\htmlref{template\_pixel\_ring}{sub:template_pixel_xxx}]
  \item[\htmlref{template\_pixel\_nest}{sub:template_pixel_xxx}] return the
  template pixel associated with any \healpix pixel
  \item[\htmlref{same\_shape\_pixels\_ring}{sub:same_shape_pixels_xxx}] 
  \item[\htmlref{same\_shape\_pixels\_nest}{sub:same_shape_pixels_xxx}] 
  return
  the ordered list of pixels having the same shape as a given pixel template
  \end{sulist}
\end{related}

\rule{\hsize}{2mm}

