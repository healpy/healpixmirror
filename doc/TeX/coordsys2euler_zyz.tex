\sloppy


\title{\healpix Fortran Subroutines Overview}
\docid{coordsys2euler\_zyz} \section[coordsys2euler\_zyz]{ }
\label{sub:coordsys2euler_zyz}
\docrv{Version 2.0}
\author{Eric Hivon}
\abstract{This document describes the \healpix Fortran90 subroutine COORDSYS2EULER\_ZYZ.}

\begin{facility}
{This routine returns the three Euler angles $\psi, \theta, \varphi
\myhtmlimage{}$, corresponding to a rotation between standard astronomical
coordinate systems. This angles can then be used in rotate\_alm}
{\modCoordVConvert}
\end{facility}

\begin{f90format}
{\mylink{sub:coordsys2euler_zyz:iepoch}{iepoch}%
, \mylink{sub:coordsys2euler_zyz:oepoch}{oepoch}%
, \mylink{sub:coordsys2euler_zyz:isys}{isys}%
, \mylink{sub:coordsys2euler_zyz:osys}{osys}%
, \mylink{sub:coordsys2euler_zyz:psi}{psi}%
, \mylink{sub:coordsys2euler_zyz:theta}{theta}%
, \mylink{sub:coordsys2euler_zyz:phi}{phi}%
}
\end{f90format}

\begin{arguments}
{
\begin{tabular}{p{0.26\hsize} p{0.05\hsize} p{0.09\hsize} p{0.50\hsize}} \hline  
\textbf{name~\&~dimensionality} & \textbf{kind} & \textbf{in/out} & \textbf{description} \\ \hline
                   &   &   &                           \\ %%% for presentation
iepoch\mytarget{sub:coordsys2euler_zyz:iepoch} & DP & IN & epoch of the input astronomical coordinate system.\\
oepoch\mytarget{sub:coordsys2euler_zyz:oepoch} & DP & IN & epoch of the output astronomical coordinate system.\\
isys\mytarget{sub:coordsys2euler_zyz:isys}(len=*) & CHR & IN & input coordinate system, should be one of 'E'=Ecliptic, 'G'=Galactic, 'C'/'Q'=Celestial/eQuatorial.\\
osys\mytarget{sub:coordsys2euler_zyz:osys}(len=*) & CHR & IN & output coordinate system, same choice as above.\\
psi\mytarget{sub:coordsys2euler_zyz:psi}	& DP & OUT & first Euler angle: rotation $\psi$ about the z-axis. \\
theta\mytarget{sub:coordsys2euler_zyz:theta}	& DP & OUT & second Euler angle: rotation $\theta$ about the original
(unrotated) y-axis; \\
phi\mytarget{sub:coordsys2euler_zyz:phi}	& DP & OUT & third Euler angle: rotation $\varphi$ about the original (unrotated) z-axis;
\end{tabular}
}
\end{arguments}

\begin{example}
{
use coord\_v\_convert, only: coordsys2euler\_zyz \\
use alm\_tools, only: rotate\_alm \\
...\\
call coordsys2euler\_zyz(2000.0\_dp, 2000.0\_dp, 'E', 'G', psi, theta, phi) \\
call rotate\_alm(64, alm\_TGC, psi, theta, phi)  \\
}
{
Rotate the $a_{\ell m}$ from Ecliptic to Galactic coordinates.
}
\end{example}

% \begin{modules}
%   \begin{sulist}{} %%%% NOTE the ``extra'' brace here %%%%
%   \item[\textbf{alm\_tools}] module, containing:
% 	\item[\htmlref{generate\_beam}{sub:generate_beam}] routine to generate beam window function
% 	\item[\htmlref{pixel\_window}{sub:pixel_window}] routine to generate pixel window function
%   \end{sulist}
% \end{modules}

\begin{related}
  \begin{sulist}{} %%%% NOTE the ``extra'' brace here %%%%
  \item[\htmlref{rotate\_alm}{sub:rotate_alm}] apply arbitrary sky rotation to a
  set of $a_{\ell m}$ coefficients.
  \item[\htmlref{xcc\_v\_convert}{sub:xcc_v_convert}] rotates a 3D coordinate
vector from one astronomical coordinate system to another.
  \end{sulist}
\end{related}

\rule{\hsize}{2mm}

\newpage
