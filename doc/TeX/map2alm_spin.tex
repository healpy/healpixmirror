
\sloppy


\title{\healpix Fortran Subroutines Overview}
\docid{map2alm\_spin*} \section[map2alm\_spin*]{ }
\label{sub:map2alm_spin}
\docrv{Version 2.1}
\author{Eric Hivon}
\abstract{This document describes the \healpix Fortran90 subroutine MAP2ALM\_SPIN*.}

\begin{facility}
{This routine extracts the alm coefficients out of maps of spin $s$ and $-s$.
%
A (complex) map $S$ of spin $s$ is a linear combination of the spin weighted harmonics ${_s}Y_{lm}$
\begin{equation}
	{_s}S(p) = \sum_{lm} {_s}a_{lm}\ \ {_s}Y_{lm}(p)
\end{equation}
for $l \ge |m|, l \ge |s|$,
and is such that ${_s}S^* = {_{-s}}S$.\\
The 
\linklatexhtml{usual phase convention for the spin weighted harmonics}%
{http://en.wikipedia.org/wiki/Spin-weighted_spherical_harmonics\#Calculating}%
{http://en.wikipedia.org/wiki/Spin-weighted_spherical_harmonics\#Calculating}
is
${_s}Y_{lm}^* = (-1)^{s+m} {_{-s}}Y_{l-m}$
and therefore 
${_s}a_{lm}^* = (-1)^{s+m} {_{-s}}a_{l-m}$.
%
The two (real) input maps for \thedocid\ are defined respectively as
\begin{eqnarray}
	{_{|s|}}S^+ &\myequal& ({_{|s|}}S + {_{-|s|}}S)/2 \\
	{_{|s|}}S^- &\myequal& ({_{|s|}}S - {_{-|s|}}S)/(2i).
\end{eqnarray}
%
\thedocid\ outputs the alm coefficients defined as
	%typo correction on spin sign of second term of RHS, 2009-09-03
\begin{eqnarray}
	{_{|s|}}a^{+}_{lm} &\myequal& - ( {_{|s|}}a_{lm} + (-1)^s {_{-|s|}}a_{lm} )/2 \\
	{_{|s|}}a^{-}_{lm} &\myequal& - ( {_{|s|}}a_{lm} - (-1)^s {_{-|s|}}a_{lm} )/(2i)
\end{eqnarray}
for $m\ge 0$, knowing that, just as for spin 0 maps, the
coefficients for $m<0$ are given by 
\begin{eqnarray}
{_{|s|}}a^{+}_{l-m} &\myequal& (-1)^m {_{|s|}}a^{+*}_{lm}, \\
{_{|s|}}a^{-}_{l-m} &\myequal& (-1)^m {_{|s|}}a^{-*}_{lm}.
\end{eqnarray}
%
With these definitions, ${_2}a^{+}$, ${_2}a^{-}$, ${_2}S^+$ and ${_2}S^-$
match \healpix  polarization $a^E, a^B, Q$ and $U$ respectively. However, for
$s=0$, $\ _{0}a^+_{lm} = -a^T_{lm}$, $\ _{0}a^-_{lm} = 0$, $\ {_0}S^+ = T$, $\
{_0}S^- = 0.$
}
{\modAlmTools}
\end{facility}

\begin{f90format}
{\mylink{sub:map2alm_spin:nsmax}{nsmax}%
, \mylink{sub:map2alm_spin:nlmax}{nlmax}%
, \mylink{sub:map2alm_spin:nmmax}{nmmax}%
, \mylink{sub:map2alm_spin:spin}{spin}%
, \mylink{sub:map2alm_spin:map}{map}%
, \mylink{sub:map2alm_spin:alm}{alm}%
 [, \mylink{sub:map2alm_spin:zbounds}{zbounds}%
, \mylink{sub:map2alm_spin:w8ring_TQU}{w8ring\_TQU}%
]}
\end{f90format}

\begin{arguments}
{
\begin{tabular}{p{0.4\hsize} p{0.05\hsize} p{0.05\hsize} p{0.40\hsize}} \hline  
\textbf{name~\&~dimensionality} & \textbf{kind} & \textbf{in/out} & \textbf{description} \\ \hline
                   &   &   &                           \\ %%% for presentation
nsmax\mytarget{sub:map2alm_spin:nsmax} & I4B & IN & the $N_{side}$ value of the map to analyse. \\
nlmax\mytarget{sub:map2alm_spin:nlmax} & I4B & IN & the maximum $\ell$ value for the analysis. \\
nmmax\mytarget{sub:map2alm_spin:nmmax} & I4B & IN & the maximum $m$ value for the analysis. \\
spin\mytarget{sub:map2alm_spin:spin} & I4B & IN & the spin $s$ of the maps to be analysed (only its absolute
value is relevant).\\
map\mytarget{sub:map2alm_spin:map}(0:12*nsmax**2-1, 1:2) & SP/ DP & IN & ${_{|s|}}S^+$ and ${_{|s|}}S^-$ input maps\\
alm\mytarget{sub:map2alm_spin:alm}(1:2, 0:nlmax, 0:nmmax) & SPC/ DPC & OUT & The ${_{|s|}}a^+_{lm}$ and
${_{|s|}}a^-_{lm}$ output values. \\
zbounds\mytarget{sub:map2alm_spin:zbounds}(1:2), \hskip 4cm OPTIONAL & DP & IN & section of the map on which to perform the $a_{lm}$
                   analysis, expressed in terms of $z=\sin({\rm latitude}) =
                   \cos(\theta).$ If zbounds(1)$<$zbounds(2), the analysis is
                   performed {\em on} the strip zbounds(1)$<z<$zbounds(2); if not,
                   it is performed {\em outside} of the strip
                   zbounds(2)$<z<$zbounds(1). \\
w8ring(1:2*nsmax,1:2), \hskip 3cm OPTIONAL & DP & IN & ring weights for quadrature corrections. If ring weights are not used, this array should be 1 everywhere.
\end{tabular}
}
\end{arguments}
%%\newpage

\begin{example}
{
use \htmlref{healpix\_types}{sub:healpix_types}\\
use alm\_tools\\
use pix\_tools\\
integer(\htmlref{i4b}{sub:healpix_types}) :: nside, lmax, spin \\
% real(\htmlref{dp}{sub:healpix_types}), allocatable, dimension(:,:) :: dw8 \\
% real(dp), dimension(2) :: z \\
real(\htmlref{sp}{sub:healpix_types}), allocatable, dimension(:,:) :: map \\
complex(\htmlref{spc}{sub:healpix_types}), allocatable, dimension(:,:,:) :: alm \\
\\
nside = 256 \\
lmax = 512 \\
spin = 5 \\
allocate(map(0:nside2npix(nside)-1,1:2)) \\
allocate(alm(1:2, 0:lmax, 0:lmax)\\
...\\
% allocate(dw8(1:2*nside, 1:2)) \\
% dw8 = 1.0\_dp \\
% z = sin(10.0\_dp * \htmlref{DEG2RAD}{sub:healpix_types}) \\
% call map2alm(nside, lmax, lmax, map, alm, ($\backslash$ z, -z $\backslash$) , dw8, plm(0:(lmax+1)*(lmax+2)*nside-1))  \\
call map2alm\_spin(nside, lmax, lmax, spin, map, alm)  \\
}
{
Analyses spin 5 and -5 maps. The maps have
an $N_{side}$ of 256, and the analysis is performed up
to 512 in $\ell$ and m. The resulting $a_{lm}$ coefficients for
are returned in alm.
}
\end{example}

\begin{modules}
  \begin{sulist}{} %%%% NOTE the ``extra'' brace here %%%%
  \item[ring\_analysis] Performs FFT for the ring analysis.
  \item[compute\_lam\_mm, get\_pixel\_layout, ]
  \item[gen\_lamfac\_der, gen\_mfac,  ] 
  \item[gen\_recfac, init\_rescale, l\_min\_ylm] Ancillary routines used
  for $\ {_s}Y_{\ell m}$ recursion
  \item[\textbf{misc\_util}] module, containing:
  \item[\htmlref{assert\_alloc}{sub:assert}] routine to print error message when an array is not
  properly allocated		
  \end{sulist}
Note: Starting with \htmlref{version 2.20}{sub:new2p20}, {\tt libpsht} routines will be called if $0<|s|\le100$.
\end{modules}

\begin{related}
  \begin{sulist}{} %%%% NOTE the ``extra'' brace here %%%%
%  \item[anafast] executable using \thedocid to analyse maps.
  \item[\htmlref{alm2map\_spin}{sub:alm2map_spin}] routine performing the inverse transform
of \thedocid.
   \item[\htmlref{map2alm}{sub:map2alm}] routine analyzing temperature and
polarization maps
%   \item[\htmlref{map2alm\_iterative}{sub:map2alm_iterative}] similar to
% \thedocid\ with iterative scheme.
  \end{sulist}
\end{related}

\rule{\hsize}{2mm}

\newpage
