% -*- LaTeX -*-

\sloppy

\title{\healpix IDL Facility User Guidelines}
\docid{ialteralm} \section[ialteralm]{ }
\label{idl:ialteralm}
\docrv{Version 1.0}
\author{Eric Hivon}
\abstract{This document describes the \healpix IDL facility \thedocid.}

\begin{facility}
{This IDL facility provides an interface to F90 '\htmlref{alteralm}{fac:alteralm}' facility. 
This program can be used to modify a set of $a_{\ell m}$ spherical harmonics
  coefficients, as those extracted by \htmlref{ianafast}{idl:ianafast} or 
  simulated by \htmlref{isynfast}{idl:isynfast}, before
  they are used as constraints on a isynfast run. Currently the alterations
  possible are %
\begin{itemize}
    \item rotation (using Wigner matrices) of the $a_{\ell m}$ from the input
    coordinate system to any other standard astrophysical coordinate system. The
    resulting $a_{\ell m}$ can be used with e.g. synfast to generate a map in the
    new coordinate system.
    \item removal of the pixel and beam window functions of the input
  $a_{\ell m}$ (corresponding to the pixel size and beam shape of the map from which
  they were extracted) and implementation of an arbitrary pixel and beam window
  function. 
 \begin{eqnarray}
 a_{\ell m}^\mathrm{OUT} = a_{\ell m}^\mathrm{IN} 
 \frac{B^\mathrm{OUT}(\ell) P^\mathrm{OUT}(\ell)}{B^\mathrm{IN}(\ell) 
 P^\mathrm{IN}(\ell)}, \label{eq:alteralm} 
\end{eqnarray}
where $P(\ell)$ is the pixel window function, and $B(\ell)$ is the beam window
 function (assuming a circular beam) or any other $\ell$ space filter (eg,
 Wiener filter). For an infinitely small pixel (or beam) one would have $P(\ell) =
 1$ (resp. $B(\ell) = 1$) for any $\ell$.
\end{itemize}
}
{src/idl/interfaces/ialteralm.pro}
\end{facility}

\begin{IDLformat}
{IALTERALM, 
\mylink{idl:ialteralm:alm_in}{alm\_in},  
\mylink{idl:ialteralm:alm_out}{alm\_out}, [
\mylink{idl:ialteralm:beam_file_in}{beam\_file\_in},
\mylink{idl:ialteralm:beam_file_out}{beam\_file\_out},
\mylink{idl:ialteralm:binpath}{binpath=},
\mylink{idl:ialteralm:coord_in}{coord\_in},
\mylink{idl:ialteralm:coord_out}{coord\_out},
\mylink{idl:ialteralm:epoch_in}{epoch\_in},
\mylink{idl:ialteralm:epoch_out}{epoch\_out},
\mylink{idl:ialteralm:fwhm_arcmin_in}{fwhm\_arcmin\_in},
\mylink{idl:ialteralm:fwhm_arcmin_out}{fwhm\_arcmin\_out},
\mylink{idl:ialteralm:help}{/help},
\mylink{idl:ialteralm:keep_tmp_files}{keep\_tmp\_files=}, 
\mylink{idl:ialteralm:lmax_out}{lmax\_out},
\mylink{idl:ialteralm:nlmax_out}{nlmax\_out},
\mylink{idl:ialteralm:nside_in}{nside\_in},
\mylink{idl:ialteralm:nside_out}{nside\_out},
\mylink{idl:ialteralm:nsmax_in}{nsmax\_in},
\mylink{idl:ialteralm:nsmax_out}{nsmax\_out},
\mylink{idl:ialteralm:silent}{/silent},
\mylink{idl:ialteralm:tmpdir}{tmpdir=},
\mylink{idl:ialteralm:windowfile_in}{windowfile\_in},
\mylink{idl:ialteralm:winfiledir_in}{winfiledir\_in},
\mylink{idl:ialteralm:windowfile_out}{windowfile\_out},
\mylink{idl:ialteralm:winfiledir_out}{winfiledir\_out}
]}
\end{IDLformat}

\begin{qualifiers}
  \begin{qulist}{} %%%% NOTE the ``extra'' brace here %%%%
   \item[alm\_in]  \mytarget{idl:ialteralm:alm_in} required input: input $a_{\ell m}$, must be a FITS file
   \item[alm\_out] \mytarget{idl:ialteralm:alm_out} required output: output $a_{\ell m}$, must be a FITS file

  \end{qulist}
\end{qualifiers}

\begin{keywords}
  \begin{kwlist}{} %%% extra brace

 \item[binpath=] \mytarget{idl:ialteralm:binpath} full path to back-end routine \default {\$HEXE/alteralm, then \$HEALPIX/bin/alteralm}\\
              -- a binpath starting with / (or $\backslash$), $~$ or \$ is interpreted as absolute\\
              -- a binpath starting with ./ is interpreted as relative to current directory\\
              -- all other binpaths are relative to \$HEALPIX

 \item[beam\_file\_in=]\mytarget{idl:ialteralm:beam_file_in} Beam window function of input $a_{\ell m}$,
                 either a FITS file or an array 
		(see \htmlref{''Beam window function files''}{fac:subsec:beamfiles} section
		in the \linklatexhtml{\healpix Fortran Facilities document}{facilities.pdf}{facilities.htm}). 
		If present, will override
	\mylink{idl:ialteralm:fwhm_arcmin_in}{{\tt fwhm\_arcmin\_in}}
                 \default{value of BEAM\_LEG  keyword read from
	\mylink{idl:ialteralm:alm_in}{\tt alm\_in}}

 \item[beam\_file\_out=] \mytarget{idl:ialteralm:beam_file_out} Beam window function of output alm, 
                 either a FITS file or an array (see \mylink{idl:ialteralm:beam_file_in}{{\tt beam\_file\_in}}. 
		If present and non-empty, will override
	\mylink{idl:ialteralm:fwhm_arcmin_out}{{\tt fwhm\_arcmin\_out}}
                 \default{'' (empty string, no beam window applied)}

 \item[coord\_in=]     \mytarget{idl:ialteralm:coord_in} Astrophysical coordinates system used to compute input $a_{\ell m}$.
                 Case-insensitive single letter code.
                 Valid choices are 'g','G' = Galactic, 'e','E' = Ecliptic,
                 'c','q','C','Q' = Celestial/eQuatorial.
                 \default{value of COORDSYS keyword read from
	\mylink{idl:ialteralm:alm_in}{\tt alm\_in}}

 \item[coord\_out=]    \mytarget{idl:ialteralm:coord_out} Astrophysical coordinates system of output alm.
                  \default{ \mylink{idl:ialteralm:coord_in}{{\tt coord\_in}} }

 \item[epoch\_in=]     \mytarget{idl:ialteralm:epoch_in} Astronomical epoch of input coordinates 
(\mylink{idl:ialteralm:coord_in}{{\tt coord\_in}})
                 \default{2000.0}
 \item[epoch\_out=]    \mytarget{idl:ialteralm:epoch_out} Astronomical epoch of output coordinates 
(\mylink{idl:ialteralm:coord_out}{{\tt coord\_out}})
                 \default{same as \mylink{idl:ialteralm:epoch_in}{\tt epoch\_in}}

 \item[fwhm\_arcmin\_in=]  \mytarget{idl:ialteralm:fwhm_arcmin_in} Full Width
Half-Maximum in arcmin of Gaussian beam applied to map from which are obtained
input $a_{\ell m}$.\\
           \default{value of FWHM keyword in \mylink{idl:ialteralm:alm_in}{\tt alm\_in}}

 \item[fwhm\_arcmin\_out=] \mytarget{idl:ialteralm:fwhm_arcmin_out} FWHM in
arcmin to be applied to output alm.\\ \default{\mylink{idl:ialteralm:fwhm_arcmin_in}{{\tt fwhm\_arcmin\_in}}}

 \item[/help]      \mytarget{idl:ialteralm:help} if set, prints extended help
\item[/keep\_tmp\_files] \mytarget{idl:ialteralm:keep_tmp_files} if set,
temporary files are not discarded at the end of the run

 \item[lmax\_out=, nlmax\_out=]    \mytarget{idl:ialteralm:lmax_out}
\mytarget{idl:ialteralm:nlmax_out} maximum multipole of output alm

 \item[nside\_in=, nsmax\_in=]     \mytarget{idl:ialteralm:nside_in}
\mytarget{idl:ialteralm:nsmax_in}
HEALPix resolution parameter of map
                  from which were computed input $a_{\ell m}$
                   \default{determined from \mylink{idl:ialteralm:alm_in}{\tt alm\_in}}

 \item[nside\_out=,nsmax\_out=]    \mytarget{idl:ialteralm:nside_out}\mytarget{idl:ialteralm:nsmax_out}
HEALPix resolution parameter Nside whose
                  window function will be applied to output alm.\\
                  Could be set to 0 for infinitely small pixels (no window)
                   \default{same as input \mylink{idl:ialteralm:nsmax_in}{{\tt nsmax\_in}}}

\item[/silent]    \mytarget{idl:ialteralm:silent} if set, works silently

\item[tmpdir=]      \mytarget{idl:ialteralm:tmpdir} directory in which are written temporary files 
\default {IDL\_TMPDIR (see IDL documentation)}

 \item[windowfile\_in=]  \mytarget{idl:ialteralm:windowfile_in}%
    FITS file containing pixel window for \mylink{idl:ialteralm:nside_in}{nside\_in}
        \default {determined automatically by back-end routine}.
      Do not set this keyword unless you really know what you are doing

  \item[winfiledir\_in=] \mytarget{idl:ialteralm:winfiledir_in}%
     directory where \mylink{idl:ialteralm:windowfile_in}{windowfile\_in} is to be found 
        \default {determined automatically by back-end routine}.
      Do not set this keyword unless you really know what you are doing

 \item[windowfile\_out=]  \mytarget{idl:ialteralm:windowfile_out}%
    FITS file containing pixel window for \mylink{idl:ialteralm:nside_out}{nside\_out}
        \default {determined automatically by back-end routine}.
      Do not set this keyword unless you really know what you are doing

  \item[winfiledir\_out=] \mytarget{idl:ialteralm:winfiledir_out}%
     directory where \mylink{idl:ialteralm:windowfile_out}{windowfile\_out} is to be found 
        \default {determined automatically by back-end routine}.
      Do not set this keyword unless you really know what you are doing

  \end{kwlist}
\end{keywords}  

\begin{codedescription}
{\thedocid\ is an interface to '\htmlref{alteralm}{fac:alteralm}' F90 facility. It
requires some disk space on which to write the parameter file and the other
temporary files. Most data can be provided/generated as an external FITS
file, or as a memory array.}
\end{codedescription}



\begin{related}
  \begin{sulist}{} %%%% NOTE the ``extra'' brace here %%%%
    \item[idl] version \idlversion or more is necessary to run \thedocid.
    \item[alteralm] F90 facility called by \thedocid.
%    \item[\htmlref{ialteralm}{idl:ialteralm}] IDL Interface to F90 \htmlref{alteralm}{fac:alteralm}
    \item[\htmlref{ianafast}{idl:ianafast}] IDL Interface to F90 \htmlref{anafast}{fac:anafast} and C++ anafast\_cxx
    \item[\htmlref{iprocess\_mask}{idl:iprocess_mask}] IDL Interface to F90 \htmlref{process\_mask}{fac:process_mask}
    \item[\htmlref{ismoothing}{idl:ismoothing}] IDL Interface to F90 \htmlref{smoothing}{fac:smoothing}
    \item[\htmlref{isynfast}{idl:isynfast}] IDL Interface to F90 \htmlref{synfast}{fac:synfast}
  \end{sulist}
\end{related}

\begin{example}
{
\begin{tabular}{l} %%%% use this tabular format %%%%
ialteralm, \htmlref{!healpix.path.test}{idl:init_healpix}+'alm.fits', '/tmp/alm\_equat.fits', \$ \\
    coord\_in='g',coord\_out='q'\\
\htmlref{isynfast}{idl:isynfast}, 0, alm\_in='/tmp/alm\_equat.fits', '/tmp/map\_equat.fits'\\
\htmlref{mollview}{idl:mollview},'/tmp/map\_equat.fits',1\\
\htmlref{mollview}{idl:mollview},'/tmp/map\_equat.fits',2
\end{tabular}
}
{
  This example script reads the test (polarised) $a_{\ell m}$ located in {\tt
\$HEALPIX/test/alm.fits} and rotates them from Galactic to Equatorial
  coordinates, it then synthetizes a map out of those,
  and finally plots its I and Q Stokes components (in Equatorial coordinates)
}
\end{example}


