% -*- LaTeX -*-


\renewcommand{\facname}{{write\_fits\_partial}}
\renewcommand{\FACNAME}{{WRITE\_FITS\_PARTIAL}}
\sloppy



\title{\healpix IDL Facility User Guidelines}
\docid{\facname} \section[\facname]{ }
\label{idl:write_fits_partial}
\docrv{Version 1.0}
\author{Eric Hivon}
\abstract{This document describes the \healpix facility \facname.}




\begin{facility}
{This IDL facility writes out a unpolarized or polarized \healpix map into a FITS file 
for a fraction of the sky.}
{src/idl/fits/write\_fits\_partial.pro}
\end{facility}

\begin{IDLformat}
{\FACNAME, \mylink{idl:write_fits_partial:File}{File}%
, \mylink{idl:write_fits_partial:Pixel}{Pixel}%
, \mylink{idl:write_fits_partial:IQU}{IQU}
 [% 
, \mylink{idl:write_fits_partial:COLNAMES}{COLNAMES=}%
, \mylink{idl:write_fits_partial:COORDSYS}{COORDSYS=}%
, \mylink{idl:write_fits_partial:EXTENSION}{EXTENSION=}%
, \mylink{idl:write_fits_partial:HDR}{HDR=}%
, \mylink{idl:write_fits_partial:HELP}{HELP=}%
, \mylink{idl:write_fits_partial:NESTED}{/NESTED}%
, \mylink{idl:write_fits_partial:NSIDE}{NSIDE=}%
, \mylink{idl:write_fits_partial:ORDERING}{ORDERING=}%
% , 
% \mylink{idl:write_fits_partial:POLARISATION}{/POLARISATION}%
, \mylink{idl:write_fits_partial:RING}{/RING}%
, \mylink{idl:write_fits_partial:UNITS}{UNITS=}%
, \mylink{idl:write_fits_partial:VERBOSE}{VERBOSE=}%
, \mylink{idl:write_fits_partial:XHDR}{XHDR=}%
]
}
\end{IDLformat}

\begin{qualifiers}
  \begin{qulist}{} %%%% NOTE the ``extra'' brace here %%%%
 	\item[{File}] \mytarget{idl:write_fits_partial:File}
          name of a FITS file in which the map is to be written

 	\item[{Pixel}] \mytarget{idl:write_fits_partial:Pixel}
	 (INT, LONG or LONG64 vector of length $n_{\rm p}$), \\ index of observed (or valid) pixels. Will be the first column of the FITS binary table

 	\item[{IQU}]  \mytarget{idl:write_fits_partial:IQU}
	 (FLOAT or DOUBLE array of size ($n_{\rm p}$, $n_{\rm c}$)), \\ 
		$I$, $Q$ and $U$ Stokes parameters of each pixel (if $n_{\rm c}=3$),
		or $I$ of each pixel (if $n_{\rm c}=1$)

  \end{qulist}
\end{qualifiers}

\begin{keywords}
  \begin{kwlist}{} %%% extra brace
       \item[{COLNAMES=}] \mytarget{idl:write_fits_partial:COLNAMES}
%		  (optional), \\
STRING vector with FITS table column names (beside PIXEL) (not case sensitive: [A-Z,0-9,\_])\\
         \default{\texttt{TEMPERATURE}, for 1 column, \\
                   \texttt{TEMPERATURE}, \texttt{Q\_POLARISATION}, \texttt{U\_POLARISATION}, for 3 columns,\\
                   or \texttt{C01}, \texttt{C02}, \texttt{C03}, \texttt{C04}, ... otherwise}\\
         If provided the number of COLNAMES must be $\ge$ the number of columns

       \item[{COORDSYS=}] \mytarget{idl:write_fits_partial:COORDSYS}
%		  (optional), \\
		if set to either 'C', 'E' or 'G',  specifies that the
		Healpix coordinate system is respectively Celestial=equatorial,
		  Ecliptic or Galactic.
		(The relevant keyword is then added/updated in the extension
		  header, but the map is NOT rotated)

	\item[{EXTENSION=}] \mytarget{idl:write_fits_partial:EXTENSION}
%		  (optional), \\
	  (0 based) extension number in which to write data. \default{0}.
	  If set to 0 (or not set) {\em a new file is written from scratch}.
	  If set to a value
		  larger than 1, the corresponding extension is added or
		  updated, as long as all previous extensions already exist.
		  All extensions of the same file should use the same ORDERING,
		  NSIDE and COORDSYS.

    	\item[HDR=] \mytarget{idl:write_fits_partial:HDR}%	
%		(optional), \\
		String array containing the information to be put in
		the primary header. 

    	\item[HELP=] \mytarget{idl:write_fits_partial:HELP}%
%		(optional), \\
		if set, an extensive help is displayed, and no file is written

	\item[{/NESTED}]\mytarget{idl:write_fits_partial:NESTED}%
%	(optional), \\
         if set, specifies that the map is in the NESTED ordering
	scheme\\
	\seealso \mylink{idl:write_fits_partial:ORDERING}{Ordering} 
	and \mylink{idl:write_fits_partial:RING}{Ring}

	\item[{NSIDE=}]  \mytarget{idl:write_fits_partial:NSIDE}
%		(optional), \\
		scalar integer, \healpix resolution parameter of the
		data set. The resolution parameter should be made
		available to the FITS file, either thru this
		qualifier, or via the header (see XHDR).

	\item[{ORDERING=}] \mytarget{idl:write_fits_partial:ORDERING}
%		  (optional), \\
		if set to either 'ring' or 'nested' (case un-sensitive),
		  specifies that the map is respectively in RING or NESTED
		  ordering scheme\\
		\seealso \mylink{idl:write_fits_partial:NESTED}{Nested} 
		and \mylink{idl:write_fits_partial:RING}{Ring} \\
	The ordering information should be made
		available to the FITS file, either thru a combination
		  of Ordering/Ring/Nested, or via the header (see XHDR).

% 	\item[{/POLARISATION}] \mytarget{idl:write_fits_partial:POLARISATION}
% 	  specifies that file will contain the I, Q and U polarisation
%            Stokes parameter in extensions 0, 1 and 2 respectively, and sets the
% FITS header keywords accordingly

	\item[{/RING}]  \mytarget{idl:write_fits_partial:RING} 
%		  (optional), \\
	if set, specifies that the map is in the RING ordering
	scheme\\
	\seealso \mylink{idl:write_fits_partial:ORDERING}{Ordering} 
	and \mylink{idl:write_fits_partial:NESTED}{Nested}

	\item[{UNITS=}] \mytarget{idl:write_fits_partial:UNITS}
%		(optional), \\
		STRING scalar or vector describing the physical units of the table columns 
	(except for the PIXEL one)
		if scalar, same units for all columns;
	if vector, each column can have its own units; if needed, 
	the last UNITS provided will be replicated for the remaining columns

	\item[{VERBOSE=}] \mytarget{idl:write_fits_partial:VERBOSE}
%		(optional), \\
		if set, the routine is verbose while writing the FITS file

    	\item[XHDR=] \mytarget{idl:write_fits_partial:XHDR}%
%		(optional), \\
		String array containing the information to be put in
		the extension header. 


   \end{kwlist}
\end{keywords}

\begin{codedescription}
{For more information on the FITS file format supported in \healpixns, 
including the one implemented in \facname,
see \url{\hpxfitsdoc}}
% {\parbox[t]{\hsize}{\facname writes out the information contained in {\tt Prim\_stc} and {\tt
% Exten\_stc} in the primary unit and extension of the FITS file
% {\tt File} respectively . Coordinate systems can also be specified by {\tt Coordsys}. Specifying the
% ordering scheme is compulsary and can be done either in {\tt Header} or by setting {\tt
% Ordering} or {\tt Nested} or {\tt Ring} to the correct value. If {\tt
% Ordering} or {\tt Nested} or {\tt Ring} is set, its value overrides what is
% given in {\tt Header}. \\

% The data is assumed to represent a full sky data set with 
% the number of data points npix = 12*Nside*Nside
% unless   
% \\Partial is set OR the input fits header contains OBJECT =
%                'PARTIAL' \\
%        AND \\
%          the Nside qualifier is given a valid value OR the FITS header contains
%                  a NSIDE}}
\end{codedescription}



\begin{related}
  \begin{sulist}{} %%%% NOTE the ``extra'' brace here %%%%
  \item[idl] version \idlversion or more is necessary to run \facname
  \item[\htmlref{read\_fits\_partial}{idl:read_fits_partial}] This \healpix IDL facility can be used to read in maps
  written by \facname.
  \item[%
\htmlref{write\_fits\_cut4}{idl:write_fits_cut4},
\htmlref{write\_fits\_partial}{idl:write_fits_partial},
\htmlref{write\_fits\_map}{idl:write_fits_map}]
  \item[%
\htmlref{write\_tqu}{idl:write_tqu},
\htmlref{write\_fits\_sb}{idl:write_fits_sb}]
\healpix IDL routines to write cut-sky and partial maps, full-sky maps, polarized full-sky maps and
arbitrary data sets into FITS files

  \item[sxaddpar] This IDL routine (included in \healpix package) can be used to update
  or add FITS keywords to the header in {\tt HDR} and {\tt XHDR}
  \end{sulist}
\end{related}


\begin{example}
{
\begin{tabular}{l} %%%% use this tabular format %%%%
nside = 512\\
pixel = lindgen(nside2npix(nside)/10)\\
\thedocid, 'map\_part\_T.fits', pixel, pixel*100., \$\\
\hspace{1em}		    nside=nside, units='K', /ring\\
\thedocid, 'map\_part\_TQU.fits', pixel, pixel{\#}[100.,1.,1.],  \$\\
\hspace{1em}		      nside=nside, units='K', /ring\\
\thedocid, 'map\_part\_xxxx.fits', pixel, pixel{\#}[1.,-2.,3.,-4.], \$\\
\hspace{1em}		       nside=nside,colnames=['c1',B2','xx','POWER'],units=['K','m','s','W'],\$\\
\hspace{1em} /ring\\
\end{tabular}
}
{will write in 'map\_part\_T.fits' a FITS binary table with the columns PIXEL and TEMPERATURE;
in 'map\_part\_TQU.fits' a table with the columns PIXEL, TEMPERATURE, Q\_POLARISATION and U\_POLARISATION;
and in 'map\_part\_xxxx.fits' a table with the columns PIXEL, C1, B2, XX and POWER.}
\end{example}
%
% \begin{examples}{2}
% {
% \begin{tabular}{l} %%%% use this tabular format %%%%
% nside = 512\\
% pixel = lindgen(nside2npix(nside)/10)\\
% \thedocid,  'map\_part\_TQU.fits',pixel, pixel#[100.,1.,1.],    nside=nside, units='K'\\
% \end{tabular}
% }
% {creates in 'map\_part\_TQU.fits'
% }% a FITS binary table with the columns PIXEL, TEMPERATURE, Q\_POLARISATION and U\_POLARISATION}
% \end{examples}
% %
% \begin{examples}{3}
% {
% \begin{tabular}{l} %%%% use this tabular format %%%%
% nside = 512\\
% pixel = lindgen(nside2npix(nside)/10)\\
% \thedocid,  'map\_part\_xxxx.fits',pixel, pixel#[1.,-2.,3.,-4.],    nside=nside, colnames=['c1',B2','xx','POWER'],units=['K','m','s','W']\\
% \end{tabular}
% }
% {creates in 'map\_part\_xxxx.fits' a FITS binary table with the columns PIXEL, C1, B2, XX and POWER}
% \end{examples}


