% -*- LaTeX -*-

% PLEASE USE THIS FILE AS A TEMPLATE FOR THE DOCUMENTATION OF YOUR OWN
% FACILITIES: IN PARTICULAR, IT IS IMPORTANT TO NOTE COMMENTS MADE IN
% THE TEXT AND TO FOLLOW THIS ORDERING. THE FORMAT FOLLOWS ONE USED BY
% THE COBE-DMR PROJECT.	
% A.J. Banday, April 1999.




\sloppy



\title{\healpix IDL Facility User Guidelines}
\docid{mollcursor} \section[mollcursor]{ }
\label{idl:\thedocid}
\docrv{Version 1.1}
\author{Eric Hivon}
\abstract{This document describes the \healpix facility mollcursor.}




\begin{facility}
{This IDL facility provides a point-and-click interface for finding
the astronomical location, value and pixel index of the pixels nearest 
to the pointed position on a Mollweide projection of a \healpix map.}
{src/idl/visu/mollcursor.pro}
\end{facility}


\begin{IDLformat}
{{MOLLCURSOR}%
, [\mylink{idl:mollcursor:cursor_type}{cursor\_type=}%
, \mylink{idl:mollcursor:file_out}{file\_out=}%
]}
\end{IDLformat}

\begin{qualifiers}
  \begin{qulist}{} %%%% NOTE the ``extra'' brace here %%%%
 	\item[{cursor\_type=}] \mytarget{idl:mollcursor:cursor_type} cursor type to be used \\
	\default {34}
 	\item[{file\_out=}] \mytarget{idl:mollcursor:file_out} file containing on output the list of
 	point selected with the cursor. \\
	If set to 1, the file will
 	take its default name: 'cursor\_catalog.txt'. \\
	If set to a non-empty character string, the file name will be that string
  \end{qulist}
\end{qualifiers}

\begin{codedescription}
{mollcursor should be run immediately following mollview. It gives the
longitude, latitude, map value and pixel number
corresponding to the cursor position in the window containing the map generated
by mollview. Mouse buttons are used to select the function :  

left button = display the information relative to the current cursor position, 

middle button = print out this information in the IDL command window 

right button = quit mollcursor\\
\\
{\bf{Note on Mac OS X, X11 and IDL cursor:}} 
depending on the Mac OS X version%
\footnote{the command {\tt sw\_vers -productVersion}
 can be used to know the Mac OS X version being used}
and most importantly on the X Window System being used,%
\footnote{the command
{\tt ls -lrt  \$HOME/Library/Preferences/*[xX]11*plist}
can be used to determine the X implementation and its configuration file}
the IDL function {\tt cursor}, and therefore \healpix \thedocid,
gnomcursor, $\ldots$ will not
work properly under X11. To solve this problem, type the relevant line below at your X11 prompt and restart X11.\\
If you are using Apple's X11, type under Tiger (10.4): \\
{\footnotesize {\tt defaults write com.apple.x11 wm\_click\_through -bool true}} \\
or, under Leopard (10.5), Snow Leopard (10.6), Lion (10.7): \\
{\footnotesize {\tt defaults write org.x.x11 wm\_click\_through -bool true}} \\
If you are using Xquartz (default under Montain Lion (10.8), Mavericks (10.9)
and Yosemite (10.10)): \\
{\footnotesize {\tt defaults write org.macosforge.xquartz.X11 wm\_click\_through -bool true}}\\
and if you are using MacPort's X11 (package xorg-server):\\
{\footnotesize {\tt defaults write org.macports.X11 wm\_click\_through -bool true}}\\
(see
%\htmladdnormallink{\small
%{http://marc.sauvage.free.fr/SApMUG/Xnotes.html}}{http://marc.sauvage.free.fr/SApMUG/Xnotes.html},
%\htmladdnormallink{\small 
%{https://sympa.obspm.fr/wws/arc/micros-mac/2008-06/msg00001.html}}{https://sympa.obspm.fr/wws/arc/micros-mac/2008-06/msg00001.html}
%and 
\htmladdnormallink{\small 
{http://www.idlcoyote.com/misc\_tips/maccursor.html}}{http://www.idlcoyote.com/misc_tips/maccursor.html}
and \linklatexhtml{''\healpix Installation Documentation''}{install.pdf}{install.htm}). \\
%To make the patch permanent, add that line into your .bashrc (or
%.cshrc, depending on your shell) file, and restart X11.\\
And finally, \thedocid\ obviously requires the '3 button mouse' to be enabled,
which can be done in the X11 Preferences menu, or if Xquartz is used (see {\tt
man Xquartz}) via:\\
{\footnotesize {\tt defaults write org.macosforge.xquartz.X11 enable\_fake\_buttons
-bool true}}
}
\end{codedescription}



% defines the field 'RELATED' for mollview, gnomview, orthview,
% cartview

\begin{related}
  \begin{sulist}{} %%%% NOTE the ``extra'' brace here %%%%
  \item[idl] version \idlversion or more is necessary to run \thedocid
  \item[ghostview] ghostview or a similar facility is required to view
	  the Postscript image generated by \thedocid.
  \item[xv] xv or a similar facility is required to view the
            GIF/PNG image generated by \thedocid  (a browser can also 
            be used).
  \item[synfast] This \healpix facility will generate the FITS format 
            sky map to be input to \thedocid.
  \item[{\htmlref{cartview}{idl:cartview}}] 
	IDL facility to generate a Cartesian projection of
  	a \healpix map.
  \item[{\htmlref{cartcursor}{idl:cartcursor}}] 
	interactive cursor to be used with cartview
  \item[{\htmlref{gnomview}{idl:gnomview}}] 
	IDL facility to generate a gnomonic projection of
  	a \healpix map.
  \item[{\htmlref{gnomcursor}{idl:gnomcursor}}] 
	interactive cursor to be used with gnomview
  \item[{\htmlref{mollview}{idl:mollview}}] 
	IDL facility to generate a Mollweide projection of
  	a \healpix map.
  \item[{\htmlref{mollcursor}{idl:mollcursor}}] interactive cursor to be used with mollview
  \item[{\htmlref{orthview}{idl:orthview}}] 
	IDL facility to generate an orthographic projection of
  	a \healpix map.
  \item[{\htmlref{orthcursor}{idl:orthcursor}}] 
	interactive cursor to be used with orthview
  \end{sulist}
\end{related}

\begin{example}
{
\begin{tabular}{ll} %%%% use this tabular format %%%%
mollcursor & \ 
\end{tabular}
}
{After mollview reads in a map and generates
its mollweide projection, mollcursor is run to know the
position and flux of bright synchrotron sources, for example.}
\end{example}


