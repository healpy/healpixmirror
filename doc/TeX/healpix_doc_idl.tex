% -*- LaTeX -*-


% \renewcommand{\facname}{{healpix\_doc }}
% \renewcommand{\FACNAME}{{HEALPIX\_DOC }}

\sloppy



\title{\healpix IDL Facility User Guidelines}
\docid{healpix\_doc} \section[healpix\_doc: PDF and HTML documentation]{ }
\label{idl:healpix_doc}
\docrv{Version 1.0}
\author{Eric Hivon}
\abstract{This document describes the \healpix IDL facility \thedocid.}




\begin{facility}
{This IDL facility displays HTML or PDF \healpix documentation}
{src/idl/misc/healpix\_doc.pro}
\end{facility}

\begin{IDLformat}
{\thedocid , [HTML={\tt |} PDF=] [, HELP=,  WHOLE=] }
\end{IDLformat}

\begin{keywords}
  \begin{kwlist}{} %%% extra brace
    \item[HELP=] if set, an extensive help on \thedocid\ is displayed.
    \item[HTML=]  if set, the \healpix (IDL) HTML documentation is shown with a web browser.
            If the browser is already in use, a new tab is open.
    \item[PDF=]   if set, the \healpix (IDL) PDF documentation is shown with a pdf viewer.\\
            Either HTML or PDF must be set.
    \item[WHOLE=]  if set, the whole \healpix documentation is accessible,
              not just the IDL related part.
  \end{kwlist}
\end{keywords}  

\begin{codedescription}
{\thedocid\  calls {\tt Online\_help} to open either the HTML or PDF \healpix
documentation. The browser and viewer used are those found by the 
{\tt \$IDL\_DIR/bin/online\_help\_html} and 
{\tt \$IDL\_DIR/bin/online\_help\_pdf} scripts respectively.
The content of the {\tt !healpix} system variable is used to
determine the documentation path.}
\end{codedescription}



\begin{related}
  \begin{sulist}{} %%%% NOTE the ``extra'' brace here %%%%
    \item[idl] version \idlversion or more is necessary to run \thedocid.	
    \item[\htmlref{!HEALPIX}{idl:init_healpix}] IDL system variable used by
\thedocid\ to locate the documentation.
  \end{sulist}
\end{related}

\begin{examples}{1}
{
\begin{tabular}{ll} %%%% use this tabular format %%%%
\thedocid, /html, /whole
\end{tabular}
}
{will open the whole \healpix HTML documentation in a web browser.
}
\end{examples}
\begin{examples}{2}
{
\begin{tabular}{ll} %%%% use this tabular format %%%%
\thedocid, /pdf
\end{tabular}
}
{will open the IDL related \healpix PDF documentation.
}
\end{examples}

