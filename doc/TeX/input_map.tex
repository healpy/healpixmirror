
\sloppy


\title{\healpix Fortran Subroutines Overview}
\docid{input\_map*} \section[input\_map*]{ }
\label{sub:input_map}
\docrv{Version 1.3}
\author{Eric Hivon \& Frode K.~Hansen}
\abstract{This document describes the \healpix Fortran90 subroutine INPUT\_MAP.}

\begin{facility}
{This routine reads a \healpix map from a FITS file. This can deal with full sky
as well as cut sky maps.}
{\modFitstools}
\end{facility}

\begin{f90format}
{\mylink{sub:input_map:filename}{filename}%
, \mylink{sub:input_map:map}{map}%
, \mylink{sub:input_map:npixtot}{npixtot}%
, \mylink{sub:input_map:nmaps}{nmaps}%
 \optional{[, \mylink{sub:input_map:fmissval}{fmissval}%
, \mylink{sub:input_map:header}{header}%
, \mylink{sub:input_map:units}{units}%
, \mylink{sub:input_map:extno}{extno}%
, \mylink{sub:input_map:ignore_polcconv}{ignore\_polcconv}%
]}}
\end{f90format}
\aboutoptional

\begin{arguments}
{
\begin{tabular}{p{0.3\hsize} p{0.05\hsize} p{0.05\hsize} p{0.5\hsize}} \hline  
\textbf{name~\&~dimensionality} & \textbf{kind} & \textbf{i/o} & \textbf{description} \\ \hline
                   &   &   &                           \\ %%% for presentation
filename\mytarget{sub:input_map:filename}(len=\filenamelen) & CHR & IN & FITS file to be read from,
                   containing a full sky or cut sky map \\
map\mytarget{sub:input_map:map}(0:npixtot-1,1:nmaps)    & SP/ DP & OUT & full sky map(s) constructed
                   from the data present in the file, missing pixels are filled
                   with fmissval \\
npixtot\mytarget{sub:input_map:npixtot}                    & I4B/ I8B & IN & number of pixels in the full sky map \\
nmaps\mytarget{sub:input_map:nmaps}     & I4B & IN &  number of maps in the file  \\
                   &   &   &                           \\ %%% for presentation
\optional{fmissval\mytarget{sub:input_map:fmissval}}  & SP/ DP & IN &  value to be given to missing pixels,
\default{0}%
%its default value is 0 
\\
\optional{header\mytarget{sub:input_map:header}}(LEN=80)(1:)     & CHR & OUT &   FITS extension header \\
\optional{units\mytarget{sub:input_map:units}}(LEN=20)(1:nmaps)  & CHR & OUT &  maps units \\
\optional{extno\mytarget{sub:input_map:extno}}  & I4B & IN & extension number to read the data from
                   (0 based).\default{0} (the first extension is read) \\
\optional{ignore\_polccconv\mytarget{sub:input_map:ignore_polcconv}}  & LGT & IN & by default 
	(\texttt{ignore\_polcconv=.false.}) the output of this routine depends on the value of the FITS keyword
	\texttt{POLCCONV} found in \mylink{sub:input_map:filename}{filename}, as described in 
	the \htmlref{note on POLCCONV}{intro:polcconv} in \linklatexhtml{The \healpix Primer}{intro.pdf}{intro.htm}. 
	Setting \texttt{ignore\_polcconv=.true.} will force input\_map to ignore that keyword.
\end{tabular}
}
\end{arguments}

\begin{example}
{
use pix\_tools, only: nside2npix \\
use fitstools, only: getsize\_fits, input\_map \\
\ldots \\
npixtot = getsize\_fits('map.fits',nmaps=nmaps, nside=nside) \\
npix = nside2npix(nside) \\
allocate(map(0:npix-1,1:nmaps)) \\
call input\_map('map.fits', map, npix, nmaps, fmissval=0.)  \\
}
{
Reads into {\tt map} the content of the FITS file 'map.fits'.
If there are
missing pixels in the input file (ie, having value {\tt NaN} (Not of Number),
$\pm$ {\tt Infinity} or matching the FITS keyword {\tt BAD\_DATA}) they will
take on output the value provided in optional {\tt fmissval} (here 0, which also
is its default value).
}
\end{example}
%%\newpage
\begin{modules}
  \begin{sulist}{} %%%% NOTE the ``extra'' brace here %%%%
  \item[\textbf{fitstools}] module, containing:
  \item[printerror] routine for printing FITS error messages.
  \item[\htmlref{read\_bintab}{sub:read_bintab}] routine to read a binary table
  from a FITS file
  \item[\htmlref{read\_fits\_cut4}{sub:read_fits_cut4}] routine to read cut sky
  map from a FITS file
  \item[\textbf{cfitsio}] library for FITS file handling.
  \end{sulist}
\end{modules}

\begin{related}
  \begin{sulist}{} %%%% NOTE the ``extra'' brace here %%%%
  \item[anafast] executable that reads a \healpix map and analyses it. 
  \item[synfast] executable that generate full sky \healpix maps
  \item[\htmlref{getsize\_fits}{sub:getsize_fits}] subroutine to know the size of a FITS file.
  \item[\htmlref{output\_map}{sub:output_map}] subroutine to write a FITS file
  from a \healpix map
  \item[\htmlref{write\_bintabh}{sub:write_bintabh}] subroutine to write a large
  array into a FITS file piece by piece
  \item[\htmlref{input\_tod*}{sub:input_tod}] subroutine to read an arbitrary subsection of
  a large binary table
  \end{sulist}
\end{related}

\rule{\hsize}{2mm}

\newpage
