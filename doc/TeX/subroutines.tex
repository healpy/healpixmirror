% -*- LaTeX -*-

% PLEASE USE THIS FILE AS A TEMPLATE FOR THE DOCUMENTATION OF YOUR OWN
% FACILITIES: IN PARTICULAR, IT IS IMPORTANT TO NOTE COMMENTS MADE IN
% THE TEXT AND TO FOLLOW THIS ORDERING. THE FORMAT FOLLOWS ONE USED BY
% THE COBE-DMR PROJECT.	
% A.J. Banday, April 1999.

\documentclass[12pt,twoside]{article}
\usepackage{xr-hyper,healpix,html,makeidx,tabularx} 
%\usepackage{healpix,xr-hyper,html,makeidx,tabularx}
\usepackage{ae,lmodern}% load vectorial font, keep PDF small *and* good quality when in T1 font
\usepackage[T1]{fontenc}% underscore searchable in PDF, but larger PDF http://latex-community.org/forum/viewtopic.php?t=8891
\begin{htmlonly}
 \renewcommand{\ell}{l}
 \renewcommand{\lq}{'}
 % -*- LaTeX -*-
% This LaTeX file sets the Healpix version
% as it will appear in the documentation
% implement it with: % -*- LaTeX -*-
% This LaTeX file sets the Healpix version
% as it will appear in the documentation
% implement it with: % -*- LaTeX -*-
% This LaTeX file sets the Healpix version
% as it will appear in the documentation
% implement it with: \input{hpxversion}
% \newcommand{\hpxversion}{3.31}
% \newcommand{\hpxverstex}{3\_31}
% \newcommand{\hpxversion}{3.40}
% \newcommand{\hpxverstex}{3\_40}
% \newcommand{\hpxversion}{3.41}
% \newcommand{\hpxverstex}{3\_41}
% \newcommand{\hpxversion}{3.50}
% \newcommand{\hpxverstex}{3\_50}
\newcommand{\hpxversion}{3.60}
\newcommand{\hpxverstex}{3\_60}

% \newcommand{\hpxversion}{3.31}
% \newcommand{\hpxverstex}{3\_31}
% \newcommand{\hpxversion}{3.40}
% \newcommand{\hpxverstex}{3\_40}
% \newcommand{\hpxversion}{3.41}
% \newcommand{\hpxverstex}{3\_41}
% \newcommand{\hpxversion}{3.50}
% \newcommand{\hpxverstex}{3\_50}
\newcommand{\hpxversion}{3.60}
\newcommand{\hpxverstex}{3\_60}

% \newcommand{\hpxversion}{3.31}
% \newcommand{\hpxverstex}{3\_31}
% \newcommand{\hpxversion}{3.40}
% \newcommand{\hpxverstex}{3\_40}
% \newcommand{\hpxversion}{3.41}
% \newcommand{\hpxverstex}{3\_41}
% \newcommand{\hpxversion}{3.50}
% \newcommand{\hpxverstex}{3\_50}
\newcommand{\hpxversion}{3.60}
\newcommand{\hpxverstex}{3\_60}

\end{htmlonly}
%\ifpdf
\hypersetup{%
	pdftitle={HEALPix F90 subroutines Overview},%
	pdfauthor={E. Hivon et al},%
	pdfkeywords={HEALPix, F90, subroutines},%
	colorlinks=true}%
% \usepackage[pdftex]{color}%
% \definecolor{mygrn}{rgb}{0.1,0.5,0.1}%
% \newcommand{\optional}[1]{\textcolor{mygrn}{\textsl{#1}}}
%\fi


\newcommand{\nside}{N_\mathrm{side}}
\newcommand{\npix}{N_\mathrm{pix}}
\newcommand{\lmax}{\ell_{\mathrm{max}}}
\newcommand{\mmax}{m_{\mathrm{max}}}
\newcommand{\smax}{s_{\mathrm{max}}}
\newcommand{\ntemplate}{N_\mathrm{template}}
%\newcommand{\myhtmlimage}[1]{\htmlimage{#1}}
\newcommand{\myhtmlimage}[1]{ }
\newcommand{\aboutoptional}{Arguments appearing in \optional{italic} are
optional.}
\renewcommand{\contentsname}{{TABLE OF CONTENTS}}

% %%%%%%%%%%%%%% to allow reference to keyword from another PDF file
% \newcounter{word}
% \makeatletter
% \newcommand*{\LBL}{%
%   \@dblarg\@LBL
% }
% \def\@LBL[#1]#2{%
%   \begingroup
%     \renewcommand*{\theword}{#2}%
%     \refstepcounter{word}%
%     \label{#1}%
%     #2%
%   \endgroup
% }
% \makeatother

% command for external link
\newcommand{\linklatexhtml}[3]{% \linklatexhtml{name}{latex_target}{html_target}
\latexhtml{\htmladdnormallink{#1}{#2}}{\htmladdnormallink{#1}{#3}}}
% commands for arbitrary link
\newcommand{\mylink}[2]{% \mylink{link_id}{link_text}
\latexhtml{\hyperlink{#1}{#2}}{\hyperref{#2}{}{}{#1}}}
\newcommand{\mylinkext}[2]{% \mylink{link_id}{link_text}  (external link)
\latexhtml{\htmlref{#2}{#1}}{\hyperref{#2}{}{}{#1}}}
% commands for targets 
% http://tex.stackexchange.com/questions/17057/hypertarget-seems-to-aim-a-line-too-low
\makeatletter
     \newcommand{\nop}[1]{\Hy@raisedlink{\hypertarget{#1}{}}}
\makeatother
\newcommand{\mytarget}[1]{\nop{#1}\phantomsection\label{#1}}%    \mytarget{id}
\begin{htmlonly}
 \newcommand{\mytarget}[1]{\label{#1}}
\end{htmlonly}

% to make sure that in html, the &=& in eqnarray appears normally,
% replace &=& with &\myequal& hereafter
\newcommand{\myequal}{=}
\begin{htmlonly}
 \renewcommand{\myequal}{$=$}
\end{htmlonly}
%path to F90 modules
\newcommand{\modAlmTools}{src/f90/mod/alm\_tools.F90}
\newcommand{\modCoordVConvert}{src/f90/mod/coord\_v\_convert.f90}
\newcommand{\modExtension}{src/f90/mod/extension.F90}
\newcommand{\modFitstools}{src/f90/mod/fitstools.F90}
\newcommand{\modHeadFits}{src/f90/mod/head\_fits.F90}
\newcommand{\modHealpixFft}{src/f90/mod/healpix\_fft.F90}
\newcommand{\modHealpixModules}{src/f90/mod/healpix\_modules.f90}
\newcommand{\modHealpixTypes}{src/f90/mod/healpix\_types.F90}
\newcommand{\modLongIntrinsic}{src/f90/mod/long\_intrinsic.F90}
\newcommand{\modMaskTools}{src/f90/mod/mask\_tools.F90}
\newcommand{\modMiscUtils}{src/f90/mod/misc\_utils.F90}
\newcommand{\modMpiAlmTools}{src/f90/mod/mpi\_alm\_tools.f90}
\newcommand{\modParamfileIo}{src/f90/mod/paramfile\_io.F90}
\newcommand{\modPixTools}{src/f90/mod/pix\_tools.F90}
\newcommand{\modRngmod}{src/f90/mod/rngmod.f90}
\newcommand{\modStatistics}{src/f90/mod/statistics.f90}
\newcommand{\modUdgradeNr}{src/f90/mod/udgrade\_nr.f90}

\newcommand{\maskToolsRelated}{%
	\item[\htmlref{dist2holes\_nest}{sub:dist2holes_nest}] angular distance to
closest invalid pixel of the given mask
	\item[\htmlref{fill\_holes\_nest}{sub:fill_holes_nest}] turn to {\em valid} all
pixels located in 'holes' containing fewer pixels than the given threshold
	\item[\htmlref{maskborder\_nest}{sub:maskborder_nest}] identify inner
boundary pixels of 'holes' for given mask
	\item[\htmlref{size\_holes\_nest}{sub:size_holes_nest}] returns size (in
pixels) of holes found in input mask
}

% macros for old changes
\newcommand{\mysmaller}{% small in LaTeX, normal in html
\latexhtml{\scriptsize}{\normalsize}}
\newcommand{\compresstext}{% smaller vertical spacing in LaTeX, normal in html
\latexhtml{%
\setlength{\parsep}{-3ex}%
\setlength{\topsep}{0ex}%
\setlength{\parskip}{0ex}}{}}
\newcommand{\compresslist}{% smaller vertical spacing in list in LaTeX, normal in html
\setlength{\itemsep}{0ex}}{}


%\includeonly{}
\sloppy
\setcounter{secnumdepth}{0}
\setcounter{tocdepth}{1}

% CROSS-LINK
%%%%http://tex.stackexchange.com/questions/41539/does-hyperref-work-between-two-files
%% add xr-hyper (part of hyperref) to used packages, *BEFORE* hyperref (or html)
\newcommand{\facname}{} % must be here because it is in idl.tex
\newcommand{\FACNAME}{} % must be here because it is in idl.tex
\externaldocument{intro}
\externaldocument{install}
\externaldocument{idl}
\externaldocument{facilities}
%\externaldocument{subroutines}
\begin{htmlonly}
\externallabels{.}{/tmp/introlabels.pl}
\externallabels{.}{/tmp/intro_labels.pl}
\externallabels{.}{/tmp/installlabels.pl}
\externallabels{.}{/tmp/csublabels.pl}
\externallabels{.}{/tmp/csub_labels.pl}
\externallabels{.}{/tmp/idllabels.pl}
\externallabels{.}{/tmp/idl_labels.pl}
\externallabels{.}{/tmp/facilitieslabels.pl}
\externallabels{.}{/tmp/fac_labels.pl}
%\externallabels{.}{/tmp/subroutineslabels.pl}
%\externallabels{.}{/tmp/sub_labels.pl}
\end{htmlonly}


%%%%
\begin{document}
\title{\healpix Fortran90 Subroutines Overview}
\label{sub:subroutines}
\docrv{Version \hpxversion}
\author{Eric Hivon, Hans K.~Eriksen, Frode K.~Hansen, Benjamin D.~Wandelt, Krzysztof M.~G\'orski,
Anthony J.~Banday and Martin Reinecke}
\abstract{This document is an overview of the \healpix Fortran90 subroutines.}
% -*- LaTeX -*-
% This LaTeX file sets the Healpix website
% as it will appear in the documentation
% implement it with: % -*- LaTeX -*-
% This LaTeX file sets the Healpix website
% as it will appear in the documentation
% implement it with: % -*- LaTeX -*-
% This LaTeX file sets the Healpix website
% as it will appear in the documentation
% implement it with: \input{hpxwebsite}
%
% DEPRECATED !!!! replaced with hpxwebsites.tex
%

%
% DEPRECATED !!!! replaced with hpxwebsites.tex
%

%
% DEPRECATED !!!! replaced with hpxwebsites.tex
%

\date{\today}
\frontpage

\tableofcontents
\newpage
\section[Conventions]{{\Large Conventions}}
Here we list some conventions which are used in this document.
\\
\hrule
\begin{tabular}{@{}p{0.3\hsize}@{\hspace{1ex}}
                        p{0.7\hsize}@{}}  &  \\


$\mathbf{*}$ & Fortran90 allows generic names which refer to several specific
subroutines. Which one of the specific routines is called depends on
the type and rank of the arguments supplied in the call. We tag
generic names with a $*$ in this document.\\
\\
$\mathbf{\nside}$ & \healpix resolution parameter --- see the
\healpix Primer.\\
\\
$\mathbf{map}$ & We use the word ``map'' referring to a function,
defined on the set of all \healpix pixels.
\\
$\mathbf{\theta}$ & The polar angle or colatitude on the sphere,
ranging from 0 at the North Pole to $\pi$ at the South Pole.\\
\\
$\mathbf{\phi}$ & The azimuthal angle on the sphere, $\phi\in[0,2\pi[$.\\
\\
\end{tabular}

%% {\bf{missing documentation
%% (, pow2alm\_units)}}

%--------------------------------------------
%-------------------------------------------- CHANGES
%--------------------------------------------
\section[Changes between releases 3.31 and \hpxversion]{Changes between releases 3.31 and
\hpxversion}
%from install.tex
\begin{itemize}
\item The subroutine 
\htmlref{\texttt{input\_map}}{sub:input_map} in its default mode
test the value of the \texttt{POLCCONV} FITS keyword when reading a polarized map,
and interpret the polarization accordingly, 
as described in the \htmlref{note on POLCCONV}{intro:polcconv} in \linklatexhtml{The \healpix Primer}{intro.pdf}{intro.htm}.
%
\item \htmlref{\texttt{median}}{sub:median} subroutine: faster by moving an internal array from heap to stack; 
does not crash anymore when dealing with empty data sets.
\end{itemize}

%--------------------------------------------
\section{Older Changes}
%--------------------------------------------

{\mysmaller%
\compresstext
%--------------------------------------------
\subsection[Changes between releases 3.00 and 3.31]{Changes between releases 3.00 and
3.31}
%--------------------------------------------

\subsubsection{Version 3.31}
\begin{itemize}\compresslist
	\item Bug correction in \htmlref{{\tt input\_map}}{sub:input_map} routine for reading of polarized multi-HDU cut sky FITS files;
	\item Introduction of 
\mylinkext{fac:alteralm:winfiledir_in}{{\tt winfiledir\_*}} and 
\mylinkext{fac:alteralm:windowfile_in}{{\tt windowfile\_*}} qualifiers in \htmlref{{\tt alteralm}}{fac:alteralm} facility.
\end{itemize}


\subsubsection{Version 3.30}
\begin{itemize}\compresslist
%
	\item new routines \htmlref{{\tt nest2uniq}}{sub:nest2uniq} 
	and \htmlref{{\tt uniq2nest}}{sub:uniq2nest} for conversion 
	of standard pixel index to/from Unique ID number. See \htmlref{''The Unique Identifier scheme''}{intro:unique} section in \linklatexhtml{''\healpix Introduction Document''}{intro.pdf}{intro.htm} for more details.
%
	\item \htmlref{{\tt alm2cl}}{sub:alm2cl} can now produces nine spectra 
      (TT, EE, BB, TE, TB, EB, ET, BT and BE), instead of six previously, when 
      called with two sets of polarized $a_{\ell m}$ and can also symmetrize 
      the output $C(\ell)$ if requested
%
	\item the $a_{\ell m}$ generated by 
	\htmlref{{\tt create\_alm}}{sub:create_alm} can now take into account
 non-zero (exotic) TB and EB cross-spectra (option \mylink{sub:create_alm:polar}{{\tt polar=2}}) if the input FITS file contains the relevant information
%
	\item addition of \mylink{sub:write_minimal_header:asym_cl}{\tt asym\_cl} optional keyword in 
	\htmlref{\tt write\_minimal\_header}{sub:write_minimal_header} routine
%
	\item addition of \mylink{sub:write_asctab:extno}{\tt extno} optional keyword in 
	\htmlref{\tt write\_asctab}{sub:write_asctab} routine to write in arbitrary HDU
%
	\item improved 
\mylink{sub:write_bintabh:repeat}{{\tt repeat}}
behavior in \htmlref{{\tt write\_bintabh}}{sub:write_bintabh} routine
%
	\item edited \htmlref{{\tt map2alm\_iterative}}{sub:map2alm_iterative} 
routine to avoid a bug specific to Intel's Ifort 15.0.2
%
	\item CFITSIO version 3.20 (August 2009) or more now required
\end{itemize}


%\subsection[Changes between releases 3.00 and 3.20]{Changes between releases 3.00 and 3.20}
\subsubsection{Version 3.20} % Dec 2014
\label{sub:new3p20}
\begin{itemize}\compresslist
	\item \healpixns-F90 routines and facilities can now also be compiled with
the free Fortran95 compiler \textbf{g95} (\htmladdnormallink{www.g95.org}{http://www.g95.org/})
	\item a separate {\tt build} directory is used to store the objects,
modules, ... produced during the compilation of the source codes
	\item bug correction in \htmlref{{\tt query\_disc}}{sub:query_disc} for
some very small discs in standard mode
	\item improved handling of long FITS keywords, now producing FITS files
fully compatible with the
\htmladdnormallink{\tt PyFITS}{http://www.stsci.edu/institute/software_hardware/pyfits} 
and 
{\tt Astropy} (\htmladdnormallink{www.astropy.org}{http://www.astropy.org/})
{\tt Python} libraries
	\item improved FITS file parsing in 
\htmlref{{\tt generate\_beam}}{sub:generate_beam},
affecting the external $B(\ell)$ reading in the F90 facilities 
\htmlref{{\tt alteralm}}{fac:alteralm}, 
\htmlref{{\tt synfast}}{fac:synfast}, 
\htmlref{{\tt sky\_ng\_sim}}{fac:sky_ng_sim}, 
\htmlref{{\tt smoothing}}{fac:smoothing}.
\end{itemize}

\subsubsection{Version 3.11} % Apr 2013
\label{sub:new3p11}
\begin{itemize}\compresslist
	\item {\tt libsharp} C routines used for Spherical Harmonics Transforms 
	and introduced in \healpix 3.10
	can now be compiled with any {\tt gcc} version.
	\item bug correction in \htmlref{{\tt query\_disc}}{sub:query_disc}
	routine in {\tt inclusive} mode
	\item bug correction in \htmlref{{\tt alm2map\_spin}}{sub:alm2map_spin} 
	routine, which had its {\tt spin} value set to 2
\end{itemize}

\subsubsection{Version 3.10} % Mar 2013
\label{sub:new3p10}
\begin{itemize}\compresslist
	\item Support for {\tt cfitsio} ''Extended File Name Syntax'', and usage of {\tt
libsharp} Spherical Harmonics Transform library. See \linklatexhtml{''Fortran
Facilities''}{facilities.pdf}{facilities.htm} for details.
%
	\item Faster Spherical Harmonics Transform routines 
thanks to \htmladdnormallink{{\tt
libsharp}}{http://sourceforge.net/projects/libsharp} C routines\footnote{
To {\em revert} to the original F90 implementation of these routines, the preprocessing
variable {\tt DONT\_USE\_SHARP} must be set during compilation.}.
\end{itemize}

% \subsection[Changes between releases 2.20 and 3.00]{Changes between releases 2.20
% and 3.00}
\subsection[Changes up to release 3.00]{Changes up to release 3.00}
\subsubsection{Version 3.00}
\label{sub:new3p00}
\begin{itemize}\compresslist
	\item all {\em input} FITS files can now be compressed (with a 
{\tt .gz}, {\tt .Z}, {\tt .z}, or {\tt .zip} 
extension) and/or remotely located (with a {\tt ftp://} or {\tt http://}
prefix). Besides, the \htmlref{fits2cl}{sub:fits2cl} routine, used to read
external beam window functions from FITS files, supports (part of) the CFTISIO 
\htmladdnormallink{Extended File Name Syntax}{http://heasarc.gsfc.nasa.gov/docs/software/fitsio/c/c_user/node81.html} in
order to read an arbitrary extension identified by its number or its name. \\
{\em Version 3.14 (March 2009) or newer of CFITSIO is required for \healpix 3.00.}
	\item%
new code {\tt process\_mask} and new module {\tt mask\_tools} containing the routines
\htmlref{dist2holes\_nest}{sub:dist2holes_nest}, 
\htmlref{fill\_holes\_nest}{sub:fill_holes_nest}, 
\htmlref{maskborder\_nest}{sub:maskborder_nest},
\htmlref{size\_holes\_nest}{sub:size_holes_nest} useful for mask apodization,
	\item improved accuracy of the co-latitude calculation in the vicinity
of the poles at high resolution in \htmlref{{\tt nest2ring, ring2nest,
pix2ang\_*, pix2vec\_*, $\ldots$}}{sub:pix_tools},
	\item the pixel query routine
 \htmlref{{\tt query\_disc}}{sub:query_disc} 
has been improved and will return fewer
false positive pixels in the 
inclusive
%\mylink{sub:query_disc:inclusive}{{\em inclusive}} 
mode.
\end{itemize}

%\subsection[Changes between releases 2.14 and 2.20]{Changes between releases 2.14 and 2.20}
\subsubsection{Version 2.20}
\label{sub:new2p20}
\begin{itemize}\compresslist
\item Spherical Harmonics Transform routines now transparently call \htmladdnormallink{\tt
libpsht}{http://libpsht.sourceforge.net} C routines, leading to a significant (2 to 4) speed-up factor. This
concerns temperature and polarized transforms (\htmlref{alm2map}{sub:alm2map}, 
\htmlref{map2alm}{sub:map2alm}) {\em without precomputation} of the $P_{\ell m}$ as
well as spin 
weighted (\htmlref{alm2map\_spin}{sub:alm2map_spin}, 
\htmlref{map2alm\_spin}{sub:map2alm_spin}) transforms for $0 < |s| \le 100$,
but {\em not} the generation of spatial derivatives
(\htmlref{alm2map\_der}{sub:alm2map_der}) which still uses the original F90 code.
The compilation and linking to {\tt libpsht}, now shipped with \healpix, is done
automatically, without any extra download or installation for the user\footnote{
To {\em revert} to the original F90 implementation of all these routines, the preprocessing
variable {\tt DONT\_USE\_PSHT} must be set during compilation.}.

\item All routines for Spherical Harmonics Transforms and most routines for
pixel manipulations (
\htmlref{ang2xxx, pix2xxx, vec2xxx, $\ldots$}{sub:pix_tools},
\htmlref{nside2npix}{sub:nside2npix}, 
\htmlref{npix2nside}{sub:npix2nside}, 
\htmlref{nside2ntemplates}{sub:nside2ntemplates},  
$\ldots$)
pixel queries (
\htmlref{query\_*}{sub:query_disc},  $\ldots$) 
and FITS I/O (%
\htmlref{input\_map}{sub:input_map}, 
\htmlref{output\_map}{sub:output_map}, 
\htmlref{read\_bintab}{sub:read_bintab}, 
\htmlref{write\_bintab}{sub:write_bintab},
$\ldots$)
of sky maps
now support resolution parameters $\nside > 8192$. 
This means that the number of
pixels and the pixel indexes can now be stored in either
{\tt integer(\htmlref{I4B}{sub:healpix_types})} or
{\tt integer(\htmlref{I8B}{sub:healpix_types})} variables (on systems
supporting 64 bit variables).\\
The reading and writing of
$a_{\ell m}$ containing files remains limited to $\ell < 46340 $, though. This
restriction does not apply to $C(\ell)$ containing files.

\item As a positive side effect of their upgrade, the F90
\htmlref{pixel/coordinate conversion
routines}{sub:pix_tools} are now up to 20\% faster.

\item Introduction of 
\htmlref{\tt long\_count}{sub:long_intrinsic} and
\htmlref{\tt long\_size}{sub:long_intrinsic} functions.

\end{itemize}

%\subsection[Changes between releases 2.13 and 2.14]{Changes between releases 2.13 and 2.14}
\subsubsection{Version 2.14}
\begin{itemize}\compresslist
\item In \htmlref{alm2map\_der}{sub:alm2map_der} routine, a numerical bug affecting the accuracy of the Stokes parameter derivatives 
$\partial X/\partial\theta$, 
$\partial^2 X/(\partial\theta\partial\phi\sin\theta)$, 
$\partial^2 X/\partial \theta^2$, 
for $X=Q,U$ has been corrected. See \linklatexhtml{''Fortran
Facilities''}{facilities.pdf}{facilities.htm} Appendix for details.
\end{itemize}

%\subsection[Changes between releases 2.0 and 2.13]{Changes between releases 2.0 and 2.13}
\subsubsection{Versions 2.10 and 2.13}
\begin{itemize}\compresslist
\item New functions in version 2.13: 
\begin{itemize}\compresslist
	\item\htmlref{%
get\_healpix\_data\_dir, %
get\_healpix\_main\_dir, %
get\_healpix\_test\_dir%
}{sub:get_healpix_xxx_dir} return full path to \healpix directories.
\end{itemize}
\item New routines in version 2.10: 
\begin{itemize}
	\item\htmlref{alm2map\_spin}{sub:alm2map_spin}: synthesis of maps of
arbitrary spin
	\item\htmlref{map2alm\_iterative}{sub:map2alm_iterative}: iterative analysis of map
	\item\htmlref{map2alm\_spin}{sub:map2alm_spin}: analysis of maps of
arbitrary spin
	\item\htmlref{healpix\_modules}{sub:healpix_modules}: meta-module
	\item\htmlref{write\_minimal\_header}{sub:write_minimal_header}: routine
to write minimal FITS header
	\item\htmlref{parse\_check\_unused}{sub:parse_xxx}: prints out
parameters present in parameter file but not used by the code.
\end{itemize}
\item Improved routines:
\begin{itemize}\compresslist
\item \htmlref{query\_strip}{sub:query_strip}: the {\tt inclusive} option now
returns {\em all} (and only) the pixels overlapping, even partially, with the
strip
\item \htmlref{query\_disc}{sub:query_disc}: when the disc center is on one of
the poles, {\em only} the pixels overlapping with the disc are now returned.
\item \htmlref{remove\_dipole}{sub:remove_dipole}: can now deal with non-uniform
pixel weights.
\item \htmlref{parse\_init}{sub:parse_xxx}: silent mode
\item \htmlref{parse\_string}{sub:parse_xxx}: can expand environment variables
(\$\{XXX\}) and leading \verb+~+$\!${\tt /}
\end{itemize}

\end{itemize}

%--------------------------------------------
%\subsection[Changes between releases 1.2 and 2.0]{Changes between releases 1.2 and 2.0}
\subsubsection{Version 2.0}
Some new features have been added
\begin{itemize}\compresslist
\item Most routines dealing with maps and $a_{\ell m}$ (eg, create\_alm, map2alm, alm2map,
  convert\_inplace, convert\_nest2ring, udgrade\_nest, udgrade\_ring) or inputting or outputting  data (read\_*, write\_*)
 now accept both single and double precision arguments.
\item The routines \htmlref{{\tt map2alm}}{sub:map2alm} and  \htmlref{{\tt remove\_dipole}}{sub:remove_dipole} can now deal with
  {\em non-symmetric} azimuthal cut sky. For backward compatibility, the former calling sequence
  is still accepted.
\item most routines are now parallelized with OpenMP (for shared memory architecture), and some of them are
also parallelized with MPI (for distributed memory architecture)
\end{itemize}


Some new routines have been introduced since version 1.2, as listed below.
\begin{itemize}\compresslist
\item New routines in version 2.0
\begin{itemize}
\item \htmlref{add\_dipole}{sub:add_dipole},
\htmlref{alm2cl}{sub:alm2cl},
\htmlref{alm2map\_der}{sub:alm2map_der},
\htmlref{fits2cl}{sub:fits2cl} (replaces read\_asctab),
\htmlref{nside2ntemplates}{sub:nside2ntemplates},
\htmlref{plm\_gen}{sub:plm_gen},
\htmlref{rand\_gauss}{sub:rand_gauss}, \htmlref{rand\_init}{sub:rand_init}, \htmlref{rand\_uni}{sub:rand_uni},
\htmlref{same\_shape\_pixels\_nest}{sub:same_shape_pixels_xxx}, \htmlref{same\_shape\_pixels\_ring}{sub:same_shape_pixels_xxx},
\htmlref{template\_pixel\_nest}{sub:template_pixel_xxx}, \htmlref{template\_pixel\_ring}{sub:template_pixel_xxx},
\htmlref{write\_plm}{sub:write_plm} (replaces write\_dbintab).
\end{itemize}
\item New modules or modules with new name
\begin{itemize}
\item \textbf{misc\_utils:} 
\htmlref{assert, 
assert\_alloc,
assert\_directory\_present,
assert\_not\_present,
assert\_present, 
fatal\_error}{sub:assert},  
    file\_present, 
\htmlref{string}{sub:string},
\htmlref{strupcase}{sub:string},
\htmlref{strlowcase}{sub:string}, 
     upcase, lowcase, wall\_clock\_time, 
\htmlref{brag\_openmp}{sub:brag_openmp}
\item \textbf {rngmod:} \htmlref{rand\_gauss}{sub:rand_gauss}, \htmlref{rand\_init}{sub:rand_init}, \htmlref{rand\_uni}{sub:rand_uni}
\end{itemize}

\item The following routines are superseded.
\begin{itemize}\compresslist
\item read\_asctab (replaced by \htmlref{fits2cl}{sub:fits2cl})
\item write\_dbintab (replaced by \htmlref{write\_plm}{sub:write_plm})
\end{itemize}

\end{itemize}

%--------------------------------------------
%\subsection{Changes between releases 1.1 and 1.2}
\subsubsection{Version 1.2}
Some new routines have been introduced since version 1.1, as listed below.
% Most of the routines that already existed now have extended
% capabilities.
% Those of them with improved or extended user interface are listed
% below. They all remain backward compatible (ie, they can be used with codes written
% around version 1.1 without any edition).

\begin{itemize}\compresslist
\item New routines in version 1.2
\begin{itemize}\compresslist
\item \htmlref{angdist}{sub:angdist}, 
\htmlref{complex\_fft}{sub:complex_fft}, 
\htmlref{concatnl}{sub:concatnl}, 
\htmlref{del\_card}{sub:del_card}, 
\htmlref{get\_card}{sub:get_card}, 
\htmlref{getargument}{sub:getargument}, 
\htmlref{getenvironment}{sub:getenvironment}, 
\htmlref{input\_tod*}{sub:input_tod}, 
\htmlref{nArguments}{sub:narguments}, 
parse\_double, parse\_init, parse\_int, parse\_lgt, parse\_long, parse\_real, parse\_string (see \htmlref{parse\_xxx}{sub:parse_xxx}), 
\htmlref{query\_disc}{sub:query_disc} (replaces \htmlref{getdisc\_ring}{sub:getdisc_ring}), 
\htmlref{query\_polygon}{sub:query_polygon}, 
\htmlref{query\_strip}{sub:query_strip}, 
\htmlref{query\_triangle}{sub:query_triangle}, 
\htmlref{read\_fits\_cut4}{sub:read_fits_cut4}, 
\htmlref{real\_fft}{sub:real_fft}, 
\htmlref{scan\_directories}{sub:scan_directories}, 
\htmlref{surface\_triangle}{sub:surface_triangle}, 
\htmlref{vect\_prod}{sub:vect_prod}, 
\htmlref{write\_bintabh}{sub:write_bintabh}, 
\htmlref{write\_fits\_cut4}{sub:write_fits_cut4}, 
\end{itemize}

\item New modules or modules with new name
\begin{itemize}\compresslist
\item the modules {\tt extension} (C extensions), {\tt healpix\_fft} (FFT
operations), {\tt paramfile\_io} (parameter parsing) have
been introduced,
\item the module {\tt wrap\_fits} has been renamed {\tt head\_fits} to
reflect its extended capabilities in manipulating FITS headers.
\end{itemize}

\item The following routines are superseded. They have been moved to the
{\tt obsolete} module.
\begin{itemize}\compresslist
\item ask\_inputmap, ask\_outputmap, ask\_lrange (initially in {\tt fitstools} module)
\item setpar, getpar, anafast\_parser, anafast\_setpar, anafast\_getpar,
hotspots\_parser, hotspots\_setpar, hotspots\_getpar, udgrade\_parser,
udgrade\_setpar, udgrade\_getpar, smoothing\_parser, smoothing\_setpar,
smoothing\_getpar (initially in {\tt utilities} module).
\end{itemize}

\end{itemize}
} % end of mysmaller



\newpage

\sloppy

\title{\healpix Fortran Subroutines Overview}
\docid{add\_card} \section[add\_card]{ }
\label{sub:add_card}
\docrv{Version 1.2}
\author{Frode K.~Hansen, Eric Hivon}
\abstract{This document describes the \healpix Fortran90 subroutine ADD\_CARD.}

\begin{facility}
{This routine writes a keyword of any kind into a FITS header. It is a wrapper to other routines that write keywords of different kinds.}
{\modHeadFits}
\end{facility}

\begin{f90format}
{\mylink{sub:add_card:header}{header}%
, \mylink{sub:add_card:kwd}{kwd}%
, \mylink{sub:add_card:value}{value}%
 \optional{[, \mylink{sub:add_card:comment}{comment}%
, \mylink{sub:add_card:update}{update}%
]} }
\end{f90format}
\aboutoptional

\begin{arguments}
{
\begin{tabular}{p{0.4\hsize} p{0.05\hsize} p{0.1\hsize} p{0.35\hsize}} \hline  
\textbf{name~\&~dimensionality} & \textbf{kind} & \textbf{in/out} & \textbf{description} \\ \hline
                   &   &   &                           \\ %%% for presentation
header\mytarget{sub:add_card:header}(LEN=80) DIMENSION(:) & CHR & INOUT & The header to write the keyword to. \\
kwd\mytarget{sub:add_card:kwd}(LEN=*) & CHR & IN & the FITS keyword to write. Should be shorter
                   or equal to 8 characters.\\
value\mytarget{sub:add_card:value} & any & IN & the value (double, real, integer, logical or
                   character string) to give to the keyword. Note that long string values
(more than 68 characters in length) are supported.\\
\optional{comment\mytarget{sub:add_card:comment}(LEN=*)} & CHR & IN & comment to the keyword. \\ 
\end{tabular}
\begin{tabular}{p{0.4\hsize} p{0.05\hsize} p{0.1\hsize} p{0.35\hsize}} \hline  
\latexhtml{\textbf{name~\&~dimensionality} & \textbf{kind} & \textbf{in/out} & \textbf{description} \\ \hline
                   &   &   &                           \\ %%% for presentation
}{}
\optional{update\mytarget{sub:add_card:update}} & LGT & IN & if set to {\tt .true.}, the first occurence of the keyword \mylink{sub:add_card:kwd}{\tt
kwd} in {\tt header} will be updated (and all other occurences removed); otherwise, the keyword will be appended at
the end (and any previous occurence removed). If the keyword is either 'HISTORY'
or 'COMMENT', {\tt update} is ignored and the keyword is peacefully appended at the end of the header.\\ 
\end{tabular}
}
\end{arguments}

\begin{example}
{
character(len=80), dimension(1:120) :: header \\
header = '' ! very important \\
call add\_card(header,'NSIDE',256,'the nside of the map')  \\
}
{
Gives the keyword `NSIDE' the value 256 in the given header-string. It is
important to make sure that the {\tt header} string array is empty before attempting
to write
anything in it.
}
\end{example}

\begin{modules}
  \begin{sulist}{} %%%% NOTE the ``extra'' brace here %%%%
  \item[write\_hl] more general routine for adding a keyword to a header.
  \item[\textbf{cfitsio}] library for FITS file handling.		
  \end{sulist}
\end{modules}

\begin{related}
  \begin{sulist}{} %%%% NOTE the ``extra'' brace here %%%%
  \item[\htmlref{write\_minimal\_header}{sub:write_minimal_header}] routine to
write \healpix compliant baseline FITS header
  \item[\htmlref{get\_card}{sub:get_card}] general purpose routine to read any keywords from a header in a FITS file.
  \item[\htmlref{del\_card}{sub:del_card}] routine to discard a keyword from a FITS header
  \item[\htmlref{read\_par}{sub:read_par}, \htmlref{number\_of\_alms}{sub:number_of_alms}] routines to read specific keywords from a
  header in a FITS file.
  \item[\htmlref{getsize\_fits}{sub:getsize_fits}] function returning the size of the data set in a fits
  file and reading some other useful FITS keywords
  \item[\htmlref{merge\_headers}{sub:merge_headers}] routine to merge two FITS headers
  \end{sulist}
\end{related}

\rule{\hsize}{2mm}

\newpage


\sloppy


\title{\healpix Fortran Subroutines Overview}
\docid{add\_dipole*} \section[add\_dipole*]{ }
\label{sub:add_dipole}
\docrv{Version 2.0}
\author{Eric Hivon}
\abstract{This document describes the \healpix Fortran90 subroutine ADD\_DIPOLE.}

\begin{facility}
{This routine provides a means to add a monopole and dipole to a \healpix map.}
{\modPixTools}
\end{facility}

\begin{f90format}
{nside, map, ordering, degree, multipoles \optional{[, fmissval]}}
\end{f90format}
\aboutoptional

\begin{arguments}
{
\begin{tabular}{p{0.32\hsize} p{0.05\hsize} p{0.08\hsize} p{0.45\hsize}} \hline  
\textbf{name~\&~dimensionality} & \textbf{kind} & \textbf{in/out} & \textbf{description} \\ \hline
                   &   &   &                           \\ %%% for presentation
nside & I4B & IN & value of $\nside$ resolution parameter for input map\\
map(0:12*nside*nside-1) & SP/ DP & INOUT & \healpix map to which the monopole and dipole will be
                   added. Those are added to {\em all unflagged pixels}. \\
ordering & I4B & IN & \healpix\ scheme 1:RING, 2: NESTED \\
degree & I4B & IN & multipoles to add. It is either 0 (nothing done),
                   1 (monopole only) or 2 (monopole and dipole) \\
multipoles(0:degree*degree-1) & DP & IN & values of monopole and
                   dipole to add.  The monopole is described as a scalar in the same
                   units as the input map, the dipole as a 3D cartesian vector, 
		   in the same units. \\
\optional{fmissval}  & SP/ DP & IN & value used to flag bad pixel on input
                   \default{-1.6375e30}. Pixels with that value are left unchanged.\\
\end{tabular}
}
\end{arguments}
\newpage
\begin{example}
{
call \thedocid (128, map, 1, 2, ($\backslash$ 10.0\_dp, 0.0\_dp, 1.2\_dp,
0.0\_dp $\backslash$) )  \\
}
{
{\tt map} is a \healpix\ map of resolution $\nside=128$, with the RING ordering scheme. A
monopole of amplitude 10 and a dipole of amplitude 1.2 and directed along the
$y$ axis will be added to it.
}
\end{example}

\begin{modules}
  \begin{sulist}{} %%%% NOTE the ``extra'' brace here %%%%
  \item[\textbf{pix\_tools}] module, containing:
%  \item[\textbf{pix\_tools}] module, containing:
  \end{sulist}
\end{modules}

\begin{related}
  \begin{sulist}{} %%%% NOTE the ``extra'' brace here %%%%
  \item[\htmlref{remove\_dipole}{sub:remove_dipole}] routine to remove the best fit monopole and
  monopole from a map.
  \end{sulist}
\end{related}

\rule{\hsize}{2mm}

\newpage


\sloppy


\docid{alm2cl*}\section[alm2cl*]{ }
\label{sub:alm2cl}
\docrv{Version 2.0}
\author{Eric Hivon}
\abstract{This document describes the \healpix Fortran90 subroutine ALM2CL*.}

\begin{facility}
{This routine computes the auto (or cross) power spectra of a one (or two) sets of spherical harmonics
  coefficients $a_{\ell m}$.
$C_{12}(\ell) = \sum_{m=-\ell}^{\ell} a_{1,\ell m}
  a_{2,\ell m}^* / (2 \ell +1) \myhtmlimage{}$ }
{\modAlmTools}
\end{facility}

\begin{f90format}
{\mylink{sub:alm2cl:nlmax}{nlmax}%
, \mylink{sub:alm2cl:nmmax}{nmmax}%
, \mylink{sub:alm2cl:alm1}{alm1}%
, \optional{[\mylink{sub:alm2cl:alm2}{alm2}%
,]} \mylink{sub:alm2cl:cl}{cl}%
}
\end{f90format}
\aboutoptional

\begin{arguments}
{
\begin{tabular}{p{0.35\hsize} p{0.05\hsize} p{0.1\hsize} p{0.40\hsize}} \hline  
\textbf{name~\&~dimensionality} & \textbf{kind} & \textbf{in/out} & \textbf{description} \\ \hline
                   &   &   &                           \\ %%% for presentation
nlmax\mytarget{sub:alm2cl:nlmax} & I4B & IN & the maximum $\ell$ value used for the $a_{\ell m}$. \\
nmmax\mytarget{sub:alm2cl:nmmax} & I4B & IN & the maximum $m$ value used for the $a_{\ell m}$. \\
alm1\mytarget{sub:alm2cl:alm1}(1:p, 0:nlmax, 0:nmmax) & SPC/ DPC & IN & First set of $a_{\ell m}$ values. $p$
                   is 3 or 1 depending on wether polarisation is included or
                   not. In the former case, the first index runs from 1 to 3 corresponding to (T,E,B). \\
\optional{alm2\mytarget{sub:alm2cl:alm2}}(1:p, 0:nlmax, 0:nmmax) & SPC/ DPC & IN & Second set of $a_{\ell m}$
                   values.  \\
%% \end{tabular}
%% \begin{tabular}{p{0.4\hsize} p{0.05\hsize} p{0.1\hsize} p{0.35\hsize}} \hline  
cl\mytarget{sub:alm2cl:cl}(0:nlmax,1:d) & SP/ DP & OUT & resulting auto or cross power spectra. 
                   If both {\tt alm1} and {\tt alm2} are present, {\tt cl} will
                   be their cross power spectrum. If only {\tt alm1} is present,
                   {\tt cl} will be its power spectrum. 
		   If $d=1$, only the temperature spectrum
                   $C_{\ell}^T$ will
                   be output. If $d=4$ and $p=3$, the output will be $C_{\ell}^T$, $C_{\ell}^E$,
                   $C_{\ell}^B$, $C_{\ell}^{T\times E}$, if $d\geq 6$ and $p=3$, $C_{\ell}^{T\times
                   B}$  $C_{\ell}^{E\times B}$ will also be output, and if $d\geq 9$ and $p=3$, $C_{\ell}^{E\times T}$, $C_{\ell}^{B\times
                   T}$  and $C_{\ell}^{B\times E}$ will be included.
\end{tabular}
}
\end{arguments}

\begin{example}
{
lmax = 128 ; mmax = lmax \\
call alm2cl(lmax, mmax, alm1, cl\_auto)  \\
call alm2cl(lmax, mmax, alm1, alm2, cl\_cross)  \\
}
{
{\tt cl\_auto} will contain the (auto) power spectrum of the $a_{\ell m}$ coefficients {\tt alm1} up to $\ell = 128$,
while {\tt cl\_cross} will be the cross power spectra of the two sets of $a_{\ell m}$ coefficients {\tt
  alm1} and {\tt alm2}.
}
\end{example}

\begin{modules}
  \begin{sulist}{} %%%% NOTE the ``extra'' brace here %%%%
  \item[none]
  \end{sulist}
\end{modules}

\begin{related}
  \begin{sulist}{} %%%% NOTE the ``extra'' brace here %%%%
  \item[\htmlref{map2alm}{sub:map2alm}] routine extracting the $a_{\ell m}$
  coefficients from a \healpix map
  \item[\htmlref{create\_alm}{sub:create_alm}] routine to generate randomly
  distributed $a_{\ell m}$ coefficients according to a given power spectrum
  \end{sulist}
\end{related}

\rule{\hsize}{2mm}

\newpage


\sloppy


\docid{alm2map*}\section[alm2map*]{ }
\label{sub:alm2map}
\docrv{Version 2.1}
\author{Eric Hivon, Frode K.~Hansen}
\abstract{This document describes the \healpix Fortran90 subroutine ALM2MAP*.}

\begin{facility}
{This routine is a wrapper to 10 other routines: alm2map\_sc\_X,
  alm2map\_sc\_pre\_X, alm2map\_pol\_X, alm2map\_pol\_pre1\_X,
  alm2map\_pol\_pre2\_X, where X stands for either s or d. These routines
  synthesize a \healpix {\em RING ordered} temperature map (and if specified, polarisation maps) 
from input $a_{\ell m}^T$ (and if specified $a_{\ell m}^E$ and $a_{\ell m}^B$) values. 
The different routines are called dependent on what parameters are passed. 
Some routines synthesize maps with or without precomputed harmonics (note that
since \healpix v2.20 precomputed harmonics most likely won't speed up computation)
and some with or without polarisation.
The routines accept both single and double precision arrays for alm\_TGC and
  map\_TQU. The precision of these arrays should match.}
{\modAlmTools}
\end{facility}

\begin{f90format}
{\mylink{sub:alm2map:nsmax}{nsmax}%
, \mylink{sub:alm2map:nlmax}{nlmax}%
, \mylink{sub:alm2map:nmmax}{nmmax}%
, \mylink{sub:alm2map:alm_TGC}{alm\_TGC}%
, \mylink{sub:alm2map:map_TQU}{map\_TQU}%
 [, \mylink{sub:alm2map:plm}{plm}%
$|$ \mylink{sub:alm2map:zbounds}{zbounds}]}
\end{f90format}

\begin{arguments}
{
\begin{tabular}{p{0.4\hsize} p{0.05\hsize} p{0.1\hsize} p{0.35\hsize}} \hline  
\textbf{name~\&~dimensionality} & \textbf{kind} & \textbf{in/out} & \textbf{description} \\ \hline
                   &   &   &                           \\ %%% for presentation
nsmax\mytarget{sub:alm2map:nsmax} & I4B & IN & the $\nside$ value of the map to synthesize. \\
nlmax\mytarget{sub:alm2map:nlmax} & I4B & IN & the maximum $\ell$ value used for the $a_{\ell m}$. \\
nmmax\mytarget{sub:alm2map:nmmax} & I4B & IN & the maximum $m$ value used for the $a_{\ell m}$. \\
alm\_TGC\mytarget{sub:alm2map:alm_TGC}(1:p, 0:nlmax, 0:nmmax) & SPC or DPC & IN & The $a_{\ell m}$ values to make
                   the map from. $p$ is 3 or 1 depending on wether polarisation is
                   respectively included or not. In the former case, the first
                   index runs from 1 to 3 corresponding to (T,E,B). \\
\end{tabular}

\begin{tabular}{p{0.4\hsize} p{0.05\hsize} p{0.1\hsize} p{0.35\hsize}}  \hline  
map\_TQU\mytarget{sub:alm2map:map_TQU}(0:12*nsmax**2-1)      & SP or DP & OUT & if only a temperature map is
to be synthesized, the map-array should be passed with this rank.  
\\ 
map\_TQU(0:12*nsmax**2-1, 1:3) & SP or DP & OUT & if both temperature an 
polarisation maps are to be synthesized, the map array should have this rank, 
where the second index is (1,2,3) corresponding to (T,Q,U). 
\\ 
plm\mytarget{sub:alm2map:plm}(0:n\_plm-1), \hskip 4cm OPTIONAL & DP & IN & If this optional matrix is passed with
this rank, precomputed $P_{\ell m}(\theta)$ are used instead of recursion. (
n\_plm = nsmax*(nmmax+1)*(2*nlmax-nmmax+2).
\\ 
plm(0:n\_plm-1,1:3), \hskip 4cm OPTIONAL & DP & IN & If this optional matrix is 
passed with this rank, precomputed $P_{\ell m}(\theta)$ AND precomputed tensor
harmonics are used instead of recursion. (n\_plm =
nsmax*(nmmax+1)*(2*nlmax-nmmax+2). 
\\
zbounds\mytarget{sub:alm2map:zbounds}(1:2), \hskip 4cm OPTIONAL & DP & IN & section of the sphere on which to perform the map synthesis, expressed in terms of $z=\sin(\mathrm{latitude}) =
                   \cos(\theta).$ %zbounds_sub.tex:  for alm2map, map2alm, remove_dipole: describe processed area
%zbounds2_sub.tex: for apply_mask: describe pixels set to 0 (=unprocessed area)
If zbounds(1)$<$zbounds(2), it is
performed {\em on} the strip zbounds(1)$<z<$zbounds(2); if not,
it is performed {\em outside} the strip
%  zbounds(2)$<z<$zbounds(1). % OLD
zbounds(2)$\le z \le$zbounds(1). % NEW ??
% The whole sphere is treated if \texttt{zbounds=(/-1.0\_dp, 1.0\_dp/)} or if 
% it is absent.
If absent, the whole map is processed.

Currently, \mylink{sub:alm2map:zbounds}{zbounds} and \mylink{sub:alm2map:plm}{plm} can not be used together.
\\
\end{tabular}
}
\end{arguments}

\begin{example}
{
use healpix\_types \\
use pix\_tools, only : nside2npix \\
use alm\_tools, only : alm2map \\
integer(I4B) :: nside, lmax, mmax, npix, n\_plm\\
real(SP), dimension(:,:), allocatable :: map \\
complex(SPC), dimension(:,:,:), allocatable :: alm \\
real(DP), dimension(:,:), allocatable :: plm \\
\ldots \\
nside=256 ; lmax=512 ; mmax=lmax\\
npix=nside2npix(nside)\\
n\_plm=nside*(mmax+1)*(2*lmax-mmax+2)\\
allocate(alm(1:3,0:lmax,0:mmax))\\
allocate(map(0:npix-1,1:3))\\
allocate(plm(0:n\_plm-1,1:3))\\
\ldots \\
call alm2map(nside, lmax, mmax, alm, map, plm)  \\
}
{
Make temperature and polarisation maps from the scalar and tensor $a_{\ell m}$
passed in alm. The maps have $\nside$ of 256, and are constructed from
$a_{\ell m}$ values up to 512 in $\ell$ and $m$. Since the optional plm array is
passed with both precomputed $P_{\ell m}(\theta)$ AND tensor harmonics, there will
be no recursions in the routine. However, this will most likely have a
\emph{negative} impact on execution speed.
}
\end{example}

\begin{modules}
  \begin{sulist}{} %%%% NOTE the ``extra'' brace here %%%%
  \item[\htmlref{ring\_synthesis}{sub:ring_synthesis}] Performs FFT over $m$ for synthesis of the rings.
  \item[compute\_lam\_mm, get\_pixel\_layout, ]
  \item[gen\_lamfac,gen\_mfac, gen\_normpol, ] 
  \item[gen\_recfac, init\_rescale, l\_min\_ylm] Ancillary routines used
  for $Y_{\ell m}$ recursion
  \item[\textbf{misc\_utils}] module, containing:
  \item[\htmlref{assert\_alloc}{sub:assert}] routine to print error message, when an array can not be
  allocated properly
  \end{sulist}
%Note: Starting with \htmlref{version 2.20}{sub:new2p20}, {\tt libpsht} routines will be called when precomputed $P_{\ell m}$ are not provided.
Note: Starting with \htmlref{version 3.10}{sub:new3p10}, {\tt libsharp} routines will be called when precomputed $P_{\ell m}$ are not provided.
\end{modules}

\begin{related}
  \begin{sulist}{} %%%% NOTE the ``extra'' brace here %%%%
   \item[\htmlref{alm2map\_der}{sub:alm2map_der}] routine generating a map and
   its derivatives from its $a_{\ell m}$
   \item[\htmlref{alm2map\_spin}{sub:alm2map_spin}] routine generating maps of
arbitrary spin from their  ${_s}a_{\ell m}$
  \item[smoothing] executable using \thedocid\ to smooth maps
  \item[synfast] executable using \thedocid\ to synthesize maps.
  \item[\htmlref{map2alm}{sub:map2alm}] routine performing the inverse transform
  of \thedocid.
  \item[\htmlref{create\_alm}{sub:create_alm}] routine to generate randomly
  distributed $a_{\ell m}$ coefficients according to a given power spectrum
  \item[\htmlref{pixel\_window}{sub:pixel_window},
\htmlref{generate\_beam}{sub:generate_beam}] return the $\ell$-space \healpix-pixel and beam window function respectively
  \item[\htmlref{alter\_alm}{sub:alter_alm}] modifies $a_{\ell m}$ to emulate effect
of real space filtering
  \end{sulist}
\end{related}

\rule{\hsize}{2mm}

\newpage

\sloppy
\docid{alm2map\_der*}\section[alm2map\_der*]{ }
\label{sub:alm2map_der}
\docrv{Version 2.0}
\author{Eric Hivon}
\abstract{This document describes the \healpix Fortran90 subroutine ALM2MAP\_DER*.}

\begin{facility}
{This routine is a wrapper to four other routines that synthesize a \healpix
  temperature (and polarisation) map(s), its (their) first derivatives, and optionally
  its (their) second derivatives.
The routines accept both single and double precision arrays for alm, map, der1 and
der2. The precision of these arrays should match. All maps produced are RING
ordered.

See \linklatexhtml{''Fortran
Facilities''}{facilities.pdf}{facilities.htm} \htmlref{Appendix}{fac:sec:bug_synder} for a note on a bug
affecting the calculation of polarisation derivatives on past versions of this routine.
}
{\modAlmTools}
\end{facility}

\begin{f90format}
{\mylink{sub:alm2map_der:nsmax}{nsmax}%
, \mylink{sub:alm2map_der:nlmax}{nlmax}%
, \mylink{sub:alm2map_der:nmmax}{nmmax}%
, \mylink{sub:alm2map_der:alm}{alm}%
, \mylink{sub:alm2map_der:map}{map}%
, \mylink{sub:alm2map_der:der1}{der1}%
 [, \mylink{sub:alm2map_der:der2}{der2}%
]}
\end{f90format}

\begin{arguments}
{
\begin{tabular}{p{0.35\hsize} p{0.05\hsize} p{0.1\hsize} p{0.40\hsize}} \hline  
\textbf{name~\&~dimensionality} & \textbf{kind} & \textbf{in/out} & \textbf{description} \\ \hline
                   &   &   &                           \\ %%% for presentation
nsmax\mytarget{sub:alm2map_der:nsmax} & I4B & IN & the $N_{side}$ value of the map to synthesize. \\
nlmax\mytarget{sub:alm2map_der:nlmax} & I4B & IN & the maximum $\ell$ value used for the $a_{lm}$. \\
nmmax\mytarget{sub:alm2map_der:nmmax} & I4B & IN & the maximum $m$ value used for the $a_{lm}$. \\
alm\mytarget{sub:alm2map_der:alm}(1:p, 0:nlmax, 0:nmmax) & SPC/ DPC & IN & The $a_{lm}$ values to make the map
                   from. p is either 1 (temperature only) or 3 (temperature+polarisation).\\
%% \end{tabular}
%% \begin{tabular}{p{0.4\hsize} p{0.05\hsize} p{0.1\hsize} p{0.35\hsize}}  \hline  
map\mytarget{sub:alm2map_der:map}(0:12*nsmax**2-1) \hskip 3cm {\em or\ \ }(0:12*nsmax**2-1,1:3)      & SP/ DP & OUT & temperature
map\mytarget{sub:alm2map_der:map} $T(p)$ or temperature + polarisation maps $T(p)$, $Q(p)$, $U(p)$ to be synthesized.  \\ 
der1\mytarget{sub:alm2map_der:der1}(0:12*nsmax**2-1, 1:2*p) & SP/ DP & OUT &  contains on output the first
derivatives of T: $\left({\partial T}/{\partial \theta}, {\partial T}/{\partial \phi}/\sin\theta \right)\myhtmlimage{}
$ or the interleaved derivatives of T, Q, and U: $\left({\partial T}/{\partial
  \theta}, {\partial Q}/{\partial \theta}, {\partial U}/{\partial \theta};\right.\myhtmlimage{}$
$\left.{\partial T}/{\partial \phi}/\sin\theta, \ldots \right)\myhtmlimage{}
$\\ 
der2\mytarget{sub:alm2map_der:der2}(0:12*nsmax**2-1,1:3*p), \hskip 3cm OPTIONAL & SP/ DP & OUT & If this optional
matrix is passed with this rank, it will contain on output the second derivatives
$\left({\partial^2 T}/{\partial \theta^2}, {\partial^2 T}/{\partial
  \theta\partial\phi}/\sin\theta,\right.\myhtmlimage{}$ 
$\left.{\partial^2 T}/{\partial \phi^2}/\sin^2\theta \right)\myhtmlimage{}$ or
$\left({\partial^2 T}/{\partial \theta^2}, {\partial^2 Q}/{\partial \theta^2},
{\partial^2 Q}/{\partial \theta^2}, \ldots \right)\myhtmlimage{}$ 

\end{tabular}
}
\end{arguments}

\begin{example}
{
use healpix\_types \\
use pix\_tools, only : nside2npix \\
use alm\_tools, only : alm2map\_der \\
integer(I4B) :: nside, lmax, mmax, npix, n\_plm\\
real(SP), dimension(:), allocatable :: map \\
real(SP), dimension(:,:), allocatable :: der1, der2 \\
complex(SPC), dimension(:,:,:), allocatable :: alm \\
\ldots \\
nside=256 ; lmax=512 ; mmax=lmax\\
npix=nside2npix(nside)\\
allocate(alm(1:1,0:lmax,0:mmax))\\
allocate(map(0:npix-1))\\
allocate(der1(0:npix-1,1:2), der2(0:npix-1,1:3))\\
\ldots \\
call alm2map\_der(nside, lmax, mmax, alm, map, der1, der2)  \\
}
{
Make temperature maps and its derivatives from the $a_{lm}$ passed in alm. The maps have $N_{side}$ of 256, and are constructed from $a_{lm}$ values up to 512 in $\ell$ and $m$.
}
\end{example}

\begin{modules}
  \begin{sulist}{} %%%% NOTE the ``extra'' brace here %%%%
  \item[\htmlref{ring\_synthesis}{sub:ring_synthesis}] Performs FFT over $m$ for synthesis of the rings.
  \item[compute\_lam\_mm, get\_pixel\_layout, ]
  \item[gen\_lamfac\_der, gen\_mfac,  ] 
  \item[gen\_recfac, init\_rescale, l\_min\_ylm] Ancillary routines used
  for $\ {_s}Y_{\ell m}$ recursion
  \item[\textbf{misc\_utils}] module, containing:
  \item[assert\_alloc] routine to print error message, when an array can not be
  allocated properly
  \end{sulist}
\end{modules}

\begin{related}
  \begin{sulist}{} %%%% NOTE the ``extra'' brace here %%%%
   \item[\htmlref{alm2map}{sub:alm2map}] routine generating maps of temperature
   and polarisation from their  $a_{\ell m}$
   \item[\htmlref{alm2map\_spin}{sub:alm2map_spin}] routine generating maps of
arbitrary spin from their  ${_s}a_{\ell m}$
%%   \item[smoothing] executable using \thedocid\ to smooth maps
  \item[synfast] executable using \thedocid\ to synthesize maps.
%%   \item[\htmlref{map2alm}{sub:map2alm}] routine performing the inverse transform
%%   of alm2map.
  \item[\htmlref{create\_alm}{sub:create_alm}] routine to generate randomly
  distributed $a_{\ell m}$ coefficients according to a given power spectrum
  \end{sulist}
\end{related}

\rule{\hsize}{2mm}

\newpage

\sloppy
\docid{alm2map\_spin*}\section[alm2map\_spin*]{ }
\label{sub:alm2map_spin}
\docrv{Version 2.0}
\author{Eric Hivon}
\abstract{This document describes the \healpix Fortran90 subroutine ALM2MAP\_SPIN*.}

\begin{facility}
{This routine produces the maps of arbitrary spin $s$ and $-s$ given their alm
coefficients.
%
A (complex) map $S$ of spin $s$ is a linear combination of the spin weighted harmonics ${_s}Y_{\ell m}$
\begin{equation}
	{_s}S(p) = \sum_{\ell m} {_s}a_{\ell m}\ \ {_s}Y_{\ell m}(p)
\end{equation}
% \begin{eqnarray}
% 	{_s}S(p) = \sum_{\ell m} {_s}a_{\ell m}\ \ {_s}Y_{\ell m}(p) \myhtmlimage{} \label{spinmap}
% \end{eqnarray}
for $l \ge |m|, l \ge |s|$,
and is such that ${_s}S^* = {_{-s}}S$.\\
The 
\linklatexhtml{usual phase convention for the spin weighted harmonics}%
{http://en.wikipedia.org/wiki/Spin-weighted_spherical_harmonics\#Calculating}%
{http://en.wikipedia.org/wiki/Spin-weighted_spherical_harmonics\#Calculating}
is
${_s}Y_{\ell m}^* = (-1)^{s+m} {_{-s}}Y_{\ell -m}$
and therefore 
${_s}a_{\ell m}^* = (-1)^{s+m} {_{-s}}a_{\ell -m}$.


%
\thedocid\ expects the alm coefficients to be provided as
	%typo correction on spin sign of second term of RHS, 2009-09-03
\begin{eqnarray}
	{_{|s|}}a^{+}_{\ell m} &\myequal& - ( {_{|s|}}a_{\ell m} + (-1)^s {_{-|s|}}a_{\ell m} )/2 \\
	{_{|s|}}a^{-}_{\ell m} &\myequal& - ( {_{|s|}}a_{\ell m} - (-1)^s {_{-|s|}}a_{\ell m} )/(2i)
\end{eqnarray}
for $m\ge 0$, knowing that, just as for spin 0 maps, the
coefficients for $m<0$ are given by 
\begin{eqnarray}
{_{|s|}}a^{+}_{l-m} &\myequal& (-1)^m {_{|s|}}a^{+*}_{\ell m}, \\
{_{|s|}}a^{-}_{l-m} &\myequal& (-1)^m {_{|s|}}a^{-*}_{\ell m}.
\end{eqnarray}
%
The two (real) maps produced by \thedocid\ are �defined respectively as
\begin{eqnarray}
	{_{|s|}}S^+ &\myequal& ({_{|s|}}S + {_{-|s|}}S)/2 \\
	{_{|s|}}S^- &\myequal& ({_{|s|}}S - {_{-|s|}}S)/(2i).
\end{eqnarray}
%
With these definitions, ${_2}a^{+}$, ${_2}a^{-}$, ${_2}S^+$ and ${_2}S^-$
match \healpix polarization $a^E, a^B, Q$ and $U$ respectively. However, for
$s=0$, $\ _{0}a^+_{\ell m} = -a^T_{\ell m}$, $\ _{0}a^-_{\ell m} = 0$, $\ {_0}S^+ = T$, $\ {_0}S^- = 0.$}
{\modAlmTools}
\end{facility}

\begin{f90format}
{\mylink{sub:alm2map_spin:nsmax}{nsmax}%
, \mylink{sub:alm2map_spin:nlmax}{nlmax}%
, \mylink{sub:alm2map_spin:nmmax}{nmmax}%
, \mylink{sub:alm2map_spin:spin}{spin}%
, \mylink{sub:alm2map_spin:alm}{alm}%
, \mylink{sub:alm2map_spin:map}{map}%
}
\end{f90format}

\begin{arguments}
{
\begin{tabular}{p{0.35\hsize} p{0.05\hsize} p{0.1\hsize} p{0.40\hsize}} \hline  
\textbf{name~\&~dimensionality} & \textbf{kind} & \textbf{in/out} & \textbf{description} \\ \hline
                   &   &   &                           \\ %%% for presentation
nsmax\mytarget{sub:alm2map_spin:nsmax} & I4B & IN & the $\nside$ value of the map to synthesize. \\
nlmax\mytarget{sub:alm2map_spin:nlmax} & I4B & IN & the maximum $\ell$ value used for the $a_{\ell m}$. \\
nmmax\mytarget{sub:alm2map_spin:nmmax} & I4B & IN & the maximum $m$ value used for the $a_{\ell m}$. \\
spin\mytarget{sub:alm2map_spin:spin} & I4B & IN & spin $s$ of the maps to be generated (only its absolute value
is relevant). \\
alm\mytarget{sub:alm2map_spin:alm}(1:2, 0:nlmax, 0:nmmax) & SPC/ DPC & IN & The ${_{|s|}}a^+_{\ell m}$ and ${_{|s|}}a^-_{\ell m}$ values to make the map
                   from.\\
map\mytarget{sub:alm2map_spin:map}(0:12*nsmax**2-1, 1:2) & SP/ DP & OUT & ${_{|s|}}S^+$ and ${_{|s|}}S^-$ output maps
\end{tabular}
}
\end{arguments}

\begin{example}
{
use healpix\_types \\
use pix\_tools, only : nside2npix \\
use alm\_tools, only : alm2map\_spin \\
integer(I4B) :: nside, lmax, mmax, npix, spin\\
real(SP), dimension(:,:), allocatable :: map \\
complex(SPC), dimension(:,:,:), allocatable :: alm \\
\ldots \\
nside=256 ; lmax=512 ; mmax=lmax ; spin=4\\
npix=nside2npix(nside)\\
allocate(alm(1:2,0:lmax,0:mmax))\\
allocate(map(0:npix-1,1:2))\\
\ldots \\
call alm2map\_spin(nside, lmax, mmax, spin, alm, map)  \\
}
{
Make spin-4 maps from the $a_{\ell m}$ passed in alm. The maps have $\nside$ of 256, and are constructed from $a_{\ell m}$ values up to 512 in $\ell$ and $m$.
}
\end{example}

\begin{modules}
  \begin{sulist}{} %%%% NOTE the ``extra'' brace here %%%%
  \item[\htmlref{ring\_synthesis}{sub:ring_synthesis}] Performs FFT over $m$ for synthesis of the rings.
  \item[compute\_lam\_mm, get\_pixel\_layout, ]
  \item[gen\_lamfac\_der, gen\_mfac, gen\_mfac\_spin, do\_lam\_lm\_spin, ] 
  \item[gen\_recfac, gen\_recfac\_spin, init\_rescale, l\_min\_ylm] Ancillary routines used
  for $Y_{\ell m}$ recursion
  \item[\textbf{misc\_utils}] module, containing:
  \item[\htmlref{assert\_alloc}{sub:assert}] routine to print error message, when an array can not be
  allocated properly
  \end{sulist}
Note: Starting with \htmlref{version 2.20}{sub:new2p20}, {\tt libpsht} routines will be called if $0<|s|\le100$.
\end{modules}

\begin{related}
  \begin{sulist}{} %%%% NOTE the ``extra'' brace here %%%%
   \item[\htmlref{alm2map}{sub:alm2map}] routine generating maps of temperature
   and polarisation from their  $a_{\ell m}$
   \item[\htmlref{alm2map\_der}{sub:alm2map_der}] routine generating maps of temperature
   and polarisation, and their spatial derivatives, from their  $a_{\ell m}$
   \item[\htmlref{map2alm\_spin}{sub:map2alm_spin}] routine performing the inverse transform
   of alm2map.
  \item[\htmlref{create\_alm}{sub:create_alm}] routine to generate randomly
  distributed $a_{\ell m}$ coefficients according to a given power spectrum
  \end{sulist}
\end{related}

\rule{\hsize}{2mm}

\newpage


\sloppy


\title{\healpix Fortran Subroutines Overview}
\docid{alms2fits*} \section[alms2fits*]{ }
\label{sub:alms2fits}
\docrv{Version 2.0}
\author{Frode K.~Hansen, Eric Hivon}
\abstract{This document describes the \healpix Fortran90 subroutine ALMS2FITS.}

\begin{facility}
{This routine stores  $a_{lm}$  values in a binary FITS file. Each FITS file
  extension created will contain one integer column with
  $index=\ell^2+\ell+m+1$, and 2 or 4 single (or double) precision columns with real/imaginary  $a_{lm}$  values and real/imaginary   standard deviation. One can store temperature $a_{lm}$ or temperature and polarisation, $a^T_{lm}$, $a^E_{lm}$ and $a^B_{lm}$. If temperature is specified, a FITS file with one extension is created. If polarisation is specified, a FITS file with 3 extensions one for each set of $a_{lm}$, $a_{lm}^T$, $a_{lm}^E$ and $a_{lm}^B$ is created.}
{\modFitstools}
\end{facility}

\begin{f90format}
{\mylink{sub:alms2fits:filename}{filename}%
, \mylink{sub:alms2fits:nalms}{nalms}%
, \mylink{sub:alms2fits:alms}{alms}%
, \mylink{sub:alms2fits:ncl}{ncl}%
, \mylink{sub:alms2fits:header}{header}%
, \mylink{sub:alms2fits:nlheader}{nlheader}%
, \mylink{sub:alms2fits:next}{next}%
}
\end{f90format}

\begin{arguments}
{
\begin{tabular}{p{0.4\hsize} p{0.05\hsize} p{0.05\hsize} p{0.40\hsize}} \hline  
\textbf{name~\&~dimensionality} & \textbf{kind} & \textbf{in/out} & \textbf{description} \\ \hline
                   &   &   &                           \\ %%% for presentation
filename\mytarget{sub:alms2fits:filename}(LEN=\filenamelen) & CHR & IN & filename for the FITS file to store the $a_{lm}$ in. \\
nalms\mytarget{sub:alms2fits:nalms} & I4B & IN & number of  $a_{lm}$  to store. \\
ncl\mytarget{sub:alms2fits:ncl} & I4B & IN & number of columns in the FITS file. If an standard deviation is given, this number is 5, otherwise it is 3. \\
next\mytarget{sub:alms2fits:next} & I4B & IN & the number of extensions. 1 for temperature only, 3
                   for temperature and polarisation. \\
\end{tabular}
\begin{tabular}{p{0.4\hsize} p{0.05\hsize} p{0.05\hsize} p{0.40\hsize}} \hline  
\textbf{name~\&~dimensionality} & \textbf{kind} & \textbf{in/out} & \textbf{description} \\ \hline
                   &   &   &                           \\ %%% for presentation
alms\mytarget{sub:alms2fits:alms}(1:nalms,1:ncl+1,1:next) & SP/ DP & IN & the $a_{lm}$ to write to the
                   file. alms(i,1,j) and alms(i,2,j) contain the $\ell$ and $m$
                   values for the ith  $a_{lm}$  (j=1,2,3 for
                   (T,E,B)). alms(i,3,j) and alms(i,4,j) contain the real and
                   imaginary value of the ith  $a_{lm}$. Finally, the standard
                   deviation for the ith  $a_{lm}$  is contained in alms(i,5,j)
                   (real) and alms(i,6,j) (imaginary). \\ 
nlheader\mytarget{sub:alms2fits:nlheader} & I4B & IN & number of header lines to write to the file. \\
header\mytarget{sub:alms2fits:header}(LEN=80) (1:nlheader, 1:next) & CHR & IN & the header to the FITS file. \\ 
\end{tabular}
}
\end{arguments}

\begin{example}
{
call alms2fits ('alms.fits', 65*66/2, alms, 3, header, 80, 3)  \\
}
{
Creates a FITS file with the $a_{lm}^T$, $a_{lm}^E$ and $a_{lm}^B$ values given in alms(1:65*66/2,1:4,1:3). The last index specifies (T,E,B). The second index gives l, m, real( $a_{lm}$ ), imaginary( $a_{lm}$ ) for each of the $a_{lm}$. The number 65*66/2 is the number of  $a_{lm}$  values up to an $\ell$ value of 64. 80 lines from header(1:80,1:3) is written to each extension.
}
\end{example}

\begin{modules}
  \begin{sulist}{} %%%% NOTE the ``extra'' brace here %%%%
  \item[write\_alms] routine called by \thedocid\ for each extension.
  \item[\textbf{fitstools}] module, containing:
  \item[printerror] routine for printing FITS error messages.
  \item[\textbf{cfitsio}] library for FITS file handling.		
  \end{sulist}
\end{modules}
\newpage
\begin{related}
  \begin{sulist}{} %%%% NOTE the ``extra'' brace here %%%%
  \item[\htmlref{fits2alms}{sub:fits2alms},
  \htmlref{read\_conbintab}{sub:read_conbintab}] routines to read $a_{lm}$ from
  a FITS file 
  \item[\htmlref{dump\_alms}{sub:dump_alms}] has the same function as \thedocid\ but with parameters passed differently.
  \end{sulist}
\end{related}

\rule{\hsize}{2mm}

\newpage


\sloppy


\title{\healpix Fortran Subroutines Overview}
\docid{alter\_alm*} \section[alter\_alm*]{ }
\label{sub:alter_alm}
\docrv{Version 2.0}
\author{Eric Hivon}
\abstract{This document describes the \healpix Fortran90 subroutine ALTER\_ALM.}

\begin{facility}
{This routine modifies scalar (and tensor) $a_{\ell m}$ by multiplying them by a beam window
  function described by a FWHM (in the case of a gaussian beam) or read from an external
  file (in the more general case of a circular beam)  $a_{\ell m}
  \longrightarrow a_{\ell m} b(\ell)\myhtmlimage{}$ . It can also be used to
  multiply the  $a_{\ell m}$ by an arbitray function of $\ell$.}
{\modAlmTools}
\end{facility}

\begin{f90format}
{\mylink{sub:alter_alm:nsmax}{nsmax}%
, \mylink{sub:alter_alm:nlmax}{nlmax}%
, \mylink{sub:alter_alm:nmmax}{nmmax}%
, \mylink{sub:alter_alm:fwhm_arcmin}{fwhm\_arcmin}%
, \mylink{sub:alter_alm:alm_TGC}{alm\_TGC}%
 [, \mylink{sub:alter_alm:beam_file}{beam\_file}%
, \mylink{sub:alter_alm:window}{window}%
]}
\end{f90format}

\begin{arguments}
{
\begin{tabular}{p{0.36\hsize} p{0.05\hsize} p{0.09\hsize} p{0.40\hsize}} \hline  
\textbf{name~\&~dimensionality} & \textbf{kind} & \textbf{in/out} & \textbf{description} \\ \hline
                   &   &   &                           \\ %%% for presentation
nsmax\mytarget{sub:alter_alm:nsmax} & I4B & IN & $\nside$ resolution parameter of the map associated with the $a_{lm}$
                   considered. Currently has no effect on the routine. \\ 
nlmax\mytarget{sub:alter_alm:nlmax} & I4B & IN & maximum $\ell$ value for the $a_{\ell m}$.   \\
nmmax\mytarget{sub:alter_alm:nmmax} & I4B & IN & maximum $m$ value for the $a_{\ell m}$.   \\
fwhm\_arcmin\mytarget{sub:alter_alm:fwhm_arcmin} & SP/ DP & IN & fwhm size of the gaussian beam in arcminutes. \\
alm\_TGC\mytarget{sub:alter_alm:alm_TGC}(1:p,0:nlmax,0:nmmax) & SPC/ DPC & INOUT & complex $a_{\ell m}$ values
                   to be altered.  The first index here runs from 1:1 for
                   temperature only, and 1:3 for polarisation. In the latter
                   case,  1=T, 2=E, 3=B. \\
\end{tabular}
\begin{tabular}{p{0.36\hsize} p{0.05\hsize} p{0.09\hsize} p{0.40\hsize}} \hline  
beam\_file\mytarget{sub:alter_alm:beam_file}(LEN=\filenamelen) \hskip 2cm (OPTIONAL)& CHR & IN & name of the file
                   containing the (non necessarily gaussian) window function
                   $B_\ell$  of a circular beam. If present, it will override
                   the argument {\tt fwhm\_arcmin}.  \\
window\mytarget{sub:alter_alm:window}(0:nlw,1:d) \hskip 5cm (OPTIONAL)& SP/ DP & IN & arbitrary window by which to multiply the
                   $a_{\ell m}$. If present, it overrides both {\tt fwhm\_arcmin}
                   and {\tt beam\_file}. If nlw $<$ nlmax, the $a_{\ell m}$ with
                   $\ell \in \{$nlw+1,nlmax$\}$ are set to 0, and a warning is issued. If $d<p$ the
                   window for temperature is replicated for polarisation.
\end{tabular}
}
\end{arguments}

\begin{example}
{
call alter\_alm(64, 128, 128, 1, 5.0, alm\_TGC)  \\
}
{
Alters scalar and tensor $a_{lm}$ of a map with $\nside=64$, $\ell_{\rm
  max}=m_{\rm max} = 128$ by multiplying them by the beam window function of a
gaussian beam with FWHM = 5 arcmin.
}
\end{example}

\begin{modules}
  \begin{sulist}{} %%%% NOTE the ``extra'' brace here %%%%
  \item[\textbf{alm\_tools}] module, containing:
	\item[\htmlref{generate\_beam}{sub:generate_beam}] routine to generate beam window function
	\item[\htmlref{pixel\_window}{sub:pixel_window}] routine to generate pixel window function
  \end{sulist}
\end{modules}

\begin{related}
  \begin{sulist}{} %%%% NOTE the ``extra'' brace here %%%%
  \item[\htmlref{create\_alm}{sub:create_alm}] Routine to create $a_{\ell m}$ coefficients.
  \item[\htmlref{rotate\_alm}{sub:rotate_alm}] Routine to rotate $a_{\ell m}$
  coefficients between 2 different arbitrary coordinate systems.
  \item[\htmlref{map2alm}{sub:map2alm}]  Routines to analyze a \healpix sky map into its $a_{\ell m}$
  coefficients.
  \item[\htmlref{alm2map}{sub:alm2map}] Routines to synthetize a \healpix sky map from its $a_{\ell m}$
  coefficients.
  \item[\htmlref{alms2fits}{sub:alms2fits}, \htmlref{dump\_alms}{sub:dump_alms}]
  Routines to save a set of $a_{lm}$ in a FITS file.  
  \end{sulist}
\end{related}

\rule{\hsize}{2mm}

\newpage



\sloppy


%%%\title{\healpix Fortran Subroutines Overview}
\docid{ang2vec} \section[ang2vec]{ }
\label{sub:ang2vec}
\docrv{Version 1.1}
\author{E. Hivon}
\abstract{This document describes the \healpix Fortran90 subroutine ANG2VEC.}

\begin{facility}
{Routine to convert the position angles $(\theta,\phi)\myhtmlimage{}$ of a point on the sphere 
into its 3D position vector $(x,y,z)$ with
$x = \sin\theta\cos\phi\myhtmlimage{}$, $y=\sin\theta\sin\phi\myhtmlimage{}$, $z=\cos\theta\myhtmlimage{}$. 
}
{\modPixTools}
\end{facility}

\begin{f90format}
{\mylink{sub:ang2vec:theta}{theta}%
, \mylink{sub:ang2vec:phi}{phi}%
, \mylink{sub:ang2vec:vector}{vector}%
}
\end{f90format}


\begin{arguments}
{
\begin{tabular}{p{0.3\hsize} p{0.05\hsize} p{0.1\hsize} p{0.45\hsize}} \hline  
\textbf{name~\&~dimensionality} & \textbf{kind} & \textbf{in/out} & \textbf{description} \\ \hline
                   &   &   &                           \\ %%% for presentation
theta\mytarget{sub:ang2vec:theta} & DP & IN & colatitude in radians measured southward from north pole (in
    $[0,\ \pi]\myhtmlimage{}$). \\
phi\mytarget{sub:ang2vec:phi}   & DP & IN & longitude in radians measured eastward (in $[0,\ 2\pi]\myhtmlimage{}$).\\
vector\mytarget{sub:ang2vec:vector}(3) & DP & OUT & three dimensional cartesian position vector
                   $(x,y,z)$ normalised to unity. The north pole is $(0,0,1)$
\end{tabular}
}
\end{arguments}

% \begin{example}
% {
% call ang2vec(theta,phi,vector) \\
% }
% {
% }
% \end{example}

\begin{related}
  \begin{sulist}{} %%%% NOTE the ``extra'' brace here %%%%
  %\item[\htmlref{ang2vec}{sub:ang2vec}] converts the position angles of a point on the sphere 
  %into its 3D position vector.
  \item[\htmlref{angdist}{sub:angdist}] computes the angular distance between 2 vectors
  \item[\htmlref{vec2ang}{sub:vec2ang}] converts the 3D position vector of point into its position
  angles on the sphere.
  \item[\htmlref{vect\_prod}{sub:vect_prod}] computes the vector product between two 3D vectors
  \end{sulist}
\end{related}

\rule{\hsize}{2mm}



\sloppy


\title{\healpix Fortran Subroutines Overview}
\docid{angdist} \section[angdist]{ }
\label{sub:angdist}
\docrv{Version 1.3}
\author{Eric Hivon}
\abstract{This document describes the \healpix Fortran90 subroutine ANGDIST.}

\begin{facility}
{Returns the angular distance in radians between two vectors. The input vectors
do not have to be normalised. For almost colinear or anti-colinear vectors, renders
numerically more accurate results than the $\cos^{-1}$ of the scalar product.} 
{\modPixTools}
\end{facility}

\begin{f90format}
{\mylink{sub:angdist:v1}{v1}%
, \mylink{sub:angdist:v2}{v2}%
, \mylink{sub:angdist:dist}{dist}%
}
\end{f90format}

\begin{arguments}
{
\begin{tabular}{p{0.3\hsize} p{0.05\hsize} p{0.1\hsize} p{0.45\hsize}} \hline 
\textbf{name~\&~dimensionality} & \textbf{kind} & \textbf{in/out} & \textbf{description} \\ \hline
                   &   &   &                           \\ %%% for presentation
v1\mytarget{sub:angdist:v1}(3) & DP & IN & cartesian vector. \\
v2\mytarget{sub:angdist:v2}(3) & DP & IN & cartesian vector. \\
dist\mytarget{sub:angdist:dist} & DP & OUT & angular distance in radians between the 2 vectors.
\end{tabular}
}
\end{arguments}

\begin{example}
{
use healpix\_types \\
use pix\_tools,    only : angdist \\
real(DP) :: dist, one = 1.0\_dp \\
call angdist((/1,2,3/)*one, (/1,2,4/)*one, dist)  \\
print*, dist
}
{
Returns the angular distance between 2 vectors.
}
\end{example}
% \newpage
% \begin{modules}
%   \begin{sulist}{} %%%% NOTE the ``extra'' brace here %%%%
%  \item[\htmlref{in\_ring}{sub:in_ring}] routine to find the pixels in a certain slice of a given ring.		
%  \item[\htmlref{ring\_num}{sub:ring_num}] function to return the ring number corresponding to the coordinate $z$
%   \end{sulist}
% \end{modules}

\begin{related}
  \begin{sulist}{} %%%% NOTE the ``extra'' brace here %%%%
  \item[\htmlref{ang2vec}{sub:ang2vec}] converts the position angles of a point on the sphere 
into its 3D position vector.
  %\item[\htmlref{angdist}{sub:angdist}] computes the angular distance between 2 vectors
  \item[\htmlref{vec2ang}{sub:vec2ang}] converts the 3D position vector of point into its position
  angles on the sphere.
  \item[\htmlref{vect\_prod}{sub:vect_prod}] computes the vector product between two 3D vectors
  \end{sulist}
\end{related}

\rule{\hsize}{2mm}

\newpage



\sloppy

%%%\title{\healpix Fortran Subroutines Overview}
\docid{assert,assert\_alloc, assert\_directory\_present,~$\ldots$} \section[assert, assert\_alloc, assert\_directory\_present, assert\_present, fatal\_error]{ }
\label{sub:assert}
\docrv{Version 2.0}
\author{Eric Hivon}
\abstract{This document describes the \healpix Fortran90 functions in module MISC\_UTILS.}

\begin{facility}
{The Fortran90 module misc\_utils contains a few routines to test an assertion and return an error
  message if it is false.}
{\modMiscUtils}
\end{facility}

%-------------------------------

\rule{\hsize}{0.7mm}
\textsc{\large{\textbf{SUBROUTINES: }}}\hfill\newline
{\tt call assert(test [, msg, errcode])} 

 \begin{tabular}{@{}p{0.3\hsize}@{\hspace{1ex}}p{0.7\hsize}@{}}
                         & if {\tt test} is true, proceeds with normal code execution. If
                        {\tt test} is false, issues a standard error message
                        (unless {\tt msg} is provided) and stops the code execution with the status
                        {\tt errcode} (or 1 by default). \\
     \end{tabular}\\

{\tt call assert\_alloc(status, code, array)} 

 \begin{tabular}{@{}p{0.3\hsize}@{\hspace{1ex}}p{0.7\hsize}@{}}
                         & if {\tt status} is 0, proceeds with normal code execution. If
                        not, issues an error message indicating a problem during memory allocation
                        of 
                        {\tt array} in program {\tt code}, and stops the code execution.\\
     \end{tabular}\\


{\tt call assert\_directory\_present(directory)} 

 \begin{tabular}{@{}p{0.3\hsize}@{\hspace{1ex}}p{0.7\hsize}@{}}
	                 & issues an error message and stops the code execution if
                        the directory named {\tt directory} can not be found\\
     \end{tabular}\\


{\tt call assert\_not\_present(filename)} 

 \begin{tabular}{@{}p{0.3\hsize}@{\hspace{1ex}}p{0.7\hsize}@{}}
                         & issues an error message and stops the code execution if
                        a file with name {\tt filename} already exists.\\
     \end{tabular}\\

{\tt call assert\_present(filename)} 

 \begin{tabular}{@{}p{0.3\hsize}@{\hspace{1ex}}p{0.7\hsize}@{}}
	                 & issues an error message and stops the code execution if
                        the file named {\tt filename} can not be found.\\
     \end{tabular}\\

{\tt call fatal\_error([msg])} 

{\tt call fatal\_error} 

 \begin{tabular}{@{}p{0.3\hsize}@{\hspace{1ex}}p{0.7\hsize}@{}}
			 & issue an (optional user defined) error message and stop the code execution.\\
     \end{tabular}\\


\vskip 3cm
%-------------------------------

\begin{arguments}
{
\begin{tabular}{p{0.30\hsize} p{0.05\hsize} p{0.08\hsize} p{0.47\hsize}} \hline  
\textbf{name~\&~dimensionality} & \textbf{kind} & \textbf{in/out} & \textbf{description} \\ \hline
                   &   &   &                           \\ %%% for presentation
test & LGT & IN & result of a logical test \\
msg \hfill OPTIONAL & CHR & IN & character string describing nature of error \\
errorcode \hfill OPTIONAL & I4B & IN & error status given to code interruption \\
status & I4B & IN & value of the {\tt stat} flag returned by the F90 {\tt allocate} command \\
code & CHR & IN & name of program or code in which allocation is made \\
array & CHR & IN & name of array allocated \\
directory & CHR & IN & directory name (contains a '/')\\
filename & CHR & IN & file name \\
\end{tabular}
}
\end{arguments}

%-------------------------------

\begin{example}
{
program my\_code \\
use misc\_utils \\
real, allocatable, dimension(:) :: vector\\
integer :: status \\
real :: a = -1. \\
\\
allocate(vector(12345),stat=status) \\
call assert\_alloc(status, 'my\_code', 'vector') \\
\\
call assert\_directory\_present('/home') \\
\\
call assert(a > 0., 'a is NEGATIVE !!!') \\
\\
end program my\_code\\
}
{ Will issue a error message and stops the code if {\tt vector} can not be allocated, will stop the
  code if '/home' is not found, and will stop the code and complain loudly about it 
because {\tt a} is actually negative.
}
\end{example}

%% \begin{modules}
%%   \begin{sulist}{} %%%% NOTE the ``extra'' brace here %%%%
%%  \item[mk\_pix2xy, mk\_xy2pix] routines used in the conversion between pixel values and ``cartesian'' coordinates on the Healpix face.
%%   \end{sulist}
%% \end{modules}

%% \begin{related}
%%   \begin{sulist}{} %%%% NOTE the ``extra'' brace here %%%%
%%   \item[\htmlref{neighbours\_nest}{sub:neighbours_nest}] find neighbouring pixels.
%%   \item[\htmlref{ang2vec}{sub:ang2vec}] convert $(\theta,\phi)$ spherical coordinates into $(x,y,z)$ cartesian coordinates.
%%   \item[\htmlref{vec2ang}{sub:vec2ang}] convert $(x,y,z)$ cartesian coordinates into $(\theta,\phi)$ spherical coordinates.
%%   \end{sulist}
%% \end{related}

\rule{\hsize}{2mm}

\newpage


\sloppy


\docid{brag\_openmp}\section[brag\_openmp]{ }
\label{sub:brag_openmp}
\docrv{Version 1.0}
\author{Eric Hivon}
\abstract{This document describes the \healpix Fortran90 subroutine BRAG_OPENMP.}

\begin{facility}
{If compiled with shared memory libraries (OpenMP), this routine prints out the number of
CPUs used (controlled by the environment variable {\tt OMP\_NUM\_THREADS}) and the number of CPUs available.}
{\modMiscUtils}
\end{facility}

\begin{f90format}
{}
\end{f90format}

% \begin{arguments}
% {
% \begin{tabular}{p{0.4\hsize} p{0.05\hsize} p{0.1\hsize} p{0.35\hsize}} \hline  
% \textbf{name~\&~dimensionality} & \textbf{kind} & \textbf{in/out} & \textbf{description} \\ \hline
%                    &   &   &                           \\ %%% for presentation
% \end{tabular}
% }
% \end{arguments}

\begin{example}
{
use misc\_utils \\
call \thedocid() \\
}
{
\parbox[t]{8.2cm}{
Will~print~out: \hfill\\
\parbox[t]{8cm}{\tt
-------------------------------------- \\
Number of OpenMP threads in use:    2 \\
Number of CPUs available:           2 \\
-------------------------------------- }
on a bi-pro (or dual core) computer}
}
\end{example}

% \begin{modules}
%   \begin{sulist}{} %%%% NOTE the ``extra'' brace here %%%%
%   \item[\htmlref{ring\_synthesis}{sub:ring_synthesis}] Performs FFT over $m$ for synthesis of the rings.
%   \item[compute\_lam\_mm, get\_pixel\_layout, ]
%   \item[gen\_lamfac,gen\_mfac, gen\_normpol, ] 
%   \item[gen\_recfac, init\_rescale, l\_min\_ylm] Ancillary routines used
%   for $Y_{\ell m}$ recursion
%   \item[\textbf{misc\_utils}] module, containing:
%   \item[assert\_alloc] routine to print error message, when an array can not be
%   allocated properly
%   \end{sulist}
% \end{modules}

% \begin{related}
%   \begin{sulist}{} %%%% NOTE the ``extra'' brace here %%%%
%   \end{sulist}
% \end{related}

\rule{\hsize}{2mm}

\newpage

\sloppy

\title{\healpix Fortran Subroutines Overview}
\docid{complex\_fft} \section[complex\_fft]{ }
\label{sub:complex_fft}
\docrv{Version 1.1}
\author{Martin Reinecke}
\abstract{This document describes the \healpix Fortran90 subroutine
complex\_fft.}

\begin{facility}
{This routine performs a forward or backward Fast Fourier Transformation
on its argument {\tt data}.}
{\modHealpixFft}
\end{facility}

\begin{f90format}
{\mylink{sub:complex_fft:data}{data}%
, \mylink{sub:complex_fft:backward}{backward}%
}
\end{f90format}

\begin{arguments}
{
\begin{tabular}{p{0.3\hsize} p{0.05\hsize} p{0.1\hsize} p{0.45\hsize}} \hline  
\textbf{name\&dimensionality} & \textbf{kind} & \textbf{in/out} & \textbf{description} \\ \hline
                   &   &   &                           \\ %%% for presentation
data\mytarget{sub:complex_fft:data}(:) & XXX & INOUT &
  array containing the input and output data. It can be of type
  real(sp), real(dp), complex(spc) or complex(dpc). If it is of type real,
  it is interpreted as an array of size(data)/2 complex variables.  \\
backward\mytarget{sub:complex_fft:backward} & LGT & IN & Optional argument. If present and true, perform backward transformation, else forward \\
\end{tabular}}
\end{arguments}

\begin{example}
{
use healpix\_fft \\
call complex\_fft (data, backward=.true.)
}
{
Performs a backward FFT on data.
}
\end{example}

\begin{related}
  \begin{sulist}{} %%%% NOTE the ``extra'' brace here %%%%
  \item[\htmlref{real\_fft}{sub:real_fft}] routine for FFT of real data
  \end{sulist}
\end{related}

\rule{\hsize}{2mm}

\newpage


\sloppy


\title{\healpix Fortran Subroutines Overview}
\docid{compute\_statistics*} \section[compute\_statistics*]{ }
\label{sub:compute_statistics}
\docrv{Version 2.0}
\author{Eric Hivon}
\abstract{This document describes the \healpix Fortran90 subroutine COMPUTE\_STATISTICS*.}

\newcommand{\myskip}{\hskip 1cm}

\begin{facility}
{This routine computes the min, max, absolute deviation and first four order moment of a data set}
{\modStatistics}
\end{facility}

\begin{f90format}
{\mylink{sub:compute_statistics:data}{data}%
 ,\mylink{sub:compute_statistics:stats}{stats}%
 \optional{[,~\mylink{sub:compute_statistics:badval}{badval}%
]}}
\end{f90format}
\aboutoptional

\begin{arguments}
{
\begin{tabular}{p{0.30\hsize} p{0.05\hsize} p{0.05\hsize} p{0.50\hsize}} \hline  
\textbf{name~\&~dimensionality} & \textbf{kind} & \textbf{in/out} & \textbf{description} \\ \hline
                   &   &   &                           \\ %%% for presentation
data\mytarget{sub:compute_statistics:data}(:) & SP/ DP & IN & data set $\{x_i\}$\\
stats\mytarget{sub:compute_statistics:stats}   & tstats & OUT & structure containing the statistics of the
                   data. The respective fields (stats\%{\em field}) are:\\
\myskip ntot & I8B & -- & total number of data points \\
\myskip nvalid & I8B & -- & number $n$ of valid data points \\
\myskip mind, maxd & DP & -- & minimum and maximum valid data \\
\myskip average & DP & -- & average of valid points $m= \sum_i x_i / n$\\
\myskip absdev & DP & -- & absolute deviation $a= \sum_i|x_i-m|/n$\\
\myskip var & DP & -- & variance $\sigma^2 = \sum(x_i-m)^2/ (n-1)$\\
\myskip rms & DP & -- & standard deviation $\sigma$ \\
\myskip skew & DP & -- & skewness factor $s = \sum(x_i-m)^3 / (n\sigma^3)$\\
\myskip kurt & DP & -- & kurtosis factor $k = \sum(x_i-m)^4 / (n\sigma^4) - 3$\\
\ & \ & \ & \\
\optional{badval\mytarget{sub:compute_statistics:badval}} \hskip 3cm (OPTIONAL) & SP/ DP & IN & sentinel value given to bad data points. Data points with this
                   value will be ignored during calculation of the statistics. If
                   not set, all points will be considered. {\bf Do not set to 0!}.
\end{tabular}
}
\end{arguments}
%%\newpage

\begin{example}
{
use statistics, only: compute\_statistics, print\_statistics, tstats \\
type(tstats) :: stats \\
... \\
compute\_statistics(map, stats)  \\
print*,stats\%average, stats\%rms\\
print\_statistics(stats) \\
}
{
Computes the statistics of {\tt map}, prints its average and {\em rms} and
prints the whole list of statistical measures.
}
\end{example}

%% \begin{modules}
%%   \begin{sulist}{} %%%% NOTE the ``extra'' brace here %%%%
%%   \item[\textbf{}] 
%%   \item[] 
%%   \end{sulist}
%% \end{modules}

\begin{related}
  \begin{sulist}{} %%%% NOTE the ``extra'' brace here %%%%
  \item[\htmlref{median}{sub:median}] routine to compute median of a data set
  \end{sulist}
\end{related}

\rule{\hsize}{2mm}

\newpage



\sloppy


%%%\title{\healpix Fortran Subroutines Overview}
\docid{concatnl} \section[concatnl]{ }
\label{sub:concatnl}
\docrv{Version 1.1}
\author{E. Hivon}
\abstract{This document describes the \healpix Fortran90 subroutine CONCATNL.}

\begin{facility}
{Function to concatenate up to 10 subtrings interspaced with LineFeed
character. Upon printing each subtring will be on a different line.
}
{\modParamfileIo}
\end{facility}

\begin{f90function}
{string1[, string2, string3, ...]}
\end{f90function}

%\ mylink: to avoid automatic processing by make_internal_links.sh

\begin{arguments}
{
\begin{tabular}{p{0.3\hsize} p{0.05\hsize} p{0.1\hsize} p{0.45\hsize}} \hline  
\textbf{name~\&~dimensionality} & \textbf{kind} & \textbf{in/out} & \textbf{description} \\ \hline
                   &   &   &                           \\ %%% for presentation
string1 & CHR & IN & the first substring to be concatenated. \\
string2 & CHR & IN \hskip 1cm optional& the second substring (if any) to be concatenated. \\
string3 & CHR & IN \hskip 1cm optional& ... up to 10 substrings can be concatenated. \\
var & CHR & OUT & concatenation of the substrings interspaced with LineFeed character.\\
\end{tabular}
}
\end{arguments}

\begin{example}
{
use paramfile\_io \\
print*,concatnl('a','bbbbbbbb','C 10 3') 
}
{\parbox[t]{2.2cm}{
Will~return:
\parbox[t]{2cm}{\tt{a\\ bbbbbbbb\\ C 10 3}}}
}
\end{example}
\begin{related}
  \begin{sulist}{} %%%% NOTE the ``extra'' brace here %%%%
  \item[\htmlref{parse\_xxx}{sub:parse_xxx}] parse an ASCII file for parameters definition
  \end{sulist}
\end{related}

\rule{\hsize}{2mm}



\sloppy

\title{\healpix Fortran Subroutines Overview}
\docid{convert\_inplace*} \section[convert\_inplace*]{ }
\label{sub:convert_inplace}
\docrv{Version 2.0}
\author{Eric Hivon}
\abstract{This document describes the \healpix Fortran90 subroutine CONVERT\_INPLACE.}


\begin{facility}
{Routine to convert a \healpix map from NESTED to RING scheme or vice
  versa. The conversion is done in place, meaning that it doesn't require memory
  for a temporary map, like the
  $\htmlref{convert\_nest2ring}{sub:convert_nest2ring}$ or
  $\htmlref{convert\_ring2nest}{sub:convert_ring2nest}$
  routines. But for that reason, this routine is slower and not parallelized. The routine is a
  wrapper for 6 different routines and can threfore process
  integer, single precision and double precision maps as well as mono or bi
  dimensional arrays.}
{\modPixTools}
\end{facility}

\begin{f90format}
{\mylink{sub:convert_inplace:subcall}{subcall}%
, \mylink{sub:convert_inplace:map}{map}%
}
\end{f90format}

\begin{arguments}
{
\begin{tabular}{p{0.4\hsize} p{0.05\hsize} p{0.1\hsize} p{0.35\hsize}} \hline  
\textbf{name~\&~dimensionality} & \textbf{kind} & \textbf{in/out} & \textbf{description} \\ \hline
                   &   &   &                           \\ %%% for presentation
subcall\mytarget{sub:convert_inplace:subcall} & --- & IN & routine to be called by convert\_inplace\_real. Set this to \htmlref{ring2nest}{sub:pix_tools} or \htmlref{nest2ring}{sub:pix_tools} dependent on wether the conversion is RING to NESTED or vice versa. \\
map\mytarget{sub:convert_inplace:map}(0:npix-1) & I4B/ SP/ DP & INOUT & mono-dimensional full sky map to be converted, the routine finds the size itself. \\
map\mytarget{sub:convert_inplace:map}(0:npix-1,1:nd) & I4B/ SP/ DP & INOUT & bi-dimensional (nd$>0$) full sky map to be
                   converted, the routine finds both dimensions
                   itself. Processing a bidimensional map with nd$>1$ should be
                   faster than each of the nd 1D-maps consecutively.\\

\end{tabular}
}
\end{arguments}

\begin{example}
{
call convert\_inplace(\htmlref{ring2nest}{sub:pix_tools},map)  \\
}
{
Converts an map from RING to NESTED scheme.
}
\end{example}
%%\newpage
\begin{modules}
  \begin{sulist}{} %%%% NOTE the ``extra'' brace here %%%%
 \item[\htmlref{nest2ring}{sub:pix_tools}] routine to convert a NESTED pixel index to RING pixel number.
 \item[\htmlref{ring2nest}{sub:pix_tools}] routine to convert a RING pixel index to NESTED pixel number.	
  \end{sulist}
\end{modules}

\begin{related}
  \begin{sulist}{} %%%% NOTE the ``extra'' brace here %%%%
  \item[\htmlref{convert\_nest2ring}{sub:convert_nest2ring}] convert from NESTED to RING scheme using a temporary array. Requires more space then convert\_inplace, but is faster.
  \item[\htmlref{convert\_ring2nest}{sub:convert_ring2nest}]
  convert from RING to NESTED scheme using a temporary array. Requires more space then convert\_inplace, but is faster.
  \end{sulist}
\end{related}

\rule{\hsize}{2mm}

\newpage


\sloppy


%%%\title{\healpix Fortran Subroutines Overview}
\docid{convert\_nest2ring*} \section[convert\_nest2ring*]{ }
\label{sub:convert_nest2ring}
\docrv{Version 2.0}
\author{Eric Hivon, Frode K.~Hansen}
\abstract{This document describes the \healpix Fortran90 subroutine CONVERT\_NEST2RING.}


\begin{facility}
{Routine to convert a \healpix map from NESTED to RING scheme. \newline
The routine is a
  wrapper for 6 different routines and can threfore process
  integer, single precision and double precision maps as well as mono or bi
  dimensional arrays. \newline This routine is fast, and is parallelized for shared memory
architecture, but requires extra memory to store a temporary map in. }
{\modPixTools}
\end{facility}

\begin{f90format}
{\mylink{sub:convert_nest2ring:nside}{nside}%
, \mylink{sub:convert_nest2ring:map}{map}%
}
\end{f90format}

\begin{arguments}
{
\begin{tabular}{p{0.4\hsize} p{0.05\hsize} p{0.1\hsize} p{0.35\hsize}} \hline  
\textbf{name~\&~dimensionality} & \textbf{kind} & \textbf{in/out} & \textbf{description} \\ \hline
                   &   &   &                           \\ %%% for presentation
nside\mytarget{sub:convert_nest2ring:nside} & I4B & IN & the $\nside$ parameter of the map to be converted. \\
map\mytarget{sub:convert_nest2ring:map}(0:12*nside**2-1) & I4B/ SP/ DP & INOUT & mono-dimensional full sky map to be converted to RING scheme. \\
map(0:12*nside**2-1,1:nd) & I4B/ SP/ DP & INOUT & bi-dimensional full sky map to
                   be converted to RING scheme. The routine finds the second
                   dimension (nd) by itself. Processing a bidimensional map with
{\tt nd}$>1$ should be
                   faster than each of the {\tt nd} 1D-maps consecutively.
\end{tabular}
}
\end{arguments}

\begin{example}
{
call convert\_nest2ring(256,map)  \\
}
{
Converts an $\nside=256$ map given in array {\tt map} from NESTED to RING scheme.
}
\end{example}

\begin{modules}
  \begin{sulist}{} %%%% NOTE the ``extra'' brace here %%%%
 \item[\htmlref{nest2ring}{sub:pix_tools}] routine to convert a NESTED pixel index to RING pixel number.		
  \end{sulist}
\end{modules}
%%%\newpage
\begin{related}
  \begin{sulist}{} %%%% NOTE the ``extra'' brace here %%%%
  \item[\htmlref{convert\_ring2nest}{sub:convert_ring2nest}] convert between RING and NESTED schemes.
  \item[\htmlref{convert\_inplace}{sub:convert_inplace}] convert between NESTED
    and RING schemes inplace. This routine is slower than \thedocid, but doesn't require as much memory.
  \end{sulist}
\end{related}

\rule{\hsize}{2mm}

\newpage


\sloppy


\title{\healpix Fortran Subroutines Overview}
\docid{convert\_ring2nest*} \section[convert\_ring2nest*]{ }
\label{sub:convert_ring2nest}
\docrv{Version 2.0}
\author{Eric Hivon, Frode K.~Hansen}
\abstract{This document describes the \healpix Fortran90 subroutine CONVERT\_RING2NEST.}


\begin{facility}
{Routine to convert a \healpix map from RING to NESTED scheme. \newline
The routine is a
  wrapper for 6 different routines and can threfore process
  integer, single precision and double precision maps as well as mono or bi
  dimensional arrays. \newline This routine is fast, and is parallelized for shared memory
architecture, but requires extra memory to store a temporary map in. }
{\modPixTools}
\end{facility}

\begin{f90format}
{\mylink{sub:convert_ring2nest:nside}{nside}%
, \mylink{sub:convert_ring2nest:map}{map}%
}
\end{f90format}

\begin{arguments}
{
\begin{tabular}{p{0.4\hsize} p{0.05\hsize} p{0.1\hsize} p{0.35\hsize}} \hline  
\textbf{name~\&~dimensionality} & \textbf{kind} & \textbf{in/out} & \textbf{description} \\ \hline
                   &   &   &                           \\ %%% for presentation
nside\mytarget{sub:convert_ring2nest:nside} & I4B & IN & the $\nside$ parameter of the map to be converted. \\
map\mytarget{sub:convert_ring2nest:map}(0:12*nside**2-1) & I4B/ SP/ DP & INOUT & mono-dimensional full sky map to be converted to RING scheme. \\
map\mytarget{sub:convert_ring2nest:map}(0:12*nside**2-1,1:nd) & I4B/ SP/ DP & INOUT & bi-dimensional full sky map to
                   be converted to RING scheme. The routine finds the second
                   dimension (nd) by itself. Processing a bidimensional map with
{\tt nd}$>1$ should be
                   faster than each of the {\tt nd} 1D-maps consecutively.
\end{tabular}
}
\end{arguments}

\begin{example}
{
call convert\_ring2nest(256,map)  \\
}
{
Converts an $\nside=256$ map given in array {\tt map} from RING to NESTED scheme.
}
\end{example}

\begin{modules}
  \begin{sulist}{} %%%% NOTE the ``extra'' brace here %%%%
 \item[\htmlref{ring2nest}{sub:pix_tools}] routine to convert a RING pixel index to NESTED pixel number.		
  \end{sulist}
\end{modules}
%%%%\newpage
\begin{related}
  \begin{sulist}{} %%%% NOTE the ``extra'' brace here %%%%
  \item[\htmlref{convert\_nest2ring}{sub:convert_ring2nest}] convert between
  NESTED and RING schemes.
  \item[\htmlref{convert\_inplace}{sub:convert_inplace}] convert between 
    RING and NESTED schemes inplace. This routine is slower than \thedocid, but doesn't require as much memory.
  \end{sulist}
\end{related}

\rule{\hsize}{2mm}

\newpage

\sloppy


\title{\healpix Fortran Subroutines Overview}
\docid{coordsys2euler\_zyz} \section[coordsys2euler\_zyz]{ }
\label{sub:coordsys2euler_zyz}
\docrv{Version 2.0}
\author{Eric Hivon}
\abstract{This document describes the \healpix Fortran90 subroutine COORDSYS2EULER\_ZYZ.}

\begin{facility}
{This routine returns the three Euler angles $\psi, \theta, \varphi
\myhtmlimage{}$, corresponding to a rotation between standard astronomical
coordinate systems. This angles can then be used in rotate\_alm}
{\modCoordVConvert}
\end{facility}

\begin{f90format}
{\mylink{sub:coordsys2euler_zyz:iepoch}{iepoch}%
, \mylink{sub:coordsys2euler_zyz:oepoch}{oepoch}%
, \mylink{sub:coordsys2euler_zyz:isys}{isys}%
, \mylink{sub:coordsys2euler_zyz:osys}{osys}%
, \mylink{sub:coordsys2euler_zyz:psi}{psi}%
, \mylink{sub:coordsys2euler_zyz:theta}{theta}%
, \mylink{sub:coordsys2euler_zyz:phi}{phi}%
}
\end{f90format}

\begin{arguments}
{
\begin{tabular}{p{0.26\hsize} p{0.05\hsize} p{0.09\hsize} p{0.50\hsize}} \hline  
\textbf{name~\&~dimensionality} & \textbf{kind} & \textbf{in/out} & \textbf{description} \\ \hline
                   &   &   &                           \\ %%% for presentation
iepoch\mytarget{sub:coordsys2euler_zyz:iepoch} & DP & IN & epoch of the input astronomical coordinate system.\\
oepoch\mytarget{sub:coordsys2euler_zyz:oepoch} & DP & IN & epoch of the output astronomical coordinate system.\\
isys\mytarget{sub:coordsys2euler_zyz:isys}(len=*) & CHR & IN & input coordinate system, should be one of 'E'=Ecliptic, 'G'=Galactic, 'C'/'Q'=Celestial/eQuatorial.\\
osys\mytarget{sub:coordsys2euler_zyz:osys}(len=*) & CHR & IN & output coordinate system, same choice as above.\\
psi\mytarget{sub:coordsys2euler_zyz:psi}	& DP & OUT & first Euler angle: rotation $\psi$ about the z-axis. \\
theta\mytarget{sub:coordsys2euler_zyz:theta}	& DP & OUT & second Euler angle: rotation $\theta$ about the original
(unrotated) y-axis; \\
phi\mytarget{sub:coordsys2euler_zyz:phi}	& DP & OUT & third Euler angle: rotation $\varphi$ about the original (unrotated) z-axis;
\end{tabular}
}
\end{arguments}

\begin{example}
{
use coord\_v\_convert, only: coordsys2euler\_zyz \\
use alm\_tools, only: rotate\_alm \\
...\\
call coordsys2euler\_zyz(2000.0\_dp, 2000.0\_dp, 'E', 'G', psi, theta, phi) \\
call rotate\_alm(64, alm\_TGC, psi, theta, phi)  \\
}
{
Rotate the $a_{lm}$ from Ecliptic to Galactic coordinates.
}
\end{example}

% \begin{modules}
%   \begin{sulist}{} %%%% NOTE the ``extra'' brace here %%%%
%   \item[\textbf{alm\_tools}] module, containing:
% 	\item[\htmlref{generate\_beam}{sub:generate_beam}] routine to generate beam window function
% 	\item[\htmlref{pixel\_window}{sub:pixel_window}] routine to generate pixel window function
%   \end{sulist}
% \end{modules}

\begin{related}
  \begin{sulist}{} %%%% NOTE the ``extra'' brace here %%%%
  \item[\htmlref{rotate\_alm}{sub:rotate_alm}] apply arbitrary sky rotation to a
  set of $a_{lm}$ coefficients.
  \item[\htmlref{xcc\_v\_convert}{sub:xcc_v_convert}] rotates a 3D coordinate
vector from one astronomical coordinate system to another.
  \end{sulist}
\end{related}

\rule{\hsize}{2mm}

\newpage


\sloppy


\title{\healpix Fortran Subroutines Overview}
\docid{create\_alm*} \section[create\_alm*]{ }
\label{sub:create_alm}
\docrv{Version 2.0}
\author{Frode K.~Hansen, Eric Hivon}
\abstract{This document describes the \healpix Fortran90 subroutine CREATE\_ALM.}

\begin{facility}
{This routine generates scalar (and tensor) $a_{\ell m}$ for a temperature (and
  polarisation) power spectrum read from an input FITS
file. The $a_{\ell m}$ are gaussian distributed with a zero mean, and their
  amplitude is multiplied with the $\ell$-space window function of a gaussian
  beam characterized by its FWHM or an arbitrary circular beam
and a pixel window read from an external file.}
{\modAlmTools}
\end{facility}

\begin{f90format}
{\mylink{sub:create_alm:nsmax}{nsmax}%
, \mylink{sub:create_alm:nlmax}{nlmax}%
, \mylink{sub:create_alm:nmmax}{nmmax}%
, \mylink{sub:create_alm:polar}{polar}%
, \mylink{sub:create_alm:filename}{filename}%
, \mylink{sub:create_alm:rng_handle}{rng\_handle}%
, \mylink{sub:create_alm:fwhm_arcmin}{fwhm\_arcmin}%
, \mylink{sub:create_alm:alm_TGC}{alm\_TGC}%
, \mylink{sub:create_alm:header}{header}%
 \optional{[,
\mylink{sub:create_alm:windowfile}{windowfile}%
, \mylink{sub:create_alm:units}{units}%
, \mylink{sub:create_alm:beam_file}{beam\_file}%
]}}
\end{f90format}
\aboutoptional

\begin{arguments}
{
\begin{tabular}{p{0.37\hsize} p{0.05\hsize} p{0.1\hsize} p{0.38\hsize}} \hline  
\textbf{name~\&~dimensionality} & \textbf{kind} & \textbf{in/out} & \textbf{description} \\ \hline
                   &   &   &                           \\ %%% for presentation
nsmax\mytarget{sub:create_alm:nsmax} & I4B & IN & $\nside$ of the map to be synthetized from the $a_{\ell m}$
                   created by this routine. \\ 
nlmax\mytarget{sub:create_alm:nlmax} & I4B & IN & maximum $\ell$ value to be considered (MAX=$4\nside$
if \mylink{sub:create_alm:windowfile}{{\tt windowfile}} is provided).   \\
nmmax\mytarget{sub:create_alm:nmmax} & I4B & IN & maximum $m$ value for the $a_{\ell m}$.   \\
%
polar\mytarget{sub:create_alm:polar} & I4B & IN & \textbf{if set to 0}, only Temperature (scalar) $a_{\ell m}$ are
generated using TT spectrum. \textbf{If set to 1}, 'conventional' polarization is added, based on EE, BB and TE spectra. \textbf{If set to 2}, and if
the relevant information is in {\tt filename}, polarization is generated
assuming non-zero correlation of Curl (B) modes with Temperature (T) and Gradient
(E) modes (TB and EB cross-spectra). Note that the \htmlref{\tt synfast}{fac:synfast} facility calls \thedocid\ with {\tt polar}=0 {\em or} {\tt polar}=1\\
%
filename\mytarget{sub:create_alm:filename}(LEN=\filenamelen) & CHR & IN & name of FITS file containing power
spectra in the order TT, [EE, BB, TE, [TB, EB]] (terms in brackets are optional, see \mylink{sub:create_alm:polar}{{\tt polar}}) \\
%% iseed & I4B & INOUT & seed for generation of gaussian distributed random numbers
%%                    to the $a_{\ell m}$. 
%%                    Set to a negative integer to (re-)initialise the
%%                    random number generator. On exit iseed will be positive
%%                    definite. If set to a positive value, the information contained in {\tt
%%                    rng\_handle} will be used to carry on an existing random sequence.\\
rng\_handle\mytarget{sub:create_alm:rng_handle} & \htmlref{planck\_rng}{sub:planck_rng} & \hskip 2cm INOUT & structure containing
information necessary to continue a random sequence
initiated {\em previously} with the 
subroutine \htmlref{{\tt rand\_init}}{sub:rand_init}. Consecutive calls to \thedocid {} can be made after a
single invocation to \htmlref{{\tt rand\_init}}{sub:rand_init}.\\
%
fwhm\_arcmin\mytarget{sub:create_alm:fwhm_arcmin} & SP/ DP & IN & FWHM size of the gaussian beam in arcminutes. \\
%
alm\_TGC\mytarget{sub:create_alm:alm_TGC}(1:p,0:nlmax,0:nmmax) & SPC/ DPC & OUT & complex $a_{\ell m}$ values
generated from the power spectrum in the FITS file. The first index here runs
form 1:1 for temperature only, and 1:3 for polarisation. In the latter case,
1=T, 2=E, 3=B. \\
%
%--------------------------------
\end{tabular} 
\begin{tabular}{p{0.37\hsize} p{0.05\hsize} p{0.1\hsize} p{0.38\hsize}} \hline  
\textbf{name~\&~dimensionality} & \textbf{kind} & \textbf{in/out} & \textbf{description} \\ \hline
                   &   &   &                           \\ %%% for presentation
%--------------------------------
header\mytarget{sub:create_alm:header}(LEN=80),dimension(60) & CHR & OUT & part of header  which
will be included in the FITS-file containing the
map  synthesised from the $a_{\ell m}$  which create\_alm generates. \\
%
\optional{windowfile\mytarget{sub:create_alm:windowfile}}(LEN=\filenamelen) & CHR & IN & full filename specification
of the FITS file with the pixel window function (defined for $\ell\le4 \nside$) \\
%
\optional{units\mytarget{sub:create_alm:units}}(LEN=80),dimension(1:) & CHR & OUT & physical units of the created
$a_{\ell m}$ (square-root of the input power spectrum units). \\
\optional{beam\_file\mytarget{sub:create_alm:beam_file}}(LEN=\filenamelen) & CHR & IN & name of the file containing
the (non necessarily gaussian) window function $B_\ell$ of a circular beam. If present, it will override
the argument \mylink{sub:create_alm:fwhm_arcmin}{{\tt fwhm\_arcmin}}. \\
\end{tabular}
}
\end{arguments}

\begin{example}
{
use alm\_tools, only: create\_alm \\
use rngmod, only: rand\_init, planck\_rng \\
type(planck\_rng) :: rng\_handle \\
\\
call rand\_init(rng\_handle, -1) \\
call create\_alm(64, 128, 128, 1, 'cl.fits', rng\_handle, 5.0, alm\_TGC, \&\\
 \&  header, 'data/pixel\_window\_n0064.fits')  \\
}
{
Creates scalar and tensor $a_{\ell m}$ from the power spectrum given in the file
`cl.fits'. The map to be created from these $a_{\ell m}$ is assumed to have
$N_{side}=64$. $C_l$s from the power spectrum are used up to an $\ell$ value of
128. 
Corresponding $a_{\ell m}$ values up to l=128 and m=128 are created as gaussian distributed
complex numbers. Their are drawn from a sequence of pseudo-random numbers
initiated with a seed of -1. 
The produced $a_{\ell m}$ are convolved with a gaussian beam of FWHM 5 arcminutes
and a pixel window read from 'data/pixel\_window\_n0064.fits'. It is assumed that after the return
from this routine, a map is generated from the created
$a_{\ell m}$. For this purpose, {\tt header} is updated with FITS format information
describing the origin and history of these $a_{\ell m}$.
}
\end{example}

\begin{modules}
  \begin{sulist}{} %%%% NOTE the ``extra'' brace here %%%%
  \item[\textbf{alm\_tools}] \underline{module}, containing:
	\item[pow2alm\_units] routine to convert from power spectrum units to
  $a_{\ell m}$ units
	\item[\htmlref{generate\_beam}{sub:generate_beam}] routine to generate beam window function
	\item[\htmlref{pixel\_window}{sub:pixel_window}] routine to read in pixel window function
  \item[\textbf{utilities}] \underline{module}, containing:
        \item[die\_alloc] routine that prints an error message if there is not enough space for allocation of variables.
  \item[\textbf{fitstools}] \underline{module}, containing:
        \item[\htmlref{fits2cl}{sub:fits2cl}] routine to read a FITS
  file containing a power spectrum.
        \item[\htmlref{read\_dbintab}{sub:read_dbintab}] routine to read a FITS-binary file containing the pixel window functions.
  \item[\textbf{head\_fits}] \underline{module}, containing:
       \item[\htmlref{add\_card}{sub:add_card}] routine to add a keyword to a FITS header.
       \item[\htmlref{get\_card}{sub:get_card}] routine to read a keyword value from
  FITS header.
       \item[\htmlref{merge\_headers}{sub:merge_headers}] routine to merge two FITS headers.
%%   \item[\textbf{ran\_tools}] module, containing:
%%   \item[randgauss\_boxmuller] function which returns a gaussian distributed random
%%   number. Please refer to the ``Comment on Random Number Generator''
%%   in the Fortran90 facilities guidelines.
  \item[\textbf{rngmod}] \underline{module}, containing:
       \item[\htmlref{rand\_gauss}{sub:rand_gauss}] function which returns a gaussian distributed random
  number. 
  \end{sulist}
\end{modules}

\begin{related}
  \begin{sulist}{} %%%% NOTE the ``extra'' brace here %%%%
  \item[\htmlref{rand\_init}{sub:rand_init}] subroutine to initiate a random number sequence. 
  \item[synfast] executable using \thedocid\ to synthesize CMB maps from a given
  power spectrum.
  \item[\htmlref{alm2map}{sub:alm2map}] Routine to transform a set of $a_{\ell m}$ created by \thedocid\ to a \healpix map.
  \item[\htmlref{alms2fits}{sub:alms2fits}, \htmlref{dump\_alms}{sub:dump_alms}]
  Routines to save a set of $a_{\ell m}$ in a FITS file.  
  \end{sulist}
\end{related}

\rule{\hsize}{2mm}

\newpage


\sloppy

%%%\title{\healpix Fortran Subroutines Overview}
\docid{del\_card} \section[del\_card]{ }
\label{sub:del_card}
\docrv{Version 2.0}
\author{Eric Hivon}
\abstract{This document describes the \healpix Fortran90 subroutine DEL\_CARD.}

\begin{facility}
{This routine removes one or several keywords from a FITS header.}
{\modHeadFits}
\end{facility}

\begin{f90format}
{\mylink{sub:del_card:header}{header}%
, \mylink{sub:del_card:kwds}{kwds}%
}
\end{f90format}

\begin{arguments}
{
\begin{tabular}{p{0.4\hsize} p{0.05\hsize} p{0.1\hsize} p{0.35\hsize}} \hline  
\textbf{name~\&~dimensionality} & \textbf{kind} & \textbf{in/out} & \textbf{description} \\ \hline
                   &   &   &                           \\ %%% for presentation
header\mytarget{sub:del_card:header}(LEN=80)(1:nlheader) & CHR & INOUT & The header to remove the keyword(s)
                   from. The routines finds out the header size.\\
kwds\mytarget{sub:del_card:kwds}(LEN=20)(1:nkws) & CHR & IN & list of FITS keywords to
                   remove. The routine accepts either a vector a keywords or a
                   single one in a scalar variable\\
kwds(LEN=20)  & CHR & IN & the one FITS keyword to
                   remove.\\
\end{tabular}
}
\end{arguments}

\begin{examples}{1}
{
call del\_card(header,(/ 'NSIDE   ','COORD   ','ORDERING' /) ) \\
}
{
Removes the keywords `NSIDE', 'COORD' and 'ORDERING' from Header
}
\end{examples}

\begin{examples}{2}
{
call del\_card(header, 'ORDERING' ) \\
}
{
Removes the keyword 'ORDERING' from Header
}
\end{examples}

\begin{modules}
  \begin{sulist}{} %%%% NOTE the ``extra'' brace here %%%%
  \item[write\_hl] more general routine for adding a keyword to a header.
  \item[\textbf{cfitsio}] library for FITS file handling.		
  \end{sulist}
\end{modules}

\begin{related}
  \begin{sulist}{} %%%% NOTE the ``extra'' brace here %%%%
  \item[\htmlref{add\_card}{sub:add_card}] general purpose routine to write any keywords into a FITS
  file header
  \item[\htmlref{get\_card}{sub:get_card}] general purpose routine to read any keywords from a header in a FITS file.
  \item[\htmlref{read\_par}{sub:read_par}, \htmlref{number\_of\_alms}{sub:number_of_alms}] routines to read specific keywords from a
  header in a FITS file.
  \item[\htmlref{getsize\_fits}{sub:getsize_fits}] function returning the size of the data set in a fits
  file and reading some other useful FITS keywords
  \item[\htmlref{merge\_headers}{sub:merge_headers}] routine to merge two FITS headers
  \end{sulist}
\end{related}

\rule{\hsize}{2mm}

\newpage

\sloppy
\docid{dist2holes\_nest}\section[dist2holes\_nest]{ }
\label{sub:dist2holes_nest}
\docrv{Version 1.0}
\author{Eric Hivon}
\abstract{This document describes the \healpix Fortran90 subroutine DIST2HOLES\_NEST.}

\begin{facility}
{For a input binary mask in NESTED ordering, \thedocid\ returns the angular distance (in
radians) from each {\em valid} (1-valued) pixel to the closest {\em invalid} (0-valued)
pixel. Distances are measured between pixel centers.}
{\modMaskTools}
\end{facility}

\begin{f90format}
{\mylink{sub:dist2holes_nest:nside}{nside}%
, \mylink{sub:dist2holes_nest:mask}{mask}%
, \mylink{sub:dist2holes_nest:distance}{distance}%
}
\end{f90format}

\begin{arguments}
{
\begin{tabular}{p{0.35\hsize} p{0.05\hsize} p{0.1\hsize} p{0.40\hsize}} \hline  
\textbf{name~\&~dimensionality} & \textbf{kind} & \textbf{in/out} & \textbf{description} \\ \hline
                   &   &   &                           \\ %%% for presentation
nside\mytarget{sub:dist2holes_nest:nside} & I4B & IN & the $N_{side}$ value of the input mask. \\
mask\mytarget{sub:dist2holes_nest:mask}(0:Npix-1) & I4B & IN & Input NESTED-ordered mask. Npix = 12*nside*nside\\
distance\mytarget{sub:dist2holes_nest:distance}(0:Npix-1) & DP & OUT & Output NESTED-ordered angular-distance map
\end{tabular}
}
\end{arguments}

\begin{example}
{
use healpix\_types \\
use healpix\_modules \\
% use pix\_tools, only : nside2npix \\
% use alm\_tools, only : dist2holes\_nest \\
% integer(I4B) :: nside, lmax, mmax, npix, spin\\
% real(SP), dimension(:,:), allocatable :: map \\
% complex(SPC), dimension(:,:,:), allocatable :: alm \\
% \ldots \\
% nside=256 ; lmax=512 ; mmax=lmax ; spin=4\\
% npix=nside2npix(nside)\\
% allocate(alm(1:2,0:lmax,0:mmax))\\
% allocate(map(0:npix-1,1:2))\\
\ldots \\
call dist2holes\_nest(nside, mask, distance)  \\
}
{???
}
\end{example}

\begin{modules}
  \begin{sulist}{} %%%% NOTE the ``extra'' brace here %%%%
  \item[\textbf{mask\_tools}] mask processing module (see related routines below)
  \end{sulist}
\end{modules}

\begin{related}
  \begin{sulist}{} %%%% NOTE the ``extra'' brace here %%%%
	\maskToolsRelated
  \end{sulist}
\end{related}

\rule{\hsize}{2mm}

\newpage


\sloppy


%%%\title{\healpix Fortran Subroutines Overview}
\docid{dump\_alms*} \section[dump\_alms*]{ }
\label{sub:dump_alms}
\docrv{Version 2.1}
\author{Frode K.~Hansen}
\abstract{This document describes the \healpix Fortran90 subroutine DUMP\_ALMS.}

\begin{facility}
{This routine stores  $a_{\ell m}$  values in a binary FITS file. The FITS file created will contain one integer column with $index=\ell^2+\ell+m+1$ and 2 single precision columns with real/imaginary  $a_{\ell m}$  values. One can store temperature $a_{\ell m}$ or polarisation, $a^E_{\ell m}$ or $a^B_{\ell m}$. If temperature is specified, a FITS file is created. If polarisation is specified, an old FITS file is opened and extra extensions is created.}
{\modFitstools}
\end{facility}

\begin{f90format}
{\mylink{sub:dump_alms:filename}{filename}%
, \mylink{sub:dump_alms:alms}{alms}%
, \mylink{sub:dump_alms:nlmax}{nlmax}%
, \mylink{sub:dump_alms:header}{header}%
, \mylink{sub:dump_alms:nlheader}{nlheader}%
, \mylink{sub:dump_alms:extno}{extno}%
}
\end{f90format}

\begin{arguments}
{
\begin{tabular}{p{0.4\hsize} p{0.05\hsize} p{0.1\hsize} p{0.35\hsize}} \hline  
\textbf{name~\&~dimensionality} & \textbf{kind} & \textbf{in/out} & \textbf{description} \\ \hline
                   &   &   &                           \\ %%% for presentation
filename\mytarget{sub:dump_alms:filename}(LEN=\filenamelen) & CHR & IN & filename for the FITS-file to store the $a_{\ell m}$ in. \\
nlmax\mytarget{sub:dump_alms:nlmax} & I4B & IN & maximum $\ell$ value to store. \\
alms\mytarget{sub:dump_alms:alms}(0:nlmax,0:nlmax) & SPC/ DPC & IN & array with $a_{\ell m}$, in the format used
by eg. \htmlref{map2alm}{sub:map2alm}, so {\tt alms(l,m)} corresponds to  $a_{\ell m}$  \\
extno\mytarget{sub:dump_alms:extno} & I4B & IN & extension number. If 0 is specified, a FITS file is created and $a_{\ell m}$ is stored in the first FITS extension as temperature $a_{\ell m}$. If 1 or 2 is specified, an already existing file is opened and a 2nd or 3rd extension is created, treating $a_{\ell m}$ as $a_{\ell m}^E$ or $a_{\ell m}^B$. \\
nlheader\mytarget{sub:dump_alms:nlheader} & I4B & IN & number of header lines to write to the file. \\
header\mytarget{sub:dump_alms:header}(LEN=80) (1:nlheader) & CHR & IN & the header to the FITS-file. \\ 
\end{tabular}
}
\end{arguments}

\begin{example}
{
call dump\_alms ('alms.fits', alms, 64, header, 100, 1)  \\
}
{
Opens an already existing FITS file which contains temperature $a_{\ell m}$. An extra extension is added to the file where the $a_{\ell m}$ array are written in a three-column format as described above. 100 header lines are written to the file from the array header(1:80). 
}
\end{example}

\begin{modules}
  \begin{sulist}{} %%%% NOTE the ``extra'' brace here %%%%
  \item[\textbf{fitstools}] module, containing:
  \item[printerror] routine for printing FITS error messages.
  \item[\textbf{cfitsio}] library for FITS file handling.		
  \end{sulist}
\end{modules}

\begin{related}
  \begin{sulist}{} %%%% NOTE the ``extra'' brace here %%%%
  \item[\htmlref{fits2alms}{sub:fits2alms}, \htmlref{read\_conbintab}{sub:read_conbintab}] routines to read $a_{\ell m}$ from a FITS-file 
  \item[\htmlref{alms2fits}{sub:alms2fits}] has the same function as \thedocid\ but is more general.
  \end{sulist}
\end{related}

\rule{\hsize}{2mm}

\newpage

\sloppy
\docid{fill\_holes\_nest}\section[fill\_holes\_nest]{ }
\label{sub:fill_holes_nest}
\docrv{Version 1.0}
\author{Eric Hivon}
\abstract{This document describes the \healpix Fortran90 subroutine FILL\_HOLES\_NEST.}

\begin{facility}
{For a input binary mask in NESTED ordering, \thedocid\ flip to 1 all pixels located
in invalid regions with fewer pixels than the threshold provided.\\
Two pixels are adjacent (and belong to the same region or hole) if they have at
least one point in common.
}
{\modMaskTools}
\end{facility}

\begin{f90format}
{\mylink{sub:fill_holes_nest:nside}{nside}%
, \mylink{sub:fill_holes_nest:new_min_size}{new\_min\_size}%
, \mylink{sub:fill_holes_nest:mask_in}{mask\_in}%
, \mylink{sub:fill_holes_nest:mask_out}{mask\_out}%
}
\end{f90format}

\begin{arguments}
{
\begin{tabular}{p{0.35\hsize} p{0.05\hsize} p{0.1\hsize} p{0.40\hsize}} \hline  
\textbf{name~\&~dimensionality} & \textbf{kind} & \textbf{in/out} & \textbf{description} \\ \hline
                   &   &   &                           \\ %%% for presentation
nside\mytarget{sub:fill_holes_nest:nside} & I4B & IN & The $N_{side}$ value of the input mask. \\
new\_min\_size\mytarget{sub:fill_holes_nest:new_min_size} & I4B & IN & Minimal size of hole (in pixels) on output\\
mask\_in\mytarget{sub:fill_holes_nest:mask_in}(0:Npix-1) & I4B & IN & Input NESTED-ordered mask. Npix = 12*nside*nside\\
mask\_out\mytarget{sub:fill_holes_nest:mask_out}(0:Npix-1) &I4B & OUT & Output NESTED-ordered mask. Can be the same
array as mask\_in.
\end{tabular}
}
\end{arguments}

\begin{example}
{
use healpix\_types \\
use healpix\_modules \\
% use pix\_tools, only : nside2npix \\
% use alm\_tools, only : fill\_holes\_nest \\
% integer(I4B) :: nside, lmax, mmax, npix, spin\\
% real(SP), dimension(:,:), allocatable :: map \\
% complex(SPC), dimension(:,:,:), allocatable :: alm \\
% \ldots \\
% nside=256 ; lmax=512 ; mmax=lmax ; spin=4\\
% npix=nside2npix(nside)\\
% allocate(alm(1:2,0:lmax,0:mmax))\\
% allocate(map(0:npix-1,1:2))\\
\ldots \\
call \thedocid(nside, new\_min\_size, mask\_in, mask\_in)  \\
}
{???
}
\end{example}

\begin{modules}
  \begin{sulist}{} %%%% NOTE the ``extra'' brace here %%%%
  \item[\textbf{mask\_tools}] mask processing module (see related routines below)
  \end{sulist}
\end{modules}

\begin{related}
  \begin{sulist}{} %%%% NOTE the ``extra'' brace here %%%%
	\maskToolsRelated
  \end{sulist}
\end{related}

\rule{\hsize}{2mm}

\newpage


\sloppy


%%%\title{\healpix Fortran Subroutines Overview}
\docid{fits2alms*} \section[fits2alms*]{ }
\label{sub:fits2alms}
\docrv{Version 2.0}
\author{Frode K.~Hansen, Eric Hivon}
\abstract{This document describes the \healpix Fortran90 subroutine FITS2ALMS.}

\begin{facility}
{This routine reads  $a_{\ell m}$  values from a binary FITS file. Each FITS file
  extension is supposed to contain one integer column with
  $index=\ell^2+\ell+m+1$  and 2 or 4 single (or double) precision columns 
with real/imaginary  $a_{\ell m}$  values and real/imaginary   standard deviation. 
One can read temperature $a_{\ell m}$ or temperature and polarisation, $a^T_{\ell m}$, $a^E_{\ell m}$ and $a^B_{\ell m}$.}
{\modFitstools}
\end{facility}

\begin{f90format}
{\mylink{sub:fits2alms:filename}{filename}%
, \mylink{sub:fits2alms:nalms}{nalms}%
, \mylink{sub:fits2alms:alms}{alms}%
, \mylink{sub:fits2alms:ncl}{ncl}%
, \mylink{sub:fits2alms:header}{header}%
, \mylink{sub:fits2alms:nlheader}{nlheader}%
, \mylink{sub:fits2alms:next}{next}%
}
\end{f90format}

\begin{arguments}
{
\begin{tabular}{p{0.39\hsize} p{0.05\hsize} p{0.06\hsize} p{0.40\hsize}} \hline  
\textbf{name~\&~dimensionality} & \textbf{kind} & \textbf{in/out} & \textbf{description} \\ \hline
                   &   &   &                           \\ %%% for presentation
filename\mytarget{sub:fits2alms:filename}(LEN=\filenamelen) & CHR & IN & filename of the FITS-file to read the $a_{\ell m}$ from. \\
nalms\mytarget{sub:fits2alms:nalms} & I4B & IN & number of $a_{\ell m}$ to read. \\
ncl\mytarget{sub:fits2alms:ncl} & I4B & IN & number of columns to read in the FITS file. If an standard
               deviation is to be read, this number is 5, otherwise it is 3. \\ 
next\mytarget{sub:fits2alms:next} & I4B & IN & the number of extensions to read. 1 for temperature only, 3
                   for temperature and polarisation. \\ 
\end{tabular}
\begin{tabular}{p{0.4\hsize} p{0.05\hsize} p{0.05\hsize} p{0.40\hsize}} \hline  
alms\mytarget{sub:fits2alms:alms}(1:nalms,1:(ncl+1),1:next) & SP/ DP & OUT & the $a_{\ell m}$ to read from the
          file. alms(i,1,j) and alms(i,2,j) contain the $\ell$ and $m$ values
          for the ith  $a_{\ell m}$  (j=1,2,3 for (T,E,B)). alms(i,3,j) and
          alms(i,4,j) contain the real and imaginary value of the ith
          $a_{\ell m}$. Finally, the   standard deviation for the ith  $a_{\ell m}$  is
          contained in alms(i,5,j) (real) and alms(i,6,j) (imaginary). \\  
nlheader\mytarget{sub:fits2alms:nlheader} & I4B & IN & number of header lines to read from the file. \\
header\mytarget{sub:fits2alms:header}(LEN=80) (1:nlheader, 1:next) & CHR & OUT & the header(s) read from the FITS-file. \\ 
\end{tabular}
}
\end{arguments}

\begin{example}
{
call fits2alms ('alms.fits', 65*66/2, alms, 3, header, 80, 3)  \\
}
{
Reads a FITS file with the $a_{\ell m}^T$, $a_{\ell m}^E$ and $a_{\ell m}^B$ values read into alms(1:65*66/2,1:4,1:3). The last index specifies (T,E,B). The second index gives l, m, real( $a_{\ell m}$ ), imaginary( $a_{\ell m}$ ) for each of the $a_{\ell m}$. The number 65*66/2 is the number of  $a_{\ell m}$  values up to an $\ell$ value of 64. 80 lines is read from the header in each extension and returned in header(1:80,1:3).
}
\end{example}

\begin{modules}
  \begin{sulist}{} %%%% NOTE the ``extra'' brace here %%%%
  \item[read\_alms] routine called by \thedocid\ for each extension.
  \item[\textbf{fitstools}] module, containing:
  \item[printerror] routine for printing FITS error messages.
  \item[\textbf{cfitsio}] library for FITS file handling.		
  \end{sulist}
\end{modules}
\newpage
\begin{related}
  \begin{sulist}{} %%%% NOTE the ``extra'' brace here %%%%
  \item[\htmlref{alms2fits}{sub:alms2fits}, \htmlref{dump\_alms}{sub:dump_alms}] routines to store $a_{\ell m}$ in a FITS-file 
  \item[\htmlref{read\_conbintab}{sub:read_conbintab}] has the same function as
  \thedocid\ but with parameters passed differently.
  \item[\htmlref{number\_of\_alms}{sub:number_of_alms}, \htmlref{getsize\_fits}{sub:getsize_fits}]
  can be used to find out the number of $a_{\ell m}$ available in the file.
  \end{sulist}
\end{related}

\rule{\hsize}{2mm}

\newpage


\sloppy


\title{\healpix Fortran Subroutines Overview}
\docid{fits2cl*} \section[fits2cl*]{ }
\label{sub:fits2cl}
\docrv{Version 2.0}
\author{Eric Hivon, Frode K.~Hansen}
\abstract{This document describes the \healpix Fortran90 subroutine FITS2CL.}

\begin{facility}
{This routine reads a power spectrum or beam window function from a FITS ASCII
or binary table. 
The routine can read temperature coefficients $C_\ell^{TT}$ or both temperature and 
polarisation coefficients $C_\ell^{TT}$, $C_\ell^{EE}$, $C_\ell^{BB}$, $C_\ell^{TE}$ (and  
$C_\ell^{TB}$, 
$C_\ell^{EB}$, 
$C_\ell^{ET}$, 
$C_\ell^{BT}$, 
$C_\ell^{BE}$ when applicable). If the 
keyword PDMTYPE is found in the header, fits2cl assumes the table to be in the 
special format used by {\em Planck} and will ignore the first data column. 
If the input FITS file contains several
extensions or HDUs, the one to be read can be specified thanks to the CFITSIO 
\htmladdnormallink{Extended File Name Syntax}{http://heasarc.gsfc.nasa.gov/docs/software/fitsio/c/c_user/node81.html}, using its number (eg, {\tt file.fits[2]} or {\tt file.fits+2}) or its
{\tt EXTNAME} value (eg. {\tt file.fits[beam\_100x100]}). By default, only the first valid
extension will be read.}
{\modFitstools}
\end{facility}

\begin{f90format}
{\mylink{sub:fits2cl:filename}{filename}%
, \mylink{sub:fits2cl:clin}{clin}%
, \mylink{sub:fits2cl:lmax}{lmax}%
, \mylink{sub:fits2cl:ncl}{ncl}%
, \mylink{sub:fits2cl:header}{header}%
, [\mylink{sub:fits2cl:units}{units}%
]}
\end{f90format}

\begin{arguments}
{
\begin{tabular}{p{0.4\hsize} p{0.05\hsize} p{0.1\hsize} p{0.35\hsize}} \hline  
\textbf{name~\&~dimensionality} & \textbf{kind} & \textbf{in/out} & \textbf{description} \\ \hline
                   &   &   &                           \\ %%% for presentation
filename\mytarget{sub:fits2cl:filename}(LEN=\filenamelen) & CHR & IN & the FITS file containing the power spectrum. \\
lmax\mytarget{sub:fits2cl:lmax} & I4B & IN & Maximum $\ell$ value to be read. \\
ncl\mytarget{sub:fits2cl:ncl} & I4B & IN & 1 for temperature coeffecients only, 4 for polarisation. \\
clin\mytarget{sub:fits2cl:clin}(0:lmax,1:ncl) & SP/ DP & OUT & the power spectrum read from the file.\\
header\mytarget{sub:fits2cl:header}(LEN=80) (1:) & CHR & OUT & the header read from the FITS-file. \\ 
units\mytarget{sub:fits2cl:units}(LEN=80) (1:) & CHR & OUT & the column units read from the FITS-file. \\ 
\end{tabular}
}
\end{arguments}

%\newpage
\begin{example}
{
use \htmlref{healpix\_modules}{sub:healpix_modules}\\
%implicit none \\
real(\mylink{sub:healpix_types:sp}{SP}), allocatable, dimension(:,:) :: cl \\
character(len=80), dimension(1:300) :: header \\
character(len=80), dimension(1:100) :: units \\
integer(\mylink{sub:healpix_types:i4b}{I4B}) :: lmax, ncl, np \\
character(len=\mylink{sub:healpix_types:filenamelen}{filenamelen}) :: fitsfile='cl.fits' \\
%fitsfile = 'cl.fits' \\
np = \htmlref{getsize\_fits}{sub:getsize_fits}(fitsfile, nmaps=ncl, mlpol=lmax) \\
allocate(cl(0:lmax, 1:ncl)) \\
call fits2cl(fitsfile, cl, lmax, ncl, header, units)  \\
%call fits2cl(fitsfile,cl,64,4,header,units)  \\
}
{
Reads a power spectrum from the FITS file {\tt `cl.fits'} and stores the result in 
{\tt \mylink{sub:fits2cl:clin}{cl}(0:lmax,1:ncl)} 
which are the \mylink{sub:fits2cl:ncl}{\tt ncl} $C_\ell$ coefficients up to 
$\ell=$\mylink{sub:fits2cl:lmax}{\tt lmax}. 
The FITS header is returned in \mylink{sub:fits2cl:header}{\tt header}, 
the column units in \mylink{sub:fits2cl:units}{\tt units}.
% Reads a power spectrum from the FITS file `cl.fits' and stores the result in cl(0:64,1:4) which are the $C_l$ coeffecients up to $l=64$ for ($T$, $E$, $B$, $T\times E$). The FITS header is returned in header, the column units in units.
}
\end{example}
\begin{modules}
  \begin{sulist}{} %%%% NOTE the ``extra'' brace here %%%%
  \item[\textbf{fitstools}] module, containing:
  \item[printerror] routine for printing FITS error messages.
  \item[\textbf{cfitsio}] library for FITS file handling.		
  \end{sulist}
\end{modules}

\begin{related}
  \begin{sulist}{} %%%% NOTE the ``extra'' brace here %%%%
  \item[\htmlref{create\_alm}{sub:create_alm}] Routine to create $a_{\ell m}$ values
  from an input power spectrum.
  \item[\htmlref{write\_asctab}{sub:write_asctab}] Routine to create an ascii
  FITS file containing a power spectrum.
  \item[\htmlref{getsize\_fits}{sub:getsize_fits}] Routine to parse FITS file header, and determine the data storage features.
% and, for instance, determine \mylink{sub:fits2cl:lmax}{lmax} and \mylink{sub:fits2cl:ncl}{ncl}
  \item[\htmlref{getnumext\_fits}{sub:getnumext_fits}] Routine to determine number of extensions of a FITS file.
  \end{sulist}
\end{related}

\rule{\hsize}{2mm}

\newpage


\sloppy


\title{\healpix Fortran Subroutines Overview}
\docid{gaussbeam} \section[gaussbeam]{ }
\label{sub:gaussbeam}
\docrv{Version 2.0}
\author{Eric Hivon}
\abstract{This document describes the \healpix Fortran90 subroutine GAUSSBEAM.}

\begin{facility}
{This routine generates the beam window function in multipole space of a
  gaussian beam parametrized by its FWHM. The
polarization beam is also provided assuming a perfectly
co-polarized beam (eg, Challinor et al 2000, 
\htmladdnormallink{astro-ph/0008228}{https://arxiv.org/abs/astro-ph/0008228v2})}
{\modAlmTools}
\end{facility}

\begin{f90format}
{\mylink{sub:gaussbeam:fwhm_arcmin}{fwhm\_arcmin}%
, \mylink{sub:gaussbeam:lmax}{lmax}%
, \mylink{sub:gaussbeam:beam}{beam}%
}
\end{f90format}

\begin{arguments}
{
\begin{tabular}{p{0.35\hsize} p{0.05\hsize} p{0.05\hsize} p{0.45\hsize}} \hline  
\textbf{name~\&~dimensionality} & \textbf{kind} & \textbf{in/out} & \textbf{description} \\ \hline
                   &   &   &                           \\ %%% for presentation
fwhm\_arcmin\mytarget{sub:gaussbeam:fwhm_arcmin} & DP & IN & FWHM of the gaussian beam in arcminutes. \\
lmax\mytarget{sub:gaussbeam:lmax} & I4B & IN & maximum $\ell$ value of the window function.   \\
beam\mytarget{sub:gaussbeam:beam}(0:lmax,1:p) & DP & OUT & beam window function generated. The second index runs form 1:1 for temperature only, and 1:3 for polarisation. In the latter case, 1=T, 2=E, 3=B.\\
\end{tabular}
}
\end{arguments}

\begin{example}
{
call gaussbeam(5.0\_dp, 1024, beam)  \\
}
{
Generates the window function of a gaussian beam of FWHM = 5 arcmin, for $\ell
\leq 1024$.
}
\end{example}

%% \begin{modules}
%%   \begin{sulist}{} %%%% NOTE the ``extra'' brace here %%%%
%%   \end{sulist}
%% \end{modules}

\begin{related}
  \begin{sulist}{} %%%% NOTE the ``extra'' brace here %%%%
  \item[\htmlref{generate\_beam}{sub:generate_beam}] Routine returning a beam
  window function.
  \item[\htmlref{pixel\_window}{sub:pixel_window}] Routine returning a pixel
  window function.
  \end{sulist}
\end{related}

\rule{\hsize}{2mm}

\newpage


\sloppy


\title{\healpix Fortran Subroutines Overview}
\docid{generate\_beam} \section[generate\_beam]{ }
\label{sub:generate_beam}
\docrv{Version 2.0}
\author{Eric Hivon}
\abstract{This document describes the \healpix Fortran90 subroutine GENERATE\_BEAM.}

\begin{facility}
{This routine generates the beam window function in multipole space. It is
  either a gaussian parametrized by its FWHM in arcmin in real space, or it is
  read from an external file.}
{\modAlmTools}
\end{facility}

\begin{f90format}
{\mylink{sub:generate_beam:fwhm_arcmin}{fwhm\_arcmin}%
, \mylink{sub:generate_beam:lmax}{lmax}%
, \mylink{sub:generate_beam:beam}{beam}%
 \optional{[, \mylink{sub:generate_beam:beam_file}{beam\_file}%
]}}
\end{f90format}
\aboutoptional

\begin{arguments}
{
\begin{tabular}{p{0.4\hsize} p{0.05\hsize} p{0.1\hsize} p{0.35\hsize}} \hline  
\textbf{name~\&~dimensionality} & \textbf{kind} & \textbf{in/out} & \textbf{description} \\ \hline
                   &   &   &                           \\ %%% for presentation
fwhm\_arcmin\mytarget{sub:generate_beam:fwhm_arcmin} & DP & IN & fwhm size of the gaussian beam in arcminutes. \\
lmax\mytarget{sub:generate_beam:lmax} & I4B & IN & maximum $\ell$ value of the window function.   \\
beam\mytarget{sub:generate_beam:beam}(0:lmax,1:p) & DP & OUT & beam window function generated. The second index runs form 1:1 for temperature only, and 1:3 for polarisation. In the latter case, 1=T, 2=E, 3=B.\\
\optional{beam\_file}\mytarget{sub:generate_beam:beam_file}(LEN=\filenamelen) (\nobreak{OPTIONAL})& CHR & IN & name of the file containing
the (non necessarily gaussian) window function $B_\ell$ of a circular beam. If present, it will override
the argument {\tt fwhm\_arcmin}. If fewer columns than requested are found in
the file, missing colums will duplicate the existing ones (based on the
assumption that $B_\ell$ is the same in T, E and B). Supports the {\tt fitsio} 'Extended Filename
Syntax' (see \mylink{sub:generate_beam:ex}{examples below}).
\end{tabular}
}
\end{arguments}

% \begin{examples}{1}
% {
% call generate\_beam(5.0\_dp, 1024, beam)  \\
% }
% {%
% Generates the window function of a gaussian beam of FWHM = 5 arcmin, for $\ell
% \leq 1024$.
% }
% \end{examples}

\begin{example}
{
use \htmlref{healpix\_modules}{sub:healpix_modules} \\
real(dp), dimension(0:1024, 1:3) :: gb0, b1, b2, b3\\
call generate\_beam(5.0\_dp, 1024, gb0)\\
call generate\_beam(0\_dp, 1024, b1, beam\_file='file.fits')\\
call generate\_beam(0\_dp, 1024, b2, beam\_file='file.fits[col 1]')\\
call generate\_beam(0\_dp, 1024, b3, beam\_file='file.fits[col 1; 2=0; 3=0]')\\
}
{%
{\tt gb0} will contain the window function of a gaussian beam of FWHM = 5 arcmin, for $\ell
\leq 1024$.
\newline
{\tt b1} will contain the first 3 columns (if available) of {\tt file.fits}. If
the file contains only two columns, then {\tt b1(:,3) = b1(:,2)}, and
if it contains a single column, then {\tt b1(:,3) = b1(:,2) = b1(:,1)}.
\newline
{\tt b2} will be
based on a virtual FITS file containg only the first column of {\tt file.fits},
and we will have {\tt b2(:,3) = b2(:,2) = b2(:,1)}. 
\newline
Finally {\tt b3} will read a
virtual FITS file in which the first column is the same as in {\tt file.fits},
while the columns 2 and 3 are set to 0. Therefore {\tt b3(:,3) = b3(:,2) = 0}.
\mytarget{sub:generate_beam:ex}
}
\end{example}



\begin{modules}
  \begin{sulist}{} %%%% NOTE the ``extra'' brace here %%%%
  \item[\textbf{alm\_tools}] module, containing:
	\item[\htmlref{gaussbeam}{sub:gaussbeam}] routine to generate a gaussian beam
  \end{sulist}
\end{modules}

\begin{related}
  \begin{sulist}{} %%%% NOTE the ``extra'' brace here %%%%
  \item[\htmlref{create\_alm}{sub:create_alm}] Routine to create $a_{\ell m}$
  coefficients using \thedocid.
  \item[\htmlref{alter\_alm}{sub:alter_alm}] Routine to alter $a_{\ell m}$
  coefficients  using \thedocid.
  \item[\htmlref{pixel\_window}{sub:pixel_window}] Routine returning a pixel
  window function.
  \end{sulist}
\end{related}

\rule{\hsize}{2mm}

\newpage


\sloppy

%%%\title{\healpix Fortran Subroutines Overview}
\docid{get\_card} \section[get\_card]{ }
\label{sub:get_card}
\docrv{Version 1.2}
\author{Eric Hivon}
\abstract{This document describes the \healpix Fortran90 subroutine GET\_CARD.}

\begin{facility}
{This routine reads a keyword of any kind from a FITS header. It is a wrapper to
other routines that read keywords of different kinds.}
{\modHeadFits}
\end{facility}

\begin{f90format}
{\mylink{sub:get_card:header}{header}%
, \mylink{sub:get_card:kwd}{kwd}%
, \mylink{sub:get_card:value}{value}%
, \mylink{sub:get_card:comment}{comment}%
}
\end{f90format}

\begin{arguments}
{
\begin{tabular}{p{0.4\hsize} p{0.05\hsize} p{0.1\hsize} p{0.35\hsize}} \hline  
\textbf{name~\&~dimensionality} & \textbf{kind} & \textbf{in/out} & \textbf{description} \\ \hline
                   &   &   &                           \\ %%% for presentation
header\mytarget{sub:get_card:header}(LEN=80) DIMENSION(:) & CHR & IN & The header to read the keyword from. \\
kwd\mytarget{sub:get_card:kwd}(LEN=8) & CHR & IN & the FITS keyword to read (NOT case sensitive). \\
value\mytarget{sub:get_card:value} & any & OUT & the value read for the keyword. 
The type of the fortran variable 'value' (double, real, integer, logical or
                   character) should match the type under which the
                   value is written in the FITS file, except if
                   'value' is a character string, in which case it can read any
                   keyword value, or if 'value' if real or double, in which case
                   it can read any numerical value. Note that long string values
(more than 68 characters in length) are supported.\\
comment\mytarget{sub:get_card:comment}(LEN=*) & CHR & OUT & comment read for the keyword. \\ 
\end{tabular}
}
\end{arguments}

\begin{example}
{
call get\_card(header,'NsIdE',nside,comment)  \\
}
{
if {\tt nside} is defined as an integer, it
will contain on output the value of NSIDE (say 256) found in header
}
\end{example}

\begin{example}
{
call get\_card(header,'ORDERING',ordering,comment)  \\
}
{
if {\tt ordering} is defined as an character string, it
will contain on output the value of ORDERING (say 'RING') found in header
}
\end{example}

\begin{modules}
  \begin{sulist}{} %%%% NOTE the ``extra'' brace here %%%%
  \item[\textbf{cfitsio}] library for FITS file handling.		
  \end{sulist}
\end{modules}

\begin{related}
  \begin{sulist}{} %%%% NOTE the ``extra'' brace here %%%%
  \item[\htmlref{add\_card}{sub:add_card}] general purpose routine to write any keywords into a FITS
  file header
  \item[\htmlref{del\_card}{sub:del_card}] routine to discard a keyword from a FITS header
  \item[\htmlref{read\_par}{sub:read_par}, \htmlref{number\_of\_alms}{sub:number_of_alms}] routines to read specific keywords from a
  header in a FITS file.
  \item[\htmlref{getsize\_fits}{sub:getsize_fits}] function returning the size of the data set in a fits
  file and reading some other useful FITS keywords
  \item[\htmlref{merge\_headers}{sub:merge_headers}] routine to merge two FITS headers
  \end{sulist}
\end{related}

\rule{\hsize}{2mm}

\newpage



\sloppy

\title{\healpix Fortran Subroutines Overview}
\docid{get\_healpix\_main\_dir,~$\ldots$} \section[get\_healpix\_data\_dir, get\_healpix\_main\_dir, get\_healpix\_test\_dir]{ }
\label{sub:get_healpix_xxx_dir}
\docrv{Version 2.0}
\author{Eric Hivon}
\abstract{This document describes the \healpix Fortran90 functions in module paramfile\_io.}

\begin{facility}
{A few functions are available to return the full path to \healpix main directory
and its {\tt data} and {\tt test} subdirectories. This allow those paths to be
controlled by preprocessing macros or environment variables in case of
non-standard installation of the \healpix directory structure.}
{\modParamfileIo}
\end{facility}

%-------------------------------

\rule{\hsize}{0.7mm}
\textsc{\large{\textbf{FUNCTIONS: }}}\hfill\newline
{\tt hmd = get\_healpix\_main\_dir()} \mytarget{sub:get_healpix_xxx_dir:ghmd}

 \begin{tabular}{@{}p{0.3\hsize}@{\hspace{1ex}}
                        p{0.7\hsize}@{}} & returns the full path to the main
			\healpix directory. It will be determined, in this
			order, from the value of the
			preprocessing macros {\tt HEALPIX} and {\tt HEALPIXDIR}
			if they are defined or the
			environment variable {\tt \$HEALPIX} otherwise.\\
     \end{tabular}\\\\

{\tt hdd = get\_healpix\_data\_dir()} \mytarget{sub:get_healpix_xxx_dir:ghdd}

 \begin{tabular}{@{}p{0.3\hsize}@{\hspace{1ex}}
                        p{0.7\hsize}@{}} & returns the full path to
			\healpix {\tt data} subdirectory. It will be determined
			from the preprocessing macro {\tt HEALPIXDATA} or the environment variable {\tt
			\$HEALPIXDATA}. If both fail, it will return the list of directories \{{\tt
			. ../data ./data .. \$HEALPIX \$HEALPIX/data \$HEALPIX/../data
			\$HEALPIX$\backslash$data}\} separated by LineFeed.
\\
     \end{tabular}\\\\


{\tt htd = get\_healpix\_test\_dir()} \mytarget{sub:get_healpix_xxx_dir:ghdd}

 \begin{tabular}{@{}p{0.3\hsize}@{\hspace{1ex}}
                        p{0.7\hsize}@{}} & returns the full path to
			\healpix {\tt test} subdirectory. It will be determined,
			in this order, from the preprocessing macro {\tt HEALPIXTEST}, the environment
			variable {\tt \$HEALPIXTEST} or {\tt \$HEALPIX/test}.\\
     \end{tabular}\\\\

\vskip 3cm
%-------------------------------

% \begin{arguments}
% {
% \begin{tabular}{p{0.30\hsize} p{0.05\hsize} p{0.08\hsize} p{0.47\hsize}} \hline  
% \textbf{name~\&~dimensionality} & \textbf{kind} & \textbf{in/out} & \textbf{description} \\ \hline
%                    &   &   &                           \\ %%% for presentation
% test & LGT & IN & result of a logical test \\
% msg \hfill OPTIONAL & CHR & IN & character string describing nature of error \\
% errorcode \hfill OPTIONAL & I4B & IN & error status given to code interruption \\
% status & I4B & IN & value of the {\tt stat} flag returned by the F90 {\tt allocate} command \\
% code & CHR & IN & name of program or code in which allocation is made \\
% array & CHR & IN & name of array allocated \\
% directory & CHR & IN & directory name (contains a '/')\\
% filename & CHR & IN & file name \\
% \end{tabular}
% }
% \end{arguments}

%-------------------------------

% \begin{example}
% {
% program my\_code \\
% use misc\_utils \\
% real, allocatable, dimension(:) :: vector\\
% integer :: status \\
% real :: a = -1. \\
% \\
% allocate(vector(12345),stat=status) \\
% call assert\_alloc(status, 'my\_code', 'vector') \\
% \\
% call assert\_directory\_present('/home') \\
% \\
% call assert(a > 0., 'a is NEGATIVE !!!') \\
% \\
% end program my\_code\\
% }
% { Will issue a error message and stops the code if {\tt vector} can not be allocated, will stop the
%   code if '/home' is not found, and will stop the code and complain loudly about it 
% because {\tt a} is actually negative.
% }
% \end{example}

%% \begin{modules}
%%   \begin{sulist}{} %%%% NOTE the ``extra'' brace here %%%%
%%  \item[mk\_pix2xy, mk\_xy2pix] routines used in the conversion between pixel values and ``cartesian'' coordinates on the Healpix face.
%%   \end{sulist}
%% \end{modules}

%% \begin{related}
%%   \begin{sulist}{} %%%% NOTE the ``extra'' brace here %%%%
%%   \item[\htmlref{neighbours\_nest}{sub:neighbours_nest}] find neighbouring pixels.
%%   \item[\htmlref{ang2vec}{sub:ang2vec}] convert $(\theta,\phi)$ spherical coordinates into $(x,y,z)$ cartesian coordinates.
%%   \item[\htmlref{vec2ang}{sub:vec2ang}] convert $(x,y,z)$ cartesian coordinates into $(\theta,\phi)$ spherical coordinates.
%%   \end{sulist}
%% \end{related}

\rule{\hsize}{2mm}

\newpage

\sloppy

\title{\healpix Fortran Subroutines Overview}
\docid{getArgument} \section[getArgument]{ }
\label{sub:getargument}
\docrv{Version 1.1}
\author{Eric Hivon}
\abstract{This document describes the \healpix Fortran90 subroutine
getArgument.}

\begin{facility}
{This subroutine emulates the C routine {\tt getarg}, which returns the value of
a given command line argument.}
{\modExtension}
\end{facility}

\begin{f90format}
{\mylink{sub:getargument:index}{index}%
, \mylink{sub:getargument:value}{value}%
}
\end{f90format}

\begin{arguments}
{
\begin{tabular}{p{0.3\hsize} p{0.05\hsize} p{0.1\hsize} p{0.45\hsize}} \hline  
\textbf{name~\&~dimensionality} & \textbf{kind} & \textbf{in/out} & \textbf{description} \\ \hline
                   &   &   &                           \\ %%% for presentation
index\mytarget{sub:getargument:index} & I4B & IN & index of the command line argument (where the first argument
                   has index 1) \\
value\mytarget{sub:getargument:value} & CHR & OUT & value of the argument 
\end{tabular}}
\end{arguments}

% \begin{example}
% {
% use extension \\
% character(len=128) :: healpixdir \\
% call getargument('HEALPIX', healpixdir) \\
% print*,healpixdir
% }
% {
% Will return the value of the {\tt \$HEALPIX} system variable (if it is defined)
% }
% \end{example}

\begin{related}
  \begin{sulist}{} %%%% NOTE the ``extra'' brace here %%%%
  \item[\htmlref{getEnvironment}{sub:getenvironment}] returns value of
  environment variable
%   \item[\htmlref{getArgument}{sub:getargument}] returns list of command line arguments
  \item[\htmlref{nArguments}{sub:narguments}] returns number of command line arguments
  \end{sulist}
\end{related}

\rule{\hsize}{2mm}

\newpage

\sloppy

\title{\healpix Fortran Subroutines Overview}
\docid{getEnvironment} \section[getEnvironment]{ }
\label{sub:getenvironment}
\docrv{Version 1.1}
\author{Eric Hivon}
\abstract{This document describes the \healpix Fortran90 subroutine
getEnvironment.}

\begin{facility}
{This subroutine emulates the C routine {\tt getenv}, which returns the value of
an environment variable.}
{\modExtension}
\end{facility}

\begin{f90format}
{\mylink{sub:getenvironment:name}{name}%
, \mylink{sub:getenvironment:value}{value}%
}
\end{f90format}

\begin{arguments}
{
\begin{tabular}{p{0.3\hsize} p{0.05\hsize} p{0.1\hsize} p{0.45\hsize}} \hline  
\textbf{name~\&~dimensionality} & \textbf{kind} & \textbf{in/out} & \textbf{description} \\ \hline
                   &   &   &                           \\ %%% for presentation
name\mytarget{sub:getenvironment:name} & CHR & IN & name of the environment variable \\
value\mytarget{sub:getenvironment:value} & CHR & OUT & value of the environment variable 
\end{tabular}}
\end{arguments}

\begin{example}
{
use extension \\
character(len=128) :: healpixdir \\
call \thedocid('HEALPIX', healpixdir) \\
print*,healpixdir
}
{
Will return the value of the {\tt \$HEALPIX} system variable (if it is defined)
}
\end{example}

\begin{related}
  \begin{sulist}{} %%%% NOTE the ``extra'' brace here %%%%
  \item[\htmlref{getArgument}{sub:getargument}] returns list of command line arguments
  \item[\htmlref{nArguments}{sub:narguments}] returns number of command line arguments
  \end{sulist}
\end{related}

\rule{\hsize}{2mm}

\newpage


\sloppy


\title{\healpix Fortran Subroutines Overview}
\docid{getdisc\_ring} \section[getdisc\_ring]{ }
\label{sub:getdisc_ring}
\docrv{Version 1.3}
% \author{Frode K.~Hansen}
\author{Eric Hivon}
\abstract{This document describes the \healpix Fortran90 subroutine GETDISC\_RING.}

\begin{facility}
{ %Routine to find the pixel index of all pixels within an angular distance radius from a defined center.
{\bf This routine is obsolete, use \htmlref{query\_disc}{sub:query_disc} instead} }
{\modPixTools}
\end{facility}

%\ mylink: to avoid automatic processing by make_internal_links.sh

% \begin{f90format}
% {nside, vector0, radius, listpix, nlist}
% \end{f90format}

% \begin{arguments}
% {
% \begin{tabular}{p{0.4\hsize} p{0.05\hsize} p{0.1\hsize} p{0.35\hsize}} \hline  
% \textbf{name\&dimensionality} & \textbf{kind} & \textbf{in/out} & \textbf{description} \\ \hline
%                    &   &   &                           \\ %%% for presentation
% nside & I4B & IN & the $N_{side}$ parameter of the map. \\
% vector0(3) & DP & IN & cartesian vector pointing at the central position. \\
% radius & DP & IN & radius from central position in radians. \\
% listpix(0:*) & I4B & OUT & the pixel indexs for all pixels inside $radius$. Make sure that the size of the array is big enough to contain all pixels. \\ 
% nlist & I4B & OUT & The number of pixels listed in $listpix$. \\
% \end{tabular}
% }
% \end{arguments}

% \begin{example}
% {
% call getdisc\_ring(256,(0,0,1),pi/2,listpix,nlist)  \\
% }
% {
% Returns the pixel indexs of all pixels north of the equatorial line in a $N_{side}=256$ map.
% }
% \end{example}
% \newpage
% \begin{modules}
%   \begin{sulist}{} %%%% NOTE the ``extra'' brace here %%%%
%  \item[\htmlref{in\_ring}{sub:in_ring}] routine to find the pixels in a certain slice of a given ring.		
%   \end{sulist}
% \end{modules}

% \begin{related}
%   \begin{sulist}{} %%%% NOTE the ``extra'' brace here %%%%
%   \item[\htmlref{pix2ang}{sub:pix_tools}, \htmlref{ang2pix}{sub:pix_tools}] convert between angle and pixel number.
%   \item[\htmlref{pix2vec}{sub:pix_tools}, \htmlref{vec2pix}{sub:pix_tools}] convert between a cartesian vector and pixel number.
%   \end{sulist}
% \end{related}

\rule{\hsize}{2mm}

\newpage


\sloppy


%%%\title{\healpix Fortran Subroutines Overview}
\docid{getnumext\_fits} \section[getnumext\_fits]{ }
\label{sub:getnumext_fits}
\docrv{Version 2.0}
\author{Eric Hivon}
\abstract{This document describes the \healpix Fortran90 subroutine GETNUMEXT\_FITS.}

\begin{facility}
{This routine returns the number of extensions present in a given FITS file.}
{\modFitstools}
\end{facility}

\begin{f90function}
{\mylink{sub:getnumext_fits:filename}{filename}%
}
\end{f90function}

\begin{arguments}
{
\begin{tabular}{p{0.3\hsize} p{0.05\hsize} p{0.05\hsize} p{0.5\hsize}} \hline  
\textbf{name~\&~dimensionality} & \textbf{kind} & \textbf{in/out} & \textbf{description} \\ \hline
                   &   &   &                           \\ %%% for presentation
var & I4B & OUT & number of extensions in the FITS file (excluding the primary
                   unit). According to the current format, \healpix files have
                   at least one extension. \\
filename\mytarget{sub:getnumext_fits:filename}(LEN=\filenamelen) & CHR & IN & filename of the FITS file. \\
\end{tabular}
}
\end{arguments}

\newpage
\begin{example}
{
next = getnumext\_fits('map.fits')  \\
}
{
Returns in {\tt next} the number of extensions present in the FITS file
'map.fits'.
}
\end{example}
\begin{modules}
  \begin{sulist}{} %%%% NOTE the ``extra'' brace here %%%%
  \item[\textbf{fitstools}] module, containing:
  \item[printerror] routine for printing FITS error messages.
  \item[\textbf{cfitsio}] library for FITS file handling.		
  \end{sulist}
\end{modules}

\begin{related}
  \begin{sulist}{} %%%% NOTE the ``extra'' brace here %%%%
  \item[\htmlref{getsize\_fits}{sub:getsize_fits}] routine returning the number
  of data points in a FITS file, as well as much more information on the file.
  \item[\htmlref{input\_map}{sub:input_map}] routine to read a \healpix FITS file
%  \item[anafast] executable that reads a \healpix map and analyses it. 
  \end{sulist}
\end{related}

\rule{\hsize}{2mm}

\newpage


\sloppy


\title{\healpix Fortran Subroutines Overview}
\docid{getsize\_fits} \section[getsize\_fits]{ }
\label{sub:getsize_fits}
\docrv{Version 2.0}
\author{Eric Hivon \& Frode K.~Hansen}
\abstract{This document describes the \healpix Fortran90 subroutine GETSIZE\_FITS.}

\begin{facility}
{This routine reads the number of maps and/or the pixel ordering of a FITS file containing a \healpix map.}
{\modFitstools}
\end{facility}

\begin{f90function}
{\mylink{sub:getsize_fits:filename}{filename}%
 \optional{[, \mylink{sub:getsize_fits:nmaps}{nmaps}%
, \mylink{sub:getsize_fits:ordering}{ordering}%
, \mylink{sub:getsize_fits:obs_npix}{obs\_npix}%
, \mylink{sub:getsize_fits:nside}{nside}%
, \mylink{sub:getsize_fits:mlpol}{mlpol}%
, \mylink{sub:getsize_fits:type}{type}%
, \mylink{sub:getsize_fits:polarisation}{polarisation}%
,
    \mylink{sub:getsize_fits:fwhm_arcmin}{fwhm\_arcmin}%
, \mylink{sub:getsize_fits:beam_leg}{beam\_leg}%
, \mylink{sub:getsize_fits:coordsys}{coordsys}%
, \mylink{sub:getsize_fits:polcconv}{polcconv}%
, \mylink{sub:getsize_fits:extno}{extno}%
]}}
\end{f90function}
\aboutoptional

\begin{arguments}
{
\begin{tabular}{p{0.28\hsize} p{0.05\hsize} p{0.07\hsize} p{0.5\hsize}} \hline  
\textbf{name~\&~dim.} & \textbf{kind} & \textbf{in/out} & \textbf{description} \\ \hline
                   &   &   &                           \\ %%% for presentation
var & I8B & OUT & number of pixels or time samples in the chosen extension of
                   the FITS file \\
filename\mytarget{sub:getsize_fits:filename}(LEN=*) & CHR & IN & filename of the FITS-file containing \healpix map(s). \\
\end{tabular}
\begin{tabular}{p{0.28\hsize} p{0.05\hsize} p{0.07\hsize} p{0.5\hsize}} \hline  
\textbf{name~\&~dim.} & \textbf{kind} & \textbf{in/out} & \textbf{description} \\ \hline
                   &   &   &                           \\ %%% for presentation
\optional{nmaps\mytarget{sub:getsize_fits:nmaps}} (OPTIONAL) & I4B & OUT & number of maps in the extension. \\
\optional{ordering\mytarget{sub:getsize_fits:ordering}} (OPTIONAL) & I4B & OUT & pixel ordering, 0=unknown, 1=RING, 2=NESTED \\
\optional{obs\_npix\mytarget{sub:getsize_fits:obs_npix}} (OPTIONAL) & I4B & OUT & number of non blanck pixels. It is set to -1 if it can not be determined from header
information alone\\
\optional{nside\mytarget{sub:getsize_fits:nside}} (OPTIONAL)  & I4B & OUT & Healpix resolution parameter Nside. Returns a negative value if not found.  \\
\optional{mlpol\mytarget{sub:getsize_fits:mlpol}} (OPTIONAL)  & I4B & OUT & maximum multipole used to generate the map
                   (for simulated map). Returns a negative value if not found.\\
\optional{type\mytarget{sub:getsize_fits:type}} (OPTIONAL)  & I4B & OUT & 
             \parbox[t]{\hsize}{Healpix/FITS file type\\
             $<$0 : file not found, or not valid\\
             0  : image only fits file, deprecated Healpix format
                   (var = 12 * nside * nside) \\
             1  : ascii table, generally used for C(l) storage \\
             2  : binary table : with implicit pixel indexing (full sky)
                   (var = 12 * nside * nside) \\
             3  : binary table : with explicit pixel indexing (generally cut sky)
                   (var $\le$ 12 * nside * nside) \\
           999  : unable to determine the type }\\
\optional{polarisation\mytarget{sub:getsize_fits:polarisation}} (\nobreak{OPTIONAL})  & I4B & OUT & 
		\parbox[t]{\hsize}{presence of polarisation data in the file\\
             $<$0 : can not find out\\
              0 : no polarisation\\
              1 : contains polarisation (Q,U or G,C)} \\
\optional{fwhm\_arcmin\mytarget{sub:getsize_fits:fwhm_arcmin}} (\nobreak{OPTIONAL}) & DP & OUT & returns the beam FWHM read from FITS header, 
                            translated from Deg (hopefully) to arcmin.
                         Returns a negative value if not found. \\
\optional{beam\_leg\mytarget{sub:getsize_fits:beam_leg}}(LEN=*) (\nobreak{OPTIONAL}) & CHR & OUT & filename of beam or
             filtering window function applied to data
	     (FITS keyword BEAM\_LEG). Returns a empty string if not found. \\
\optional{coordsys\mytarget{sub:getsize_fits:coordsys}}(LEN=20) (\nobreak{OPTIONAL}) & CHR & OUT & string describing the pixelation
                   astrophysical coordinates. 
		'G' = Galactic, 'E' = ecliptic, 'C' = celestial = equatorial. 
		Returns a empty string if not found. \\
\optional{polcconv\mytarget{sub:getsize_fits:polcconv}} (OPTIONAL) & I4B & OUT & polarisation coordinate convention
% (see Healpix primer for details) 0=unknown, 1=COSMO, 2=IAU \\
 (see Healpix primer for details) 0=unknown, 1=COSMO, 2=IAU, 3=neither COSMO nor IAU \\ % 2017-01-09
\optional{extno\mytarget{sub:getsize_fits:extno}} (OPTIONAL)  & I4B & IN & extension number (0 based) for which information
             is provided. Default = 0 (first extension). 
\end{tabular}
}
\end{arguments}

\newpage
\begin{example}
{
npix= getsize\_fits('map.fits', nmaps=nmaps, ordering=ordering, obs\_npix=obs\_npix, nside=nside, mlpol=mlpol, type=type, polarisation=polarisation)  \\
}
{
Returns 1 or 3 in nmaps, dependent on wether 'map.fits' contain only
temperature or both temperature and polarisation maps. The pixel ordering number is found by reading the keyword ORDERING in the FITS file. If this keyword does not exist, 0 is returned.
}
\end{example}
\begin{modules}
  \begin{sulist}{} %%%% NOTE the ``extra'' brace here %%%%
  \item[\textbf{fitstools}] module, containing:
  \item[printerror] routine for printing FITS error messages.
  \item[\textbf{cfitsio}] library for FITS file handling.		
  \end{sulist}
\end{modules}

\begin{related}
  \begin{sulist}{} %%%% NOTE the ``extra'' brace here %%%%
  \item[\htmlref{getnumext\_fits}{sub:getnumext_fits}] routine returning the number of extension in a FITS
  file
  \item[\htmlref{input\_map}{sub:input_map}] routine to read a \healpix FITS file
%  \item[anafast] executable that reads a \healpix map and analyses it. 
  \end{sulist}
\end{related}

\rule{\hsize}{2mm}

\newpage

% -*- LaTeX -*-

\sloppy


%%%\title{\healpix Fortran Subroutines Overview}
\docid{healpix\_modules} \section[healpix\_modules module]{ }
\label{sub:healpix_modules}
\docrv{Version 2.0}
\author{Eric Hivon}
\abstract{This document describes the \healpix Fortran90 module HEALPIX\_MODULES.}

\begin{facility}
{This module is a meta module containing most of the \healpix modules. It currently includes
\begin{itemize}
\setlength{\itemsep}{-5pt}
  \item {\tt{alm\_tools}},
  \item {\tt{bit\_manipulation}},
  \item {\tt{coord\_v\_convert}},
  \item {\tt{extension}},
  \item {\tt{fitstools}},
  \item {\tt{head\_fits}},
  \item {\tt{healpix\_fft}},
  \item \htmlref{{\tt{healpix\_types}}}{sub:healpix_types},
  \item {\tt{long\_intrinsic}},
%  \item m\_indmed
  \item {\tt{mask\_tools}},
  \item {\tt{misc\_utils}},
  \item {\tt{num\_rec}},
  \item {\tt{obsolete}},
  \item {\tt{paramfile\_io}},
  \item {\tt{pix\_tools}},
  \item {\tt{ran\_tools}},
  \item {\tt{rngmod}},
  \item {\tt{statistics}},
  \item {\tt{udgrade\_nr}},
  \item {\tt{utilities}}.
\end{itemize}

Note that {\tt{mpi\_alm\_tools}} is not included since it requires the MPI library for compilation.
}
{\modHealpixModules}
\end{facility}


\begin{example}
{
use healpix\_modules \\
print*,' pi = ',PI \\
print*,' number of pixels in a Nside=64 map:',nside2npix(64)
}
{
Invoking {\tt{healpix\_modules}} gives access to all \healpix routines and parameters.
}
\end{example}


\rule{\hsize}{2mm}

\newpage


\sloppy


\title{\healpix Fortran Subroutines Overview}
\docid{healpix\_types} \section[healpix\_types module]{ }
\label{sub:healpix_types}
\docrv{Version 2.0}
\author{Eric Hivon}
\abstract{This document describes the \healpix Fortran90 module HEALPIX\_TYPES.}

\begin{facility}
{This module defines a set of parameters used by most other
\healpix modules.}
{\modHealpixTypes}
\end{facility}


%---------------------
\newenvironment{mytable}[1]{%
\begin{minipage}[b]{\linewidth}{%
\renewcommand{\thefootnote}{\fnsymbol{footnote}}
\renewcommand{\footnoterule}{}
{#1}
}%
\end{minipage}
}
%---------------------

The parameters defined in \thedocid\ include

\begin{itemize}
\item
'kind' parameters, used when defining the type of a variable,

\begin{mytable}{%
\begin{tabularx}{\linewidth}{lcc X}
name & type & value\footnote{actual value may depend on hardware or compiler} & definition \\
\hline
I1B & integer & 1 & number of bytes in the hardware-supported signed integers covering the range -99 to
99 with the least margin\\
I2B & integer & 2 & same as above for the range -9999 to 9999 (ie, 4 digits)\\
I4B & integer & 4 & same as above for 9 digits \\
I8B & integer & 8 & same as above for 16 digits\footnote{may not be supported by
  some hardware or compiler; on those systems, the user should set the
preprocessing variable {\tt NO64BITS} to 1 during compilation to demote
automatically {\tt I8B} to {\tt I4B}} \\
SP & integer & 4 & number of bytes in the hardware-supported floating-point
numbers covering the range $10^{-30}$ to $10^{30}$ with the least margin
(hereafter single precision)\\
DP & integer & 8 & same as above for the range $10^{-200}$ to $10^{200}$
(double precision)\\
SPC & integer & 4 & number of bytes in real ({\em or} imaginary) part of single precision complex numbers\\
DPC & integer & 8 & same as above for double precision complex numbers\\
LGT & integer & 4 & number of bytes in logical variables \\
\hline
\end{tabularx}
}%
\end{mytable}


\item
largest accessible numbers,

\begin{mytable}{%
\begin{tabularx}{\linewidth}{lcc X}
name & type or kind & value\footnote{actual value may depend on hardware or compiler} & definition \\
\hline
MAX\_I1B & integer & 127 & largest number accessible to integers of kind {\tt I1B}\\
MAX\_I2B & integer & $32767$ &
same as above for {\tt I2B} integers\\
MAX\_I4B & integer & $2^{31}-1 \simeq 2.1\ 10^9$& same as above for {\tt I4B} integers \\
MAX\_I8B & I8B & $2^{63}-1 \simeq 9.2\ 10^{18}$& same as above for {\tt I8B} integers \\
MAX\_SP & SP & $\simeq 3.40\ 10^{38}$ & same as above for {\tt SP} floating-point\\
MAX\_DP & DP & $\simeq 1.80\ 10^{308}$ & same as above for {\tt DP} floating-point\\
\hline
\end{tabularx}
}
\end{mytable}

\item
mathematical definitions, \\
\begin{mytable}{%
\begin{tabularx}{\linewidth}{lcc X}
name & kind & value & definition \\
\hline
QUARTPI & DP & $\pi/4$ & \\
HALFPI & DP & $\pi/2$ & \\
PI & DP & $\pi \simeq 3.14159\ldots$ & \\
TWOPI & DP & $2\pi$ & \\
FOURPI & DP & $4\pi$ & \\
SQRT2 & DP & $\sqrt{2}$ & \\
EULER & DP & $\gamma \simeq 0.577\ldots$ & Euler constant \\
SQ4PI\_INV & DP & $1/\sqrt{4\pi}$ & \\
TWOTHIRD & DP & $2/3$ & \\
DEG2RAD & DP & $\pi/180$ & Degrees to Radians conversion factor\\
RAD2DEG & DP & $180/\pi$ & Radians to Degrees conversion factor\\
\hline
\end{tabularx}
}
\end{mytable}

\item
and \healpix specific definitions, \\
\begin{mytable}{%
\begin{tabularx}{\linewidth}{lcc X}
name & type or kind & value & definition \\
\hline
\mytarget{sub:healpix_types:hpx_sbadval}HPX\_SBADVAL & SP & $-1.6375\ 10^{30}$ & default sentinel value given to missing
pixels in single precision data sets \\
\mytarget{sub:healpix_types:hpx_dbadval}HPX\_DBADVAL & DP & $-1.6375\ 10^{30}$ & same as above for double precision data
sets\\
\mytarget{sub:healpix_types:filenamelen}{FILENAMELEN} & integer & 1024 & default length in character of file names. \\
HEALPIX\_VERSION & character & ``\hpxversion'' & current \healpix package version.\\
%% HPX\_MXL0 & I4B & $40$ & parameter $m_0$ of $(l,m)$ range of spherical harmonics computed. \\
%% HPX\_MXL1 & DP & $1.35$ & parameter $r$ in equation above\\
\hline
\end{tabularx}
}
\end{mytable}

\end{itemize}

\begin{example}
{
use healpix\_types \\
real(kind=DP) :: dx \\
print*,' pi = ',PI
}
{
The value of {\tt PI}, as well as all other \thedocid\ parameters are made known
to the code
}
\end{example}


\rule{\hsize}{2mm}

\newpage


\sloppy


%%%\title{\healpix Fortran Subroutines Overview}
\docid{in\_ring} \section[in\_ring]{ }
\label{sub:in_ring}
\docrv{Version 1.3}
% \author{Frode K.~Hansen}
\author{Eric Hivon}
\abstract{This document describes the \healpix Fortran90 subroutine IN\_RING.}

\begin{facility}
{Routine to find the pixel index of all pixels on a slice of a given
ring. The output indices can be either in the RING or NESTED scheme,
depending on the {\tt nest} keyword.}
{\modPixTools}
\end{facility}

\begin{f90format}
{\mylink{sub:in_ring:nside}{nside}%
, \mylink{sub:in_ring:iz}{iz}%
, \mylink{sub:in_ring:phi0}{phi0}%
, \mylink{sub:in_ring:dphi}{dphi}%
, \mylink{sub:in_ring:listir}{listir}%
, \mylink{sub:in_ring:nir}{nir}%
, \mylink{sub:in_ring:nest}{nest}%
}
\end{f90format}

\begin{arguments}
{
\begin{tabular}{p{0.4\hsize} p{0.05\hsize} p{0.1\hsize} p{0.35\hsize}} \hline  
\textbf{name~\&~dimensionality} & \textbf{kind} & \textbf{in/out} & \textbf{description} \\ \hline
                   &   &   &                           \\ %%% for presentation
nside\mytarget{sub:in_ring:nside} & I4B & IN & the $\nside$ parameter of the map. \\
iz\mytarget{sub:in_ring:iz} & I4B & IN & ring number, counted southwards from the north pole. \\
phi0\mytarget{sub:in_ring:phi0} & DP & IN & central $\phi$ position in the slice. \\
dphi\mytarget{sub:in_ring:dphi} & DP & IN & defines the size of the slice. The slice has length $2\times dphi$ along the ring with center at $phi0$. \\ 
listir\mytarget{sub:in_ring:listir}(0:4*nside-1) & I4B/ I8B & OUT & The pixel indexes in the slice. \\
nir\mytarget{sub:in_ring:nir} & I4B & OUT & the number of pixels in the slice. {\tt nir}$\le 4\nside$\\
nest\mytarget{sub:in_ring:nest}\ \ (OPTIONAL) & I4B & IN &  The pixel indexes are in the NESTED numbering
scheme if {\tt nest}=1, and in RING scheme otherwise. \\
\end{tabular}
}
\end{arguments}

\begin{example}
{
call in\_ring(256, 10, 0, 0.1, listir, nir, nest=1)  \\
}
{
Returns the NESTED pixel index of all pixels within 0.1 radians on each side of $\phi=0$ on the 10th ring.
}
\end{example}

\newpage
\begin{modules}
  \begin{sulist}{} %%%% NOTE the ``extra'' brace here %%%%
 \item[\htmlref{ring2nest}{sub:pix_tools}] conversion from RING scheme pixel index to NESTED scheme pixel index
 \item[next\_in\_line\_nest] returns NESTED index of pixel lying to the East of the
 current pixel and on the same ring
  \end{sulist}
\end{modules}

\begin{related}
  \begin{sulist}{} %%%% NOTE the ``extra'' brace here %%%%
  \item[\htmlref{pix2ang}{sub:pix_tools}, \htmlref{ang2pix}{sub:pix_tools}] convert between angle and pixel number.
  \item[\htmlref{pix2vec}{sub:pix_tools}, \htmlref{vec2pix}{sub:pix_tools}] convert between a cartesian vector and pixel number.
  \item[\htmlref{getdisc\_ring}{sub:getdisc_ring}] find all pixels within a certain radius.
  \end{sulist}
\end{related}

\rule{\hsize}{2mm}

\newpage


\sloppy


\title{\healpix Fortran Subroutines Overview}
\docid{input\_map*} \section[input\_map*]{ }
\label{sub:input_map}
\docrv{Version 1.3}
\author{Eric Hivon \& Frode K.~Hansen}
\abstract{This document describes the \healpix Fortran90 subroutine INPUT\_MAP.}

\begin{facility}
{This routine reads a \healpix map from a FITS file. This can deal with full sky
as well as cut sky maps, but always outputs a full sky map (with possibly many empty pixels).}
{\modFitstools}
\end{facility}

\begin{f90format}
{\mylink{sub:input_map:filename}{filename}%
, \mylink{sub:input_map:map}{map}%
, \mylink{sub:input_map:npixtot}{npixtot}%
, \mylink{sub:input_map:nmaps}{nmaps}%
 \optional{[, \mylink{sub:input_map:fmissval}{fmissval}%
, \mylink{sub:input_map:header}{header}%
, \mylink{sub:input_map:units}{units}%
, \mylink{sub:input_map:extno}{extno}%
, \mylink{sub:input_map:ignore_polcconv}{ignore\_polcconv}%
]}}
\end{f90format}
\aboutoptional

\begin{arguments}
{
\begin{tabular}{p{0.3\hsize} p{0.05\hsize} p{0.05\hsize} p{0.5\hsize}} \hline  
\textbf{name~\&~dimensionality} & \textbf{kind} & \textbf{i/o} & \textbf{description} \\ \hline
                   &   &   &                           \\ %%% for presentation
filename\mytarget{sub:input_map:filename}(len=\filenamelen) & CHR & IN & FITS file to be read from,
                   containing a full sky or cut sky map \\
map\mytarget{sub:input_map:map}(0:npixtot-1,1:nmaps)    & SP/ DP & OUT & full sky map(s) constructed
                   from the data present in the file, missing pixels are filled
                   with fmissval \\
npixtot\mytarget{sub:input_map:npixtot}                    & I4B/ I8B & IN & number of pixels in the full sky map \\
nmaps\mytarget{sub:input_map:nmaps}     & I4B & IN &  number of maps in the file  \\
                   &   &   &                           \\ %%% for presentation
\optional{fmissval\mytarget{sub:input_map:fmissval}}  & SP/ DP & IN &  value to be given to missing pixels,
\default{0}%
%its default value is 0 
\\
\optional{header\mytarget{sub:input_map:header}}(LEN=80)(1:)     & CHR & OUT &   FITS extension header \\
\optional{units\mytarget{sub:input_map:units}}(LEN=20)(1:nmaps)  & CHR & OUT &  maps units \\
\optional{extno\mytarget{sub:input_map:extno}}  & I4B & IN & extension number to read the data from
                   (0 based).\default{0} (the first extension is read) \\
\optional{ignore\_polccconv\mytarget{sub:input_map:ignore_polcconv}}  & LGT & IN & by default 
	(\texttt{ignore\_polcconv=.false.}) the output of this routine depends on the value of the FITS keyword
	\texttt{POLCCONV} found in \mylink{sub:input_map:filename}{filename}, as described in 
	the \htmlref{note on POLCCONV}{intro:polcconv} in \linklatexhtml{The \healpix Primer}{intro.pdf}{intro.htm}. 
	Setting \texttt{ignore\_polcconv=.true.} will force input\_map to ignore that keyword.
\end{tabular}
}
\end{arguments}

\begin{example}
{
use pix\_tools, only: nside2npix \\
use fitstools, only: getsize\_fits, input\_map \\
\ldots \\
npixtot = getsize\_fits('map.fits',nmaps=nmaps, nside=nside) \\
npix = nside2npix(nside) \\
allocate(map(0:npix-1,1:nmaps)) \\
call input\_map('map.fits', map, npix, nmaps, fmissval=0.)  \\
}
{
Reads into {\tt map} the content of the FITS file 'map.fits'.
If there are
missing pixels in the input file (ie, having value {\tt NaN} (Not of Number),
$\pm$ {\tt Infinity} or matching the FITS keyword {\tt BAD\_DATA}) they will
take on output the value provided in optional {\tt fmissval} (here 0, which also
is its default value).
}
\end{example}
%%\newpage
\begin{modules}
  \begin{sulist}{} %%%% NOTE the ``extra'' brace here %%%%
  \item[\textbf{fitstools}] module, containing:
  \item[printerror] routine for printing FITS error messages.
  \item[\htmlref{read\_bintab}{sub:read_bintab}] routine to read a binary table
  from a FITS file
  \item[\htmlref{read\_fits\_cut4}{sub:read_fits_cut4}] routine to read cut sky
  map from a FITS file
  \item[\htmlref{read\_fits\_partial}{sub:read_fits_partial}] routine to read a partial sky
  map from a FITS file
  \item[\textbf{cfitsio}] library for FITS file handling.
  \end{sulist}
\end{modules}

\begin{related}
  \begin{sulist}{} %%%% NOTE the ``extra'' brace here %%%%
  \item[anafast] executable that reads a \healpix map and analyses it. 
  \item[synfast] executable that generate full sky \healpix maps
  \item[\htmlref{getsize\_fits}{sub:getsize_fits}] subroutine to know the size of a FITS file.
  \item[\htmlref{output\_map}{sub:output_map}] subroutine to write a FITS file
  from a \healpix map
  \item[\htmlref{write\_bintabh}{sub:write_bintabh}] subroutine to write a large
  array into a FITS file piece by piece
  \item[\htmlref{input\_tod*}{sub:input_tod}] subroutine to read an arbitrary subsection of
  a large binary table
  \end{sulist}
\end{related}

\rule{\hsize}{2mm}

\newpage


\sloppy


%%%\title{\healpix Fortran Subroutines Overview}
\docid{input\_tod*} \section[input\_tod*]{ }
\label{sub:input_tod}
\docrv{Version 2.0}
\author{Eric Hivon \& Frode K.~Hansen}
\abstract{This document describes the \healpix Fortran90 subroutine INPUT\_TOD*.}

\begin{facility}
{This routine reads a large binary table (for instance a Time Ordered Data
 set) from a FITS file. The user can choose to read only a section of the table,
 starting from an arbitrary position. 
The data can be read into a single or double precision array.}
{\modFitstools}
\end{facility}

\begin{f90format}
{\mylink{sub:input_tod:filename}{filename}%
, \mylink{sub:input_tod:tod}{tod}%
, \mylink{sub:input_tod:npix}{npix}%
, \mylink{sub:input_tod:ntods}{ntods}%
 \optional{[, \mylink{sub:input_tod:header}{header}%
, \mylink{sub:input_tod:firstpix}{firstpix}%
, \mylink{sub:input_tod:fmissval}{fmissval}%
]}}
\end{f90format}
\aboutoptional

\begin{arguments}
{
\begin{tabular}{p{0.3\hsize} p{0.05\hsize} p{0.05\hsize} p{0.5\hsize}} \hline  
\textbf{name~\&~dimensionality} & \textbf{kind} & \textbf{in/out} & \textbf{description} \\ \hline
                   &   &   &                           \\ %%% for presentation
filename\mytarget{sub:input_tod:filename}(LEN=\filenamelen) & CHR & IN & FITS file to be read from \\
tod\mytarget{sub:input_tod:tod}(0:npix-1,1:ntods)    & SP/ DP & OUT & array constructed
                   from the data present in the file (from the sample {\tt
                   firstpix} to {\tt firstpix + npix} - 1. Missing pixels or time
                   samples are filled with {\tt fmissval}. \\
npix\mytarget{sub:input_tod:npix}      & I8B & IN & number of pixels or samples to be read. See Note below. \\
ntods\mytarget{sub:input_tod:ntods}     & I4B & IN &  number of columns to read  \\
\optional{header\mytarget{sub:input_tod:header}}(LEN=80)(1:)    & CHR & OUT &   FITS extension header \\
\optional{firstpix\mytarget{sub:input_tod:firstpix}}  & I8B & IN & first pixel (or time sample) to read from
                   (0 based). \default 0. See Note below. \\
\optional{fmissval\mytarget{sub:input_tod:fmissval}}  & SP/ DP & IN &  value to be given to missing pixels, its default
                   value is 0. Should be of the same type as {\tt tod}.
\end{tabular}
{\bf Note :} Indices and number of data elements larger than
                   $2^{31}$ are only accessible in FITS files on computers with 64 bit
                   enabled compilers and with some specific compilation options of
                   cfitsio (see cfitsio documentation).
}
\end{arguments}

\begin{modules}
  \begin{sulist}{} %%%% NOTE the ``extra'' brace here %%%%
  \item[\textbf{fitstools}] module, containing:
  \item[printerror] routine for printing FITS error messages.
  \item[\textbf{cfitsio}] library for FITS file handling.		
  \end{sulist}
\end{modules}

\begin{related}
  \begin{sulist}{} %%%% NOTE the ``extra'' brace here %%%%
  \item[anafast] executable that reads a \healpix map and analyses it. 
  \item[synfast] executable that generate full sky \healpix maps
  \item[\htmlref{getsize\_fits}{sub:getsize_fits}] subroutine to know the size of a FITS file.
  \item[\htmlref{write\_bintabh}{sub:write_bintabh}] subroutine to write large arrays into FITS files
  \item[\htmlref{output\_map}{sub:output_map}] subroutine to write a FITS file from a \healpix map
  \item[\htmlref{input\_map}{sub:input_map}] subroutine to read a \healpix map
  (either full sky of cut sky) from a FITS file
  \end{sulist}
\end{related}

\rule{\hsize}{2mm}

\newpage



\sloppy

%%%\title{\healpix Fortran Subroutines Overview}
\docid{long\_count,~long\_size} \section[long\_count,~long\_size]{ }
\label{sub:long_intrinsic}
\docrv{Version 2.0}
\author{Eric Hivon}
\abstract{This document describes the \healpix Fortran90 functions in module LONG\_INTRINSIC.}

\begin{facility}
{The Fortran90 module {\tt long\_intrinsic} contains a subset of  
intrinsic functions (currently {\tt count} and {\tt size}) compiled so that they return \htmlref{I8B}{sub:healpix_types} variables
instead of the default integer (generally \htmlref{I4B}{sub:healpix_types}),
therefore allowing the handling of arrays with more than $2^{31}-1$
elements.}
{\modLongIntrinsic}
\end{facility}

%-------------------------------

\rule{\hsize}{0.7mm}
\textsc{\large{\textbf{FUNCTIONS: }}}\hfill\newline
{\tt \mylink{sub:long_intrinsic:cnt}{cnt} = long\_count(\mylink{sub:long_intrinsic:mask1}{mask1})} 

 \begin{tabular}{@{}p{0.3\hsize}@{\hspace{1ex}}p{0.7\hsize}@{}}
                         & returns the I8B integer value that is
the number of elements of the logical array {\tt mask1} that have the value {\tt
true}. 
     \end{tabular}\\

{\tt \mylink{sub:long_intrinsic:sz}{sz} = long\_size(\mylink{sub:long_intrinsic:array1}{array1} 
[,\mylink{sub:long_intrinsic:dim}{dim}])} \\
{\tt \mylink{sub:long_intrinsic:sz}{sz} = long\_size(\mylink{sub:long_intrinsic:array2}{array2} 
[,\mylink{sub:long_intrinsic:dim}{dim}])} 

 \begin{tabular}{@{}p{0.3\hsize}@{\hspace{1ex}}p{0.7\hsize}@{}}
                         & returns the I8B integer value that is
the size of the 1D array {\tt array1} or 2D array {\tt array2} or their
extent along the dimension {\tt dim} if the scalar integer {\tt dim} is provided.
     \end{tabular}\\

%\vskip 0.1cm
%-------------------------------

\begin{arguments}
{
\begin{tabular}{p{0.30\hsize} p{0.05\hsize} p{0.08\hsize} p{0.47\hsize}} \hline  
\textbf{name~\&~dimensionality} & \textbf{kind} & \textbf{in/out} & \textbf{description} \\ \hline
                   &   &   &                           \\ %%% for presentation
cnt \mytarget{sub:long_intrinsic:cnt} & I8B & OUT & number of elements with value {\tt true} \\
sz \mytarget{sub:long_intrinsic:sz} & I8B & OUT & size or extent of array \\
mask1(:) \mytarget{sub:long_intrinsic:mask1} & LGT & IN & 1D logical array \\
array1(:) \mytarget{sub:long_intrinsic:array1}& I4B/ I8B/ SP/ DP & IN & 1D integer or real array \\
array2(:,:) \mytarget{sub:long_intrinsic:array2} & I4B/ I8B/ SP/ DP & IN & 2D integer or real array \\
dim \mytarget{sub:long_intrinsic:dim}\ \ \ (OPTIONAL) & I4B & IN & dimension (starting at 1) along which the array
extent is measured.
\end{tabular}
}
\end{arguments}

%-------------------------------

\begin{example}
{
 use \htmlref{healpix\_modules}{sub:healpix_modules} \\
 real(SP), dimension(:,:), allocatable :: bigarray \\
 allocate(bigarray(2\_i8b**31+5, 3)) \\
 print*,       size(bigarray),       size(bigarray,1),       size(bigarray,dim=2) \\
 print*, long\_size(bigarray), long\_size(bigarray,1), long\_size(bigarray,dim=2) \\
 deallocate(bigarray)
}
{Will return (with default compilation options)
\begin{tabbing}
     -2147483633 \= -2147483643  \= 3 \\
     6442450959  \> 2147483653   \> 3
\end{tabbing}
meaning that {\tt long\_size} handles correctly this large array while by default
{\tt size} does not.}
\end{example}


%% \begin{modules}
%%   \begin{sulist}{} %%%% NOTE the ``extra'' brace here %%%%
%%  \item[mk\_pix2xy, mk\_xy2pix] routines used in the conversion between pixel values and ``cartesian'' coordinates on the Healpix face.
%%   \end{sulist}
%% \end{modules}

%% \begin{related}
%%   \begin{sulist}{} %%%% NOTE the ``extra'' brace here %%%%
%%   \item[\htmlref{neighbours\_nest}{sub:neighbours_nest}] find neighbouring pixels.
%%   \item[\htmlref{ang2vec}{sub:ang2vec}] convert $(\theta,\phi)$ spherical coordinates into $(x,y,z)$ cartesian coordinates.
%%   \item[\htmlref{vec2ang}{sub:vec2ang}] convert $(x,y,z)$ cartesian coordinates into $(\theta,\phi)$ spherical coordinates.
%%   \end{sulist}
%% \end{related}

\rule{\hsize}{2mm}

\newpage
% special format, hand processed

\sloppy


\title{\healpix Fortran Subroutines Overview}
\docid{map2alm*} \section[map2alm*]{ }
\label{sub:map2alm}
\docrv{Version 2.1}
\author{Frode K.~Hansen, Eric Hivon}
\abstract{This document describes the \healpix Fortran90 subroutine MAP2ALM*.}

\begin{facility}
{This routine is a wrapper to 5 internal routines:map2alm\_sc,
map2alm\_sc\_pre, map2alm\_pol, map2alm\_pol\_pre1,
map2alm\_pol\_pre2. These routines analyse a \healpix {\em RING ordered} map and return
$a_{\ell m}^T$ (and if specified $a_{\ell m}^E$ and $a_{\ell m}^B$) values up to
the desired order in $\ell$ (maximum 3*$\nside$). The different
routines are called depending on what parameters are passed. Some
routines analyse with or without precomputed harmonics and some with
or without polarisation. }
{\modAlmTools}
\end{facility}

\begin{f90format}
{\mylink{sub:map2alm:nsmax}{nsmax}%
, \mylink{sub:map2alm:nlmax}{nlmax}%
, \mylink{sub:map2alm:nmmax}{nmmax}%
, \mylink{sub:map2alm:map_TQU}{map\_TQU}%
, \mylink{sub:map2alm:alm_TGC}{alm\_TGC}%
, \mylink{sub:map2alm:zbounds}{zbounds}%
, \mylink{sub:map2alm:w8ring_TQU}{w8ring\_TQU}%
 [, \mylink{sub:map2alm:plm}{plm}%
]}
\end{f90format}

\begin{arguments}
{
\begin{tabular}{p{0.4\hsize} p{0.05\hsize} p{0.05\hsize} p{0.40\hsize}} \hline  
\textbf{name~\&~dimensionality} & \textbf{kind} & \textbf{in/out} & \textbf{description} \\ \hline
                   &   &   &                           \\ %%% for presentation
nsmax\mytarget{sub:map2alm:nsmax} & I4B & IN & the $\nside$ value of the map to analyse. \\
nlmax\mytarget{sub:map2alm:nlmax} & I4B & IN & the maximum $\ell$ value for the analysis. \\
nmmax\mytarget{sub:map2alm:nmmax} & I4B & IN & the maximum $m$ value for the analysis. \\
map\_TQU\mytarget{sub:map2alm:map_TQU}(0:12*nsmax**2-1) & SP/ DP & IN & if only the temperature map is to be analyse, the map-array should be passed with this rank. \\ 
map\_TQU(0:12*nsmax**2-1, 1:3) & SP/ DP & IN & if both temperature an polarisation maps are to be analysed, the map array should have this rank, where the second index is (1,2,3) corresponding to (T,Q,U). \\ 
\end{tabular}
\begin{tabular}{p{0.4\hsize} p{0.05\hsize} p{0.05\hsize} p{0.40\hsize}}   \hline  
alm\_TGC\mytarget{sub:map2alm:alm_TGC}(1:p, 0:nlmax, 0:nmmax) & SPC/ DPC & OUT & The $a_{\ell m}$ values output from the analysis. p is 1 or 3 dependent on wether polarisation is included or not. In the former case, the first index is (1,2,3) corresponding to (T,E,B). \\
%% cos\_theta\_cut & DP & IN & the cosine of the cutting angle for a cut sky
%%    analysis. {\bf Note}: in order to have no cut at all cos\_theta\_cut needs to be
%%    set to a {\em negative} value, as cos\_theta\_cut=$\cos(90^\circ)=0$ still
%%    removes the equatorial ring.\\
zbounds\mytarget{sub:map2alm:zbounds}(1:2) & DP & IN & section of the map on which to perform the $a_{\ell m}$
                   analysis, expressed in terms of $z=\sin(\mathrm{latitude}) =
                   \cos(\theta).$ If zbounds(1)$<$zbounds(2), the analysis is
                   performed {\em on} the strip zbounds(1)$<z<$zbounds(2); if not,
                   it is performed {\em outside} of the strip
                   zbounds(2)$<z<$zbounds(1). \\
w8ring\_TQU\mytarget{sub:map2alm:w8ring_TQU}(1:2*nsmax, 1:p) & DP & IN & ring weights for quadrature corrections. If ring weights are not used, this array should be 1 everywhere. p is 1 for a temperature analysis and 3 for (T,Q,U). \\
{\small{plm(0:(nlmax+1)(nlmax+2)nsmax-1)}}\mytarget{sub:map2alm:plm}, \hskip 6cm OPTIONAL & DP & IN & If this optional matrix is passed with this rank, precomputed $P_{\ell m}(\theta)$ are used instead of recursion. Note that since version 2.20 this feature has become obsolete
because of algorithm optimizations.\\ 
{\small{plm(0:(nlmax+1)(nlmax+2)nsmax-1,1:3)}}, \hskip 6cm OPTIONAL & DP & IN & If this optional matrix is passed with this rank, precomputed $P_{\ell m}(\theta)$ AND precomputed tensor harmonics are used instead of recursion. \\
\end{tabular}
}
\end{arguments}
%%\newpage

\begin{example}
{
use \htmlref{healpix\_types}{sub:healpix_types}\\
use alm\_tools\\
use pix\_tools\\
integer(\htmlref{i4b}{sub:healpix_types}) :: nside, lmax \\
real(\htmlref{dp}{sub:healpix_types}), allocatable, dimension(:,:) :: dw8 \\
real(dp), dimension(2) :: z \\
real(\htmlref{sp}{sub:healpix_types}), allocatable, dimension(:,:) :: map \\
complex(\htmlref{spc}{sub:healpix_types}), allocatable, dimension(:,:,:) :: alm \\
\\
nside = 256 \\
lmax = 512 \\
allocate(dw8(1:2*nside, 1:3)) \\
allocate(map(0:nside2npix(nside)-1,1:3)) \\
allocate(alm(1:3, 0:lmax, 0:lmax)\\
dw8 = 1.0\_dp \\
z = sin(10.0\_dp * \htmlref{DEG2RAD}{sub:healpix_types}) \\
call map2alm(nside, lmax, lmax, map, alm, ($\backslash$ z, -z $\backslash$) , dw8, plm(0:(lmax+1)*(lmax+2)*nside-1))  \\
}
{
Analyses temperature and polarisation maps passed in map. The map has
an $\nside$ of 256, and the analysis is performed up
to 512 in $\ell$ and $m$. The resulting $a_{\ell m}$ coefficients for
temperature and polarisation are returned in alm. A $10^\circ$ cut on
each side of the equator is applied. Uniform weights are used. Since
the optional plm array is provided with rank one, precomputed scalar $P_{\ell m}(\theta)$ are
used while tensor harmonics are computed with a recursion.
}
\end{example}

\begin{modules}
  \begin{sulist}{} %%%% NOTE the ``extra'' brace here %%%%
  \item[ring\_analysis] Performs FFT for the ring analysis.
  \item[\textbf{misc\_util}] module, containing:
  \item[\htmlref{assert\_alloc}{sub:assert}] routine to print error message when an array is not
  properly allocated		
  \end{sulist}
Note: Starting with \htmlref{version 2.20}{sub:new2p20}, {\tt libpsht} routines will be called when
precomputed $P_{\ell m}$ are not provided.
\end{modules}

\begin{related}
  \begin{sulist}{} %%%% NOTE the ``extra'' brace here %%%%
  \item[anafast] executable using \thedocid{} to analyse maps.
  \item[\htmlref{alm2map}{sub:alm2map}] routine performing the inverse transform
of \thedocid.
  \item[\htmlref{dump\_alms}{sub:dump_alms}] write $a_{\ell m}$ coefficients
computed by \thedocid{} into a FITS file
  \item[\htmlref{map2alm\_iterative}{sub:map2alm_iterative}] similar to
\thedocid{} with iterative scheme.
  \end{sulist}
\end{related}

\rule{\hsize}{2mm}

\newpage


\sloppy


%\title{\healpix Fortran Subroutines Overview}
\docid{map2alm\_iterative} \section[map2alm\_iterative*]{ }
\label{sub:map2alm_iterative}
\docrv{Version 2.1}
\author{Eric Hivon}
\abstract{This document describes the \healpix Fortran90 subroutine MAP2ALM\_ITERATIVE*.}

\begin{facility}
{This routine covers and extends the functionalities of \htmlref{map2alm}{sub:map2alm}: it
analyzes a (polarised) \healpix {\em RING ordered} map and returns
its $a_{\ell m}$ coefficients for temperature (and polarisation) up to a specified
multipole, and use precomputed harmonics if those
are provided, but it also can also perform an iterative (Jacobi) determination of the $a_{\ell m}$, and
apply a pixel mask if one is provided.\\
\newcommand{\bA}{\textbf{A}}
\newcommand{\bS}{\textbf{S}}
\newcommand{\ba}{\textbf{a}}
\newcommand{\bm}{\textbf{m}}
\newcommand{\bw}{\textbf{w}}
Denoting $\bA$ and $\bS$ the 
analysis  (\htmlref{map2alm}{sub:map2alm}) and
synthesis (\htmlref{alm2map}{sub:alm2map})
operators and  $\ba, \bm$ and $\bw$, the $a_{\ell m}$, map and pixel mask vectors, the
Jacobi iterative process reads 
\begin{eqnarray}
	\label{eq:map2alm_it_a}
	\ba^{(n)} = \ba^{(n-1)} + \bA. \left( \bw.\bm - \bS .\ba^{(n-1)} \right),
\end{eqnarray}
with
\begin{eqnarray}
	\label{eq:map2alm_it_b}
	\ba^{(0)} = \bA.\bw.\bm.
\end{eqnarray}
%
During the processing, the standard deviation of the input map $\left(\bw.\bm\right)$ 
and the current residual map $\left(\bw.\bm - \bS .\ba^{(n-1)}\right)$ is printed out, with the latter expected
to get smaller and smaller as $n$ increases.\\
The standard deviation of map $x$ has the usual definition
$\sigma \equiv \sqrt{\sum_{p=1}^{N}\frac{(x(p)-\bar{x})^2}{N-1}}$, where the mean is
$\bar{x} \equiv  \sum_{p=1}^{N} \frac{x(p)}{N}$, and the index $p$ runs over all pixels.%
% Note that this iterative process is meaningful only if the 
% input map has (almost) no power outside the multipole range $[0, \lmax]$ of analysis.
% The application of a mask to the map will most likely break this assumption, 
% reducing the improvement to be expected from the iterations.
\\
In \htmlref{version 3.50}{sub:new3p50} a bug affecting previous versions of \thedocid{} has been fixed.
(It occured when 
\mylink{sub:map2alm_iterative:iter_order}{\texttt{iter\_order}}$>0$ 
was used in conjonction with a 
\mylink{sub:map2alm_iterative:mask}{\texttt{mask}} 
and/or a restrictive 
\mylink{sub:map2alm_iterative:zbounds}{\texttt{zbounds}}, 
with a magnitude that depended on each of those factors and was larger for non-boolean masks (ie, $\bw^2 \ne \bw$).
To assess the impact of this bug on previous results, the old implementation remains available in 
\texttt{map2alm\_iterative\_old}). 
The result was correct when the mask (if any) was applied to the map prior to the 
\thedocid{} calling, or when no iteration was requested.}
{\modAlmTools}
\end{facility}

\begin{f90format}
{\mylink{sub:map2alm_iterative:nsmax}{nsmax}%
, \mylink{sub:map2alm_iterative:nlmax}{nlmax}%
, \mylink{sub:map2alm_iterative:nmmax}{nmmax}%
, \mylink{sub:map2alm_iterative:iter_order}{iter\_order}%
, \mylink{sub:map2alm_iterative:map_TQU}{map\_TQU}%
, \mylink{sub:map2alm_iterative:alm_TGC}{alm\_TGC}%
 [, \mylink{sub:map2alm_iterative:zbounds}{zbounds}%
, \mylink{sub:map2alm_iterative:w8ring_TQU}{w8ring\_TQU}%
 ,
\mylink{sub:map2alm_iterative:plm}{plm}%
, \mylink{sub:map2alm_iterative:mask}{mask}%
]}
\end{f90format}

%\newpage
\begin{arguments}
{
\begin{tabular}{p{0.38\hsize} p{0.05\hsize} p{0.07\hsize} p{0.40\hsize}} \hline  
\textbf{name~\&~dimensionality} & \textbf{kind} & \textbf{in/out} & \textbf{description} \\ \hline
                   &   &   &                           \\ %%% for presentation
nsmax\mytarget{sub:map2alm_iterative:nsmax} & I4B & IN & the $\nside$ value of the map to analyse. \\
nlmax\mytarget{sub:map2alm_iterative:nlmax} & I4B & IN & the maximum $\ell$ value ($\lmax$) for the analysis. \\
nmmax\mytarget{sub:map2alm_iterative:nmmax} & I4B & IN & the maximum $m$ value for the analysis. \\
%
iter\_order\mytarget{sub:map2alm_iterative:iter_order} & I4B & IN & the order of Jacobi iteration. Increasing that order
improves the accuracy of the final $a_{\ell m}$ but increases the computation time $
T_{\mathrm{CPU}} \propto 1 + 2 \times $iter\_order. 
iter\_order  $=0$ is a straight analysis, while iter\_order $=3$ is usually a
good compromise. \\
%
map\_TQU\mytarget{sub:map2alm_iterative:map_TQU}(0:12*nsmax**2-1, 1:p) & SP/ DP & INOUT & input map. $p$ is 1 or 3
depending if temperature (T) only or temperature and polarisation (T, Q, U) are
to be analysed. It will be altered on output if a \mylink{sub:map2alm_iterative:mask}{mask} is provided and/or if \mylink{sub:map2alm_iterative:iter_order}{iter\_order}$>0$ and \mylink{sub:map2alm_iterative:zbounds}{zbounds} is provided.\\
%
alm\_TGC\mytarget{sub:map2alm_iterative:alm_TGC}(1:p, 0:nlmax, 0:nmmax) & SPC/ DPC & OUT & The $a_{\ell m}$ values output
from the analysis. 
$p$ is 1 or 3 depending on whether polarisation is included or not. In the former
case, the first index is (1,2,3) corresponding to (T,E,B). \\
%
zbounds\mytarget{sub:map2alm_iterative:zbounds}(1:2), \hskip 6cm OPTIONAL  & DP & IN & section of the map on which to perform the $a_{\ell m}$
                   analysis, expressed in terms of $z=\sin(\mathrm{latitude}) =
                   \cos(\theta).$ %zbounds_sub.tex:  for alm2map, map2alm, remove_dipole: describe processed area
%zbounds2_sub.tex: for apply_mask: describe pixels set to 0 (=unprocessed area)
If zbounds(1)$<$zbounds(2), it is
performed {\em on} the strip zbounds(1)$<z<$zbounds(2); if not,
it is performed {\em outside} the strip
%  zbounds(2)$<z<$zbounds(1). % OLD
zbounds(2)$\le z \le$zbounds(1). % NEW ??
% The whole sphere is treated if \texttt{zbounds=(/-1.0\_dp, 1.0\_dp/)} or if 
% it is absent.
If absent, the whole map is processed.
\\
%
\end{tabular}
%
\begin{tabular}{p{0.38\hsize} p{0.05\hsize} p{0.07\hsize} p{0.40\hsize}}   \hline  
w8ring\_TQU\mytarget{sub:map2alm_iterative:w8ring_TQU}(1:2*nsmax,1:p), \hskip 6cm OPTIONAL  & DP & IN & ring weights for
quadrature corrections. p is 1 for a temperature analysis and 3 for (T,Q,U). If absent, the
ring weights are all set to 1.\\
%
plm\mytarget{sub:map2alm_iterative:plm}(0:,1:p), \hskip 6cm OPTIONAL & DP & IN & If this
optional matrix is passed, precomputed scalar (and tensor) $P_{\ell m}(\theta)$ are
used instead of recursion. \\
%
mask\mytarget{sub:map2alm_iterative:mask}(0:12*nsmax**2-1,1:q), \hskip 6cm OPTIONAL & SP/ DP & IN & pixel mask,
assumed to have the same resolution (and RING ordering) as the map. The map {\tt
map\_TQU} is
multiplied by that mask before being analyzed, and will therefore be altered on
output. 
$q$ should be in $\{1,2,3\}$. If $p=q=3$, then each of
the 3 masks is applied to the respective map. If $p=3$ and $q=2$, the first mask
is applied to the first map, and the second mask to the second (Q) and third (U)
map. If $p=3$ and $q=1$, the same mask is applied to the 3 maps. Note: the output
$a_{\ell m}$ are computed directly on the masked map, and are {\em not} corrected for the
loss of power, correlation or leakage created by the mask.
\end{tabular}
}
\end{arguments}
%%\newpage

\begin{example}
{
use healpix\_types\\
use alm\_tools\\
use pix\_tools\\
integer(i4b) :: nside, lmax, npix, iter \\
real(sp), allocatable, dimension(:,:) :: map \\
real(sp), allocatable, dimension(:) :: mask \\
complex(spc), allocatable, dimension(:,:,:) :: alm \\
\\
nside = 256 \\
lmax = 512 \\
iter = 2\\
npix = nside2npix(nside) \\
allocate(map(0:npix-1,1:3)) \\
allocate(mask(0:npix-1)) \\
mask(0:) = 0. ! set unvalid pixels to 0\\
mask(0:10000-1) = 1. ! valid pixels \\
allocate(alm(1:3, 0:lmax, 0:lmax)\\
call map2alm\_iterative(nside, lmax, lmax, iter, map, alm, mask=mask)  \\
}
{
Analyses temperature and polarisation signals in the first 10000 pixels of {\tt map} (as
determined by {\tt mask}). The map has
an $\nside$ of 256, and the analysis is supposed to be performed up
to 512 in $\ell$ and $m$. The resulting $a_{\ell m}$ coefficients for
temperature and polarisation are returned in {\tt alm}. Uniform weights are
assumed. In order to improve the $a_{\ell m}$ accuracy, 2 Jacobi iterations are performed.
}
\end{example}

\begin{modules}
  \begin{sulist}{} %%%% NOTE the ``extra'' brace here %%%%
%  \item[ring\_analysis] Performs FFT for the ring analysis.
  \item[\htmlref{map2alm}{sub:map2alm}] Performs the alm analysis
  \item[\htmlref{alm2map}{sub:alm2map}] Performs the map synthesis
  \item[\textbf{misc\_util}] module, containing:
  \item[\htmlref{assert\_alloc}{sub:assert}] routine to print error message when an array is not
  properly allocated		
  \end{sulist}
\end{modules}

\begin{related}
  \begin{sulist}{} %%%% NOTE the ``extra'' brace here %%%%
  \item[\htmlref{anafast}{fac:anafast}] executable using \thedocid \ to analyse maps.
  \item[\htmlref{alm2map}{sub:alm2map}] routine performing the inverse transform of \thedocid.
  \item[\htmlref{alm2map\_spin}{sub:alm2map_spin}] synthesize spin weighted
maps.
  \item[\htmlref{dump\_alms}{sub:dump_alms}] write $a_{\ell m}$ coefficients
computed by \thedocid\ into a FITS file
  \item[\htmlref{map2alm\_spin}{sub:map2alm_spin}] analyze spin weighted maps.
  \end{sulist}
\end{related}

\rule{\hsize}{2mm}

\newpage


\sloppy


\title{\healpix Fortran Subroutines Overview}
\docid{map2alm\_spin*} \section[map2alm\_spin*]{ }
\label{sub:map2alm_spin}
\docrv{Version 2.1}
\author{Eric Hivon}
\abstract{This document describes the \healpix Fortran90 subroutine MAP2ALM\_SPIN*.}

\begin{facility}
{This routine extracts the alm coefficients out of maps of spin $s$ and $-s$.
%
A (complex) map $S$ of spin $s$ is a linear combination of the spin weighted harmonics ${_s}Y_{\ell m}$
\begin{equation}
	{_s}S(p) = \sum_{\ell m} {_s}a_{\ell m}\ \ {_s}Y_{\ell m}(p)
\end{equation}
for $\ell \ge |m|, \ell \ge |s|$,
and is such that ${_s}S^* = {_{-s}}S$.\\
The 
\linklatexhtml{usual phase convention for the spin weighted harmonics}%
{https://en.wikipedia.org/wiki/Spin-weighted_spherical_harmonics\#Calculating}%
{https://en.wikipedia.org/wiki/Spin-weighted_spherical_harmonics\#Calculating}
is
${_s}Y_{\ell m}^* = (-1)^{s+m} {_{-s}}Y_{\ell -m}$
and therefore 
${_s}a_{\ell m}^* = (-1)^{s+m} {_{-s}}a_{\ell -m}$.
%
The two (real) input maps for \thedocid\ are defined respectively as
\begin{eqnarray}
	{_{|s|}}S^+ &\myequal& ({_{|s|}}S + {_{-|s|}}S)/2 \\
	{_{|s|}}S^- &\myequal& ({_{|s|}}S - {_{-|s|}}S)/(2i).
\end{eqnarray}
%
\thedocid\ outputs the alm coefficients defined as
	%typo correction on spin sign of second term of RHS, 2009-09-03
\begin{eqnarray}
	{_{|s|}}a^{+}_{\ell m} &\myequal& - ( {_{|s|}}a_{\ell m} + (-1)^s {_{-|s|}}a_{\ell m} )/2 \\
	{_{|s|}}a^{-}_{\ell m} &\myequal& - ( {_{|s|}}a_{\ell m} - (-1)^s {_{-|s|}}a_{\ell m} )/(2i)
\end{eqnarray}
for $m\ge 0$, knowing that, just as for spin 0 maps, the
coefficients for $m<0$ are given by 
\begin{eqnarray}
{_{|s|}}a^{+}_{\ell-m} &\myequal& (-1)^m {_{|s|}}a^{+*}_{\ell m}, \\
{_{|s|}}a^{-}_{\ell-m} &\myequal& (-1)^m {_{|s|}}a^{-*}_{\ell m}.
\end{eqnarray}
%
With these definitions, ${_2}a^{+}$, ${_2}a^{-}$, ${_2}S^+$ and ${_2}S^-$
match \healpix  polarization $a^E, a^B, Q$ and $U$ respectively. However, for
$s=0$, $\ _{0}a^+_{\ell m} = -a^T_{\ell m}$, $\ _{0}a^-_{\ell m} = 0$, $\ {_0}S^+ = T$, $\
{_0}S^- = 0.$
}
{\modAlmTools}
\end{facility}

\begin{f90format}
{\mylink{sub:map2alm_spin:nsmax}{nsmax}%
, \mylink{sub:map2alm_spin:nlmax}{nlmax}%
, \mylink{sub:map2alm_spin:nmmax}{nmmax}%
, \mylink{sub:map2alm_spin:spin}{spin}%
, \mylink{sub:map2alm_spin:map}{map}%
, \mylink{sub:map2alm_spin:alm}{alm}%
 [, \mylink{sub:map2alm_spin:zbounds}{zbounds}=%
, \mylink{sub:map2alm_spin:w8ring_TQU}{w8ring\_TQU}=%
]}
\end{f90format}

\begin{arguments}
{
\begin{tabular}{p{0.4\hsize} p{0.05\hsize} p{0.05\hsize} p{0.40\hsize}} \hline  
\textbf{name~\&~dimensionality} & \textbf{kind} & \textbf{in/out} & \textbf{description} \\ \hline
                   &   &   &                           \\ %%% for presentation
nsmax\mytarget{sub:map2alm_spin:nsmax} & I4B & IN & the $\nside$ value of the map to analyse. \\
nlmax\mytarget{sub:map2alm_spin:nlmax} & I4B & IN & the maximum $\ell$ value for the analysis. \\
nmmax\mytarget{sub:map2alm_spin:nmmax} & I4B & IN & the maximum $m$ value for the analysis. \\
spin\mytarget{sub:map2alm_spin:spin} & I4B & IN & the spin $s$ of the maps to be analysed (only its absolute
value is relevant).\\
map\mytarget{sub:map2alm_spin:map}(0:12*nsmax**2-1, 1:2) & SP/ DP & IN & ${_{|s|}}S^+$ and ${_{|s|}}S^-$ input maps\\
alm\mytarget{sub:map2alm_spin:alm}(1:2, 0:nlmax, 0:nmmax) & SPC/ DPC & OUT & The ${_{|s|}}a^+_{\ell m}$ and
${_{|s|}}a^-_{\ell m}$ output values. \\
zbounds\mytarget{sub:map2alm_spin:zbounds}(1:2), \hskip 4cm OPTIONAL & DP & IN & section of the map on which to perform the $a_{\ell m}$
                   analysis, expressed in terms of $z=\sin(\mathrm{latitude}) =
                   \cos(\theta).$ %zbounds_sub.tex:  for alm2map, map2alm, remove_dipole: describe processed area
%zbounds2_sub.tex: for apply_mask: describe pixels set to 0 (=unprocessed area)
If zbounds(1)$<$zbounds(2), it is
performed {\em on} the strip zbounds(1)$<z<$zbounds(2); if not,
it is performed {\em outside} the strip
%  zbounds(2)$<z<$zbounds(1). % OLD
zbounds(2)$\le z \le$zbounds(1). % NEW ??
% The whole sphere is treated if \texttt{zbounds=(/-1.0\_dp, 1.0\_dp/)} or if 
% it is absent.
If absent, the whole map is processed.
 \\
w8ring\_TQU(1:2*nsmax,1:2)\mytarget{sub:map2alm_spin:w8ring_TQU}, \hskip 3cm OPTIONAL & DP & IN & ring weights for quadrature corrections. If ring weights are not used, this array should be 1 everywhere.
\end{tabular}
}
\end{arguments}
%%\newpage

\begin{example}
{
use \htmlref{healpix\_types}{sub:healpix_types}\\
use alm\_tools\\
use pix\_tools\\
integer(\htmlref{i4b}{sub:healpix_types}) :: nside, lmax, spin \\
% real(\htmlref{dp}{sub:healpix_types}), allocatable, dimension(:,:) :: dw8 \\
% real(dp), dimension(2) :: z \\
real(\htmlref{sp}{sub:healpix_types}), allocatable, dimension(:,:) :: map \\
complex(\htmlref{spc}{sub:healpix_types}), allocatable, dimension(:,:,:) :: alm \\
\\
nside = 256 \\
lmax = 512 \\
spin = 5 \\
allocate(map(0:nside2npix(nside)-1,1:2)) \\
allocate(alm(1:2, 0:lmax, 0:lmax)\\
...\\
% allocate(dw8(1:2*nside, 1:2)) \\
% dw8 = 1.0\_dp \\
% z = sin(10.0\_dp * \htmlref{DEG2RAD}{sub:healpix_types}) \\
% call map2alm(nside, lmax, lmax, map, alm, ($\backslash$ z, -z $\backslash$) , dw8, plm(0:(lmax+1)*(lmax+2)*nside-1))  \\
call map2alm\_spin(nside, lmax, lmax, spin, map, alm)  \\
}
{
Analyses spin 5 and -5 maps. The maps have
an $\nside$ of 256, and the analysis is performed up
to 512 in $\ell$ and $m$. The resulting $a_{\ell m}$ coefficients for
are returned in alm.
}
\end{example}

\begin{modules}
  \begin{sulist}{} %%%% NOTE the ``extra'' brace here %%%%
  \item[ring\_analysis] Performs FFT for the ring analysis.
  \item[compute\_lam\_mm, get\_pixel\_layout, ]
  \item[gen\_lamfac\_der, gen\_mfac,  ] 
  \item[gen\_recfac, init\_rescale, l\_min\_ylm] Ancillary routines used
  for $\ {_s}Y_{\ell m}$ recursion
  \item[\textbf{misc\_util}] module, containing:
  \item[\htmlref{assert\_alloc}{sub:assert}] routine to print error message when an array is not
  properly allocated		
  \end{sulist}
%Note: Starting with \htmlref{version 2.20}{sub:new2p20}, {\tt libpsht} routines will be called if $0<|s|\le100$.
Note: Starting with \htmlref{version 3.10}{sub:new3p10}, {\tt libsharp} routines will be called if $0<|s|\le100$.
\end{modules}

\begin{related}
  \begin{sulist}{} %%%% NOTE the ``extra'' brace here %%%%
%  \item[anafast] executable using \thedocid to analyse maps.
  \item[\htmlref{alm2map\_spin}{sub:alm2map_spin}] routine performing the inverse transform
of \thedocid.
   \item[\htmlref{map2alm}{sub:map2alm}] routine analyzing temperature and
polarization maps
%   \item[\htmlref{map2alm\_iterative}{sub:map2alm_iterative}] similar to
% \thedocid\ with iterative scheme.
  \end{sulist}
\end{related}

\rule{\hsize}{2mm}

\newpage

\sloppy
\docid{maskborder\_nest}\section[maskborder\_nest]{ }
\label{sub:maskborder_nest}
\docrv{Version 1.0}
\author{Eric Hivon}
\abstract{This document describes the \healpix Fortran90 subroutine MASKBORDER\_NEST.}

\begin{facility}
{For a input binary mask in NESTED ordering, \thedocid\ identifies the pixels
located on the inner boundary of {\em invalid} regions
}
{\modMaskTools}
\end{facility}

\begin{f90format}
{\mylink{sub:maskborder_nest:nside}{nside}%
, \mylink{sub:maskborder_nest:mask_in}{mask\_in}%
, \mylink{sub:maskborder_nest:mask_out}{mask\_out}%
, \mylink{sub:maskborder_nest:nbordpix}{nbordpix}%
, \optional{[ \mylink{sub:maskborder_nest:border_value}{border\_value}%
]}}
\end{f90format}
\aboutoptional

\begin{arguments}
{
\begin{tabular}{p{0.35\hsize} p{0.05\hsize} p{0.1\hsize} p{0.40\hsize}} \hline  
\textbf{name~\&~dimensionality} & \textbf{kind} & \textbf{in/out} & \textbf{description} \\ \hline
                   &   &   &                           \\ %%% for presentation
nside\mytarget{sub:maskborder_nest:nside} & I4B & IN & The $N_{side}$ value of the input mask. \\
mask\_in\mytarget{sub:maskborder_nest:mask_in}(0:Npix-1) & I4B & IN & Input binary NESTED-ordered mask. Npix = 12*nside*nside\\
mask\_out\mytarget{sub:maskborder_nest:mask_out}(0:Npix-1) &I4B & OUT & Output NESTED-ordered mask, in which inner border
pixels (defined as 0-valued and adjacent to 1-valued pixels) take the value 2
(or {\tt border\_value}). Can be the same
array as mask\_in.\\
nbordpix\mytarget{sub:maskborder_nest:nbordpix} & I4B & OUT & Number of border pixels found\\
\optional{border\_value\mytarget{sub:maskborder_nest:border_value}} & I4B & IN & value to be given to border pixels in
output mask. \default{2}.
\end{tabular}
}
\end{arguments}

\begin{example}
{
use healpix\_types \\
use healpix\_modules \\
% use pix\_tools, only : nside2npix \\
% use alm\_tools, only : maskborder\_nest \\
% integer(I4B) :: nside, lmax, mmax, npix, spin\\
% real(SP), dimension(:,:), allocatable :: map \\
% complex(SPC), dimension(:,:,:), allocatable :: alm \\
% \ldots \\
% nside=256 ; lmax=512 ; mmax=lmax ; spin=4\\
% npix=nside2npix(nside)\\
% allocate(alm(1:2,0:lmax,0:mmax))\\
% allocate(map(0:npix-1,1:2))\\
\ldots \\
call \thedocid(nside, mask\_in, mask\_in, nbordpix)  \\
}
{For a binary input mask {\tt mask\_in}, it will look for border pixels and output
their number in {\tt nborpix}. In this example the {\tt mask\_in} will be
modified so that border pixels take value 2 on output.
}
\end{example}

\begin{modules}
  \begin{sulist}{} %%%% NOTE the ``extra'' brace here %%%%
  \item[\textbf{mask\_tools}] mask processing module (see related routines below)
  \end{sulist}
\end{modules}

\begin{related}
  \begin{sulist}{} %%%% NOTE the ``extra'' brace here %%%%
	\maskToolsRelated
  \end{sulist}
\end{related}

\rule{\hsize}{2mm}

\newpage


\sloppy


\title{\healpix Fortran Subroutines Overview}
\docid{medfiltmap*} \section[medfiltmap*]{ }
\label{sub:medfiltmap}
\docrv{Version 2.1}
\author{Eric Hivon}
\abstract{This document describes the \healpix Fortran90 subroutine MEDFILTMAP*.}

\begin{facility}
{This routine performs the median filtering of a \healpix full sky map for a
  given neighborhood radius }
{\modPixTools}
\end{facility}

\begin{f90format}
{\mylink{sub:medfiltmap:in_map}{in\_map}%
, \mylink{sub:medfiltmap:radius}{radius}%
, \mylink{sub:medfiltmap:med_map}{med\_map}%
 \hfill [,~\mylink{sub:medfiltmap:nest}{nest}%
, \mylink{sub:medfiltmap:fmissval}{fmissval}%
, \mylink{sub:medfiltmap:fill_holes}{fill\_holes}%
]}
\end{f90format}

\begin{arguments}
{
\begin{tabular}{p{0.30\hsize} p{0.05\hsize} p{0.05\hsize} p{0.50\hsize}} \hline  
\textbf{name~\&~dimensionality} & \textbf{kind} & \textbf{in/out} & \textbf{description} \\ \hline
                   &   &   &                           \\ %%% for presentation
in\_map\mytarget{sub:medfiltmap:in_map}(0:npix-1) & SP/ DP & IN & Full sky \healpix map to filter. {\tt npix}
                   should be valid \healpix pixel number. \\
radius\mytarget{sub:medfiltmap:radius} & DP & IN & Radius in RADIANS of the disk on which the median is
                   computed. \\
med\_map\mytarget{sub:medfiltmap:med_map}(0:npix-1) & SP/ DP & OUT & Median filtered map: each pixel is the
                   median of the input map valid neighboring pixels contained
                   within a distance {\tt radius} \\
nest\mytarget{sub:medfiltmap:nest} \hskip 1cm OPTIONAL & I4B & IN & set to 1 if the map ordering is NESTED, set to 0 if
                   it is RING. \\
fmissval\mytarget{sub:medfiltmap:fmissval} \hskip 1cm OPTIONAL & SP/ DP & IN & sentinel value given to missing or
                   non-valid pixels. Default: {\tt HPX\_SBADVAL} or {\tt
                   HPX\_DBADVAL} $ = -1.6375\ 10^{30}$ \\
fill\_holes\mytarget{sub:medfiltmap:fill_holes} \hskip 1cm OPTIONAL & LGT & IN & if set to {\tt .true.} will replace
                   non-valid pixels by median of neighbors; if set to {\tt .false.}
                   will leave non-valid pixels unchanged. Default: {\tt .false.}
\end{tabular}
}
\end{arguments}
%%\newpage

\begin{example}
{
use healpix\_types \\
use pix\_tools \\
... \\
call medfiltmap(map, 0.5*DEG2RAD, med)  \\
}
{
Output in {\tt med} the median filter of {\tt map}, using a filter radius of 0.5 Deg
}
\end{example}

\begin{modules}
  \begin{sulist}{} %%%% NOTE the ``extra'' brace here %%%%
  \item[\textbf{statistics}] module, containing:
  \item[\htmlref{median}{sub:median}] routine to compute the median of a data set		
  \item[\textbf{pix\_tools}] module, containing:
  \item[\htmlref{pix2vec\_ring, pix2vec\_nest}{sub:pix_tools}] routines to find
  the location of a pixel on the sky
  \item[\htmlref{query\_disc}{sub:query_disc}] routine to find pixels lying
  within a radius of a given point
  \end{sulist}
\end{modules}

%% \begin{related}
%%   \begin{sulist}{} %%%% NOTE the ``extra'' brace here %%%%
%%   \item[anafast] executable using \thedocid to analyse maps.
%%   \item[\htmlref{alm2map}{sub:alm2map}] routine performing the inverse transform of \thedocid.
%%   \end{sulist}
%% \end{related}

\rule{\hsize}{2mm}

\newpage


\sloppy


\title{\healpix Fortran Subroutines Overview}
\docid{median*} \section[median*]{ }
\label{sub:median}
\docrv{Version 2.0}
\author{Eric Hivon}
\abstract{This document describes the \healpix Fortran90 subroutine MEDIAN*.}

\begin{facility}
{This function computes the median of a data set}
{\modStatistics}
\end{facility}

\begin{f90function}
{\mylink{sub:median:data}{data}%
 [,~\mylink{sub:median:badval}{badval}%
, \mylink{sub:median:even}{even}%
]}
\end{f90function}

\begin{arguments}
{
\begin{tabular}{p{0.30\hsize} p{0.05\hsize} p{0.05\hsize} p{0.50\hsize}} \hline  
\textbf{name~\&~dimensionality} & \textbf{kind} & \textbf{in/out} & \textbf{description} \\ \hline
                   &   &   &                           \\ %%% for presentation
var & SP/ DP & OUT & median of the data set, defined as the middle number (or
                   the average of the 2 middle numbers) once the valid data points are
                   sorted in monotonous order\\
data\mytarget{sub:median:data}(:) & SP/ DP & IN & data set \\
badval\mytarget{sub:median:badval} \hskip 3cm (OPTIONAL) & SP/ DP & IN & sentinel value given to bad data points. Data points with this
                   value will be ignored during calculation of the median. If
                   not set, all points will be considered. {\bf Do not set to 0!}.\\
even\mytarget{sub:median:even} \hskip 4cm (OPTIONAL) & LGT & IN & if set to {\tt .true.} and the number of
                   valid data points is even, will output the average of the 2
                   middle points (which doubles the calculation time). If the
                   number of points is odd, the single middle point is output
                   and this keyword is ignored.
\end{tabular}
}
\end{arguments}
%%\newpage

\begin{example}
{
use statistics, only: median \\
... \\
med = median(map, even=.true.)  \\
}
{
Outputs in {\tt med} the median of {\tt map}
}
\end{example}

\begin{modules}
  \begin{sulist}{} %%%% NOTE the ``extra'' brace here %%%%
  \item[\textbf{m\_indmed}] module of the Orderpack 2.0 package, written by:
  Michel Olagnon,  \htmladdnormallink{http://www.fortran-2000.com/rank/}{http://www.fortran-2000.com/rank/}
  \item[indmed] routine to output rank of median
  \end{sulist}
\end{modules}

\begin{related}
  \begin{sulist}{} %%%% NOTE the ``extra'' brace here %%%%
  \item[\htmlref{compute\_statistics}{sub:compute_statistics}] routine min, max,
  absolute deviation, and first four order moments of a data set
  \end{sulist}
\end{related}

\rule{\hsize}{2mm}

\newpage


\sloppy

%%%\title{\healpix Fortran Subroutines Overview}
\docid{merge\_headers} \section[merge\_headers]{ }
\label{sub:merge_headers}
\docrv{Version 1.2}
\author{Frode K.~Hansen, Eric Hivon}
\abstract{This document describes the \healpix Fortran90 subroutine MERGE\_HEADERS.}

\begin{facility}
{This routine merges two FITS headers.}
{\modHeadFits}
\end{facility}

\begin{f90format}
{\mylink{sub:merge_headers:header1}{header1}%
, \mylink{sub:merge_headers:header2}{header2}%
}
\end{f90format}

\begin{arguments}
{
\begin{tabular}{p{0.4\hsize} p{0.05\hsize} p{0.1\hsize} p{0.35\hsize}} \hline  
\textbf{name\&dimensionality} & \textbf{kind} & \textbf{in/out} & \textbf{description} \\ \hline
                   &   &   &                           \\ %%% for presentation
header1\mytarget{sub:merge_headers:header1}(LEN=80) DIMENSION(:) & CHR & IN & First header. \\
header2\mytarget{sub:merge_headers:header2}(LEN=80) DIMENSION(:) & CHR & INOUT & Second header. On output,
                   will contain the concatenation of (in that order) header2 and
                   header1. If header2 is too short to allow the
                   merging the output will be truncated\\
\end{tabular}
}
\end{arguments}

\begin{example}
{
call merge\_headers(header1, header2)  \\
}
{
On output header2 will contain the original header2, followed by the
content of header1
}
\end{example}

\begin{modules}
  \begin{sulist}{} %%%% NOTE the ``extra'' brace here %%%%
  \item[write\_hl] more general routine for adding a keyword to a header.
  \item[\textbf{cfitsio}] library for FITS file handling.		
  \end{sulist}
\end{modules}

\begin{related}
  \begin{sulist}{} %%%% NOTE the ``extra'' brace here %%%%
  \item[\htmlref{add\_card}{sub:add_card}] general purpose routine to write any keywords into a FITS
  file header
  \item[\htmlref{get\_card}{sub:get_card}] general purpose routine to read any keywords from a header in a FITS file.
  \item[\htmlref{del\_card}{sub:del_card}] routine to discard a keyword from a FITS header
  \item[\htmlref{read\_par}{sub:read_par}, \htmlref{number\_of\_alms}{sub:number_of_alms}] routines to read specific keywords from a
  header in a FITS file.
  \item[\htmlref{getsize\_fits}{sub:getsize_fits}] function returning the size of the data set in a fits
  file and reading some other useful FITS keywords
  \end{sulist}
\end{related}

\rule{\hsize}{2mm}

\newpage


\sloppy


%%%\title{\healpix Fortran Subroutines Overview}
\docid{mpi\_alm\_tools*} \section[mpi\_alm\_tools*]{ }
\label{sub:mpi_alm_tools}
\docrv{Version 1.0}
\author{Hans K. Eriksen}
\abstract{This document describes the \healpix Fortran 90 module MPI\_ALM\_TOOLS*.}

\begin{facility}
{This module implements MPI parallelization of the alm2map and map2alm routines. 
It is not compiled by default during installation, but rather intended for
users who need massive parallelization in their own programming. Typical
applications are Monte Carlo simulations and Markov chain type
analyses.

The routines can be called in two modes, either simple or
advanced. The former mimics the interface of the standard routines,
but with an additional MPI handle as a first argument, and is intended
for applications which requires only one or a few transforms. The
latter interface provides both more flexibility (in particular the
option of pre-computation of the Legendre polynomials) and a simpler
interface when multiple transforms are required. This interface is
particularly well suited for Monte Carlo simulations and Markov chain
type analyses.  } 
{\modMpiAlmTools}
\end{facility}

\begin{example}
{ 
\begin{itemize}

\item Simple one-line interfaces:
\begin{itemize}
\item mpi\_map2alm\_simple
\item mpi\_alm2map\_simple
\end{itemize}

\item Three-step advanced interfaces:
\begin{enumerate}
   \item Initialization: \\mpi\_initialize\_alm\_tools
   \item Execution of spherical harmonics transforms
   \begin{itemize}
	   \item mpi\_map2alm (root processor)
	   \item mpi\_alm2map (root processor)
	   \item mpi\_map2alm\_slave (slave processor)
	   \item mpi\_alm2map\_slave (slave processor)
    \end{itemize}
    \item Finalizing: \\ mpi\_cleanup\_alm\_tools
\end{enumerate}
\end{itemize}
}
{
}
\end{example}

\rule{\hsize}{2mm}

\newpage


\sloppy


%%%\title{\healpix Fortran Subroutines Overview}
\docid{mpi\_alm2map*} \section[mpi\_alm2map*]{ }
\label{sub:mpi_alm2map}
\docrv{Version 1.0}
\author{Hans K. Eriksen}
\abstract{This document describes the \healpix Fortran 90 subroutine MPI\_ALM2MAP*.}

\begin{facility}
{This subroutine implements MPI parallelization of the serial alm2map
routine. It supports both temperature and polarization inputs in both
single and double precision. It must only be run by the root node of
the MPI communicator.
}
{\modMpiAlmTools}
\end{facility}

\begin{f90format}
{\mylink{sub:mpi_alm2map:alms}{alms}%
, \mylink{sub:mpi_alm2map:map}{map}%
}
\end{f90format}

\begin{arguments}
{
\begin{tabular}{p{0.4\hsize} p{0.05\hsize} p{0.05\hsize} p{0.40\hsize}} \hline  
\textbf{name~\&~dimensionality} & \textbf{kind} & \textbf{in/out} & \textbf{description} \\ \hline
                   &   &   &                           \\ %%% for presentation
alms\mytarget{sub:mpi_alm2map:alms}(1:nmaps,0:lmax,0:nmax) & SPC or DPC & IN & Input alms. If
nmaps=1, only temperature information is included; if nmaps=3,
polarization information is included \\ 
map\mytarget{sub:mpi_alm2map:map}(0:npix,1:nmaps) & SP or DP & OUT & Output map. nmaps must match 
that of the input alms array.\\
\end{tabular}
}
\end{arguments}
%%\newpage

\begin{example}
{
call mpi\_comm\_rank(comm, myid, ierr)\\
if (myid == root) then\\
\hspace*{1cm}call mpi\_initialize\_alm\_tools(comm, nsmax, nlmax, nmmax, \\
\hspace*{3cm}zbounds,polarization, precompute\_plms)\\
\hspace*{1cm}call mpi\_alm2map(alms, map)\\
else \\
\hspace*{1cm}call mpi\_initialize\_alm\_tools(comm)\\
\hspace*{1cm}call mpi\_alm2map\_slave\\
end\\
call mpi\_cleanup\_alm\_tools\\
}
{
This example 1) initializes the mpi\_alm\_tools module (i.e.,
allocates internal arrays and defines required parameters), 2)
executes a parallel alm2map operation, and 3) frees the previously
allocated memory.
}
\end{example}

\begin{modules}
  \begin{sulist}{} %%%% NOTE the ``extra'' brace here %%%%
  \item[\textbf{alm\_tools}] module
  \end{sulist}
\end{modules}

\begin{related}
  \begin{sulist}{} %%%% NOTE the ``extra'' brace here %%%%
   \item[\htmlref{mpi\_cleanup\_alm\_tools}{sub:mpi_cleanup_alm_tools}] Frees memory that is allocated by the current routine. 
   \item[\htmlref{mpi\_initialize\_alm\_tools}{sub:mpi_initialize_alm_tools}] Allocates memory and defines variables for the mpi\_alm\_tools module. 
  \item[\htmlref{mpi\_alm2map\_slave}{sub:mpi_alm2map_slave}] Routine for executing a parallel inverse spherical harmonics transform (slave processor interface)
  \item[\htmlref{mpi\_map2alm}{sub:mpi_map2alm}] Routine for executing a parallel spherical harmonics transform (root processor interface)
  \item[\htmlref{mpi\_map2alm\_slave}{sub:mpi_map2alm_slave}] Routine for executing a parallel spherical harmonics transform (slave processor interface)
  \item[\htmlref{mpi\_alm2map\_simple}{sub:mpi_alm2map_simple}] One-line interface to the parallel inverse spherical harmonics transform 
  \item[\htmlref{mpi\_map2alm\_simple}{sub:mpi_map2alm_simple}] One-line interface to the parallel spherical harmonics transform 
  \end{sulist}
\end{related}

\rule{\hsize}{2mm}

\newpage


\sloppy


\title{\healpix Fortran Subroutines Overview}
\docid{mpi\_alm2map\_simple*} \section[mpi\_alm2map\_simple*]{ }
\label{sub:mpi_alm2map_simple}
\docrv{Version 1.0}
\author{Hans K. Eriksen}
\abstract{This document describes the \healpix Fortran 90 subroutine MPI\_ALM2MAP\_SIMPLE*.}

\begin{facility}
{This subroutine provides a simplified (one-line) interface to the MPI version of
alm2map. It supports both temperature and polarization inputs in both
single and double precision. It must only be run by all nodes in
the MPI communicator.  } 
{\modMpiAlmTools}
\end{facility}

\begin{f90format}
{\mylink{sub:mpi_alm2map_simple:comm}{comm}%
, \mylink{sub:mpi_alm2map_simple:alms}{alms}%
, \mylink{sub:mpi_alm2map_simple:map}{map}%
}
\end{f90format}

\begin{arguments}
{
\begin{tabular}{p{0.4\hsize} p{0.05\hsize} p{0.05\hsize} p{0.40\hsize}} \hline  
\textbf{name~\&~dimensionality} & \textbf{kind} & \textbf{in/out} & \textbf{description} \\ \hline
                   &   &   &                           \\ %%% for presentation
comm\mytarget{sub:mpi_alm2map_simple:comm} & I4B & IN & MPI communicator. \\
alms\mytarget{sub:mpi_alm2map_simple:alms}(1:nmaps,0:lmax,0:nmax) & SPC or DPC & IN & Input alms. If
nmaps=1, only temperature information is included; if nmaps=3,
polarization information is included \\ 
map\mytarget{sub:mpi_alm2map_simple:map}(0:npix,1:nmaps) & SP or DP & OUT & Output map. nmaps must match 
that of the input alms array.\\
\end{tabular}
}
\end{arguments}
%%\newpage

\begin{example}
{
\hspace*{1cm}call mpi\_alm2map\_simple(comm, map, alms)\\
}
{
This example executes a parallel map2alm operation through the
one-line interface. Although all processors must supply allocated
arrays to the routine, only the root processor's information will be
used as input, and only the root processor's alms will be complete
after execution. 
}
\end{example}

\begin{modules}
  \begin{sulist}{} %%%% NOTE the ``extra'' brace here %%%%
  \item[\textbf{alm\_tools}] module
  \end{sulist}
\end{modules}

\begin{related}
  \begin{sulist}{} %%%% NOTE the ``extra'' brace here %%%%
   \item[\htmlref{mpi\_cleanup\_alm\_tools}{sub:mpi_cleanup_alm_tools}] Frees memory that is allocated by the current routine. 
   \item[\htmlref{mpi\_initialize\_alm\_tools}{sub:mpi_initialize_alm_tools}] Allocates memory and defines variables for the mpi\_alm\_tools module. 
  \item[\htmlref{mpi\_alm2map}{sub:mpi_alm2map}] Routine for executing a parallel inverse spherical harmonics transform (root processor interface)
  \item[\htmlref{mpi\_alm2map\_slave}{sub:mpi_alm2map_slave}] Routine for executing a parallel inverse spherical harmonics transform (slave processor interface)
  \item[\htmlref{mpi\_map2alm}{sub:mpi_map2alm}] Routine for executing a parallel spherical harmonics transform (root processor interface)
  \item[\htmlref{mpi\_map2alm\_slave}{sub:mpi_map2alm_slave}] Routine for executing a parallel spherical harmonics transform (slave processor interface)
  \item[\htmlref{mpi\_map2alm\_simple}{sub:mpi_map2alm_simple}] One-line interface to the parallel spherical harmonics transform 
  \end{sulist}
\end{related}


\rule{\hsize}{2mm}

\newpage


\sloppy


%%%\title{\healpix Fortran Subroutines Overview}
\docid{mpi\_alm2map\_slave} \section[mpi\_alm2map\_slave]{ }
\label{sub:mpi_alm2map_slave}
\docrv{Version 1.0}
\author{Hans K. Eriksen}
\abstract{This document describes the \healpix Fortran 90 subroutine MPI\_ALM2MAP\_SLAVE.}

\begin{facility}
{This subroutine complements the master routine mpi\_alm2map, and
should be run by all slaves in the current MPI communicator. It is run
without arguments, but after an appropriate call to
initialize\_mpi\_alm\_tools. 
}
{\modMpiAlmTools}
\end{facility}

\begin{f90format}
{}
\end{f90format}

\begin{arguments}
{
None.
}
\end{arguments}
%%\newpage

\begin{example}
{
call mpi\_comm\_rank(comm, myid, ierr)\\
if (myid == root) then\\
\hspace*{1cm}call mpi\_initialize\_alm\_tools(comm, nsmax, nlmax, nmmax, \\
\hspace*{3cm}zbounds,polarization, precompute\_plms)\\
\hspace*{1cm}call mpi\_alm2map(alms, map)\\
else \\
\hspace*{1cm}call mpi\_initialize\_alm\_tools(comm)\\
\hspace*{1cm}call mpi\_alm2map\_slave\\
end\\
call mpi\_cleanup\_alm\_tools\\
}
{
This example 1) initializes the mpi\_alm\_tools module (i.e.,
allocates internal arrays and defines required parameters), 2)
executes a parallel alm2map operation, and 3) frees the previously
allocated memory.
}
\end{example}

\begin{modules}
  \begin{sulist}{} %%%% NOTE the ``extra'' brace here %%%%
  \item[\textbf{alm\_tools}] module
  \end{sulist}
\end{modules}

\begin{related}
  \begin{sulist}{} %%%% NOTE the ``extra'' brace here %%%%
   \item[\htmlref{mpi\_cleanup\_alm\_tools}{sub:mpi_cleanup_alm_tools}] Frees memory that is allocated by the current routine. 
   \item[\htmlref{mpi\_initialize\_alm\_tools}{sub:mpi_initialize_alm_tools}] Allocates memory and defines variables for the mpi\_alm\_tools module. 
  \item[\htmlref{mpi\_alm2map}{sub:mpi_alm2map}] Routine for executing a parallel inverse spherical harmonics transform (root processor interface)
  \item[\htmlref{mpi\_map2alm}{sub:mpi_map2alm}] Routine for executing a parallel spherical harmonics transform (root processor interface)
  \item[\htmlref{mpi\_map2alm\_slave}{sub:mpi_map2alm_slave}] Routine for executing a parallel spherical harmonics transform (slave processor interface)
  \item[\htmlref{mpi\_alm2map\_simple}{sub:mpi_alm2map_simple}] One-line interface to the parallel inverse spherical harmonics transform 
  \item[\htmlref{mpi\_map2alm\_simple}{sub:mpi_map2alm_simple}] One-line interface to the parallel spherical harmonics transform 
  \end{sulist}
\end{related}

\rule{\hsize}{2mm}

\newpage


\sloppy


%%%\title{\healpix Fortran Subroutines Overview}
\docid{mpi\_cleanup\_alm\_tools} \section[mpi\_cleanup\_alm\_tools]{ }
\label{sub:mpi_cleanup_alm_tools}
\docrv{Version 1.0}
\author{Hans K. Eriksen}
\abstract{This document describes the \healpix Fortran 90 subroutine MPI\_CLEANUP\_ALM\_TOOLS.}

\begin{facility}
{This subroutine deallocates any private arrays previously allocated
in the mpi\_alm\_tools module. It should be run (without arguments) by
all processors in the current communicator after the last call to any
of the working routines. 
}
{\modMpiAlmTools}
\end{facility}

\begin{f90format}
{}
\end{f90format}

\begin{arguments}
{
None.
}
\end{arguments}
%%\newpage

\begin{example}
{
call mpi\_comm\_rank(comm, myid, ierr)\\
if (myid == root) then\\
\hspace*{1cm}call mpi\_initialize\_alm\_tools(comm, nsmax, nlmax, nmmax, \\
\hspace*{3cm}zbounds,polarization, precompute\_plms)\\
\hspace*{1cm}call mpi\_map2alm(map, alms)\\
else \\
\hspace*{1cm}call mpi\_initialize\_alm\_tools(comm)\\
\hspace*{1cm}call mpi\_map2alm\_slave\\
end\\
call mpi\_cleanup\_alm\_tools\\
}
{
This example 1) initializes the mpi\_alm\_tools module (i.e.,
allocates internal arrays and defines required parameters), 2)
executes a parallel map2alm operation, and 3) frees the previously
allocated memory.
}
\end{example}

\begin{related}
  \begin{sulist}{} %%%% NOTE the ``extra'' brace here %%%%
   \item[\htmlref{mpi\_initialize\_alm\_tools}{sub:mpi_initialize_alm_tools}] Allocates memory and defines variables for the mpi\_alm\_tools module. 
  \item[\htmlref{mpi\_alm2map}{sub:mpi_alm2map}] Routine for executing a parallel inverse spherical harmonics transform (root processor interface)
  \item[\htmlref{mpi\_alm2map\_slave}{sub:mpi_alm2map_slave}] Routine for executing a parallel inverse spherical harmonics transform (slave processor interface)
  \item[\htmlref{mpi\_map2alm}{sub:mpi_map2alm}] Routine for executing a parallel spherical harmonics transform (root processor interface)
  \item[\htmlref{mpi\_map2alm\_slave}{sub:mpi_map2alm_slave}] Routine for executing a parallel spherical harmonics transform (slave processor interface)
  \item[\htmlref{mpi\_alm2map\_simple}{sub:mpi_alm2map_simple}] One-line interface to the parallel inverse spherical harmonics transform 
  \item[\htmlref{mpi\_map2alm\_simple}{sub:mpi_map2alm_simple}] One-line interface to the parallel spherical harmonics transform 
  \end{sulist}
\end{related}

\rule{\hsize}{2mm}

\newpage


\sloppy


\title{\healpix Fortran Subroutines Overview}
\docid{mpi\_initialize\_alm\_tools} \section[mpi\_initialize\_alm\_tools]{ }
\label{sub:mpi_initialize_alm_tools}
\docrv{Version 1.0}
\author{Hans K. Eriksen}
\abstract{This document describes the \healpix Fortran 90 subroutine
MPI\_INITIALIZE\_ALM\_TOOLS*.}  

\begin{facility}
{This subroutine initializes the mpi\_alm\_tools module, and must be
run prior to any of the advanced interface working routines by all
processors in the MPI communicator. The root processor must supply all arguments, 
while it is optional for the slaves. However, the information is disregarded 
if they do.\\
A major advantage of MPI parallelization is large quantities
of memory, allowing for pre-computation of the Legendre 
polynomials even with high $\nside$ and
$\lmax$, since each processor only needs a fraction
$(1/N_{\mathrm{procs}})$ of the complete table. This feature is
controlled by the ``precompute\_plms'' parameter. In general, the CPU
time can be expected to decrease by roughly 50\% using pre-computed
Legendre polynomials for temperature calculations, and by about 30\%
for polarization calculations.
}
{\modMpiAlmTools}
\end{facility}

\begin{f90format}
{\mylink{sub:mpi_initialize_alm_tools:comm}{comm}%
, [\mylink{sub:mpi_initialize_alm_tools:nsmax}{nsmax}%
], [\mylink{sub:mpi_initialize_alm_tools:nlmax}{nlmax}%
], [\mylink{sub:mpi_initialize_alm_tools:nmmax}{nmmax}%
], [\mylink{sub:mpi_initialize_alm_tools:zbounds}{zbounds}%
], [\mylink{sub:mpi_initialize_alm_tools:polarization}{polarization}%
], [\mylink{sub:mpi_initialize_alm_tools:precompute_plms}{precompute\_plms}%
], [\mylink{sub:mpi_initialize_alm_tools:w8ring}{w8ring}%
]}
\end{f90format}

\begin{arguments}
{
\begin{tabular}{p{0.4\hsize} p{0.05\hsize} p{0.05\hsize} p{0.40\hsize}} \hline  
\textbf{name~\&~dimensionality} & \textbf{kind} & \textbf{in/out} & \textbf{description} \\ \hline
                   &   &   &                           \\ %%% for presentation
comm\mytarget{sub:mpi_initialize_alm_tools:comm}  & I4B & IN & MPI communicator. \\
nsmax\mytarget{sub:mpi_initialize_alm_tools:nsmax} & I4B & IN & the $\nside$ value of the HEALPix map. (OPTIONAL) \\
nlmax\mytarget{sub:mpi_initialize_alm_tools:nlmax} & I4B & IN & the maximum $\ell$ value used for the $a_{\ell m}$. (OPTIONAL) \\
nmmax\mytarget{sub:mpi_initialize_alm_tools:nmmax} & I4B & IN & the maximum $m$ value used for the $a_{\ell m}$. (OPTIONAL) \\
\end{tabular}
\begin{tabular}{p{0.4\hsize} p{0.05\hsize} p{0.05\hsize} p{0.40\hsize}} \hline  
zbounds\mytarget{sub:mpi_initialize_alm_tools:zbounds}(1:2) & DP & IN & section of the map on which to perform the $a_{\ell m}$
                   analysis, expressed in terms of $z=\sin({\rm latitude}) =
                   \cos(\theta)$. If zbounds(1)$<$zbounds(2), the analysis is
                   performed {\em on} the strip zbounds(1)$<z<$zbounds(2); if not,
                   it is performed {\em outside} of the strip
                   zbounds(2)$<z<$zbounds(1). (OPTIONAL) \\
polarization\mytarget{sub:mpi_initialize_alm_tools:polarization} & LGT & IN & if polarization is required, this should be
set to true, else it should be set to false. (OPTIONAL) \\
precompute\_plms\mytarget{sub:mpi_initialize_alm_tools:precompute_plms} & I4B & IN & 0 = do not pre-compute any $P_{\ell
m}$'s; 1 = pre-compute $P_{\ell m}^\mathrm{T}$; 2 = pre-compute
$P_{\ell m}^\mathrm{T}$ and $P_{\ell m}^\mathrm{P}$.  (OPTIONAL)\\
w8ring\mytarget{sub:mpi_initialize_alm_tools:w8ring}\_TQU(1:2*nsmax, 1:p) & DP & IN & ring weights for quadrature corrections. If ring weights are not used, this array should be 1 everywhere. p is 1 for a temperature analysis and 3 for (T,Q,U). (OPTIONAL)\\
\end{tabular}
}
\end{arguments}


\begin{example}
{
call mpi\_comm\_rank(comm, myid, ierr)\\
if (myid == root) then\\
\hspace*{1cm}call mpi\_initialize\_alm\_tools(comm, nsmax, nlmax, nmmax, \\
\hspace*{3cm}zbounds,polarization, precompute\_plms)\\
\hspace*{1cm}call mpi\_map2alm(map, alms)\\
else \\
\hspace*{1cm}call mpi\_initialize\_alm\_tools(comm)\\
\hspace*{1cm}call mpi\_map2alm\_slave\\
end\\
call mpi\_cleanup\_alm\_tools\\
}
{
This example 1) initializes the mpi\_alm\_tools module (i.e.,
allocates internal arrays and defines required parameters), 2)
executes a parallel map2alm operation, and 3) frees the previously
allocated memory.
}
\end{example}

\begin{related}
  \begin{sulist}{} %%%% NOTE the ``extra'' brace here %%%%
   \item[\htmlref{mpi\_cleanup\_alm\_tools}{sub:mpi_cleanup_alm_tools}] Frees memory that is allocated by the current routine. 
  \item[\htmlref{mpi\_alm2map}{sub:mpi_alm2map}] Routine for executing a parallel inverse spherical harmonics transform (root processor interface)
  \item[\htmlref{mpi\_alm2map\_slave}{sub:mpi_alm2map_slave}] Routine for executing a parallel inverse spherical harmonics transform (slave processor interface)
  \item[\htmlref{mpi\_map2alm}{sub:mpi_map2alm}] Routine for executing a parallel spherical harmonics transform (root processor interface)
  \item[\htmlref{mpi\_map2alm\_slave}{sub:mpi_map2alm_slave}] Routine for executing a parallel spherical harmonics transform (slave processor interface)
  \item[\htmlref{mpi\_alm2map\_simple}{sub:mpi_alm2map_simple}] One-line interface to the parallel inverse spherical harmonics transform 
  \item[\htmlref{mpi\_map2alm\_simple}{sub:mpi_map2alm_simple}] One-line interface to the parallel spherical harmonics transform 
  \end{sulist}
\end{related}

\rule{\hsize}{2mm}

\newpage


\sloppy


\title{\healpix Fortran Subroutines Overview}
\docid{mpi\_map2alm*} \section[mpi\_map2alm*]{ }
\label{sub:mpi_map2alm}
\docrv{Version 1.0}
\author{Hans K. Eriksen}
\abstract{This document describes the \healpix Fortran 90 subroutine MPI\_MAP2ALM*.}

\begin{facility}
{This subroutine implements MPI parallelization of the serial map2alm
routine. It supports both temperature and polarization inputs in both
single and double precision. It must only be run by the root node of
the MPI communicator.
}
{\modMpiAlmTools}
\end{facility}

\begin{f90format}
{\mylink{sub:mpi_map2alm:map}{map}%
, \mylink{sub:mpi_map2alm:alms}{alms}%
}
\end{f90format}

\begin{arguments}
{
\begin{tabular}{p{0.4\hsize} p{0.05\hsize} p{0.05\hsize} p{0.40\hsize}} \hline  
\textbf{name~\&~dimensionality} & \textbf{kind} & \textbf{in/out} & \textbf{description} \\ \hline
                   &   &   &                           \\ %%% for presentation
map\mytarget{sub:mpi_map2alm:map}(0:npix,1:nmaps) & SP or DP & IN & map to analyse. If
nmaps=1, only temperature information is included; if nmaps=3,
polarization information is included \\
alms\mytarget{sub:mpi_map2alm:alms}(1:nmaps,0:lmax,0:nmax) & SPC or DPC & OUT & output alms. nmaps must
equal that of the input map\\
\end{tabular}
}
\end{arguments}
%%\newpage

\begin{example}
{
call mpi\_comm\_rank(comm, myid, ierr)\\
if (myid == root) then\\
\hspace*{1cm}call mpi\_initialize\_alm\_tools(comm, nsmax, nlmax, nmmax, \\
\hspace*{3cm}zbounds,polarization, precompute\_plms)\\
\hspace*{1cm}call mpi\_map2alm(map, alms)\\
else \\
\hspace*{1cm}call mpi\_initialize\_alm\_tools(comm)\\
\hspace*{1cm}call mpi\_map2alm\_slave\\
end\\
call mpi\_cleanup\_alm\_tools\\
}
{
This example 1) initializes the mpi\_alm\_tools module (i.e.,
allocates internal arrays and defines required parameters), 2)
executes a parallel map2alm operation, and 3) frees the previously
allocated memory.
}
\end{example}

\begin{modules}
  \begin{sulist}{} %%%% NOTE the ``extra'' brace here %%%%
  \item[\textbf{alm\_tools}] module
  \end{sulist}
\end{modules}

\begin{related}
  \begin{sulist}{} %%%% NOTE the ``extra'' brace here %%%%
   \item[\htmlref{mpi\_cleanup\_alm\_tools}{sub:mpi_cleanup_alm_tools}] Frees memory that is allocated by the current routine. 
   \item[\htmlref{mpi\_initialize\_alm\_tools}{sub:mpi_initialize_alm_tools}] Allocates memory and defines variables for the mpi\_alm\_tools module. 
  \item[\htmlref{mpi\_alm2map}{sub:mpi_alm2map}] Routine for executing a parallel inverse spherical harmonics transform (root processor interface)
  \item[\htmlref{mpi\_alm2map\_slave}{sub:mpi_alm2map_slave}] Routine for executing a parallel inverse spherical harmonics transform (slave processor interface)
  \item[\htmlref{mpi\_map2alm\_slave}{sub:mpi_map2alm_slave}] Routine for executing a parallel spherical harmonics transform (slave processor interface)
  \item[\htmlref{mpi\_alm2map\_simple}{sub:mpi_alm2map_simple}] One-line interface to the parallel inverse spherical harmonics transform 
  \item[\htmlref{mpi\_map2alm\_simple}{sub:mpi_map2alm_simple}] One-line interface to the parallel spherical harmonics transform 
  \end{sulist}
\end{related}

\rule{\hsize}{2mm}

\newpage


\sloppy


\title{\healpix Fortran Subroutines Overview}
\docid{mpi\_map2alm\_simple*} \section[mpi\_map2alm\_simple*]{ }
\label{sub:mpi_map2alm_simple}
\docrv{Version 1.0}
\author{Hans K. Eriksen}
\abstract{This document describes the \healpix Fortran 90 subroutine MPI\_MAP2ALM\_SIMPLE*.}

\begin{facility}
{This subroutine provides a simplified (one-line) interface to the MPI version of
map2alm. It supports both temperature and polarization inputs in both
single and double precision. It must only be run by all processors in 
the MPI communicator.
}
{\modMpiAlmTools}
\end{facility}

\begin{f90format}
{\mylink{sub:mpi_map2alm_simple:comm}{comm}%
, \mylink{sub:mpi_map2alm_simple:map}{map}%
, \mylink{sub:mpi_map2alm_simple:alms}{alms}%
, [\mylink{sub:mpi_map2alm_simple:zbounds}{zbounds}%
], [\mylink{sub:mpi_map2alm_simple:w8ring}{w8ring}%
]}
\end{f90format}

\begin{arguments}
{
\begin{tabular}{p{0.4\hsize} p{0.05\hsize} p{0.05\hsize} p{0.40\hsize}} \hline  
\textbf{name~\&~dimensionality} & \textbf{kind} & \textbf{in/out} & \textbf{description} \\ \hline
                   &   &   &                           \\ %%% for presentation
comm\mytarget{sub:mpi_map2alm_simple:comm} & I4B & IN & MPI communicator. \\
map\mytarget{sub:mpi_map2alm_simple:map}(0:npix-1,1:nmaps) & SP or DP & IN & input map. If
nmaps=1, only temperature information is included; if nmaps=3,
polarization information is included\\
alms\mytarget{sub:mpi_map2alm_simple:alms}(1:nmaps,0:lmax,0:nmax) & SPC or DPC & IN & output alms. 
nmaps must
equal that of the input map\\
zbounds\mytarget{sub:mpi_map2alm_simple:zbounds}(1:2) & DP & IN & section of the map on which to perform the $a_{lm}$
                   analysis, expressed in terms of $z=\sin({\rm latitude}) =
                   \cos(\theta)$. If zbounds(1)$<$zbounds(2), the analysis is
                   performed {\em on} the strip zbounds(1)$<z<$zbounds(2); if not,
                   it is performed {\em outside} of the strip
                   zbounds(2)$<z<$zbounds(1). (OPTIONAL) \\
\end{tabular}
\begin{tabular}{p{0.4\hsize} p{0.05\hsize} p{0.05\hsize} p{0.40\hsize}} \hline  
w8ring\mytarget{sub:mpi_map2alm_simple:w8ring}\_TQU(1:2*nsmax, 1:p) & DP & IN & ring weights for quadrature corrections. If ring weights are not used, this array should be 1 everywhere. p is 1 for a temperature analysis and 3 for (T,Q,U). (OPTIONAL)\\
\end{tabular}
}
\end{arguments}
%%\newpage

\begin{example}
{
\hspace*{1cm}call mpi\_map2alm\_simple(comm, map, alms)\\
}
{
This example executes a parallel map2alm operation through the
one-line interface. Although all processors must supply allocated
arrays to the routine, only the root processor's information will be
used as input, and only the root processor's alms will be complete
after execution. 
}
\end{example}

\begin{modules}
  \begin{sulist}{} %%%% NOTE the ``extra'' brace here %%%%
  \item[\textbf{alm\_tools}] module
  \end{sulist}
\end{modules}

\begin{related}
  \begin{sulist}{} %%%% NOTE the ``extra'' brace here %%%%
   \item[\htmlref{mpi\_cleanup\_alm\_tools}{sub:mpi_cleanup_alm_tools}] Frees memory that is allocated by the current routine. 
   \item[\htmlref{mpi\_initialize\_alm\_tools}{sub:mpi_initialize_alm_tools}] Allocates memory and defines variables for the mpi\_alm\_tools module. 
  \item[\htmlref{mpi\_alm2map}{sub:mpi_alm2map}] Routine for executing a parallel inverse spherical harmonics transform (root processor interface)
  \item[\htmlref{mpi\_alm2map\_slave}{sub:mpi_alm2map_slave}] Routine for executing a parallel inverse spherical harmonics transform (slave processor interface)
  \item[\htmlref{mpi\_map2alm}{sub:mpi_map2alm}] Routine for executing a parallel spherical harmonics transform (root processor interface)
  \item[\htmlref{mpi\_map2alm\_slave}{sub:mpi_map2alm_slave}] Routine for executing a parallel spherical harmonics transform (slave processor interface)
  \item[\htmlref{mpi\_alm2map\_simple}{sub:mpi_alm2map_simple}] One-line interface to the parallel inverse spherical harmonics transform 
  \end{sulist}
\end{related}


\rule{\hsize}{2mm}

\newpage


\sloppy


%%%\title{\healpix Fortran Subroutines Overview}
\docid{mpi\_map2alm\_slave} \section[mpi\_map2alm\_slave]{ }
\label{sub:mpi_map2alm_slave}
\docrv{Version 1.0}
\author{Hans K. Eriksen}
\abstract{This document describes the \healpix Fortran 90 subroutine MPI\_MAP2ALM\_SLAVE.}

\begin{facility}
{This subroutine complements the master routine mpi\_map2alm, and
should be run by all slaves in the current MPI communicator. It is run
without arguments, but after an appropriate call to
initialize\_mpi\_alm\_tools. 
}
{\modMpiAlmTools}
\end{facility}

\begin{f90format}
{}
\end{f90format}

\begin{arguments}
{
None.
}
\end{arguments}
%%\newpage

\begin{example}
{
call mpi\_comm\_rank(comm, myid, ierr)\\
if (myid == root) then\\
\hspace*{1cm}call mpi\_initialize\_alm\_tools(comm, nsmax, nlmax, nmmax, \\
\hspace*{3cm}zbounds,polarization, precompute\_plms)\\
\hspace*{1cm}call mpi\_map2alm(map, alms)\\
else \\
\hspace*{1cm}call mpi\_initialize\_alm\_tools(comm)\\
\hspace*{1cm}call mpi\_map2alm\_slave\\
end\\
call mpi\_cleanup\_alm\_tools\\
}
{
This example 1) initializes the mpi\_alm\_tools module (i.e.,
allocates internal arrays and defines required parameters), 2)
executes a parallel map2alm operation, and 3) frees the previously
allocated memory.
}
\end{example}

\begin{modules}
  \begin{sulist}{} %%%% NOTE the ``extra'' brace here %%%%
  \item[\textbf{alm\_tools}] module
  \end{sulist}
\end{modules}

\begin{related}
  \begin{sulist}{} %%%% NOTE the ``extra'' brace here %%%%
   \item[\htmlref{mpi\_cleanup\_alm\_tools}{sub:mpi_cleanup_alm_tools}] Frees memory that is allocated by the current routine. 
   \item[\htmlref{mpi\_initialize\_alm\_tools}{sub:mpi_initialize_alm_tools}] Allocates memory and defines variables for the mpi\_alm\_tools module. 
  \item[\htmlref{mpi\_alm2map}{sub:mpi_alm2map}] Routine for executing a parallel inverse spherical harmonics transform (root processor interface)
  \item[\htmlref{mpi\_alm2map\_slave}{sub:mpi_alm2map_slave}] Routine for executing a parallel inverse spherical harmonics transform (slave processor interface)
  \item[\htmlref{mpi\_map2alm}{sub:mpi_map2alm}] Routine for executing a parallel spherical harmonics transform (root processor interface)
  \item[\htmlref{mpi\_alm2map\_simple}{sub:mpi_alm2map_simple}] One-line interface to the parallel inverse spherical harmonics transform 
  \item[\htmlref{mpi\_map2alm\_simple}{sub:mpi_map2alm_simple}] One-line interface to the parallel spherical harmonics transform 
  \end{sulist}
\end{related}

\rule{\hsize}{2mm}

\newpage

\sloppy

\title{\healpix Fortran Subroutines Overview}
\docid{nArguments} \section[nArguments]{ }
\label{sub:narguments}
\docrv{Version 1.0}
\author{Eric Hivon}
\abstract{This document describes the \healpix Fortran90 subroutine
narguments.}

\begin{facility}
{This function emulates the C routine {\tt iargc}, which returns the number of
command line arguments provided.\\
Starting with release 3.60, it calls the F2003 extension function \texttt{command\_argument\_count}.}
{\modExtension}
\end{facility}

\begin{f90function}
{\ }
\end{f90function}

\begin{arguments}
{
\begin{tabular}{p{0.3\hsize} p{0.05\hsize} p{0.1\hsize} p{0.45\hsize}} \hline  
\textbf{name\&dimensionality} & \textbf{kind} & \textbf{in/out} & \textbf{description} \\ \hline
                   &   &   &                           \\ %%% for presentation
var & I4B & OUT & number of command line arguments

\end{tabular}}
\end{arguments}

% \begin{example}
% {
% use extension \\
% character(len=128) :: healpixdir \\
% call getargument('HEALPIX', healpixdir) \\
% print*,healpixdir
% }
% {
% Will return the value of the {\tt \$HEALPIX} system variable (if it is defined)
% }
% \end{example}

\begin{related}
  \begin{sulist}{} %%%% NOTE the ``extra'' brace here %%%%
  \item[\htmlref{getEnvironment}{sub:getenvironment}] returns value of
  environment variable
  \item[\htmlref{getArgument}{sub:getargument}] returns list of command line arguments
%  \item[\htmlref{nArguments}{sub:narguments}] returns number of command line arguments
  \end{sulist}
\end{related}

\rule{\hsize}{2mm}

\newpage


\sloppy

\title{\healpix Fortran Subroutines Overview}
\docid{neighbours\_nest} \section[neighbours\_nest]{ }
\label{sub:neighbours_nest}
\docrv{Version 2.0}
\author{}
\abstract{This document describes the \healpix Fortran90 subroutine NEIGHBOURS\_NEST.}

\begin{facility}
{This subroutine returns the number and locations (in terms of pixel
numbers) of the topological neighbours of a central pixel. The pixels
are ordered in a clockwise sense about the central pixel with the
southernmost pixel in first element. For the 4 pixels in the southern corners of the
equatorial faces which have two equally southern neighbours the
routine returns the southwestern pixel first and proceeds clockwise.}
{\modPixTools}
\end{facility}

\begin{f90format}
{\mylink{sub:neighbours_nest:nside}{nside}%
, \mylink{sub:neighbours_nest:ipix}{ipix}%
, \mylink{sub:neighbours_nest:list}{list}%
, \mylink{sub:neighbours_nest:nneigh}{nneigh}%
}
\end{f90format}

\begin{arguments}
{
\begin{tabular}{p{0.4\hsize} p{0.05\hsize} p{0.1\hsize} p{0.35\hsize}} \hline  
\textbf{name~\&~dimensionality} & \textbf{kind} & \textbf{in/out} & \textbf{description} \\ \hline
                   &   &   &                           \\ %%% for presentation
nside\mytarget{sub:neighbours_nest:nside} & I4B & IN & The $\nside$ parameter of the map. \\
ipix\mytarget{sub:neighbours_nest:ipix} & I4B/ I8B & IN & The NESTED pixel index of the central pixel. \\
list\mytarget{sub:neighbours_nest:list}(8) & I4B/ I8B & OUT & On exit, the vector of neighbouring pixels. This
                   contains {\tt nneigh} relevant elements.\\
nneigh\mytarget{sub:neighbours_nest:nneigh} & I4B & OUT & The number of neighbours (mostly 8, except at
                   8 sites, where there are only 7 neighbours).\\
\end{tabular}
}
\end{arguments}

\begin{example}
{
use pix\_tools \\
integer :: nneigh, list(1:8) \\
call neighbours\_nest(4, 1, list, nneigh)  \\
print*,nneigh \\
print*,list(1:nneigh)
}
{
This returns {\tt nneigh}$=8$ and a vector {\tt list}, which contains the pixel
numbers ( 90,  0,  2,  3,  6,  4,  94,  91).}
\end{example}

\begin{modules}
  \begin{sulist}{} %%%% NOTE the ``extra'' brace here %%%%
 \item[mk\_xy2pix, mk\_pix2xy] precomputing arrays for the conversion
 of NESTED pixel numbers to Cartesian coords in each face.
 \item[pix2xy\_nest, xy2pix\_nest] Conversion between NESTED pixel numbers to Cartesian coords in each face.
 \item[\textbf{bit\_manipulation}] module, containing:
 \item[invMSB, invLSB,swapLSBMSB,invswapLSBMSB] functions which manipulate the bit vector which
 represents the NESTED pixel numbers. They correspond to
 NorthWest<->SouthEast, SouthWest<->NorthEast, East<->West and
 North-South flips of the diamond faces, respectively.
  \end{sulist}
\end{modules}

\begin{related}
  \begin{sulist}{} %%%% NOTE the ``extra'' brace here %%%%
  \item[\htmlref{pix2ang}{sub:pix_tools}, \htmlref{ang2pix}{sub:pix_tools}] convert between angle and pixel number.
  \item[\htmlref{pix2vec}{sub:pix_tools}, \htmlref{vec2pix}{sub:pix_tools}] convert between a cartesian vector and pixel number.
  \end{sulist}
\end{related}

\rule{\hsize}{2mm}




\sloppy


\title{\healpix Fortran Subroutines Overview}
\docid{nest2uniq} \section[nest2uniq]{ }
\label{sub:nest2uniq}
\docrv{Version 1.0}
\author{E. Hivon}
\abstract{This document describes the \healpix Fortran90 subroutines
  NEST2UNIQ.}

\begin{facility}
{This F90 facility turns the
parameter $\nside$ (a power of 2) and the pixel index $p$ into the Unique ID number $u = p + 4 \nside^2$.
See \htmlref{''The Unique Identifier scheme''}{intro:unique} section in 
\linklatexhtml{''\healpix Introduction Document''}{intro.pdf}{intro.htm} for more details.
}
{\modPixTools}
\end{facility}

\begin{f90format}
{%
\mylink{sub:nest2uniq:nside}{nside}, 
\mylink{sub:nest2uniq:pnest}{pnest},
\mylink{sub:nest2uniq:puniq}{puniq}}
\end{f90format}

\begin{arguments}
{
\begin{tabular}{p{0.10\hsize} p{0.1\hsize} p{0.1\hsize} p{0.60\hsize}} \hline  
\textbf{name} & \textbf{kind} & \textbf{in/out} & \textbf{description} \\ \hline
                   &   &   &                           \\ %%% for presentation
nside \mytarget{sub:nest2uniq:nside} & I4B     & IN & The \healpix $\nside$ parameter. \\
pnest \mytarget{sub:nest2uniq:pnest} & I4B/I8B & IN & (NESTED scheme) pixel identification number over the range \{0,$12\nside^2-1$\}.\\
puniq \mytarget{sub:nest2uniq:puniq} & I4B/I8B & OUT & The \healpix Unique pixel identifier. 
\end{tabular}
}
\end{arguments}

\begin{example}
{use \htmlref{healpix\_modules}{sub:healpix_modules}\\
integer(I4B) :: puniq \\
call nest2uniq(1, 0, puniq)\\
print*,puniq
}
{
\begin{minipage}{11cm}
returns  \\
     4 \\
since the first pixel ($p=0$) at $\nside=$ 1 is the pixel with Unique ID number 4.
\end{minipage}
}
\end{example}

\begin{related}
  \begin{sulist}{} %%%% NOTE the ``extra'' brace here %%%%
  \item[\htmlref{uniq2nest}{sub:uniq2nest}] ] Transforms  Unique \healpix pixel ID number into Nside and Nested pixel number
  \item[\htmlref{pix2xxx, ...}{sub:pix_tools}] to turn NESTED pixel index into sky coordinates and back
  \end{sulist}
\end{related}

\rule{\hsize}{2mm}




\sloppy


%%%\title{\healpix Fortran Subroutines Overview}
\docid{npix2nside} \section[npix2nside]{ }
\label{sub:npix2nside}
\docrv{Version 1.1}
\author{E. Hivon}
\abstract{This document describes the \healpix Fortran90 subroutine NPIX2NSIDE.}

\begin{facility}
{Function to provide the resolution parameter $\nside$ correspoonding to $\npix$
pixels over the full sky. 
}
{\modPixTools}
\end{facility}

\begin{f90function}
{\mylink{sub:npix2nside:npix}{npix}%
}
\end{f90function}

\begin{arguments}
{
\begin{tabular}{p{0.3\hsize} p{0.05\hsize} p{0.1\hsize} p{0.45\hsize}} \hline  
\textbf{name~\&~dimensionality} & \textbf{kind} & \textbf{in/out} & \textbf{description} \\ \hline
                   &   &   &                           \\ %%% for presentation
npix\mytarget{sub:npix2nside:npix} & I4B/ I8B & IN & the number $\npix$ of pixels over the whole sky. \\
var & I4B & OUT & the parameter $\nside$. If $\npix$ is valid (12 times a power of 2 in
$\{1,\ldots,2^{28}\}$), $\nside=\sqrt{\npix/12}$ is returned; if not, an error message is
issued and -1 is returned.\\
\end{tabular}
}
\end{arguments}

\begin{example}
{
use \htmlref{healpix\_modules}{sub:healpix_modules} \\
integer(\htmlref{I4B}{sub:healpix_types}) :: nside \\
nside= npix2nside(786432)  \\
}
{
Returns the resolution parameter $\nside$ (256) corresponding to 786432 pixels
on the sky.
}
\end{example}
\begin{related}
  \begin{sulist}{} %%%% NOTE the ``extra'' brace here %%%%
  \item[\htmlref{nside2npix}{sub:nside2npix}] returns the number of pixels $\npix$ correspondinng to
  resolution parameter $\nside$
  \end{sulist}
\end{related}

\rule{\hsize}{2mm}



\sloppy


\title{\healpix Fortran Subroutines Overview}
\docid{nside2npix} \section[nside2npix]{ }
\label{sub:nside2npix}
\docrv{Version 1.1}
\author{E. Hivon}
\abstract{This document describes the \healpix Fortran90 subroutine NSIDE2NPIX.}

\begin{facility}
{Function to provide the number of pixels $\npix$ over the full sky corresponding
to resolution parameter $\nside$. 
}
{\modPixTools}
\end{facility}

\begin{f90function}
{\mylink{sub:nside2npix:nside}{nside}%
}
\end{f90function}

\begin{arguments}
{
\begin{tabular}{p{0.3\hsize} p{0.05\hsize} p{0.1\hsize} p{0.45\hsize}} \hline  
\textbf{name~\&~dimensionality} & \textbf{kind} & \textbf{in/out} & \textbf{description} \\ \hline
                   &   &   &                           \\ %%% for presentation
nside\mytarget{sub:nside2npix:nside} & I4B & IN & the $\nside$ parameter of the map. \\
var & I8B & OUT & the number of pixels $\npix$ of the map. If $\nside$ is valid (a power of 2 in
$\{1,\ldots,2^{28}=268435456\}$), $\npix=12\nside^2$ is returned; if not, an error message is
issued and -1 is returned.\\
\end{tabular}
}
\end{arguments}

\begin{example}
{
use \htmlref{healpix\_modules}{sub:healpix_modules} \\
integer(\htmlref{I8B}{sub:healpix_types}) :: npix \\
npix= nside2npix(256)  \\
}
{
Returns the number of \healpix pixels (786432) for the resolution
parameter 256.
}
\end{example}
\begin{related}
  \begin{sulist}{} %%%% NOTE the ``extra'' brace here %%%%
  \item[\htmlref{npix2nside}{sub:npix2nside}] returns resolution parameter corresponding to the number of pixels.
%   \item[pix2xxx] conversion between pixel index and position on the sky.
  \end{sulist}
\end{related}

\rule{\hsize}{2mm}



\sloppy


%%%\title{\healpix Fortran Subroutines Overview}
\docid{nside2ntemplates} \section[nside2ntemplates]{ }
\label{sub:nside2ntemplates}
\docrv{Version 1.1}
\author{E. Hivon}
\abstract{This document describes the \healpix Fortran90 subroutine NSIDE2NTEMPLATES.}

\begin{facility}
{Function to provide the number of template pixels $$\ntemplate=\frac{1+\nside(\nside+6)}{4}$$ corresponding
to resolution parameter $\nside$. Each template pixel has a different shape that
{\em can not} be matched (by rotation or reflexion) to that of any of the other templates.
}
{\modPixTools}
\end{facility}

\begin{f90function}
{\mylink{sub:nside2ntemplates:nside}{nside}%
}
\end{f90function}

\begin{arguments}
{
\begin{tabular}{p{0.3\hsize} p{0.05\hsize} p{0.1\hsize} p{0.45\hsize}} \hline  
\textbf{name~\&~dimensionality} & \textbf{kind} & \textbf{in/out} & \textbf{description} \\ \hline
                   &   &   &                           \\ %%% for presentation
nside\mytarget{sub:nside2ntemplates:nside} & I4B & IN & the $\nside$ parameter. \\
var & I8B & OUT & the number of template pixels $\ntemplate$.\\
\end{tabular}
}
\end{arguments}

\begin{example}
{
use \htmlref{healpix\_modules}{sub:healpix_modules} \\
integer(\htmlref{I8B}{sub:healpix_types}) :: ntpl \\
ntpl= nside2ntemplates(256)  \\
}
{
Returns in {\tt ntpl} the number of \healpix template pixels (16768) for the resolution
parameter 256.
}
\end{example}
\begin{related}
  \begin{sulist}{} %%%% NOTE the ``extra'' brace here %%%%
  \item[\htmlref{template\_pixel\_ring}{sub:template_pixel_xxx}]
  \item[\htmlref{template\_pixel\_nest}{sub:template_pixel_xxx}] return the
  template pixel associated with any \healpix pixel
  \item[\htmlref{same\_shape\_pixels\_ring}{sub:same_shape_pixels_xxx}] 
  \item[\htmlref{same\_shape\_pixels\_nest}{sub:same_shape_pixels_xxx}] 
  return
  the ordered list of pixels having the same shape as a given pixel template
  \end{sulist}
\end{related}

\rule{\hsize}{2mm}



\sloppy


\title{\healpix Fortran Subroutines Overview}
\docid{number\_of\_alms} \section[number\_of\_alms]{ }
\label{sub:number_of_alms}
\docrv{Version 1.2}
\author{Frode K.~Hansen, Eric Hivon}
\abstract{This document describes the \healpix Fortran90 function NUMBER\_OF\_ALMS.}

\begin{facility}
{This function returns the number of $a_{lm}$ values stored in each FITS extension in a FITS file containing $a_{lm}$}
{\modFitstools}
\end{facility}

\begin{f90function}
{\mylink{sub:number_of_alms:filename}{filename}%
[, \mylink{sub:number_of_alms:extnum}{extnum}%
]}
\end{f90function}

\begin{arguments}
{
\begin{tabular}{p{0.4\hsize} p{0.05\hsize} p{0.1\hsize} p{0.35\hsize}} \hline  
\textbf{name~\&~dimensionality} & \textbf{kind} & \textbf{in/out} & \textbf{description} \\ \hline
                   &   &   &                           \\ %%% for presentation
filename\mytarget{sub:number_of_alms:filename}(LEN=\filenamelen) & CHR & IN & filename of the FITS-file containing
                   $a_{\ell m}$. \\
extnum\mytarget{sub:number_of_alms:extnum} & I4B & OUT & number of extensions in the file \\
\end{tabular}
}
\end{arguments}

\begin{example}
{
print*,number\_of\_alms('alms.fits')  \\
}
{
Prints the number of $a_{lm}$ stored in each extension of the file 'alms.fits'
}
\end{example}

\begin{modules}
  \begin{sulist}{} %%%% NOTE the ``extra'' brace here %%%%
  \item[\textbf{fitstools}] module, containing:
  \item[printerror] routine for printing FITS error messages.
  \item[\textbf{cfitsio}] library for FITS file handling.		
  \end{sulist}
\end{modules}
\newpage
\begin{related}
  \begin{sulist}{} %%%% NOTE the ``extra'' brace here %%%%
  \item[\htmlref{fits2alms}{sub:fits2alms}, \htmlref{read\_conbintab}{sub:read_conbintab}] routines that read $a_{lm}$ values from FITS files. 
  \end{sulist}
\end{related}

\rule{\hsize}{2mm}

\newpage


\sloppy

% corrected example, 2007-Jan
\title{\healpix Fortran Subroutines Overview}
\docid{output\_map*} \section[output\_map*]{ }
\label{sub:output_map}
\docrv{Version 1.3}
\author{Eric Hivon}
\abstract{This document describes the \healpix Fortran90 subroutine OUTPUT\_MAP.}

\begin{facility}
{This routine writes a full sky \healpix map into a FITS file. The map can be
  either single or double precision real. It {\em has} to be 2-dimensional.}
{\modFitstools}
\end{facility}

\begin{f90format}
{\mylink{sub:output_map:map}{map}%
, \mylink{sub:output_map:header}{header}%
, \mylink{sub:output_map:filename}{filename}%
 \optional{[,\mylink{sub:output_map:extno}{extno}%
]}}
\end{f90format}

\begin{arguments}
{
\begin{tabular}{p{0.3\hsize} p{0.05\hsize} p{0.08\hsize} p{0.5\hsize}} \hline  
\textbf{name~\&~dimensionality} & \textbf{kind} & \textbf{in/out} & \textbf{description} \\ \hline
                   &   &   &                           \\ %%% for presentation
map\mytarget{sub:output_map:map}(0:,1:) 		& SP/ DP 	& IN & full sky map(s) to be output \\
header\mytarget{sub:output_map:header}(LEN=80)(1:) 	& CHR 	& IN & string array containing the
                   FITS header to be included in the file \\
filename\mytarget{sub:output_map:filename}(LEN=*) & CHR & IN & filename of the FITS-file to
                   contain \healpix map(s). \\
\optional{extno\mytarget{sub:output_map:extno}} \hskip 4cm & I4B & IN & extension number in which to write the data (0
                   based). \default 0
\end{tabular}
}
\end{arguments}

\begin{example}
{
use healpix\_types \\
use fits\_tools, only : output\_map \\
real(sp), dimension(0:12*16**2-1, 1:1) :: map \\
character(len=80), dimension(1:10) :: header \\
header(:) = '' \\
map(:,:) = 1. \\
call output\_map(map, header, 'map.fits')
}
{generates a simple map (made of 1s) and outputs it into the FITS file {\tt map.fits}
}
\end{example}
\newpage
\begin{modules}
  \begin{sulist}{} %%%% NOTE the ``extra'' brace here %%%%
  \item[\textbf{fitstools}] module, containing:
  \item[printerror] routine for printing FITS error messages.
  \item[write\_bintab] routine to write a binary table into a FITS file.
  \item[\textbf{cfitsio}] library for FITS file handling.		
  \end{sulist}
\end{modules}

\begin{related}
  \begin{sulist}{} %%%% NOTE the ``extra'' brace here %%%%
  \item[anafast] executable that reads a \healpix map from a FITS file
  and analyses it. 
  \item[synfast] executable that generate full sky \healpix maps
  \item[\htmlref{input\_map}{sub:input_map}] subroutine to read a \healpix map from a a FITS file
  \item[\htmlref{write\_bintabh}{sub:write_bintabh}] subroutine to write a large
  array into a FITS file piece by piece
  \item[\htmlref{input\_tod*}{sub:input_tod}] subroutine to read an arbitrary subsection of
  a large binary table
  \item[\htmlref{write\_minimal\_header}{sub:write_minimal_header}] routine to
write minimal FITS header
  \end{sulist}
\end{related}

\rule{\hsize}{2mm}

\newpage



\sloppy

\title{\healpix Fortran Subroutines Overview}
\docid{parse\_init,~parse\_int,~...,~parse\_finish} \section[parse\_init, parse\_int, $\ldots$, parse\_finish]{ }
\label{sub:parse_xxx}
\docrv{Version 1.1}
\author{Martin Reinecke, Eric Hivon}
\abstract{This document describes the \healpix Fortran90 subroutines in module paramfile\_io.}

\begin{facility}
{The Fortran90 module paramfile\_io contains functions to obtain
parameters from parameter files or interactively}
{\modParamfileIo}
\end{facility}

\begin{arguments}
{
\begin{tabular}{p{0.3\hsize} p{0.05\hsize} p{0.1\hsize} p{0.45\hsize}} \hline  
\textbf{name\&dimensionality} & \textbf{kind} & \textbf{in/out} & \textbf{description} \\ \hline
                   &   &   &                           \\ %%% for presentation
fname & CHR & IN & file containing the simulation parameters.
                   If empty, parameters are obtained interactively.\\
handle & PMF & INOUT & Object of type (paramfile\_handle) used to store parameter information \\
keyname & CHR & IN & name of the required parameter \\
default & XXX & IN & optional argument containing the default value for
                     a given simulation parameter; must be of
                     appropriate type. \\
vmin & XXX & IN & optional argument containing the minimum value for
                     a given simulation parameter; must be of
                     appropriate type. \\
vmax & XXX & IN & optional argument containing the maximum value for
                     a given simulation parameter; must be of
                     appropriate type. \\
descr & CHR & IN & optional argument containing a description of the
                   required simulation parameter \\
filestatus & CHR & IN & optional argument. If present, the parameter
                   must be a valid filename. If filestatus=='new',
                   the file must not exist; if filestatus=='old',
                   the file must exist. \\
code & CHR & IN & optional argument. Contains the name of the executable.\\
silent & LGT & IN & optional argument. If set to {\tt .true.} the parsing
routines will run silently in non-interactive mode (except for warning or error
messages, which will always appear). This is mainly intended for MPI usage where
many processors parse the same parameter file: {\tt silent} can be set to
{\tt .true.} on all CPUs except one.
\end{tabular}
}
\end{arguments}
\newpage

\rule{\hsize}{0.7mm}
\textsc{\large{\textbf{ROUTINES: }}}\hfill\newline
{\tt handle = parse\_init (fname [,silent])} 

\quad initializes the parser to work on the file fname, or interactively, if fname is empty

{\tt intval = parse\_int (handle, keyname [, default, vmin, vmax, descr])} 

{\tt longval = parse\_long (handle, keyname [, default, vmin, vmax, descr])} 

{\tt realval = parse\_real (handle, keyname [, default, vmin, vmax, descr])} 

{\tt doubleval = parse\_double (handle, keyname [, default, vmin, vmax, descr])} 

{\tt stringval = parse\_string (handle, keyname [, default, descr, filestatus])} 

{\tt logicval = parse\_lgt (handle, keyname [, default, descr])} 

\quad These routines obtain integer(i4b), integer(i8b), real(sp), real(dp), character(len=*) and logical values,
respectively. \\
Note: {\tt parse\_string} will expand all environment variables of
the form \$\{XXX\} (eg: \$\{HOME\}). It will also replace a {\em leading} 
\verb+~+$\!${\tt /}
%%%{\tt \tilde{}/} 
by the user's home directory.

{\tt call parse\_summarize (handle [, code])}

\quad if the parameters were set interactively, this routine will print out a 
parameter file performing the same settings.

{\tt call parse\_check\_unused (handle [, code])}

\quad if a parameter file was read, this routine will print out all the parameters
found in the file but not used by the code. Intended at detecting typos in
parameter names.

{\tt call parse\_finish (handle)}

\quad frees the memory

\begin{example}
{
program who\_r\_u \\
use healpix\_types \\
use paramfile\_io \\
use extension \\
\\
implicit none \\
type(paramfile\_handle) :: handle \\
character(len=256) :: parafile, name \\
real(DP) :: age \\
\\
parafile = ''  \\
if (\htmlref{nArguments()}{sub:narguments} == 1) call \htmlref{getArgument}{sub:getargument}(1, parafile)  \\
handle = parse\_init(parafile)  \\
name  = parse\_string(handle, 'name',descr='Enter your last name: ')  \\
age   = parse\_double(handle, 'age', descr='Enter your age in years: ', \&   \\
   \& default=18.d0,vmin=0.d0)  \\
call parse\_summarize(handle, 'who\_r\_u')  \\
end program who\_r\_u 
}
{If a file is provided as command line argument when running the executable {\tt who\_r\_u}, that file
  will be parsed in search of the lines starting with 'name =' and 'age =',
  otherwise the same questions will be asked interactively.
}
\end{example}

\begin{related}
  \begin{sulist}{} %%%% NOTE the ``extra'' brace here %%%%
  \item[\htmlref{concatnl}{sub:concatnl}] generates from a set of strings the
  multi-line description
  \item[\htmlref{nArguments}{sub:narguments}] returns the number of
  command line arguments
  \item[\htmlref{getArgument}{sub:getArgument}] returns the list of command line arguments
  \end{sulist}
\end{related}

\rule{\hsize}{2mm}

\newpage
% special format, TBD

\sloppy


\title{\healpix Fortran Subroutines Overview}
\docid{pixel\_window} \section[pixel\_window]{ }
\label{sub:pixel_window}
\docrv{Version 2.0}
\author{Eric Hivon}
\abstract{This document describes the \healpix Fortran90 subroutine
PIXEL\_WINDOW.}

\newcommand{\wpix}{w_{\rm pix}(\ell)}
\newcommand{\alm}{a_{lm}}
\newcommand{\almpix}{a_{lm}^{\rm (pix)}}

\begin{facility}
{This routine returns the {\em averaged} $\ell$-space window function $\wpix$ (for temperature and
  polarisation) associated to \healpix\
  pixels of resolution parameter $\nside$. Because of the integration of the
signal over the
pixel area, the $\almpix$ coefficients of a pixelated map
are related to the unpixelated underlying $\alm$ by $\almpix = \alm \wpix$.\\
Unless specified otherwise, the $\wpix$ are read from the files
  \$HEALPIX/data/pixel\_window\_n????.fits.}
{\modAlmTools}
\end{facility}

\begin{f90format}
{\mylink{sub:pixel_window:pixlw}{pixlw}%
, \mylink{sub:pixel_window:nside}{nside}%
 [, \mylink{sub:pixel_window:windowfile}{windowfile}%
]}
\end{f90format}

\begin{arguments}
{
\begin{tabular}{p{0.30\hsize} p{0.05\hsize} p{0.05\hsize} p{0.50\hsize}} \hline  
\textbf{name~\&~dimensionality} & \textbf{kind} & \textbf{in/out} & \textbf{description} \\ \hline
                   &   &   &                           \\ %%% for presentation
pixlw\mytarget{sub:pixel_window:pixlw}(0:lmax,1:p) & DP & OUT & pixel window function(s) $\wpix$ generated. The first index
                   must be $\ell_{\rm max}\leq 4\nside$. The second index runs from 1:1 for
                   temperature only, and 1:3 for polarisation. In the latter
                   case, 1=T, 2=E, 3=B.\\
nside\mytarget{sub:pixel_window:nside} & I4B & IN & \healpix\ $\nside$ resolution parameter. Unless {\tt
                   windowfile} is set, the file associated
                   with $\nside$ and shipped with the package is read by
                   default. If {\tt nside} = 0, the pixel is assumed infinitely
                   small and {\tt pixlw} is returned with value 1.\\
windowfile\mytarget{sub:pixel_window:windowfile} \hskip 2cm
(OPTIONAL) & CHR & IN & FITS file containing the pixel window to be read instead
                   of the default.
\end{tabular}
}
\end{arguments}

\begin{example}
{
call pixel\_window(pixlw, 64)  \\
}
{
returns in pixlw the pixel window function for $\nside = 64$.
}
\end{example}

\begin{modules}
  \begin{sulist}{} %%%% NOTE the ``extra'' brace here %%%%
  \item[\textbf{misc\_utils}] module, containing:
      \item[\htmlref{assert, fatal\_error}{sub:assert}] interrupt code in case of error
  \item[\textbf{extension}] module, containing:
     \item[\htmlref{getEnvironment}{sub:getenvironment}] read environment variable
  \item[\textbf{fitstools}] module, containing:
     \item[\htmlref{read\_dbintab}{sub:read_dbintab}] reads double precision binary table
  \end{sulist}
\end{modules}

\begin{related}
  \begin{sulist}{} %%%% NOTE the ``extra'' brace here %%%%
  \item[\htmlref{gaussbeam}{sub:gaussbeam}] routine to generate a gaussian
beam window function
  \item[\htmlref{generate\_beam}{sub:generate_beam}] returns a beam window function
  \item[\htmlref{alter\_alm}{sub:alter_alm}, \htmlref{rotate\_alm}{sub:rotate_alm}] modifies $a_{lm}$ to emulate effect
of real space filtering and coordinate rotation respectively
  \item[\htmlref{alm2map}{sub:alm2map}] synthetize a \healpix map from its $\alm$
(or $\almpix$).
  \item[\htmlref{alm2map\_der}{sub:alm2map_der}] synthetize a map and its
derivatives from its $\alm$ (or $\almpix$).
  \end{sulist}
\end{related}

\rule{\hsize}{2mm}

\newpage



\sloppy

%%%\title{\healpix Fortran Subroutines Overview}
\docid{pix2xxx,ang2xxx,vec2xxx, nest2ring,ring2nest} \section[pix2xxx,ang2xxx,vec2xxx, nest2ring,ring2nest]{ }
\label{sub:pix_tools}
\docrv{Version 1.2}
\author{Frode K.~Hansen, Eric Hivon}
\abstract{This document describes the \healpix Fortran90 subroutines in module PIX_TOOLS.}

\begin{facility}
{The Fortran90 module pix\_tools contains some subroutines to convert between pixel number in the \healpix map and $(\theta,\phi)$ or $(x,y,z)$ coordinates on the sphere. Some of these routines are listed here.}
{\modPixTools}
\end{facility}

\begin{arguments}
{
\begin{tabular}{p{0.30\hsize} p{0.05\hsize} p{0.08\hsize} p{0.47\hsize}} \hline  
\textbf{name~\&~dimensionality} & \textbf{kind} & \textbf{in/out} & \textbf{description} \\ \hline
                   &   &   &                           \\ %%% for presentation
nside\mytarget{sub:pix_tools:nside} & I4B & IN & $\nside$ parameter for the \healpix map. \\
ipnest\mytarget{sub:pix_tools:ipnest} & I4B/ I8B & --- & pixel identification number in NESTED scheme over the range \{0,$\npix-1$\}. \\
ipring\mytarget{sub:pix_tools:ipring} & I4B/ I8B & --- & pixel identification number in RING scheme over the range \{0,$\npix-1$\}. \\
theta\mytarget{sub:pix_tools:theta} & DP & --- & colatitude in radians measured southward from north pole in \{0,$\pi$\}. \\
phi\mytarget{sub:pix_tools:phi} & DP & --- & longitude in radians, measured eastward in $[0,2\pi]$. \\ 
vector(3)\mytarget{sub:pix_tools:vector} & DP & --- & three dimensional cartesian position vector
                   $(x,y,z)$. The north pole is $(0,0,1)$. An output vector is normalised to unity. \\
vertex(3,4)\mytarget{sub:pix_tools:vertex} \hskip 3cm OPTIONAL & DP & OUT & three dimensional cartesian position vectors
                   $(x,y,z)$ (normalised to unity) pointing to the 4 vertices of a given pixel. The four vertices are listed in the order North, West, South, East.
\end{tabular}
}
\end{arguments}
\newpage
%\ mylink: to avoid automatic processing by make_internal_links.sh

\rule{\hsize}{0.7mm}
\textsc{\large{\textbf{ROUTINES: }}}\hfill\newline
{\tt call pix2ang\_ring(nside, ipring, theta, phi)} 

 \begin{tabular}{@{}p{0.25\hsize}@{\hspace{1ex}}p{0.75\hsize}@{}}
                                         & renders \mylink{sub:pix_tools:theta}{{\tt theta}} and \mylink{sub:pix_tools:phi}{{\tt phi}} coordinates of the nominal pixel center given the pixel number \mylink{sub:pix_tools:ipring}{{\tt ipring}} and a map resolution parameter \mylink{sub:pix_tools:nside}{{\tt nside}}. \\
     \end{tabular}\\\\
%
{\tt call pix2vec\_ring(nside, ipring, vector [,vertex])} 

 \begin{tabular}{@{}p{0.25\hsize}@{\hspace{1ex}}p{0.75\hsize}@{}}
                                         & renders cartesian vector coordinates of the nominal pixel center given the pixel number \mylink{sub:pix_tools:ipring}{{\tt ipring}} and a map resolution parameter \mylink{sub:pix_tools:nside}{{\tt nside}}. Optionally renders cartesian vector coordinates of the considered pixel four vertices.\\
     \end{tabular}\\\\
%
{\tt call ang2pix\_ring(nside, theta, phi, ipring)} 

 \begin{tabular}{@{}p{0.25\hsize}@{\hspace{1ex}}p{0.75\hsize}@{}}
                                         & renders the pixel number \mylink{sub:pix_tools:ipring}{{\tt ipring}} for a pixel which, given the map resolution parameter \mylink{sub:pix_tools:nside}{{\tt nside}}, contains the point on the sphere at angular coordinates \mylink{sub:pix_tools:theta}{{\tt theta}} and \mylink{sub:pix_tools:phi}{{\tt phi}}. \\
     \end{tabular}\\\\
%
{\tt call vec2pix\_ring(nside, vector, ipring)} 

 \begin{tabular}{@{}p{0.25\hsize}@{\hspace{1ex}}p{0.75\hsize}@{}}
                                         & renders the pixel number \mylink{sub:pix_tools:ipring}{{\tt ipring}} for a pixel which, given the map resolution parameter \mylink{sub:pix_tools:nside}{{\tt nside}}, contains the point on the sphere at cartesian coordinates \mylink{sub:pix_tools:vector}{{\tt vector}}. \\
     \end{tabular}\\\\
%
{\tt call pix2ang\_nest(nside, ipnest, theta, phi)} 

 \begin{tabular}{@{}p{0.25\hsize}@{\hspace{1ex}}p{0.75\hsize}@{}}
                                         & renders \mylink{sub:pix_tools:theta}{{\tt theta}} and \mylink{sub:pix_tools:phi}{{\tt phi}} coordinates of the nominal pixel center given the pixel number \mylink{sub:pix_tools:ipnest}{{\tt ipnest}} and a map resolution parameter \mylink{sub:pix_tools:nside}{{\tt nside}}. \\
     \end{tabular}\\\\
%
{\tt call pix2vec\_nest(nside, ipnest, vector [,vertex])} 

 \begin{tabular}{@{}p{0.25\hsize}@{\hspace{1ex}}p{0.75\hsize}@{}}
                                         & renders cartesian vector coordinates of the nominal pixel center given the pixel number \mylink{sub:pix_tools:ipnest}{{\tt ipnest}} and a map resolution parameter \mylink{sub:pix_tools:nside}{{\tt nside}}. Optionally renders cartesian vector coordinates of the considered pixel four vertices.\\
     \end{tabular}\\\\
%
{\tt call ang2pix\_nest(nside, theta, phi, ipnest)} 

 \begin{tabular}{@{}p{0.25\hsize}@{\hspace{1ex}}p{0.75\hsize}@{}}
                                         & renders the pixel number \mylink{sub:pix_tools:ipnest}{{\tt ipnest}} for a pixel which, given the map resolution parameter \mylink{sub:pix_tools:nside}{{\tt nside}}, contains the point on the sphere at angular coordinates \mylink{sub:pix_tools:theta}{{\tt theta}} and \mylink{sub:pix_tools:phi}{{\tt phi}}. \\
     \end{tabular}\\\\
%
{\tt call vec2pix\_nest(nside, vector, ipnest)} 

 \begin{tabular}{@{}p{0.25\hsize}@{\hspace{1ex}}p{0.75\hsize}@{}}
                                         & renders the pixel number
                        \mylink{sub:pix_tools:ipnest}{{\tt ipnest}} for a pixel which, given the map
                        resolution parameter \mylink{sub:pix_tools:nside}{{\tt nside}}, contains the
                        point on the sphere at cartesian coordinates
                        \mylink{sub:pix_tools:vector}{{\tt vector}} . \\
     \end{tabular}\\\\
%
{\tt call nest2ring(nside, ipnest, ipring)} 

 \begin{tabular}{@{}p{0.25\hsize}@{\hspace{1ex}}p{0.75\hsize}@{}}
                                         & performs conversion from NESTED to RING pixel number. \\
     \end{tabular}\\\\
%
{\tt call ring2nest(nside, ipring, ipnest)} 

 \begin{tabular}{@{}p{0.25\hsize}@{\hspace{1ex}}p{0.75\hsize}@{}}
                                         & performs conversion from RING to NESTED pixel number. %\\
     \end{tabular}\\\\

\begin{modules}
  \begin{sulist}{} %%%% NOTE the ``extra'' brace here %%%%
 \item[mk\_pix2xy, mk\_xy2pix] routines used in the conversion between pixel values and ``cartesian'' coordinates on the Healpix face.
  \end{sulist}
\end{modules}

\begin{related}
  \begin{sulist}{} %%%% NOTE the ``extra'' brace here %%%%
  \item[\htmlref{neighbours\_nest}{sub:neighbours_nest}] find neighbouring pixels.
  \item[\htmlref{ang2vec}{sub:ang2vec}] convert $(\theta,\phi)$ spherical coordinates into $(x,y,z)$ cartesian coordinates.
  \item[\htmlref{vec2ang}{sub:vec2ang}] convert $(x,y,z)$ cartesian coordinates into $(\theta,\phi)$ spherical coordinates.
  \item[\htmlref{convert\_inplace}{sub:convert_inplace}] in-place conversion  between RING and NESTED for integer/real/double maps.
  \item[\htmlref{convert\_nest2ring}{sub:convert_nest2ring}] convert from NESTED to RING scheme using a temporary array.
    \item[\htmlref{nest2uniq}{sub:nest2uniq}, \htmlref{uniq2nest}{sub:uniq2nest}] conversion of standard pixel index to/from Unique ID number
  \end{sulist}
\end{related}

\rule{\hsize}{2mm}

\newpage
% special format, TBD

\sloppy


%%%\title{\healpix Fortran Subroutines Overview}
\docid{planck\_rng} \section[planck\_rng derived type]{ }
\label{sub:planck_rng}
\docrv{Version 2.0}
\author{Eric Hivon}
\abstract{This document describes the \healpix Fortran90 type \thedocid.}

\begin{facility}
{The derived type \thedocid\ is used by the Random Number Generation routines 
\htmlref{rand\_init}{sub:rand_init},
\htmlref{rand\_uni}{sub:rand_uni}, 
\htmlref{rand\_gauss}{sub:rand_gauss} to describe fully the current RNG
sequence.\\
Most users do not need to know about the \thedocid\ definition. It may be
useful for those wanting to take a snapshot of the RNG sequence they are using (by eg,
dumping the latest values of \thedocid\ structure on disk) so that the same sequence can be resumed
later on from that same point.}
{\modRngmod}
\end{facility}


% %---------------------
% \newenvironment{mytable}[1]{%
% \begin{minipage}[b]{\linewidth}{%
% % \renewcommand{\thefootnote}{\fnsymbol{footnote}}
% % \renewcommand{\footnoterule}{}
% {#1}
% }%
% \end{minipage}
% }
%---------------------

The type \thedocid\ is a structure defined as

\begin{mytable}{%
\begin{tabularx}{\linewidth}{lcX}
name & type  & definition \\
\hline
x, y, z, w & I4B & internal variables of uniform RNG\\
gset & DP & internal variable for Gaussian RNG\\
empty & LGT & flag used by Gaussian RNG\\
\hline
\end{tabularx}
}%
\end{mytable}



% \begin{example}
% {
% use planck\_rng \\
% real(kind=DP) :: dx \\
% print*,' pi = ',PI
% }
% {
% The value of {\tt PI}, as well as all other \thedocid\ parameters are made known
% to the code
% }
% \end{example}

\begin{related}
  \begin{sulist}{} %%%% NOTE the ``extra'' brace here %%%%
  \item[\htmlref{rand\_gauss}{sub:rand_gauss}] function which returns a  random normal deviate.
  \item[\htmlref{rand\_uni}{sub:rand_uni}] function which returns a random uniform deviate.
   \item[\htmlref{rand\_init}{sub:rand_init}] subroutine to initiate a random number sequence. 
  \end{sulist}
\end{related}

\rule{\hsize}{2mm}

\newpage

\sloppy
\docid{plm\_gen}\section[plm\_gen]{ }
\label{sub:plm_gen}
\docrv{Version 2.0}
\author{Eric Hivon}
\abstract{This document describes the \healpix Fortran90 subroutine PLM\_GEN.}

\begin{facility}
{This routine computes the latitude dependent part $\lambda_{\ell m}$ of the
  spherical harmonics ($Y_{\ell m}(\theta,\phi) = \lambda_{\ell m}(\theta) e^{i m \phi}$) of spin 0 and 2
  (see \healpix primer)
  used to synthetize or analyze \healpix maps of temperature and polarisation.
  Since version 2.20, which introduced optimized algorithms for spherical
  harmonic transforms, it has become obsolete and should no longer be used.}
{\modAlmTools}
\end{facility}

\begin{f90format}
{\mylink{sub:plm_gen:nsmax}{nsmax}%
, \mylink{sub:plm_gen:nlmax}{nlmax}%
, \mylink{sub:plm_gen:nmmax}{nmmax}%
, \mylink{sub:plm_gen:plm}{plm}%
}
\end{f90format}

\begin{arguments}
{
\begin{tabular}{p{0.4\hsize} p{0.05\hsize} p{0.1\hsize} p{0.35\hsize}} \hline  
\textbf{name~\&~dimensionality} & \textbf{kind} & \textbf{in/out} & \textbf{description} \\ \hline
                   &   &   &                           \\ %%% for presentation
nsmax\mytarget{sub:plm_gen:nsmax} & I4B & IN & The $\nside$ value for which to compute the $\lambda_{\ell m}$. \\
nlmax\mytarget{sub:plm_gen:nlmax} & I4B & IN & The maximum multipole order $\ell$ of the generated $\lambda_{\ell m}$. \\
nmmax\mytarget{sub:plm_gen:nmmax} & I4B & IN & The maximum degree $m$ of the generated $\lambda_{\ell m}$. \\
plm\mytarget{sub:plm_gen:plm}(0:n\_plm-1, 1:p) & DP & OUT & The $\lambda_{\ell m}$ values, either for temperature only
                   ($p=1$) or temperature and polarisation ($p=3$). The number
                    of $\lambda_{\ell m}$ is n\_plm =
                    nsmax*(nmmax+1)*(2*nlmax-nmmax+2). They are stored in the
                    order of increasing order $\ell$, increasing degree $m$, for
                    all the \healpix ring colatitudes $\theta$ from North Pole to Equator, ie
 		   $\lambda_{00}(\theta_1)$, $\lambda_{10}(\theta_1)$, $\lambda_{20}(\theta_1)$,
                    \ldots, $\lambda_{11}(\theta_1)$, $\lambda_{21}(\theta_1)$;
                    \ldots,  $\lambda_{00}(\theta_2)$ \ldots \\
\end{tabular}
}
\end{arguments}

\begin{example}
{
use healpix\_types \\
use alm\_tools, only : plm\_gen \\
integer(I4B) :: nside, lmax, mmax, n\_plm\\
real(DP), dimension(:,:), allocatable :: plm \\
\ldots \\
nside=256 ; lmax=512 ; mmax=lmax\\
npix=nside2npix(nside)\\
n\_plm=nside*(mmax+1)*(2*lmax-mmax+2)\\
allocate(plm(0:n\_plm-1,1:3))\\
\ldots \\
call \thedocid(nside, lmax, mmax, plm)  \\
}
{
Computes the spherical harmonics for temperature and polarisation for $\nside= 256$, up to 512 in $\ell$ and $m$.
}
\end{example}

\begin{modules}
  \begin{sulist}{} %%%% NOTE the ``extra'' brace here %%%%
  \item[compute\_lam\_mm, get\_pixel\_layout, ]
  \item[gen\_lamfac,gen\_mfac, gen\_normpol, ] 
  \item[gen\_recfac, init\_rescale, l\_min\_ylm] Ancillary routines used
  for $\lambda_{\ell m}$ recursion
  \item[\textbf{misc\_utils}] module, containing:
  \item[assert\_alloc] routine to print error message, when an array can not be
  allocated properly
  \end{sulist}
\end{modules}

\begin{related}
  \begin{sulist}{} %%%% NOTE the ``extra'' brace here %%%%
   \item[\htmlref{alm2map}{sub:alm2map}] routine generating maps of temperature
   and polarisation from their $a_{\ell m}$ that can use precomputed $\lambda_{\ell
   m}$ generated by \thedocid
   \item[\htmlref{map2alm}{sub:map2alm}] routine analysing maps of temperature
   and polarisation that can use precomputed $\lambda_{\ell
   m}$ generated by \thedocid
  \item[plmgen] executable using \thedocid\ to compute the $\lambda_{\ell m}$ and
  writting them on disc 
%%   \item[anafast] executable that may use the $\lambda_{\ell m}$ produced by \thedocid\ to analyse maps
%%   \item[smoothing] executable that may use the $\lambda_{\ell m}$ produced by \thedocid\ to smooth maps
%%   \item[synfast] executable that may use the $\lambda_{\ell m}$ produced by \thedocid\ to synthesize maps.
  \end{sulist}
\end{related}

\rule{\hsize}{2mm}

\newpage


\sloppy


%%%\title{\healpix Fortran Subroutines Overview}
\docid{query\_disc} \section[query\_disc]{ }
\label{sub:query_disc}
\docrv{Version 1.3}
\author{Eric Hivon}
\abstract{This document describes the \healpix Fortran90 subroutine QUERY\_DISC.}

\begin{facility}
{Routine to find the index of all pixels within an angular distance radius from a defined
center. The output indices can be either in the RING or NESTED scheme} 
{\modPixTools}
\end{facility}

\begin{f90format}
{\mylink{sub:query_disc:nside}{nside}%
, \mylink{sub:query_disc:vector0}{vector0}%
, \mylink{sub:query_disc:radius}{radius}%
, \mylink{sub:query_disc:listpix}{listpix}%
, \mylink{sub:query_disc:nlist}{nlist}%
 [, \mylink{sub:query_disc:nest}{nest}%
, \mylink{sub:query_disc:inclusive}{inclusive}%
]}
\end{f90format}

\begin{arguments}
{
\begin{tabular}{p{0.28\hsize} p{0.05\hsize} p{0.1\hsize} p{0.47\hsize}} \hline 
\textbf{name~\&~dimensionality} & \textbf{kind} & \textbf{in/out} & \textbf{description} \\ \hline
                   &   &   &                           \\ %%% for presentation
nside\mytarget{sub:query_disc:nside} & I4B & IN & the $\nside$ parameter of the map. \\
vector0\mytarget{sub:query_disc:vector0}(3) & DP & IN & cartesian vector pointing at the disc center. \\
radius\mytarget{sub:query_disc:radius} & DP & IN & disc radius in radians. \\
listpix\mytarget{sub:query_disc:listpix}(0:*) & I4B/ I8B & OUT & the index for all pixels within {\tt radius}. Make sure that the size of the array is big enough to contain all pixels. \\ 
nlist\mytarget{sub:query_disc:nlist} & I4B/ I8B & OUT & The number of pixels listed in {\tt listpix}. \\
nest\mytarget{sub:query_disc:nest}\ \ (OPTIONAL) & I4B & IN &  The pixel indices are in the NESTED numbering
                   scheme if nest=1, and in RING scheme otherwise. \\
inclusive\mytarget{sub:query_disc:inclusive}\ \ (OPTIONAL) & I4B & IN & If set to 1, all the pixels overlapping
                   (even partially)
                   with the disc are listed, otherwise only those whose
                   center lies within the disc are listed. \\

\end{tabular}
}
\end{arguments}

\begin{example}
{
use \htmlref{healpix\_modules}{sub:healpix_modules} \\
call query\_disc(256,(/0,0,1/),pi/2,listpix,nlist,nest=1)  \\
}
{
Returns the NESTED pixel index of all pixels north of the equatorial line in a $\nside=256$ map.
}
\end{example}
% \newpage
\begin{modules}
  \begin{sulist}{} %%%% NOTE the ``extra'' brace here %%%%
 \item[\htmlref{in\_ring}{sub:in_ring}] routine to find the pixels in a certain slice of a given ring.		
 \item[\htmlref{ring\_num}{sub:ring_num}] function to return the ring number corresponding to the coordinate $z$
  \end{sulist}
\end{modules}

\begin{related}
  \begin{sulist}{} %%%% NOTE the ``extra'' brace here %%%%
  \item[\htmlref{pix2ang}{sub:pix_tools}, \htmlref{ang2pix}{sub:pix_tools}] convert between angle and pixel number.
  \item[\htmlref{pix2vec}{sub:pix_tools}, \htmlref{vec2pix}{sub:pix_tools}] convert between a cartesian vector and pixel number.
  \item[query\_disc, \htmlref{query\_polygon}{sub:query_polygon},]
  \item[\htmlref{query\_strip}{sub:query_strip}, \htmlref{query\_triangle}{sub:query_triangle}] render the list of pixels enclosed
  respectively in a given disc, polygon, latitude strip and triangle
  \item[\htmlref{surface\_triangle}{sub:surface_triangle}] computes the surface
in steradians of a spherical triangle defined by 3 vertices

  \end{sulist}
\end{related}

\rule{\hsize}{2mm}

\newpage


\sloppy


\title{\healpix Fortran Subroutines Overview}
\docid{query\_polygon} \section[query\_polygon]{ }
\label{sub:query_polygon}
\docrv{Version 1.3}
\author{Eric Hivon}
\abstract{This document describes the \healpix Fortran90 subroutine QUERY\_POLYGON.}

\begin{facility}
{Routine to find the index of all pixels enclosed in a polygon. The polygon should be convex, 
or have only one concave vertex. The edges should not intersect each other. 
The output indices can be either in the RING or NESTED scheme} 
{\modPixTools}
\end{facility}

\begin{f90format}
{\mylink{sub:query_polygon:nside}{nside}%
, \mylink{sub:query_polygon:vlist}{vlist}%
, \mylink{sub:query_polygon:nv}{nv}%
, \mylink{sub:query_polygon:listpix}{listpix}%
, \mylink{sub:query_polygon:nlist}{nlist}%
 [, \mylink{sub:query_polygon:nest}{nest}%
, \mylink{sub:query_polygon:inclusive}{inclusive}%
]}
\end{f90format}

\begin{arguments}
{
\begin{tabular}{p{0.25\hsize} p{0.05\hsize} p{0.1\hsize} p{0.5\hsize}} \hline  
\textbf{name~\&~dimensionality} & \textbf{kind} & \textbf{in/out} & \textbf{description} \\ \hline
                   &   &   &                           \\ %%% for presentation
nside\mytarget{sub:query_polygon:nside} & I4B & IN & the $\nside$ parameter of the map. \\
vlist\mytarget{sub:query_polygon:vlist}(1:3,0:*) & DP & IN & cartesian vector pointing at polygon
                   respective vertices. \\
nv\mytarget{sub:query_polygon:nv} & I4B & IN & number of vertices, should be equal to 3 or larger. \\
listpix\mytarget{sub:query_polygon:listpix}(0:*) & I4B/ I8B & OUT & the index for all pixels enclosed in the triangle. Make sure that the size of the array is big enough to contain all pixels. \\ 
nlist\mytarget{sub:query_polygon:nlist} & I4B/ I8B & OUT & The number of pixels listed in {\tt listpix}. \\
nest\mytarget{sub:query_polygon:nest}\ \ (OPTIONAL) & I4B & IN &  The pixel indices are in the NESTED numbering scheme if nest=1, and in RING scheme otherwise. \\
inclusive\mytarget{sub:query_polygon:inclusive}\ \ (OPTIONAL) & I4B & IN & If set to 1, all the pixels overlapping
                   (even partially)
                   with the polygon are listed, otherwise only those whose
                   center lies within the polygon are listed. \\
\end{tabular}
}
\end{arguments}

\begin{example}
{
use \htmlref{healpix\_modules}{sub:healpix_modules} \\
real(dp), dimension(1:3,0:9) :: vertices \\
vertices(:,0) = (/0.,0.,1./)  ! +z \\
vertices(:,1) = (/1.,0.,0./)  ! +x \\
vertices(:,2) = (/1.,1.,-1./) ! x+y-z \\
vertices(:,3) = (/0.,1.,0./)  ! +y \\
 \\
call query\_polygon(256,vertices,4,listpix,nlist,nest=0)  \\
}
{
Returns the RING pixel index of all pixels in the polygon with vertices of
cartesian coordinates (0,0,1), (1,0,0), (1,1,-1) and (0,1,0) in a $\nside=256$ map.
}
\end{example}
%\newpage
\begin{modules}
  \begin{sulist}{} %%%% NOTE the ``extra'' brace here %%%%
 \item[isort] routine to sort integer number
 \item[\htmlref{query\_triangle}{sub:query_triangle}] render the list of pixels enclosed
  in a given triangle
 \item[\htmlref{surface\_triangle}{sub:surface_triangle}] computes the surface of a spherical triangle defined by 3 vertices
 \item[\htmlref{vect\_prod}{sub:vect_prod}] routine to compute the vectorial product of two 3D vectors
  \end{sulist}
\end{modules}

\begin{related}
  \begin{sulist}{} %%%% NOTE the ``extra'' brace here %%%%
  \item[\htmlref{pix2ang}{sub:pix_tools}, \htmlref{ang2pix}{sub:pix_tools}] convert between angle and pixel number.
  \item[\htmlref{pix2vec}{sub:pix_tools}, \htmlref{vec2pix}{sub:pix_tools}] convert between a cartesian vector and pixel number.
  \item[\htmlref{query\_disc}{sub:query_disc}, query\_polygon,]
  \item[\htmlref{query\_strip}{sub:query_strip}, \htmlref{query\_triangle}{sub:query_triangle}] render the list of pixels enclosed
  respectively in a given disc, polygon, latitude strip and triangle
  \item[\htmlref{surface\_triangle}{sub:surface_triangle}] computes the surface
in steradians of a spherical triangle defined by 3 vertices

  \end{sulist}
\end{related}

\rule{\hsize}{2mm}

\newpage


\sloppy


%%%\title{\healpix Fortran Subroutines Overview}
\docid{query\_strip} \section[query\_strip]{ }
\label{sub:query_strip}
\docrv{Version 1.3}
\author{Eric Hivon}
\abstract{This document describes the \healpix Fortran90 subroutine QUERY\_STRIP.}

\begin{facility}
{Routine to find the index of all pixels enclosed in a latitude strip. The output indices can be either in the RING or NESTED scheme} 
{\modPixTools}
\end{facility}

\begin{f90format}
{\mylink{sub:query_strip:nside}{nside}%
, \mylink{sub:query_strip:theta1}{theta1}%
, \mylink{sub:query_strip:theta2}{theta2}%
, \mylink{sub:query_strip:listpix}{listpix}%
, \mylink{sub:query_strip:nlist}{nlist}%
 [, \mylink{sub:query_strip:nest}{nest}%
, \mylink{sub:query_strip:inclusive}{inclusive}%
]}
\end{f90format}

\begin{arguments}
{
\begin{tabular}{p{0.28\hsize} p{0.05\hsize} p{0.1\hsize} p{0.47\hsize}} \hline  
\textbf{name\&dimensionality} & \textbf{kind} & \textbf{in/out} & \textbf{description} \\ \hline
                   &   &   &                           \\ %%% for presentation
nside\mytarget{sub:query_strip:nside} & I4B & IN & the $\nside$ parameter of the map. \\
theta1\mytarget{sub:query_strip:theta1} & DP & IN & colatitude lower bound in radians measured from North Pole
                   (between 0 and $\pi$). \\
theta2\mytarget{sub:query_strip:theta2} & DP & IN & colatitude upper bound in radians measured from North Pole (between 0 and $\pi$). If
                   theta1$<$ theta2, the pixels lying in [theta1,theta2]
                   are output, otherwise, the pixel lying in [0,
                   theta2] and those lying in [theta1, $\pi$] are output.\\
listpix\mytarget{sub:query_strip:listpix}(0:*) & I4B/ I8B & OUT & the index for all pixels enclosed in the
                   strip(s). Make sure that the size of the array is big enough to contain all pixels. \\ 
nlist\mytarget{sub:query_strip:nlist} & I4B/ I8B & OUT & The number of pixels listed in {\tt listpix}. \\
nest\mytarget{sub:query_strip:nest}\ \ (OPTIONAL) & I4B & IN &  The pixel indices are in the NESTED numbering scheme if nest=1, and in RING scheme otherwise. \\
inclusive\mytarget{sub:query_strip:inclusive}\ \ (OPTIONAL) & I4B & IN & If set to 1, all the pixels overlapping
                   (even partially)
                   with the strip are listed; otherwise only those whose
                   center lies within the strip are listed. \\
\end{tabular}
}
\end{arguments}

\begin{example}
{
call query\_strip(256,0.75*PI,0.2*PI,listpix,nlist,nest=1)  \\
}
{
Returns the NESTED pixel index of all pixels with colatitude in
[0,$\pi/5$] and those with colatitude in [$3\pi/4$,$\pi$]
}
\end{example}
%\newpage
\begin{modules}
  \begin{sulist}{} %%%% NOTE the ``extra'' brace here %%%%
 \item[\htmlref{in\_ring}{sub:in_ring}] routine to find the pixels in a certain slice of a given ring.		
 \item[intrs\_intrv] routine to compute the intersection of 2 intervals on a circle
 \item[\htmlref{ring\_num}{sub:ring_num}] function to return the ring number corresponding to the coordinate $z$
 \item[\htmlref{vect\_prod}{sub:vect_prod}] routine to compute the vectorial product of two 3D vectors
  \end{sulist}
\end{modules}

\begin{related}
  \begin{sulist}{} %%%% NOTE the ``extra'' brace here %%%%
  \item[\htmlref{pix2ang}{sub:pix_tools}, \htmlref{ang2pix}{sub:pix_tools}] convert between angle and pixel number.
  \item[\htmlref{pix2vec}{sub:pix_tools}, \htmlref{vec2pix}{sub:pix_tools}] convert between a cartesian vector and pixel number.
  \item[\htmlref{query\_disc}{sub:query_disc}, \htmlref{query\_polygon}{sub:query_polygon},]
  \item[query\_strip, \htmlref{query\_triangle}{sub:query_triangle}] render the list of pixels enclosed
  respectively in a given disc, polygon, latitude strip and triangle
  \item[\htmlref{surface\_triangle}{sub:surface_triangle}] computes the surface
in steradians of a spherical triangle defined by 3 vertices

  \end{sulist}
\end{related}

\rule{\hsize}{2mm}

\newpage


\sloppy


\title{\healpix Fortran Subroutines Overview}
\docid{query\_triangle} \section[query\_triangle]{ }
\label{sub:query_triangle}
\docrv{Version 1.3}
\author{Eric Hivon}
\abstract{This document describes the \healpix Fortran90 subroutine QUERY\_TRIANGLE.}

\begin{facility}
{Routine to find the index of all pixels enclosed in a spherical triangle described by its three vertices. The output indices can be either in the RING or NESTED scheme} 
{\modPixTools}
\end{facility}

\begin{f90format}
{\mylink{sub:query_triangle:nside}{nside}%
, \mylink{sub:query_triangle:v1}{v1}%
, \mylink{sub:query_triangle:v2}{v2}%
, \mylink{sub:query_triangle:v3}{v3}%
, \mylink{sub:query_triangle:listpix}{listpix}%
, \mylink{sub:query_triangle:nlist}{nlist}%
 [, \mylink{sub:query_triangle:nest}{nest}%
, \mylink{sub:query_triangle:inclusive}{inclusive}%
]}
\end{f90format}

\begin{arguments}
{
\begin{tabular}{p{0.28\hsize} p{0.05\hsize} p{0.1\hsize} p{0.47\hsize}} \hline 
\textbf{name\&dimensionality} & \textbf{kind} & \textbf{in/out} & \textbf{description} \\ \hline
                   &   &   &                           \\ %%% for presentation
nside\mytarget{sub:query_triangle:nside} & I4B & IN & the $\nside$ parameter of the map. \\
v1\mytarget{sub:query_triangle:v1}(3) & DP & IN & cartesian vector pointing at the triangle first vertex. \\
v2\mytarget{sub:query_triangle:v2}(3) & DP & IN & cartesian vector pointing at the triangle second vertex. \\
v3\mytarget{sub:query_triangle:v3}(3) & DP & IN & cartesian vector pointing at the triangle third vertex. \\
listpix\mytarget{sub:query_triangle:listpix}(0:*) & I4B/ I8B & OUT & the index for all pixels enclosed in the triangle. Make sure that the size of the array is big enough to contain all pixels. \\ 
nlist\mytarget{sub:query_triangle:nlist} & I4B/ I8B & OUT & The number of pixels listed in {\tt listpix}. \\
nest\mytarget{sub:query_triangle:nest}\ \ (OPTIONAL) & I4B & IN &  The pixel indices are in the NESTED numbering scheme if nest=1, and in RING scheme otherwise. \\
inclusive\mytarget{sub:query_triangle:inclusive}\ \ (OPTIONAL) & I4B & IN & If set to 1, all the pixels overlapping
                   (even partially)
                   with the triangle are listed, otherwise only those whose
                   center lies within the triangle are listed. \\
\end{tabular}
}
\end{arguments}

\begin{example}
{
call query\_triangle(256,(/1,0,0/),(/0,1,0/),(/0,0,1/),listpix,nlist)  \\
}
{
Returns the RING pixel index of the (98560) pixels in the octant ($x,y,z>0$) in a $\nside=256$ map.
}
\end{example}
%\newpage
\begin{modules}
  \begin{sulist}{} %%%% NOTE the ``extra'' brace here %%%%
 \item[\htmlref{in\_ring}{sub:in_ring}] routine to find the pixels in a certain slice of a given ring.		
 \item[intrs\_intrv] routine to compute the intersection of 2 intervals on a circle
 \item[\htmlref{ring\_num}{sub:ring_num}] function to return the ring number corresponding to the coordinate $z$
 \item[\htmlref{vect\_prod}{sub:vect_prod}] routine to compute the vectorial product of two 3D vectors
  \end{sulist}
\end{modules}

\begin{related}
  \begin{sulist}{} %%%% NOTE the ``extra'' brace here %%%%
  \item[\htmlref{pix2ang}{sub:pix_tools}, \htmlref{ang2pix}{sub:pix_tools}] convert between angle and pixel number.
  \item[\htmlref{pix2vec}{sub:pix_tools}, \htmlref{vec2pix}{sub:pix_tools}] convert between a cartesian vector and pixel number.
  \item[\htmlref{query\_disc}{sub:query_disc}, \htmlref{query\_polygon}{sub:query_polygon},]
  \item[\htmlref{query\_strip}{sub:query_strip}, query\_triangle] render the list of pixels enclosed
  respectively in a given disc, polygon, latitude strip and triangle
  \item[\htmlref{surface\_triangle}{sub:surface_triangle}] computes the surface
in steradians of a spherical triangle defined by 3 vertices

  \end{sulist}
\end{related}

\rule{\hsize}{2mm}

\newpage


\sloppy


%%%\title{\healpix Fortran Subroutines Overview}
\docid{rand\_gauss} \section[rand\_gauss]{ }
\label{sub:rand_gauss}
\docrv{Version 2.0}
\author{Eric Hivon}
\abstract{This document describes the \healpix Fortran90 subroutine RAND\_GAUSS.}

\begin{facility}
{This routine returns a number out of a pseudo-random normal deviate (ie, its
  probability distribution is a Gaussian of mean 0 and variance 1).
}
{\modRngmod}
\end{facility}

\begin{f90function}
{\mylink{sub:rand_gauss:rng_handle}{rng\_handle}%
}
\end{f90function}

\begin{arguments}
{
\begin{tabular}{p{0.3\hsize} p{0.15\hsize} p{0.1\hsize} p{0.35\hsize}} \hline  
\textbf{name~\&~dimensionality} & \textbf{kind} & \textbf{in/out} & \textbf{description} \\ \hline
                   &   &   &                           \\ %%% for presentation
rng\_handle\mytarget{sub:rand_gauss:rng_handle} & planck\_rng & INOUT & structure of type {\tt planck\_rng}
                   containing on all information necessary to continue same
                   random sequence. \\ 
var & DP & OUT & number belonging to a pseudo-random normal deviate.
\end{tabular}
}
\end{arguments}

\begin{example}
{
use healpix\_types \\
use rngmod \\
type(planck\_rng) :: rng\_handle \\
real(dp) :: gauss \\
\\
call rand\_init(rng\_handle, 12345, 6789012)  \\
gauss = rand\_gauss(rng\_handle)
}
{
initiates a random sequence with the pair of seeds (12345, 6789012), and
generates one number out of the normal deviate.
}
\end{example}

\begin{related}
  \begin{sulist}{} %%%% NOTE the ``extra'' brace here %%%%
  \item[\htmlref{planck\_rng}{sub:planck_rng}] derived type describing RNG state
%%  \item[\htmlref{rand\_gauss}{sub:rand_gauss}] function which returns a  random normal deviate.
  \item[\htmlref{rand\_uni}{sub:rand_uni}] function which returns a random uniform deviate.
   \item[\htmlref{rand\_init}{sub:rand_init}] subroutine to initiate a random number sequence. 
  \end{sulist}
\end{related}

\rule{\hsize}{2mm}

\newpage


\sloppy


\title{\healpix Fortran Subroutines Overview}
\docid{rand\_init} \section[rand\_init]{ }
\label{sub:rand_init}
\docrv{Version 2.0}
\author{Eric Hivon}
\abstract{This document describes the \healpix Fortran90 subroutine RAND\_INIT.}

\begin{facility}
{This routine initializes, with up to 4 seeds, a randomn number sequence. 
 The generator being primed is an F90 port of an xorshift generator described
  in Marsaglia, Journal of Statistical Software 2003, vol 8.
  It has a theoretical period of $2^{128} - 1 \approx 3.4 10^{38}$.
Please refer to the ``Comment on Random Number Generator''
  in the Fortran90 facilities guidelines.
}
{\modRngmod}
\end{facility}

\begin{f90format}
{\mylink{sub:rand_init:rng_handle}{rng\_handle}%
, [\mylink{sub:rand_init:seed1}{seed1}%
, \mylink{sub:rand_init:seed2}{seed2}%
, \mylink{sub:rand_init:seed3}{seed3}%
, \mylink{sub:rand_init:seed4}{seed4}%
]}
\end{f90format}

\begin{arguments}
{
\begin{tabular}{p{0.3\hsize} p{0.15\hsize} p{0.1\hsize} p{0.35\hsize}} \hline  
\textbf{name~\&~dimensionality} & \textbf{kind} & \textbf{in/out} & \textbf{description} \\ \hline
                   &   &   &                           \\ %%% for presentation
rng\_handle\mytarget{sub:rand_init:rng_handle} & planck\_rng & OUT & structure of type {\tt planck\_rng}
                   containing on output all information necessary to continue same random sequence. \\ 
seed1\mytarget{sub:rand_init:seed1} (OPTIONAL)& I4B & IN & first seed of the random sequence. Can be of arbitray
                   sign. If set to
                   zero or not provided will be replaced internally by a non-zero hard coded value.   \\
seed2\mytarget{sub:rand_init:seed2} (OPTIONAL)& I4B & IN & second seed. Same properties as above  \\
seed3\mytarget{sub:rand_init:seed3} (OPTIONAL)& I4B & IN & third seed. Same as above.  \\
seed4\mytarget{sub:rand_init:seed4} (OPTIONAL)& I4B & IN & fourth seed. Same as above.  \\
\end{tabular}
}
\end{arguments}

\begin{example}
{
use rngmod \\
type(planck\_rng) :: rng\_handle \\
call rand\_init(rng\_handle, 12345, 6789012)  \\
}
{
initiates a random sequence with the pair of seeds (12345, 6789012).
}
\end{example}

\begin{related}
  \begin{sulist}{} %%%% NOTE the ``extra'' brace here %%%%
  \item[\htmlref{planck\_rng}{sub:planck_rng}] derived type describing RNG state
  \item[\htmlref{rand\_gauss}{sub:rand_gauss}] function which returns a  random normal deviate.
  \item[\htmlref{rand\_uni}{sub:rand_uni}] function which returns a random uniform deviate.
%%   \item[\htmlref{rand\_init}{sub:rand_init}] subroutine to initiate a random number sequence. 
  \end{sulist}
\end{related}

\rule{\hsize}{2mm}

\newpage


\sloppy


%%%\title{\healpix Fortran Subroutines Overview}
\docid{rand\_uni} \section[rand\_uni]{ }
\label{sub:rand_uni}
\docrv{Version 2.0}
\author{Eric Hivon}
\abstract{This document describes the \healpix Fortran90 subroutine RAND\_UNI.}

\begin{facility}
{This routine returns a number out of a pseudo-random uniform deviate (ie, its
  probability distribution is uniform in the range ]0,1[).
}
{\modRngmod}
\end{facility}

\begin{f90function}
{\mylink{sub:rand_uni:rng_handle}{rng\_handle}%
}
\end{f90function}

\begin{arguments}
{
\begin{tabular}{p{0.3\hsize} p{0.15\hsize} p{0.1\hsize} p{0.35\hsize}} \hline  
\textbf{name~\&~dimensionality} & \textbf{kind} & \textbf{in/out} & \textbf{description} \\ \hline
                   &   &   &                           \\ %%% for presentation
rng\_handle\mytarget{sub:rand_uni:rng_handle} & planck\_rng & INOUT & structure of type {\tt planck\_rng}
                   containing on all information necessary to continue same
                   random sequence. \\ 
var & DP & OUT & number belonging to a pseudo-random uniform deviate.
\end{tabular}
}
\end{arguments}

\begin{example}
{
use healpix\_types \\
use rngmod \\
type(planck\_rng) :: rng\_handle \\
real(dp) :: uni \\
\\
call rand\_init(rng\_handle, 12345, 6789012)  \\
uni = rand\_uni(rng\_handle)
}
{
initiates a random sequence with the pair of seeds (12345, 6789012), and
generates one number out of the uniform deviate.
}
\end{example}

\begin{related}
  \begin{sulist}{} %%%% NOTE the ``extra'' brace here %%%%
  \item[\htmlref{planck\_rng}{sub:planck_rng}] derived type describing RNG state
  \item[\htmlref{rand\_gauss}{sub:rand_gauss}] function which returns a  random normal deviate.
%%  \item[\htmlref{rand\_uni}{sub:rand_uni}] function which returns a random uniform deviate.
  \item[\htmlref{rand\_init}{sub:rand_init}] subroutine to initiate a random number sequence. 
  \end{sulist}
\end{related}

\rule{\hsize}{2mm}

\newpage


\sloppy


%%%\title{\healpix Fortran Subroutines Overview}
\docid{read\_asctab*} \section[read\_asctab*]{ }
\label{sub:read_asctab}
\docrv{Version 2.0}
\author{Eric Hivon, Frode K.~Hansen}
\abstract{This document describes the \healpix Fortran90 subroutine READ\_ASCTAB.}

\begin{facility}
{ {\bf This routine is obsolete, use \htmlref{fits2cl}{sub:fits2cl} instead} }
{\modFitstools}
\end{facility}

\rule{\hsize}{2mm}

\newpage


\sloppy


%%%\title{\healpix Fortran Subroutines Overview}
\docid{read\_bintab*} \section[read\_bintab*]{ }
\label{sub:read_bintab}
\docrv{Version 2.1}
\author{Frode K.~Hansen, Eric Hivon}
\abstract{This document describes the \healpix Fortran90 subroutine READ\_BINTAB.}

\begin{facility}
{This routine reads a \healpix map from a binary FITS-file. The routine can read a temperature map or both temperature and polarisation maps (T,Q,U)}
{\modFitstools}
\end{facility}

\begin{f90format}
{\mylink{sub:read_bintab:filename}{filename}%
, \mylink{sub:read_bintab:map}{map}%
, \mylink{sub:read_bintab:npixtot}{npixtot}%
, \mylink{sub:read_bintab:nmaps}{nmaps}%
, \mylink{sub:read_bintab:nullval}{nullval}%
, \mylink{sub:read_bintab:anynull}{anynull}%
 \optional{[,\mylink{sub:read_bintab:header}{header}%
, \mylink{sub:read_bintab:units}{units}%
, \mylink{sub:read_bintab:extno}{extno}%
]}}
\end{f90format}
\aboutoptional

\begin{arguments}
{
\begin{tabular}{p{0.4\hsize} p{0.05\hsize} p{0.05\hsize} p{0.40\hsize}} \hline  
\textbf{name~\&d~imensionality} & \textbf{kind} & \textbf{in/out} & \textbf{description} \\ \hline
                   &   &   &                           \\ %%% for presentation
filename\mytarget{sub:read_bintab:filename}(LEN=\filenamelen) & CHR & IN & filename of FITS-file containing the map(s). \\
npixtot\mytarget{sub:read_bintab:npixtot} & I4B & IN & Number of pixels to be read from map.\\
nmaps\mytarget{sub:read_bintab:nmaps}     & I4B & IN & number of maps to be read, 1 for temperature only, and 3 for (T,Q,U). \\
map\mytarget{sub:read_bintab:map}(0:npixtot-1,1:nmaps) & SP/ DP & OUT & the map read from the FITS-file.\\
nullval\mytarget{sub:read_bintab:nullval} & SP/ DP & OUT & value of missing pixels in the map. \\
anynull\mytarget{sub:read_bintab:anynull} & LGT & OUT & {\tt .TRUE.}, if there are missing pixels, and {\tt .FALSE.}
                   otherwise. \\
\optional{header\mytarget{sub:read_bintab:header}}(LEN=80)(1:)\hskip 3cm (OPTIONAL) & CHR & OUT & character string array
                   containing the FITS header read from the file. Its
                   dimension has to be defined prior to calling the
                   routine \\
\optional{units\mytarget{sub:read_bintab:units}}(LEN=*)(1:nmaps) & CHR & OUT & character string array
                   containing the physical units of each map read \\
\optional{extno\mytarget{sub:read_bintab:extno}} & I4B & IN & extension number to read the data from
                   (0 based).\default 0 (the first extension is read) 
\end{tabular}
}
\end{arguments}
\newpage

\begin{example}
{
call read\_bintab ('map.fits', map, 12*32**2, 1, nullval, anynull)  \\
}
{
Reads a \healpix temperature map from the file `map.fits' to the array
map(0:12*32**2-1,1:1). The pixel number 12*32**2 is the number of pixels in a
$\nside=32$ \healpix map. 
If there are missing pixels in the input file (with
value {\tt NaN} (Not a Number), $\pm${\tt Infinity}, or matching the FITS
keyword {\tt BAD\_DATA}) then {\tt
anynull} is {\tt .TRUE.} and these pixels get the value returned in {\tt nullval}. 
}
\end{example}

\begin{modules}
  \begin{sulist}{} %%%% NOTE the ``extra'' brace here %%%%
  \item[\textbf{fitstools}] module, containing:
  \item[printerror] routine for printing FITS error messages.
  \item[\textbf{cfitsio}] library for FITS file handling.		
  \end{sulist}
\end{modules}

\begin{related}
  \begin{sulist}{} %%%% NOTE the ``extra'' brace here %%%%
  \item[\htmlref{input\_map}{sub:input_map}] Routine which reads a map using \thedocid\ and fills missing pixels with a given value.
  \item[\htmlref{map2alm}{sub:map2alm}] Routine which analyse a map and returns the $a_{lm}$
  coefficients.
  \item[\htmlref{read\_fits\_cut4}{sub:read_fits_cut4}] Routine to read cut sky \healpix FITS maps
  \item[\htmlref{write\_plm}{sub:write_plm}, \htmlref{write\_bintab}{sub:write_bintab}] Routines to write \healpix FITS maps
  \end{sulist}
\end{related}

\rule{\hsize}{2mm}

\newpage


\sloppy


\title{\healpix Fortran Subroutines Overview}
\docid{read\_conbintab*} \section[read\_conbintab*]{ }
\label{sub:read_conbintab}
\docrv{Version 2.0}
\author{Frode K.~Hansen, Eric Hivon}
\abstract{This document describes the \healpix Fortran90 subroutine READ\_CONBINTAB.}

\begin{facility}
{This routine reads a FITS file containing  $a_{lm}$  values for constained
  realisation. The FITS file is supposed to contain one integer column with
  $index=\ell^2+\ell+m+1$ and 2 or 4 single (or double) precision columns with
  real/imaginary  $a_{lm}$  values and real/imaginary   standard deviation on
  these $a_{lm}$. It is supposed to contain either 1 or 3 extension(s) containing
  respectively the $a_{lm}$ for T and if applicable E and B.}
{\modFitstools}
\end{facility}

\begin{f90format}
{\mylink{sub:read_conbintab:filename}{filename}%
, \mylink{sub:read_conbintab:alms}{alms}%
, \mylink{sub:read_conbintab:nalms}{nalms}%
 [, \mylink{sub:read_conbintab:units}{units}%
, \mylink{sub:read_conbintab:extno}{extno}%
]}
\end{f90format}

\begin{arguments}
{
\begin{tabular}{p{0.39\hsize} p{0.05\hsize} p{0.06\hsize} p{0.40\hsize}} \hline  
\textbf{name\&dimensionality} & \textbf{kind} & \textbf{in/out} & \textbf{description} \\ \hline
                   &   &   &                           \\ %%% for presentation
filename\mytarget{sub:read_conbintab:filename}(LEN=\filenamelen) & CHR & IN & filename of FITS file containing $a_{lm}$. \\
nalms\mytarget{sub:read_conbintab:nalms} & I4B & IN & Number of  $a_{lm}$  values to read from the file. \\
alms\mytarget{sub:read_conbintab:alms}(0:nalms-1,1:6) & SP/ DP & OUT & the $a_{lm}$ read from the file. alms(i,1)
                   and alms(i,2) contain the $\ell$ and $m$ values for the ith
                   $a_{lm}$ . alms(i,3) and alms(i,4) contain the real and
                   imaginary value of the ith  $a_{lm}$ . Finally, the
                   standard deviation for the ith  $a_{lm}$  is contained in
                   alms(i,5) (real) and alms(i,6) (imaginary). \\
units\mytarget{sub:read_conbintab:units}(len=20)(1:) \hskip 6cm (OPTIONAL)& CHR & OUT & character string containing the units of the
                   $a_{\ell m}$ \\
extno\mytarget{sub:read_conbintab:extno} \hskip 8cm (OPTIONAL) & I4B & IN & extension (0 based) of the FITS file to be read

\end{tabular}
}
\end{arguments}
\newpage

\begin{example}
{
call read\_conbintab ('alms.fits',alms,65*66/2)  \\
}
{
Read 65*66/2 (the number of $a_{lm}$ needed to fill the whole range from l=0 to l=64)  $a_{lm}$  values from the file `alms.fits' into the array alms(0:65*66/2-1,1:6). 
}
\end{example}

\begin{modules}
  \begin{sulist}{} %%%% NOTE the ``extra'' brace here %%%%
  \item[\textbf{fitstools}] module, containing:
  \item[printerror] routine for printing FITS error messages.
  \item[\textbf{cfitsio}] library for FITS file handling.		
  \end{sulist}
\end{modules}

\begin{related}
  \begin{sulist}{} %%%% NOTE the ``extra'' brace here %%%%
  \item[\htmlref{alms2fits}{sub:alms2fits}, \htmlref{dump\_alms}{sub:dump_alms}] routines to write $a_{lm}$ to a FITS-file 
  \item[\htmlref{fits2alms}{sub:fits2alms}] has the same function as read\_conbintab but is more general.
  \item[\htmlref{number\_of\_alms}{sub:number_of_alms}, \htmlref{getsize\_fits}{sub:getsize_fits}]
  can be used to find out the number of $a_{lm}$ available in the file.
  \end{sulist}
\end{related}

\rule{\hsize}{2mm}

\newpage


\sloppy


%%%\title{\healpix Fortran Subroutines Overview}
\docid{read\_dbintab} \section[read\_dbintab]{ }
\label{sub:read_dbintab}
\docrv{Version 1.1}
\author{Frode K.~Hansen}
\abstract{This document describes the \healpix Fortran90 subroutine READ\_DBINTAB.}

\begin{facility}
{This routine reads a double precision binary array from a FITS-file. It is used by \healpix to read precomputed $P_{\ell m}(\theta)$ values and pixel window functions.}
{\modFitstools}
\end{facility}

\begin{f90format}
{\mylink{sub:read_dbintab:filename}{filename}%
, \mylink{sub:read_dbintab:map}{map}%
, \mylink{sub:read_dbintab:npixtot}{npixtot}%
, \mylink{sub:read_dbintab:nmaps}{nmaps}%
, \mylink{sub:read_dbintab:nullval}{nullval}%
, \mylink{sub:read_dbintab:anynull}{anynull}%
, \mylink{sub:read_dbintab:units}{units}%
}
\end{f90format}

\begin{arguments}
{
\begin{tabular}{p{0.4\hsize} p{0.05\hsize} p{0.1\hsize} p{0.35\hsize}} \hline  
\textbf{name~\&~dimensionality} & \textbf{kind} & \textbf{in/out} & \textbf{description} \\ \hline
                   &   &   &                           \\ %%% for presentation
filename\mytarget{sub:read_dbintab:filename}(LEN=\filenamelen) & CHR & IN & filename of FITS-file containing the double precision array. \\
npixtot\mytarget{sub:read_dbintab:npixtot} & I4B & IN & Number of values to be read from the file.\\
nmaps\mytarget{sub:read_dbintab:nmaps} & I4B & IN & number of 1-dim. arrays, 1 for scalar $P_{\ell m}\!\!$ s and pixel windows, 3 for scalar and tensor $P_{\ell m}\!\!$ s. \\
map\mytarget{sub:read_dbintab:map}(0:npixtot-1,1:nmaps) & DP & OUT & the array read from the FITS-file.\\
nullval\mytarget{sub:read_dbintab:nullval} & DP & OUT & value of missing pixels in the array. \\
anynull\mytarget{sub:read_dbintab:anynull} & LGT & OUT & TRUE, if there are missing pixels, and FALSE otherwise. \\
units\mytarget{sub:read_dbintab:units}(len=20)(1:nmaps) & CHR & OUT & respective physical units of the maps in the FITS file.
\end{tabular}
}
\end{arguments}
\newpage

\begin{example}
{
call read\_dbintab ('plm\_32.fits',plm,65*66*32,1,nullval,anynull)  \\
}
{
Reads precomputed scalar $P_{\ell m}(\theta)$ from the file `plm\_32.fits'. The values are returned in the array plm(0:65*66*32,1:1). The number of values 65*66*32 is the number of precomputed $P_{\ell m}(\theta)$ for a $\nside=32$, $\lmax=64$ map. If there are missing values in the file, anynull is TRUE and nullval contains the values of these pixels.
}
\end{example}

\begin{modules}
  \begin{sulist}{} %%%% NOTE the ``extra'' brace here %%%%
  \item[\textbf{fitstools}] module, containing:
  \item[printerror] routine for printing FITS error messages.
  \item[\textbf{cfitsio}] library for FITS file handling.		
  \end{sulist}
\end{modules}

\begin{related}
  \begin{sulist}{} %%%% NOTE the ``extra'' brace here %%%%
  \item[plmgen] Executable to create files with precomputed $P_{\ell m}(\theta)$.
  \item[\htmlref{write\_plm}{sub:write_plm}] Routine to create a file to be read by read\_dbintab.
  \end{sulist}
\end{related}

\rule{\hsize}{2mm}

\newpage


\sloppy


\title{\healpix Fortran Subroutines Overview}
\docid{read\_fits\_cut4} \section[read\_fits\_cut4]{ }
\label{sub:read_fits_cut4}
\docrv{Version 1.3}
\author{Eric Hivon \& Frode K.~Hansen}
\abstract{This document describes the \healpix Fortran90 subroutine READ\_FITS\_CUT4.}

\begin{facility}
{This routine reads a cut sky \healpix map from a FITS file. The format used for the
FITS file follows the one used for Boomerang98 and is adapted from COBE/DMR}
{\modFitstools}
\end{facility}

\begin{f90format}
{\mylink{sub:read_fits_cut4:filename}{filename}%
, \mylink{sub:read_fits_cut4:np}{np}%
, \mylink{sub:read_fits_cut4:pixel}{pixel}%
, \optional{[\mylink{sub:read_fits_cut4:signal}{signal}%
, \mylink{sub:read_fits_cut4:n_obs}{n\_obs}%
, \mylink{sub:read_fits_cut4:serror}{serror}%
, \mylink{sub:read_fits_cut4:header}{header}%
, \mylink{sub:read_fits_cut4:units}{units}%
, \mylink{sub:read_fits_cut4:extno}{extno}%
]}}
\end{f90format}
\aboutoptional

\begin{arguments}
{
\begin{tabular}{p{0.3\hsize} p{0.05\hsize} p{0.05\hsize} p{0.5\hsize}} \hline  
\textbf{name\&dimensionality} & \textbf{kind} & \textbf{in/out} & \textbf{description} \\ \hline
                   &   &   &                           \\ %%% for presentation
filename\mytarget{sub:read_fits_cut4:filename}(LEN=\filenamelen) & CHR & IN & FITS file to be read from,
                   containing a cut sky map \\
np\mytarget{sub:read_fits_cut4:np}               & I4B & IN & number of pixels to be read from the file \\
pixel\mytarget{sub:read_fits_cut4:pixel}(0:np-1)    & I4B & OUT & index of observed (or valid) pixels \\
\optional{signal\mytarget{sub:read_fits_cut4:signal}}(0:np-1)\hskip 2cm  (OPTIONAL)     & SP & OUT & value of signal in each observed pixel\\
\optional{n\_obs\mytarget{sub:read_fits_cut4:n_obs}}(0:np-1)     & I4B & OUT & number of observation per pixel \\
\optional{serror\mytarget{sub:read_fits_cut4:serror}}(0:np-1)     & SP  & OUT & {\em rms} of signal in pixel. (For white noise,
                   this would be $\myhtmlimage{}\propto 1/\sqrt{{\rm n\_obs}}$) \\
\optional{header\mytarget{sub:read_fits_cut4:header}}(LEN=80)(1:)    & CHR & OUT &   FITS extension header \\
\optional{units\mytarget{sub:read_fits_cut4:units}}(LEN=20)       & CHR & OUT &  maps units (applies only to
                   Signal and Serror, which are assumed to have the same units) \\
\optional{extno\mytarget{sub:read_fits_cut4:extno}}  & I4B & IN & extension number (0 based) for which map
             is read. Default = 0 (first extension). 
\end{tabular}
}
\end{arguments}

% \begin{example}
% {
% npix= read\_fits\_cut4('map.fits', nmaps=nmaps, ordering=ordering,obs\_npix=obs\_npix, nside=nside, mlpol=mlpol, type=type, polarisation=polarisation)  \\
% }
% {
% Returns 1 or 3 in nmaps, dependent on wether 'map.fits' contain only
% temperature or both temperature and polarisation maps. The pixel ordering number is found by reading the keyword ORDERING in the FITS file. If this keyword does not exist, 0 is returned.
% }
% \end{example}
\newpage
\begin{modules}
  \begin{sulist}{} %%%% NOTE the ``extra'' brace here %%%%
  \item[\textbf{fitstools}] module, containing:
  \item[printerror] routine for printing FITS error messages.
  \item[\textbf{cfitsio}] library for FITS file handling.		
  \end{sulist}
\end{modules}

\begin{related}
  \begin{sulist}{} %%%% NOTE the ``extra'' brace here %%%%
  \item[anafast] executable that reads a \healpix map and analyses it. 
  \item[synfast] executable that generate full sky \healpix maps
  \item[\htmlref{getsize\_fits}{sub:getsize_fits}] routine to know the size of a FITS file and its type (eg, full sky vs cut sky)
  \item[\htmlref{input\_map}{sub:input_map}] all purpose routine to input a map of any kind from a FITS file
  \item[\htmlref{output\_map}{sub:output_map}] subroutine to write a FITS file from a \healpix map
  \item[\htmlref{write\_fits\_cut4}{sub:write_fits_cut4}] subroutine to write a cut sky map into a FITS file
  \end{sulist}
\end{related}

\rule{\hsize}{2mm}

\newpage


\sloppy


%%%\title{\healpix Fortran Subroutines Overview}
\docid{read\_par} \section[read\_par]{ }
\label{sub:read_par}
\docrv{Version 2.0}
\author{Frode K.~Hansen, Eric Hivon}
\abstract{This document describes the \healpix Fortran90 subroutine READ\_PAR.}

\begin{facility}
{This routine reads the `NSIDE', `TFIELDS' , `MAX-LPOL', and optionally `MAX-MPOL'
  keywords from a FITS-file. These keywords desribe $\nside$, number of
  datasets (maps) and maximum multipole $\ell$ (order) and $m$ (degree) value
  for the file. If a keyword is not found in the FITS file, a value of -1 is
  returned instead. The file could eg. be a \healpix map, or a set of $a_{\ell m}$  or precomputed $P_{\ell m}(\theta)$}
{\modFitstools}
\end{facility}

\begin{f90format}
{ \mylink{sub:read_par:filename}{filename}%
, \mylink{sub:read_par:nside}{nside}%
, \mylink{sub:read_par:lmax}{lmax}%
, \mylink{sub:read_par:tfields}{tfields}%
 \optional{[, \mylink{sub:read_par:mmax}{mmax}%
]} }
\end{f90format}

\begin{arguments}
{
\begin{tabular}{p{0.35\hsize} p{0.05\hsize} p{0.05\hsize} p{0.45\hsize}} \hline  
\textbf{name~\&~dimensionality} & \textbf{kind} & \textbf{in/out} & \textbf{description} \\ \hline
                   &   &   &                           \\ %%% for presentation
filename\mytarget{sub:read_par:filename}(LEN=\filenamelen) & CHR & IN & filename of the FITS file. \\
nside\mytarget{sub:read_par:nside} & I4B & OUT & `NSIDE' keyword value from the FITS header.\\
lmax\mytarget{sub:read_par:lmax} & I4B & OUT & `MAX-LPOL' keyword value from the FITS header. \\
tfields\mytarget{sub:read_par:tfields} & I4B & OUT & `TFIELDS' keyword value from the FITS header. \\ 
\optional{mmax\mytarget{sub:read_par:mmax}} (OPTIONAL) & I4B & OUT & `MAX-MPOL' keyword value from the FITS header. \\
\end{tabular}
}
\end{arguments}

\begin{example}
{
call read\_par('plm\_128p.fits', nside, lmax, nhar)  \\
}
{
Checks the $\nside$ and maximum $\ell$ value used for the precomputed $P_{\ell
  m}(\theta)$ that are stored in the file `plm\_128p.fits'. If the file also contains tensor harmonics, nhar is returned with the value 3, otherwise it is 1.
}
\end{example}
%%%\newpage
\begin{modules}
  \begin{sulist}{} %%%% NOTE the ``extra'' brace here %%%%
  \item[\textbf{fitstools}] module, containing:
  \item[printerror] routine for printing FITS error messages.
  \item[\textbf{cfitsio}] library for FITS file handling.		
  \end{sulist}
\end{modules}

\begin{related}
  \begin{sulist}{} %%%% NOTE the ``extra'' brace here %%%%
  \item[synfast, plmgen] executables that produce FITS-files with keywords relevant to this routine.
  \end{sulist}
\end{related}

\rule{\hsize}{2mm}

\newpage

\sloppy

\title{\healpix Fortran Subroutines Overview}
\docid{real\_fft} \section[real\_fft]{ }
\label{sub:real_fft}
\docrv{Version 1.1}
\author{Martin Reinecke}
\abstract{This document describes the \healpix Fortran90 subroutine
real\_fft.}

\begin{facility}
{This routine performs a forward or backward Fast Fourier Transformation
on its argument {\tt data}.}
{\modHealpixFft}
\end{facility}

\begin{f90format}
{\mylink{sub:real_fft:data}{data}%
, \mylink{sub:real_fft:backward}{backward}%
}
\end{f90format}

\begin{arguments}
{
\begin{tabular}{p{0.3\hsize} p{0.05\hsize} p{0.1\hsize} p{0.45\hsize}} \hline  
\textbf{name~\&~dimensionality} & \textbf{kind} & \textbf{in/out} & \textbf{description} \\ \hline
                   &   &   &                           \\ %%% for presentation
data\mytarget{sub:real_fft:data}(:) & XXX & INOUT &
  array containing the input and output data.
  Can be of type real(sp) or real(dp) \\
backward\mytarget{sub:real_fft:backward} & LGT & IN & Optional argument. If present and true, perform backward transformation, else forward 
\end{tabular}}
\end{arguments}

\begin{example}
{
use healpix\_fft \\
call real\_fft (data, backward=.true.)
}
{
Performs a backward FFT on data.
}
\end{example}

\begin{related}
  \begin{sulist}{} %%%% NOTE the ``extra'' brace here %%%%
  \item[\htmlref{complex\_fft}{sub:complex_fft}] routine for FFT of complex data
  \end{sulist}
\end{related}

\rule{\hsize}{2mm}

\newpage


\sloppy


%%%\title{\healpix Fortran Subroutines Overview}
\docid{remove\_dipole*} \section[remove\_dipole*]{ }
\label{sub:remove_dipole}
\docrv{Version 2.1}
\author{Eric Hivon}
\abstract{This document describes the \healpix Fortran90 subroutine
REMOVE\_DIPOLE.}
\newcommand{\vecf}{{\rm{ \bf f}}}
\newcommand{\vecb}{{\rm{ \bf b}}}
\newcommand{\matA}{{\rm{ \bf A}}}
\newcommand{\calP}{\cal{P}}

\begin{facility}
{This routine provides a means to fit and remove the dipole and monopole
from a \healpix map. The fit is obtained by solving the linear system
\begin{equation}
	\label{eq:remove_dipole_a}
	\sum_{j=0}^{d^2-1}\ A_{ij}\ f_j = b_i
\end{equation}
 with, $d=1$ or $2$, and
\begin{equation}
	\label{eq:remove_dipole_b}
	b_i \equiv \sum_{p \in \calP} s_i(p) w(p) m(p),
\end{equation}
\begin{equation}
	\label{eq:remove_dipole_c}
	A_{ij} \equiv \sum_{p \in \calP} s_i(p) w(p) s_j(p),
\end{equation}
 where $\calP$ is the set of
valid, unmasked pixels, $m$ is the input map, $w$ is pixel weighting, while
$s_0(p) = 1$ and $s_1(p)=x,\ s_2(p)=y,\ s_3(p)=z$ are
respectively the monopole and dipole templates. The output map is then
\begin{equation}
	\label{eq:remove_dipole_d}
	m'(p) = m(p) - \sum_{i=0}^{d^2-1} f_i s_i(p).
\end{equation}
}
{\modPixTools}
\end{facility}

\begin{f90format}
{\mylink{sub:remove_dipole:nside}{nside}%
, \mylink{sub:remove_dipole:map}{map}%
, \mylink{sub:remove_dipole:ordering}{ordering}%
, \mylink{sub:remove_dipole:degree}{degree}%
, \mylink{sub:remove_dipole:multipoles}{multipoles}%
, \mylink{sub:remove_dipole:zbounds}{zbounds}%
 [, \mylink{sub:remove_dipole:fmissval}{fmissval}%
, \mylink{sub:remove_dipole:mask}{mask}%
, \mylink{sub:remove_dipole:weights}{weights}%
]}
\end{f90format}

\begin{arguments}
{
\begin{tabular}{p{0.32\hsize} p{0.05\hsize} p{0.08\hsize} p{0.45\hsize}} \hline  
\textbf{name~\&~dimensionality} & \textbf{kind} & \textbf{in/out} & \textbf{description} \\ \hline
                  &   &   &                           \\ %%% for presentation
nside\mytarget{sub:remove_dipole:nside} & I4B & IN & value of $\nside$ resolution parameter for input map\\
map\mytarget{sub:remove_dipole:map}(0:12*nside*nside-1) & SP/ DP & INOUT & \healpix map from which the monopole and dipole will be
                   removed. Those are removed from {\em all unflagged pixels},
                   even those excluded by the cut {\tt zounds} or the {\tt mask}. \\
ordering\mytarget{sub:remove_dipole:ordering} & I4B & IN & \healpix\ scheme 1:RING, 2: NESTED \\
degree\mytarget{sub:remove_dipole:degree}   & I4B & IN & multipoles to fit and remove. It is either 0 (nothing done),
                   1 (monopole only) or 2 (monopole and dipole). \\
multipoles\mytarget{sub:remove_dipole:multipoles}(0:degree*degree-1) & DP & OUT & values of best fit monopole and
                   dipole. The monopole is described as a scalar in the same
                   units as the input map, the dipole as a 3D cartesian vector, in the same units. \\
zbounds\mytarget{sub:remove_dipole:zbounds}(1:2) & DP & IN & section of the map on which to perform the
                   fit, expressed in terms of $z=\sin({\rm latitude}) =
                   \cos(\theta)$. %zbounds_sub.tex:  for alm2map, map2alm, remove_dipole: describe processed area
%zbounds2_sub.tex: for apply_mask: describe pixels set to 0 (=unprocessed area)
If zbounds(1)$<$zbounds(2), it is
performed {\em on} the strip zbounds(1)$<z<$zbounds(2); if not,
it is performed {\em outside} the strip
%  zbounds(2)$<z<$zbounds(1). % OLD
zbounds(2)$\le z \le$zbounds(1). % NEW ??
% The whole sphere is treated if \texttt{zbounds=(/-1.0\_dp, 1.0\_dp/)} or if 
% it is absent.
If absent, the whole map is processed.
 \\
fmissval\mytarget{sub:remove_dipole:fmissval}  \hskip 4cm (OPTIONAL) & SP/ DP & IN & value used to flag bad pixel on input
                   \default{-1.6375e30}. Pixels with that value are ignored
                   during the fit, and left unchanged on output.\\
mask\mytarget{sub:remove_dipole:mask}(0:12*nside*nside-1)  \hskip 4cm (OPTIONAL)& SP/ DP & IN & mask of valid pixels. 
                       Pixels with $|$mask$|<10^{-10}$ are not used for fit. Note:
                   the map is {\em not} multiplied by the mask. \\
weights\mytarget{sub:remove_dipole:weights}(0:12*nside*nside-1)  \hskip 4cm (OPTIONAL)& SP/ DP & IN & weight to be
given to each map pixel before doing the fit. By default pixels are given
a uniform weight of 1. Note:
                   the output map is {\em not} multiplied by the weights. \\

\end{tabular}
}
\end{arguments}

\newpage
\begin{example}
{
s = sin(15.0\_dp * \mylink{sub:healpix_types:deg2rad}{DEG2RAD}) \\
call \thedocid (128, map, 1, 2, multipoles, (/ s, -s /) )  \\
}
{
Will compute and remove the best fit monopole and dipole from a map with
$\nside=128$ in RING ordering scheme. The fit is performed on pixels with $|b|>15^o$.
}
\end{example}

\begin{modules}
  \begin{sulist}{} %%%% NOTE the ``extra'' brace here %%%%
  \item[\textbf{pix\_tools}] module, containing:
%  \item[\textbf{pix\_tools}] module, containing:
  \end{sulist}
\end{modules}

\begin{related}
  \begin{sulist}{} %%%% NOTE the ``extra'' brace here %%%%
  \item[\htmlref{add\_dipole}{sub:add_dipole}] routine to add a dipole and
  monopole to a map.
  \end{sulist}
\end{related}

\rule{\hsize}{2mm}

\newpage


\sloppy


\title{\healpix Fortran Subroutines Overview}
\docid{ring\_analysis} \section[ring\_analysis]{ }
\label{sub:ring_analysis}
\docrv{Version 1.1}
\author{Benjamin D.~Wandelt, Frode K.~Hansen}
\abstract{This document describes the \healpix Fortran90 subroutine RING\_ANALYSIS.}

\begin{facility}
{This subroutine computes the Fast Fourier Transform of a single ring
 of pixels
 and extends the computed coefficients up to the maximum
 $m$ of the transform.}
{\modAlmTools}
\end{facility}

\begin{f90format}
{\mylink{sub:ring_analysis:nsmax}{nsmax}%
, \mylink{sub:ring_analysis:nlmax}{nlmax}%
, \mylink{sub:ring_analysis:nmmax}{nmmax}%
, \mylink{sub:ring_analysis:datain}{datain}%
, \mylink{sub:ring_analysis:nph}{nph}%
, \mylink{sub:ring_analysis:dataout}{dataout}%
, \mylink{sub:ring_analysis:kphi0}{kphi0}%
}
\end{f90format}

\begin{arguments}
{
\begin{tabular}{p{0.4\hsize} p{0.05\hsize} p{0.1\hsize} p{0.35\hsize}} \hline  
\textbf{name~\&~dimensionality} & \textbf{kind} & \textbf{in/out} & \textbf{description} \\ \hline
                   &   &   &                           \\ %%% for presentation
nsmax\mytarget{sub:ring_analysis:nsmax} & I4B & IN & $\nside$ of the map. \\
nlmax\mytarget{sub:ring_analysis:nlmax} & I4B & IN & Maximum $\ell$ of the analysis.\\
nmmax\mytarget{sub:ring_analysis:nmmax} & I4B & IN & Maximum $m$ of the analysis.\\
nph\mytarget{sub:ring_analysis:nph} & I4B & IN & The number of points on the ring. \\ 
datain\mytarget{sub:ring_analysis:datain}(0:nph-1) & DP & IN & Function values on the ring. \\
dataout\mytarget{sub:ring_analysis:dataout}(0:nmmax) & DPC & OUT & Fourier components, replicated to $Nmmax$.\\
kphi0\mytarget{sub:ring_analysis:kphi0} & I4B & IN & 0 if the first pixel on the ring is  at
                   $\phi=0$; 1 otherwise. \\
\end{tabular}
}
\end{arguments}

\begin{example}
{
call ring\_analysis(64,128,128,datain,8,dataout,0)  \\
}
{
Returns the periodically extended complex 
Fourier Transform of datain in
dataout. They are returned in the following order: 0 1 2 3 4 5 6 7
6 5 4 3 2 1 $0\dots$. The value $kphi0=0$ specifies that no phase
factor needed to be applied, because the ring starts at $\phi=0$.
}
\end{example}

\begin{modules}
  \begin{sulist}{} %%%% NOTE the ``extra'' brace here %%%%
  \item[\textbf{healpix\_fft}] module.
  \end{sulist}
\end{modules}

\begin{related}
  \begin{sulist}{} %%%% NOTE the ``extra'' brace here %%%%
  \item[\htmlref{ring\_synthesis}{sub:ring_synthesis}] Inverse transform (complex-to-real), used in
  \htmlref{alm2map}{sub:alm2map},
  \htmlref{alm2map\_der}{sub:alm2map_der} and synfast
  \end{sulist}
\end{related}

\rule{\hsize}{2mm}

\newpage



\sloppy

\title{\healpix Fortran Subroutines Overview}
\docid{ring\_num} \section[ring\_num]{ }
\label{sub:ring_num}
\docrv{Version 1.1}
\author{Frode K.~Hansen}
\abstract{This document describes the \healpix Fortran90 function RING\_NUM.}


\begin{facility}
{This function returns the ring number for a z coordinate.}
{\modPixTools}
\end{facility}

\begin{f90function}
{\mylink{sub:ring_num:nside}{nside}%
, \mylink{sub:ring_num:z}{z}%
}
\end{f90function}

\begin{arguments}
{
\begin{tabular}{p{0.4\hsize} p{0.05\hsize} p{0.1\hsize} p{0.35\hsize}} \hline  
\textbf{name\&dimensionality} & \textbf{kind} & \textbf{in/out} & \textbf{description} \\ \hline
                   &   &   &                           \\ %%% for presentation
nside\mytarget{sub:ring_num:nside} & I4B & IN & the $\nside$ parameter of the map. \\
z\mytarget{sub:ring_num:z} & DP & IN & the z coordinate to find the ring number for. \\

\end{tabular}
}
\end{arguments}

\begin{example}
{
print*,ring\_num(256, 0.5)  \\
}
{
Prints the ring number of the ring at position $z=0.5$.
}
\end{example}

\begin{modules}
  \begin{sulist}{} %%%% NOTE the ``extra'' brace here %%%%
 \item[None]	
  \end{sulist}
\end{modules}
\newpage
\begin{related}
  \begin{sulist}{} %%%% NOTE the ``extra'' brace here %%%%
 \item[\htmlref{in\_ring}{sub:in_ring}] Returns the pixels in a slice on a given ring.
  \end{sulist}
\end{related}

\rule{\hsize}{2mm}

\newpage


\sloppy


%%%\title{\healpix Fortran Subroutines Overview}
\docid{ring\_synthesis} \section[ring\_synthesis]{ }
\label{sub:ring_synthesis}
\docrv{Version 1.1}
\author{Frode K.~Hansen}
\abstract{This document describes the \healpix Fortran90 subroutine RING\_SYNTHESIS.}

\begin{facility}
{}
{\modAlmTools}
\end{facility}

\begin{f90format}
{\mylink{sub:ring_synthesis:nsmax}{nsmax}%
, \mylink{sub:ring_synthesis:nlmax}{nlmax}%
, \mylink{sub:ring_synthesis:nmmax}{nmmax}%
, \mylink{sub:ring_synthesis:datain}{datain}%
, \mylink{sub:ring_synthesis:nph}{nph}%
, \mylink{sub:ring_synthesis:dataout}{dataout}%
, \mylink{sub:ring_synthesis:kphi0}{kphi0}%
}
\end{f90format}

\begin{arguments}
{
\begin{tabular}{p{0.4\hsize} p{0.05\hsize} p{0.1\hsize} p{0.35\hsize}} \hline  
\textbf{name~\&~dimensionality} & \textbf{kind} & \textbf{in/out} & \textbf{description} \\ \hline
                   &   &   &                           \\ %%% for presentation
nsmax\mytarget{sub:ring_synthesis:nsmax} & I4B & IN & $\nside$ of the map.  \\
nlmax\mytarget{sub:ring_synthesis:nlmax} & I4B & IN & Maximum $\ell$ of the analysis. \\
nmmax\mytarget{sub:ring_synthesis:nmmax} & I4B & IN & Maximum $m$ of the analysis. \\
nph\mytarget{sub:ring_synthesis:nph} & I4B & IN & The number of points on the ring. \\ 
datain\mytarget{sub:ring_synthesis:datain}(0:nmmax) & DPC & IN & Fourier components as computed from the $a_{lm}$. \\
dataout\mytarget{sub:ring_synthesis:dataout}(0:nph-1) & DP & OUT & Synthesized function values on the ring. \\
kphi0\mytarget{sub:ring_synthesis:kphi0} & I4B & IN &  0 if the first pixel on the ring is  at
                   $\phi=0$; 1 otherwise. \\
\end{tabular}
}
\end{arguments}

\begin{example}
{
call ring\_synthesis(64,128,128,datain,8,dataout,1)   \\
}
{
This computes the inverse (complex-to-real) Fast Fourier Transform for
the second ring from the pole, containing $8$ pixels, for a map
resolution of $\nside=64$. $128$ complex Fourier
compoments contribute to these 8 pixels. The value $kphi0=1$ specifies
that a phase factor needed to be applied to correctly
rotate the ring into position on the \healpix grid.
}
\end{example}

\begin{modules}
  \begin{sulist}{} %%%% NOTE the ``extra'' brace here %%%%
  \item[\textbf{healpix\_fft}] module.
  \end{sulist}
\end{modules}

\begin{related}
  \begin{sulist}{} %%%% NOTE the ``extra'' brace here %%%%
  \item[\htmlref{ring\_analysis}{sub:ring_analysis}] Forward transform, used in
  \htmlref{map2alm}{sub:map2alm} and anafast 
  \end{sulist}
\end{related}

\rule{\hsize}{2mm}

\newpage


\sloppy


%%%\title{\healpix Fortran Subroutines Overview}
\docid{rotate\_alm*} \section[rotate\_alm*]{ }
\label{sub:rotate_alm}
\docrv{Version 2.0}
\author{Eric Hivon}
\abstract{This document describes the \healpix Fortran90 subroutine ROTATE\_ALM.}

\begin{facility}
{This routine transform the scalar (and tensor) $a_{\ell m}$ coefficients to
emulate the effect of an arbitrary rotation of the underlying map. The rotation is done
directly on the $a_{\ell m}$ using the Wigner rotation matrices, computed by
recursion.
To rotate the $a_{\ell m}$ for $\ell \leq \lmax$ the number of
operations scales like $\lmax^3$.}
{\modAlmTools}
\end{facility}

\begin{f90format}
{\mylink{sub:rotate_alm:nlmax}{nlmax}%
, \mylink{sub:rotate_alm:alm_TGC}{alm\_TGC}%
, \mylink{sub:rotate_alm:psi}{psi}%
, \mylink{sub:rotate_alm:theta}{theta}%
, \mylink{sub:rotate_alm:phi}{phi}%
}
\end{f90format}

\begin{arguments}
{
\begin{tabular}{p{0.36\hsize} p{0.05\hsize} p{0.09\hsize} p{0.40\hsize}} \hline  
\textbf{name~\&~dimensionality} & \textbf{kind} & \textbf{in/out} & \textbf{description} \\ \hline
                   &   &   &                           \\ %%% for presentation
nlmax\mytarget{sub:rotate_alm:nlmax} & I4B & IN & maximum $\ell$ value for the $a_{\ell m}$.\\
alm\_TGC\mytarget{sub:rotate_alm:alm_TGC}(1:p,0:nlmax,0:nlmax) & SPC/ DPC & INOUT & complex $a_{\ell m}$ values
                   before and after rotation of the coordinate system.  
	The first index here runs from 1:1 for
                   temperature only, and 1:3 for polarisation. In the latter
                   case,  1=T, 2=E, 3=B. \\
% \end{tabular}
% \begin{tabular}{p{0.36\hsize} p{0.05\hsize} p{0.09\hsize} p{0.40\hsize}}
%                    \hline  
psi\mytarget{sub:rotate_alm:psi}	& DP & IN & first rotation: angle $\psi$ about the z-axis.
All angles are in radians and should lie in [-2$\pi$,2$\pi$], the rotations are
active and the referential system is assumed to be right handed, the routine
\htmlref{coordsys2euler\_zyz}{sub:coordsys2euler_zyz} can be used to generate
the Euler angles
$\psi, \theta, \varphi$ for rotation between standard astronomical coordinate
systems; \\
theta\mytarget{sub:rotate_alm:theta}	& DP & IN & second rotation: angle $\theta$ about the original
(unrotated) y-axis; \\
phi\mytarget{sub:rotate_alm:phi}	& DP & IN & third rotation: angle $\varphi$ about the original (unrotated) z-axis;
\end{tabular}
}
\end{arguments}

\begin{example}
{
use alm\_tools, only: rotate\_alm \\
...\\
call rotate\_alm(64, alm\_TGC, PI/3., 0.5\_dp, 0.0\_dp)  \\
}
{
Transforms scalar and tensor $a_{\ell m}$ for $\lmax = \mmax = 64$ to emulate a rotation of the underlying map by
($\psi=\pi/3, \theta=0.5, \varphi=0\myhtmlimage{}$).
}
\end{example}

\begin{example}
{
use coord\_v\_convert, only: coordsys2euler\_zyz \\
use alm\_tools, only: rotate\_alm \\
...\\
call coordsys2euler\_zyz(2000.0\_dp, 2000.0\_dp, 'E', 'G', psi, theta, phi) \\
call rotate\_alm(64, alm\_TGC, psi, theta, phi)  \\
}
{
Rotate the $a_{\ell m}$ from Ecliptic to Galactic coordinates.
}
\end{example}

% \begin{modules}
%   \begin{sulist}{} %%%% NOTE the ``extra'' brace here %%%%
%   \item[\textbf{alm\_tools}] module, containing:
% 	\item[\htmlref{generate\_beam}{sub:generate_beam}] routine to generate beam window function
% 	\item[\htmlref{pixel\_window}{sub:pixel_window}] routine to generate pixel window function
%   \end{sulist}
% \end{modules}

\begin{related}
  \begin{sulist}{} %%%% NOTE the ``extra'' brace here %%%%
  \item[\htmlref{coordsys2euler\_zyz}{sub:coordsys2euler_zyz}] can be used to generate
the Euler angles $\psi, \theta, \varphi \myhtmlimage{}$ for rotation between standard astronomical coordinate systems
  \item[\htmlref{create\_alm}{sub:create_alm}] Routine to create $a_{\ell m}$ coefficients.
  \item[\htmlref{alter\_alm}{sub:alter_alm}] Routine to modify $a_{\ell m}$
  coefficients to apply or remove the effect of an instrumental beam.
  \item[\htmlref{map2alm}{sub:map2alm}]  Routines to analyze a \healpix sky map into its $a_{\ell m}$
  coefficients.
  \item[\htmlref{alm2map}{sub:alm2map}] Routines to synthetize a \healpix sky map from its $a_{\ell m}$
  coefficients.
  \item[\htmlref{alms2fits}{sub:alms2fits}, \htmlref{dump\_alms}{sub:dump_alms}]
  Routines to save a set of $a_{\ell m}$ in a FITS file.  
  \item[\htmlref{xcc\_v\_convert}{sub:xcc_v_convert}] rotates a 3D coordinate
vector from one astronomical coordinate system to another.
  \end{sulist}
\end{related}

\rule{\hsize}{2mm}

\newpage


\sloppy


%%%\title{\healpix Fortran Subroutines Overview}
\docid{same\_shape\_pixels\_nest, same\_shape\_pixels\_ring} \section[same\_shape\_pixels\_nest, same\_shape\_pixels\_ring]{ }
\label{sub:same_shape_pixels_xxx}
\docrv{Version 1.0}
\author{E. Hivon}
\abstract{This document describes the \healpix Fortran90 subroutines
  SAME\_SHAPE\_PIXELS\_RING and SAME\_SHAPE\_PIXELS\_NEST.}

\begin{facility}
{These routines provide the ordered list of all \healpix pixels having the same shape
  as a given template, for a resolution parameter $\nside$. Depending on the
  template considered the number of such pixels is either 8, 16, 4$\nside$ or
  8$\nside$.

%% Any pixel can be {\em matched in shape}
%%   to a single of these templates by a combination of  a rotation around the polar axis with 
%%   reflexion(s) around a meridian and/or the equator. 

The template pixels are all located in the Northern Hemisphere, or on the
 Equator.
They are chosen to have their center located at
\begin{eqnarray}
	\label{eq:same_shape_pixel_xxx}
     z=\cos(\theta)\ge 2/3 \mycomma    0< \phi \leq \pi/2,   \nonumber \\            %[Nside*(Nside+2)/4]
     2/3 > z \geq 0 \mycomma \phi=0, \quad{\rm or}\quad  \phi=\frac{\pi}{4\nside}.  %\nonumber %[Nside]
\myhtmlimage{}
\end{eqnarray}
 They are numbered continuously from 0, starting at the North Pole, with the index
 increasing in $\phi$, and then increasing for decreasing $z$.
}
{\modPixTools}
\end{facility}

\docid{same\_shape\_pixels\_nest}
\begin{f90format}
{%
 \mylink{sub:same_shape_pixels_xxx:nside}{nside}, 
 \mylink{sub:same_shape_pixels_xxx:template}{template}
[,~\mylink{sub:same_shape_pixels_xxx:list}{list}, 
 \mylink{sub:same_shape_pixels_xxx:reflexion}{reflexion}, 
 \mylink{sub:same_shape_pixels_xxx:nrep}{nrep}]}
\end{f90format}
\docid{same\_shape\_pixels\_ring}
\begin{f90format}
%{ nside, template [,~list, reflexion, nrep]}
{%
 \mylink{sub:same_shape_pixels_xxx:nside}{nside}, 
 \mylink{sub:same_shape_pixels_xxx:template}{template}
[,~\mylink{sub:same_shape_pixels_xxx:list}{list}, 
 \mylink{sub:same_shape_pixels_xxx:reflexion}{reflexion}, 
 \mylink{sub:same_shape_pixels_xxx:nrep}{nrep}]}
\end{f90format}
%\ mylink: to avoid automatic processing by make_internal_links.sh

\begin{arguments}
{
\begin{tabular}{p{0.28\hsize} p{0.05\hsize} p{0.1\hsize} p{0.47\hsize}} \hline  
\textbf{name~\&~dimensionality} & \textbf{kind} & \textbf{in/out} & \textbf{description} \\ \hline
                   &   &   &                           \\ %%% for presentation
nside\mytarget{sub:same_shape_pixels_xxx:nside} & I4B & IN & the \healpix $\nside$ parameter. \\
template\mytarget{sub:same_shape_pixels_xxx:template} & I4B/ I8B & IN & identification number of the
                   template pixel (the numbering
                   scheme of the pixel templates is the same for both routines). \\
list(0:nrep-1)\mytarget{sub:same_shape_pixels_xxx:list} \hskip 3cm OPTIONAL & I4B/ I8B & OUT & pointer containing the ordered list of NESTED/RING scheme
                   identification numbers (in \{0,$12\nside^2-1$\})
  of all pixels having the same shape as the template provided. The routines
                   will allocate the {\tt list} array if it is not allocated
                   upon calling. \\
reflexion(0:nrep-1)\mytarget{sub:same_shape_pixels_xxx:reflexion} \hskip 3cm OPTIONAL & I4B & OUT & pointer containing the transformation(s) (in
                   \{0, 3\}) to
                   apply to each of the returned pixels to match exactly in
                   shape and position its respective template. 0: rotation around the polar axis only,
                   1: rotation + East-West swap (ie, reflexion around meridian),
                   2: rotation + North-South swap (ie, reflexion around
                   Equator), 3: rotation + East-West and North-South swaps. The routines
                   will allocate the {\tt list} array if it is not allocated
                   upon calling. \\
nrep\mytarget{sub:same_shape_pixels_xxx:nrep} \hskip 4cm OPTIONAL & I4B/ I8B  & OUT & number of pixels having the same template (either 8, 16, 4$\nside$ or
  8$\nside$).
\end{tabular}
}
\end{arguments}

\begin{example}
{
use \htmlref{healpix\_modules}{sub:healpix_modules} \\
integer, parameter :: IXB = I4B  ! for nside <= 8192\\
!integer, parameter :: IXB = I8B  ! for any valid nside\\
integer(I4B):: nside\\
integer(IXB):: template, nrep\\
integer(I4B), dimension(:), pointer :: listref\\
integer(IXB), dimension(:), pointer :: listpix\\
\\
allocate(listref(0:0)) ! only the lower bound matters\\
allocate(listpix(0:0)) ! only the lower bound matters\\
nside = 256\\
template = 1234\\
call same\_shape\_pixels\_ring(nside, template, list=listpix, reflexion=listref, nrep=nrep) \\
print*,nrep\\
print*,listpix(0:nrep-1)\\
print*,listref(0:nrep-1)\\
}
{
Returns in {\tt listpix} the RING-scheme index of the all the pixels having
the same shape as the template \#1234 for $\nside=256$. Upon return {\tt listref} will
contain the rotation/reflexions to apply to each pixel returned to match the template,
and {\tt nrep} will contain the number of pixels having that same shape (16 in that case).
Note that some variables (corresponding to arguments \texttt{template}, \texttt{list} and \texttt{nrep})
must be of type \texttt{I8B} instead of \texttt{I4B} if $\nside>8192$ is to be used.
}
\end{example}
\begin{related}
  \begin{sulist}{} %%%% NOTE the ``extra'' brace here %%%%
  \item[\htmlref{nside2templates}{sub:nside2ntemplates}] returns the
  number of template pixel shapes available for a given $\nside$.
  \item[\htmlref{template\_pixel\_ring}{sub:template_pixel_xxx}] 
  \item[\htmlref{template\_pixel\_nest}{sub:template_pixel_xxx}] 
  return
  the template shape matching the pixel provided
  \end{sulist}
\end{related}

\rule{\hsize}{2mm}

% special format, TBD


\sloppy


%%%\title{\healpix Fortran Subroutines Overview}
\docid{scan\_directories} \section[scan\_directories]{ }
\label{sub:scan_directories}
\docrv{Version 1.0}
\author{E. Hivon}
\abstract{This document describes the \healpix Fortran90 subroutine SCAN\_DIRECTORIES.}

\begin{facility}
{Function to scan a set of directories for a given file
}
{\modParamfileIo}
\end{facility}

\begin{f90function}
{\mylink{sub:scan_directories:directories}{directories}%
, \mylink{sub:scan_directories:filename}{filename}%
, \mylink{sub:scan_directories:fullpath}{fullpath}%
}
\end{f90function}

\begin{arguments}
{
\begin{tabular}{p{0.3\hsize} p{0.05\hsize} p{0.1\hsize} p{0.45\hsize}} \hline  
\textbf{name\&dimensionality} & \textbf{kind} & \textbf{in/out} & \textbf{description} \\ \hline
                   &   &   &                           \\ %%% for presentation
directories\mytarget{sub:scan_directories:directories} & CHR & IN & contains the set of directories (up to 20), separated by an ASCII
                   character of value $<$ 32  (see {\tt{\htmlref{concatnl}{sub:concatnl}}}). During the
                   search, it is assumed that the
                   given directories and filename can be separated by nothing,
                   a $/$ (slash) or a $\backslash$ (backslash)\\
filename\mytarget{sub:scan_directories:filename} & CHR & IN & the file to be found. \\
fullpath\mytarget{sub:scan_directories:fullpath} & CHR & OUT & returns the full path to the first occurrence of the
                   file among the directories provided. Empty if the file is not
                   found. The search is not recursive.  \\
var & LGT & OUT & set to true if the file is found, to false otherwise.\\
\end{tabular}
}
\end{arguments}

\begin{example}
{
use paramfile\_io \\
character(len=filenamelen) :: dirs, full \\
logical(lgt) :: found \\
dirs = concatnl('dir1','$/$dir2','$/$dir2$/$subdir1$/$') !build directories list\\
found = \thedocid(dirs, 'myfile', full) ! do the search \\
if (found) print*,trim(full)
}
{{Search for 'myfile' in the directories  'dir1', '$/$dir2', '$/$dir2$/$subdir1$/$'}}
\end{example}

\begin{related}
  \begin{sulist}{} %%%% NOTE the ``extra'' brace here %%%%
  \item[\htmlref{parse\_xxx}{sub:parse_xxx}] parse an ASCII file for parameters definition
  \item[\htmlref{concatnl}{sub:concatnl}] concatenates a set of substrings into one string, interspaced
  with LineFeed character
  \end{sulist}
\end{related}

\rule{\hsize}{2mm}


%   !=========================================================================
%    subroutine size_holes_nest(nside, mask, nholes, nph, &
%         &                     tags, sizeholes, listpix)
%      !=========================================================================
%      ! SIZE_HOLES_NEST: subroutine to determine size (=number of pixels) of holes
%      ! A hole is the set of all adjacent pixels initially set to 0.
%      ! 2 pixels are adjacent if they have at least one point in common.
%      !
%      ! Nside: integer, IN, resolution parameter
%      ! Mask(0:): integer 1D array, IN, each pixel must be either 1 (=valid) or 0 (=invalid)
%      ! Nholes: integer, OUT, number of holes
%      ! Nph:    integer, OUT, total number of pixels in holes
%      ! Tags(0:npix-1): integer 1D array, OPTIONAL, OUT:
%      !                   invalid pixels belonging to largest hole have value -1,
%      !                   invalid pixels belonging to second largest hole: -2,
%      !                   and so on, while valid pixels have value 1
%      ! Sizeholes(0:Nholes-1): integer pointer, OPTIONAL, OUT: respective size of each hole
%      ! Listpix(0:Nph+Nholes): integer pointer, OPTIONAL, OUT: list of pixels in each hole
%      !
%      !========================================================================
\sloppy
\docid{size\_holes\_nest}\section[size\_holes\_nest]{ }
\label{sub:size_holes_nest}
\docrv{Version 1.0}
\author{Eric Hivon}
\abstract{This document describes the \healpix Fortran90 subroutine SIZE\_HOLES\_NEST.}

\begin{facility}
{For a input binary mask in NESTED ordering, \thedocid\ identifies the pixels
located on the inner boundary of {\em invalid} regions
}
{\modMaskTools}
\end{facility}

\begin{f90format}
{\mylink{sub:size_holes_nest:nside}{nside}%
, \mylink{sub:size_holes_nest:mask}{mask}%
, \mylink{sub:size_holes_nest:nholes}{nholes}%
, \mylink{sub:size_holes_nest:nph}{nph}%
,  \optional{[\mylink{sub:size_holes_nest:tags}{tags}%
, \mylink{sub:size_holes_nest:sizeholes}{sizeholes}%
, \mylink{sub:size_holes_nest:listpix}{listpix}%
]}}
\end{f90format}
\aboutoptional

\begin{arguments}
{
\begin{tabular}{p{0.25\hsize} p{0.05\hsize} p{0.08\hsize} p{0.50\hsize}} \hline  
\textbf{name~\&~dim.} & \textbf{kind} & \textbf{in/out} & \textbf{description} \\ \hline
                   &   &   &                           \\ %%% for presentation
nside\mytarget{sub:size_holes_nest:nside} & I4B & IN & The $nside$ value of the input mask. \\
mask\mytarget{sub:size_holes_nest:mask}(0:Npix-1) & I4B & IN & Input binary NESTED-ordered mask. Npix =
12*nside*nside\\
nholes\mytarget{sub:size_holes_nest:nholes} & I4B & OUT & Number of holes found \\
nph\mytarget{sub:size_holes_nest:nph} & I4B & OUT & Number of pixels in holes 
\\
\optional{tags\mytarget{sub:size_holes_nest:tags}}(0:Npix-1) \hskip 2cm  (OPTIONAL) & I4B & OUT & Pointer allocated by \thedocid, containing
a sky map in which {\em invalid} pixels belonging to the largest hole have
value -1, those belonging to the second largest hole have value -2, and so on,
while valid pixels keep value +1.
\\
\optional{sizeholes\mytarget{sub:size_holes_nest:sizeholes}}(0:nholes-1) &I4B & OUT & Pointer allocated by \thedocid,
containing on output the respective size of each hole (in decreasing order).
Eg, {\tt sizeholes}(0) is the number of pixels in the largest hole (taking value -1 in
{\tt tags}).
\\
\optional{listpix\mytarget{sub:size_holes_nest:listpix}}(0:nholes+nph) & I4B & OUT & Pointer allocated by \thedocid,
containing on output the indexed list of pixels in each hole. Pixels located in the first (and largest)
hole are given by {\tt listpix(listpix(0):listpix(1)-1)}
\end{tabular}
}
\end{arguments}

\begin{example}
{
use healpix\_types \\
use healpix\_modules \\
% use pix\_tools, only : nside2npix \\
% use alm\_tools, only : size\_holes\_nest \\
% integer(I4B) :: nside, lmax, mmax, npix, spin\\
% real(SP), dimension(:,:), allocatable :: map \\
% complex(SPC), dimension(:,:,:), allocatable :: alm \\
% \ldots \\
% nside=256 ; lmax=512 ; mmax=lmax ; spin=4\\
% npix=nside2npix(nside)\\
% allocate(alm(1:2,0:lmax,0:mmax))\\
% allocate(map(0:npix-1,1:2))\\
\ldots \\
call \thedocid(nside, mask, nholes, nph)  \\
}
{???
}
\end{example}

\begin{modules}
  \begin{sulist}{} %%%% NOTE the ``extra'' brace here %%%%
  \item[\textbf{mask\_tools}] mask processing module (see related routines below)
  \end{sulist}
\end{modules}

\begin{related}
  \begin{sulist}{} %%%% NOTE the ``extra'' brace here %%%%
	\maskToolsRelated
  \end{sulist}
\end{related}

\rule{\hsize}{2mm}

\newpage



\sloppy

\title{\healpix Fortran Subroutines Overview}
\docid{string, strlowcase, strupcase} \section[string, strlowcase, strupcase]{ }
\label{sub:string}
\docrv{Version 2.1}
\author{Eric Hivon}
\abstract{This document describes the \healpix Fortran90 functions in module MISC\_UTILS.}

\begin{facility}
{The Fortran90 module misc\_utils contains three functions to create or
  manipulate character strings.}
{\modMiscUtils}
\end{facility}

\begin{arguments}
{
\begin{tabular}{p{0.28\hsize} p{0.05\hsize} p{0.10\hsize} p{0.47\hsize}} \hline  
\textbf{name~\&~dimensionality} & \textbf{kind} & \textbf{in/out} & \textbf{description} \\ \hline
                   &   &   &                           \\ %%% for presentation
number & LGT/ I4B/ SP/ DP & IN & number or boolean flag to be turned into a character string. \\
instring & CHR & IN & arbitrary character string. \\
outstring & CHR & --- & output character string. \\
format \hskip 3cm OPTIONAL & CHR & IN & character string describing Fortran
                   format of output. %% \\
%% upstring & CHR & --- & uppercase character string. \\
%% lowstring & CHR & --- & lowercase character string. 
\end{tabular}
}
\end{arguments}

\rule{\hsize}{0.7mm}
\textsc{\large{\textbf{FUNCTIONS: }}}\hfill\newline
{\tt outstring = string(number [,format])} 

 \begin{tabular}{@{}p{0.3\hsize}@{\hspace{1ex}}p{0.7\hsize}@{}}
                         & returns in {\tt outstring} its argument {\tt number} converted to a
                                         character string. If {\tt format} is provided it is used to
                                         format the output, if not, the fortran default format
                                         matching {\tt number}'s type is used. \\
     \end{tabular}\\\\

{\tt outstring = strlowcase(instring)} 

 \begin{tabular}{@{}p{0.3\hsize}@{\hspace{1ex}}p{0.7\hsize}@{}}
                         & returns in {\tt outstring} its argument {\tt instring}
                                         converted to lowercase. ASCII characters in the [A-Z] range
                                         are mapped to [a-z], while all others remain unchanged.\\
     \end{tabular}\\\\

{\tt outstring = strupcase(instring)} 

 \begin{tabular}{@{}p{0.3\hsize}@{\hspace{1ex}}p{0.7\hsize}@{}}
                         & returns in {\tt outstring} its argument {\tt instring}
                                         converted to uppercase. ASCII characters in the [a-z] range
                                         are mapped to [A-Z], while all others remain unchanged.\\
     \end{tabular}\\\\


\begin{example}
{
use misc\_utils \\
character(len=24) :: s1 \\
s1 = string(123,'(i5.5)') \\
print*, trim(s1) \\
print*,trim(strupcase('*aBcD-123')) \\
print*,trim(strlowcase('*aBcD-123')) \\
}
{ Will printout {\tt 00123}, {\tt *ABCD-123} and {\tt *abcd-123}.
}
\end{example}

%% \begin{modules}
%%   \begin{sulist}{} %%%% NOTE the ``extra'' brace here %%%%
%%  \item[] 
%%   \end{sulist}
%% \end{modules}

%% \begin{related}
%%   \begin{sulist}{} %%%% NOTE the ``extra'' brace here %%%%
%%   \item[]
%%   \end{sulist}
%% \end{related}

\rule{\hsize}{2mm}

\newpage


\sloppy


%%%\title{\healpix Fortran Subroutines Overview}
\docid{surface\_triangle} \section[surface\_triangle]{ }
\label{sub:surface_triangle}
\docrv{Version 1.2}
\author{Eric Hivon}
\abstract{This document describes the \healpix Fortran90 subroutine SURFACE\_TRIANGLE.}

\begin{facility}
{Returns the surface in steradians of the spherical triangle described by its
three vertices} 
{\modPixTools}
\end{facility}

\begin{f90format}
{\mylink{sub:surface_triangle:v1}{v1}%
, \mylink{sub:surface_triangle:v2}{v2}%
, \mylink{sub:surface_triangle:v3}{v3}%
, \mylink{sub:surface_triangle:surface}{surface}%
}
\end{f90format}

\begin{arguments}
{
\begin{tabular}{p{0.25\hsize} p{0.05\hsize} p{0.1\hsize} p{0.5\hsize}} \hline 
\textbf{name\&dimensionality} & \textbf{kind} & \textbf{in/out} & \textbf{description} \\ \hline
                   &   &   &                           \\ %%% for presentation
v1\mytarget{sub:surface_triangle:v1}(3) & DP & IN & cartesian vector pointing at the triangle first vertex. \\
v2\mytarget{sub:surface_triangle:v2}(3) & DP & IN & cartesian vector pointing at the triangle second vertex. \\
v3\mytarget{sub:surface_triangle:v3}(3) & DP & IN & cartesian vector pointing at the triangle third vertex. \\
surface\mytarget{sub:surface_triangle:surface} & DP & OUT & surface of the triangle in steradians.
\end{tabular}
}
\end{arguments}

\begin{example}
{
use healpix\_types \\
use pix\_tools,    only : surface\_triangle \\
real(DP) :: surface, one = 1.0\_dp \\
call surface\_triangle((/1,0,0/)*one, (/0,1,0/)*one, (/0,0,1/)*one, surface)  \\
print*, surface
}
{
Returns the surface in steradians of the triangle defined by the octant ($x,y,z>0$) : 1.5707963267948966
}
\end{example}
% \newpage
% \begin{modules}
%   \begin{sulist}{} %%%% NOTE the ``extra'' brace here %%%%
%  \item[\htmlref{in\_ring}{sub:in_ring}] routine to find the pixels in a certain slice of a given ring.		
%  \item[\htmlref{ring\_num}{sub:ring_num}] function to return the ring number corresponding to the coordinate $z$
%   \end{sulist}
% \end{modules}

\begin{related}
  \begin{sulist}{} %%%% NOTE the ``extra'' brace here %%%%
  \item[\htmlref{pix2ang}{sub:pix_tools}, \htmlref{ang2pix}{sub:pix_tools}] convert between angle and pixel number.
  \item[\htmlref{pix2vec}{sub:pix_tools}, \htmlref{vec2pix}{sub:pix_tools}] convert between a cartesian vector and pixel number.
  \item[\htmlref{query\_disc}{sub:query_disc}, \htmlref{query\_polygon}{sub:query_polygon},]
  \item[\htmlref{query\_strip}{sub:query_strip}, \htmlref{query\_triangle}{sub:query_triangle}] render the list of pixels enclosed
  respectively in a given disc, polygon, latitude strip and triangle

  \end{sulist}
\end{related}

\rule{\hsize}{2mm}

\newpage


\sloppy


\title{\healpix Fortran Subroutines Overview}
\docid{template\_pixel\_nest, template\_pixel\_ring}
\section[template\_pixel\_nest, template\_pixel\_ring]{ }
\label{sub:template_pixel_xxx}
\docrv{Version 1.0}
\author{E. Hivon}
\abstract{This document describes the \healpix Fortran90 subroutines
  TEMPLATE\_PIXEL\_RING and TEMPLATE\_PIXEL\_NEST.}

\begin{facility}
{Routines to provide the index of the template pixel associated with a given
  \healpix pixel, for a resolution parameter $\nside$. 

Any pixel can be {\em matched in shape}
  to a single of these templates by a combination of  a rotation around the polar axis with 
  reflexion(s) around a meridian and/or the equator. 

The template pixels are all located in the Northern Hemisphere, or on the
 Equator.
They are chosen to have their center located at
\begin{eqnarray}
     z=\cos(\theta)\ge 2/3,  &\ &    0< \phi \leq \pi/2,   \nonumber\\            %[Nside*(Nside+2)/4]
     2/3 > z \geq 0,  &\ & \phi=0, \quad{\rm or}\quad  \phi=\frac{\pi}{4\nside}.  \nonumber %[Nside]
\myhtmlimage{}
\end{eqnarray}
 They are numbered continuously from 0, starting at the North Pole, with the index
 increasing in $\phi$, and then increasing for decreasing $z$.
}
{\modPixTools}
\end{facility}

\docid{template\_pixel\_nest}
\begin{f90format}
{%
\mylink{sub:template_pixel_xxx:nside}{nside}, 
\mylink{sub:template_pixel_xxx:pixel_nest}{pixel\_nest}, 
\mylink{sub:template_pixel_xxx:template}{template}, 
\mylink{sub:template_pixel_xxx:reflexion}{reflexion}%
}
\end{f90format}
\docid{template\_pixel\_ring}
\begin{f90format}
{%
\mylink{sub:template_pixel_xxx:nside}{nside}, 
\mylink{sub:template_pixel_xxx:pixel_ring}{pixel\_ring}, 
\mylink{sub:template_pixel_xxx:template}{template}, 
\mylink{sub:template_pixel_xxx:reflexion}{reflexion}%
}
\end{f90format}
%\ mylink: to avoid automatic processing by make_internal_links.sh

\begin{arguments}
{
\begin{tabular}{p{0.3\hsize} p{0.05\hsize} p{0.1\hsize} p{0.45\hsize}} \hline  
\textbf{name~\&~dimensionality} & \textbf{kind} & \textbf{in/out} & \textbf{description} \\ \hline
                   &   &   &                           \\ %%% for presentation
nside\mytarget{sub:template_pixel_xxx:nside} & I4B & IN & the \healpix $\nside$ parameter. \\
pixel\_nest\mytarget{sub:template_pixel_xxx:pixel_nest} & I4B/ I8B & IN & NESTED scheme pixel identification number over the range \{0,$12\nside^2-1$\}.\\
pixel\_ring\mytarget{sub:template_pixel_xxx:pixel_ring} & I4B/ I8B & IN & RING scheme pixel identification number over the
                   range \{0,$12\nside^2-1$\}.\\
template\mytarget{sub:template_pixel_xxx:template} & I4B/ I8B & OUT & identification number of the
                   template matching in shape the pixel provided (the numbering
                   scheme of the pixel templates is the same for both routines). \\
reflexion\mytarget{sub:template_pixel_xxx:reflexion} & I4B & OUT & in \{0, 3\} encodes the transformation(s) to
                   apply to each pixel provided to match exactly in
                   shape and position its respective template. 0: rotation around the polar axis only,
                   1: rotation + East-West swap (ie, reflexion around meridian),
                   2: rotation + North-South swap (ie, reflexion around
                   Equator), 3: rotation + East-West and North-South swaps
\end{tabular}
}
\end{arguments}

\begin{example}
{
call template\_pixel\_ring(256, 500000, template, reflexion)  \\
}
{
Returns in {\tt template} the index of the template pixel (16663) whose shape matches
that of the pixel \#500000 for $\nside=256$. Upon return {\tt reflexion} will
contain 2, meaning that the template must be reflected around a meridian and
around the equator (and then rotated around the polar axis) in order to match
the pixel.
}
\end{example}
\begin{related}
  \begin{sulist}{} %%%% NOTE the ``extra'' brace here %%%%
  \item[\htmlref{nside2templates}{sub:nside2ntemplates}] returns the
  number of template pixel shapes available for a given $\nside$.
  \item[\htmlref{same\_shape\_pixels\_ring}{sub:same_shape_pixels_xxx}] 
  \item[\htmlref{same\_shape\_pixels\_nest}{sub:same_shape_pixels_xxx}] 
  return
  the ordered list of pixels having the same shape as a given pixel template
  \end{sulist}
\end{related}

\rule{\hsize}{2mm}

% special format, TBD

\sloppy


%%%\title{\healpix Fortran Subroutines Overview}
\docid{udgrade\_nest*} \section[udgrade\_nest*]{ }
\label{sub:udgrade_nest}
\docrv{Version 2.0}
\author{Eric Hivon}
\abstract{This document describes the \healpix Fortran90 subroutine UDGRADE\_NEST.}


\begin{facility}
{Routine to degrade or prograde the pixel size of a \healpix map indexed with
  the NESTED scheme. The degradation/progradation is done assuming an
intensive quantity (like temperature) that does NOT scale with surface area. \newline
In case of degradation, a big pixel that contains one or several bad pixels will
take the average of the valid small pixels, unless a 'pessimistic' behavior
is assumed in which case the big pixel will take the bad pixel sentinel value.
In case of progradation, a bad pixel only spawns bad pixels. \newline
The routine accepts both mono and bi-dimensional maps.
}
{\modUdgradeNr}
\end{facility}

\begin{f90format}
{\mylink{sub:udgrade_nest:map_in}{map\_in}%
, \mylink{sub:udgrade_nest:nside_in}{nside\_in}%
, \mylink{sub:udgrade_nest:map_out}{map\_out}%
, \mylink{sub:udgrade_nest:nside_out}{nside\_out}%
 \optional{[, \mylink{sub:udgrade_nest:fmissval}{fmissval}%
, \mylink{sub:udgrade_nest:pessimistic}{pessimistic}%
]}}
\end{f90format}
\aboutoptional

\begin{arguments}
{
\begin{tabular}{p{0.3\hsize} p{0.05\hsize} p{0.1\hsize} p{0.45\hsize}} \hline  
\textbf{name~\&~dimensionality} & \textbf{kind} & \textbf{in/out} & \textbf{description} \\ \hline
                   &   &   &                           \\ %%% for presentation
map\_in\mytarget{sub:udgrade_nest:map_in}(0:12*nside\_in**2-1) & SP/ DP & IN & mono-dimensional full sky map to be
                   prograded or degraded. \\
map\_in                                  (0:12*nside\_in**2-1,1:nd) & SP/ DP & IN & bi-dimensional full sky map to be
                   prograded or degraded. The routine finds the second
                   dimension (nd) by itself. \\
nside\_in\mytarget{sub:udgrade_nest:nside_in} & I4B & IN & the $\nside$ resolution parameter of the input
                   map. Must be a power of 2.\\
map\_out\mytarget{sub:udgrade_nest:map_out}(0:12*nside\_out**2-1) & SP/ DP & OUT & mono-dimensional full sky map after
                   degradation or progradation. \\
map\_out                                   (0:12*nside\_out**2-1,1:nd) & SP/ DP & OUT & bi-dimensional full sky map after
                   degradation or progradation. The second dimension
                   (nd) should match that of the input map.\\
nside\_out\mytarget{sub:udgrade_nest:nside_out} & I4B & IN & the $\nside$ resolution parameter of the output
                   map. Must be a power of 2. If nside\_out $>$ nside\_in, the
                   map is prograded (ie, more and smaller pixels) with each
                   pixel having the same value as its parent; otherwise, the
                   map in degraded (ie, fewer larger pixels), with each pixel
                   being the average of its $($nside\_in/nside\_out$)^2$ components.\\
\optional{fmissval\mytarget{sub:udgrade_nest:fmissval}}  & SP/ DP & IN & sentinel value given to bad pixels in input and output
                   maps.%
\default{\htmlref{\tt HPX\_SBADVAL}{sub:healpix_types} or %
\htmlref{\tt HPX\_DBADVAL}{sub:healpix_types}%
} \\
\optional{pessimistic\mytarget{sub:udgrade_nest:pessimistic}}   & LGT & IN & if set to {\tt .true.}, during a degradation, a big pixel containing at least a small
                   bad pixel will be returned as bad as well, instead of taking
                   the average of the remaing valid pixels. \default{\tt .false.}
\end{tabular}
}
\end{arguments}

\begin{example}
{
use udgrade\_nr \\
call udgrade\_nest(map\_hi, 256, map\_low, 64)  \\
}
{
Degrades a NESTED ordered map with $\nside=256$ into a NESTED map with $\nside=64$
}
\end{example}

%% \begin{modules}
%%   \begin{sulist}{} %%%% NOTE the ``extra'' brace here %%%%
%%  \item[\htmlref{ring2nest}{sub:pix_tools}] routine to udgrade a RING pixel index to NESTED pixel number.
%%   \end{sulist}
%% \end{modules}
%%%%\newpage
\begin{related}
  \begin{sulist}{} %%%% NOTE the ``extra'' brace here %%%%
  \item[\htmlref{udgrade\_ring}{sub:udgrade_ring}] prograde or degrade a RING
  ordered map.
%%   \item[\htmlref{udgrade\_inplace}{sub:udgrade_inplace}] udgrade between 
%%     RING and NESTED schemes inplace. This routine is slower than \thedocid, but doesn't require as much memory.
  \end{sulist}
\end{related}

\rule{\hsize}{2mm}

\newpage


\sloppy


\title{\healpix Fortran Subroutines Overview}
\docid{udgrade\_ring*} \section[udgrade\_ring*]{ }
\label{sub:udgrade_ring}
\docrv{Version 2.0}
\author{Eric Hivon}
\abstract{This document describes the \healpix Fortran90 subroutine UDGRADE\_RING.}


\begin{facility}
{Routine to degrade or prograde the pixel size of a \healpix map indexed with
  the RING scheme. The degradation/progradation is done assuming an
intensive quantity (like temperature) that does NOT scale with surface area. \newline
In case of degradation, a big pixel that contains one or several bad pixels will
take the average of the valid small pixels, unless a 'pessimistic' behavior
is assumed in which case the big pixel will take the bad pixel sentinel value.
In case of progradation, a bad pixel only spawns bad pixels.\newline
The routine accepts both mono and bi-dimensional maps.
}
{\modUdgradeNr}
\end{facility}

\begin{f90format}
{\mylink{sub:udgrade_ring:map_in}{map\_in}%
, \mylink{sub:udgrade_ring:nside_in}{nside\_in}%
, \mylink{sub:udgrade_ring:map_out}{map\_out}%
, \mylink{sub:udgrade_ring:nside_out}{nside\_out}%
 \optional{[, \mylink{sub:udgrade_ring:fmissval}{fmissval}%
, \mylink{sub:udgrade_ring:pessimistic}{pessimistic}%
]}}
\end{f90format}
\aboutoptional

\begin{arguments}
{
\begin{tabular}{p{0.3\hsize} p{0.05\hsize} p{0.1\hsize} p{0.45\hsize}} \hline  
\textbf{name~\&~dimensionality} & \textbf{kind} & \textbf{in/out} & \textbf{description} \\ \hline
                   &   &   &                           \\ %%% for presentation
map\_in\mytarget{sub:udgrade_ring:map_in}(0:12*nside\_in**2-1) & SP/ DP & INOUT & mono-dimensional full sky map to be
                   prograded or degraded. The routine finds the second
                   dimension (nd) by itself. \textbf{Note that the map is modified on
                   output (reordered into NESTED scheme).}\\
map\_in\mytarget{sub:udgrade_ring:map_in}(0:12*nside\_in**2-1,1:nd) & SP/ DP & INOUT & bi-dimensional full sky map to be
                   prograded or degraded. \textbf{Note that the map is modified on
                   output (reordered into NESTED scheme).}\\
nside\_in\mytarget{sub:udgrade_ring:nside_in} & I4B & IN & the $\nside$ resolution parameter of the input
                   map. Must be a power of 2.\\
map\_out\mytarget{sub:udgrade_ring:map_out}(0:12*nside\_out**2-1) & SP/ DP & OUT & mono-dimensional full sky map after
                   degradation or progradation. \\
map\_out\mytarget{sub:udgrade_ring:map_out}(0:12*nside\_out**2-1,1:nd) & SP/ DP & OUT & bi-dimensional full sky map after
                   degradation or progradation. The second dimension
                   (nd) should match that of the input map.\\
nside\_out\mytarget{sub:udgrade_ring:nside_out} & I4B & IN & the $\nside$ resolution parameter of the output
                   map. Must be a power of 2. If nside\_out $>$ nside\_in, the
                   map is prograded (ie, more and smaller pixels) with each
                   pixel having the same value as its parent; otherwise, the
                   map in degraded (ie, fewer larger pixels), with each pixel
                   being the average of its $($nside\_in/nside\_out$)^2$ components.\\
\optional{fmissval\mytarget{sub:udgrade_ring:fmissval}}  & SP/ DP & IN & sentinel value given to bad pixels in input and output
                   maps.%
\default{\htmlref{\tt HPX\_SBADVAL}{sub:healpix_types} or %
\htmlref{\tt HPX\_DBADVAL}{sub:healpix_types}%
} \\
\optional{pessimistic\mytarget{sub:udgrade_ring:pessimistic}}  & LGT & IN & if set to {\tt .true.}, during a degradation, a big pixel containing at least a small
                   bad pixel will be returned as bad as well, instead of taking
                   the average of the remaing valid pixels. \default{\tt .false.}
\end{tabular}
}
\end{arguments}

\begin{example}
{
use udgrade\_nr \\
call udgrade\_ring(map\_hi, 256, map\_low, 64)  \\
}
{
Degrades a RING ordered map with $\nside=256$ into a RING map with $\nside=64$
}
\end{example}

%% \begin{modules}
%%   \begin{sulist}{} %%%% NOTE the ``extra'' brace here %%%%
%%  \item[\htmlref{ring2nest}{sub:pix_tools}] routine to udgrade a RING pixel index to NESTED pixel number.
%%   \end{sulist}
%% \end{modules}
%%%%\newpage
\begin{related}
  \begin{sulist}{} %%%% NOTE the ``extra'' brace here %%%%
  \item[\htmlref{udgrade\_nest}{sub:udgrade_nest}] prograde or degrade a NESTED
  ordered map.
%%   \item[\htmlref{udgrade\_inplace}{sub:udgrade_inplace}] udgrade between 
%%     RING and NESTED schemes inplace. This routine is slower than \thedocid, but doesn't require as much memory.
  \end{sulist}
\end{related}

\rule{\hsize}{2mm}

\newpage


\sloppy


\title{\healpix Fortran Subroutines Overview}
\docid{uniq2nest} \section[uniq2nest]{ }
\label{sub:uniq2nest}
\docrv{Version 1.0}
\author{E. Hivon}
\abstract{This document describes the \healpix Fortran90 subroutines
  UNIQ2NEST.}

\begin{facility}
{This F90 facility turns the Unique Identifier $u = p + 4 \nside^2$, into the parameter $\nside$ (a power of 2) and the pixel index $p$. See \htmlref{''The Unique Identifier scheme''}{intro:unique} in 
\linklatexhtml{''\healpix Introduction Document''}{intro.pdf}{intro.htm} for more details.
}
{\modPixTools}
\end{facility}

\begin{f90format}
{%
\mylink{sub:uniq2nest:puniq}{puniq}, 
\mylink{sub:uniq2nest:nside}{nside}, 
\mylink{sub:uniq2nest:pnest}{pnest}}
\end{f90format}

\begin{arguments}
{
\begin{tabular}{p{0.10\hsize} p{0.1\hsize} p{0.1\hsize} p{0.60\hsize}} \hline  
\textbf{name} & \textbf{kind} & \textbf{in/out} & \textbf{description} \\ \hline
                   &   &   &                           \\ %%% for presentation
puniq \mytarget{sub:uniq2nest:puniq} & I4B/I8B & IN & The \healpix Unique pixel identifier. Must be $\ge 4$. \\
nside \mytarget{sub:uniq2nest:nside} & I4B      & OUT & The \healpix $\nside$ parameter. \\
pnest \mytarget{sub:uniq2nest:pnest} & I4B/I8B & OUT & (NESTED scheme) pixel identification number over the range \{0,$12\nside^2-1$\}.
\end{tabular}
}
\end{arguments}

\begin{example}
{use \htmlref{healpix\_modules}{sub:healpix_modules}\\
integer(I4B) :: nside, pnest \\
call uniq2nest(4, nside, pnest)\\
print*,nside,pnest
}
{
\begin{minipage}{11cm}
returns  \\
     1 \hskip 1cm 0 \\
since the pixel with Unique ID number 4 is the first pixel ($p=0$) at $\nside=$ 1.
\end{minipage}
}
\end{example}

\begin{related}
  \begin{sulist}{} %%%% NOTE the ``extra'' brace here %%%%
  \item[\htmlref{nest2uniq}{sub:nest2uniq}] Transforms Nside and Nested pixel number into Unique \healpix pixel ID number
  \item[\htmlref{pix2xxx, ...}{sub:pix_tools}] to turn NESTED pixel index into sky coordinates and back
  \end{sulist}
\end{related}

\rule{\hsize}{2mm}




\sloppy


\title{\healpix Fortran Subroutines Overview}
\docid{vec2ang} \section[vec2ang]{ }
\label{sub:vec2ang}
\docrv{Version 1.0}
\author{E. Hivon}
\abstract{This document describes the \healpix Fortran90 subroutine VEC2ANG.}

\begin{facility}
{Routine to convert the 3D position vector $(x,y,z)$ of point into its position
  angles  $(\theta,\phi)\myhtmlimage{}$ on the sphere with
$x = \sin\theta\cos\phi\myhtmlimage{}$, $y=\sin\theta\sin\phi\myhtmlimage{}$, $z=\cos\theta\myhtmlimage{}$.
}
{\modPixTools}
\end{facility}

\begin{f90format}
{\mylink{sub:vec2ang:vector}{vector}%
, \mylink{sub:vec2ang:theta}{theta}%
, \mylink{sub:vec2ang:phi}{phi}%
}
\end{f90format}


\begin{arguments}
{
\begin{tabular}{p{0.3\hsize} p{0.05\hsize} p{0.1\hsize} p{0.45\hsize}} \hline  
\textbf{name\&dimensionality} & \textbf{kind} & \textbf{in/out} & \textbf{description} \\ \hline
                   &   &   &                           \\ %%% for presentation
vector\mytarget{sub:vec2ang:vector}(3) & DP & IN & three dimensional cartesian position vector
                   $(x,y,z)$. The north pole is $(0,0,1)$\\
theta\mytarget{sub:vec2ang:theta} & DP & OUT & colatitude in radians measured southward from north pole (in
    [0,$\pi$]). \\
phi\mytarget{sub:vec2ang:phi}   & DP & OUT & longitude in radians measured eastward (in [0, $2\pi$]).\\
\end{tabular}
}
\end{arguments}

% \begin{example}
% {
% call vec2ang(vector,theta,phi) \\
% }
% {
% }
% \end{example}
\begin{related}
  \begin{sulist}{} %%%% NOTE the ``extra'' brace here %%%%
  \item[\htmlref{ang2vec}{sub:ang2vec}] converts the position angles of a point on the sphere 
into its 3D position vector.
  \item[\htmlref{angdist}{sub:angdist}] computes the angular distance between 2 vectors
  %\item[\htmlref{vec2ang}{sub:vec2ang}] converts the 3D position vector of
  %point into its position angles on the sphere.
  \item[\htmlref{vect\_prod}{sub:vect_prod}] computes the vector product between two 3D vectors
  \end{sulist}
\end{related}

\rule{\hsize}{2mm}



\sloppy


%%%\title{\healpix Fortran Subroutines Overview}
\docid{vect\_prod} \section[vect\_prod]{ }
\label{sub:vect_prod}
\docrv{Version 1.2}
\author{Eric Hivon}
\abstract{This document describes the \healpix Fortran90 subroutine VECT\_PROD.}

\begin{facility}
{Returns the vectorial product of two vectors.} 
{\modPixTools}
\end{facility}

\begin{f90format}
{\mylink{sub:vect_prod:v1}{v1}%
, \mylink{sub:vect_prod:v2}{v2}%
, \mylink{sub:vect_prod:v3}{v3}%
}
\end{f90format}

\begin{arguments}
{
\begin{tabular}{p{0.3\hsize} p{0.05\hsize} p{0.1\hsize} p{0.45\hsize}} \hline 
\textbf{name~\&~dimensionality} & \textbf{kind} & \textbf{in/out} & \textbf{description} \\ \hline
                   &   &   &                           \\ %%% for presentation
v1\mytarget{sub:vect_prod:v1}(3) & DP & IN & cartesian vector ${\bf v}_1$. \\
v2\mytarget{sub:vect_prod:v2}(3) & DP & IN & cartesian vector ${\bf v}_2$. \\
v3\mytarget{sub:vect_prod:v3}(3) & DP & OUT & cartesian vector ${\bf v}_3 = {\bf v}_1 \times {\bf v}_2$ \\
\end{tabular}
}
\end{arguments}

\begin{example}
{
use healpix\_types \\
use pix\_tools,    only : vect\_prod \\
real(DP), dimension(3) :: vec\\
real(DP) :: one = 1.0\_dp \\
call vect\_prod((/2,0,0/)*one, (/0,1,0/)*one, vec)  \\
print*, vec
}
{
will return : 0.00E+000  0.00E+000   2.00
}
\end{example}
% \newpage
% \begin{modules}
%   \begin{sulist}{} %%%% NOTE the ``extra'' brace here %%%%
%  \item[\htmlref{in\_ring}{sub:in_ring}] routine to find the pixels in a certain slice of a given ring.		
%  \item[\htmlref{ring\_num}{sub:ring_num}] function to return the ring number corresponding to the coordinate $z$
%   \end{sulist}
% \end{modules}

\begin{related}
  \begin{sulist}{} %%%% NOTE the ``extra'' brace here %%%%
  \item[\htmlref{ang2vec}{sub:ang2vec}] converts the position angles of a point on the sphere 
into its 3D position vector.
  \item[\htmlref{angdist}{sub:angdist}] computes the angular distance between 2 vectors
  \item[\htmlref{vec2ang}{sub:vec2ang}] converts the 3D position vector of point into its position
  angles on the sphere.
  %\item[\htmlref{vect\_prod}{sub:vect_prod}] computes the vector product between two 3D vectors
  \end{sulist}
\end{related}

\rule{\hsize}{2mm}

\newpage


\sloppy


%%%\title{\healpix Fortran Subroutines Overview}
\docid{write\_asctab*} \section[write\_asctab*]{ }
\label{sub:write_asctab}
\docrv{Version 2.0}
\author{Eric Hivon, Frode K.~Hansen}
\abstract{This document describes the \healpix Fortran90 subroutine WRITE\_ASCTAB.}

\begin{facility}
{This routine stores a power spectrum in an ascii FITS-file. The routine can store temperature coeffecients $C_\ell^T$ or both temperature and polarisation coeffecients $C_\ell^T$, $C_\ell^E$, $C_\ell^B$, $C_\ell^{T\times E}$.}
{\modFitstools}
\end{facility}

\begin{f90format}
{\mylink{sub:write_asctab:clout}{clout}%
, \mylink{sub:write_asctab:lmax}{lmax}%
, \mylink{sub:write_asctab:ncl}{ncl}%
, \mylink{sub:write_asctab:header}{header}%
, \mylink{sub:write_asctab:nlheader}{nlheader}%
, \mylink{sub:write_asctab:filename}{filename}%
 \optional{[, \mylink{sub:write_asctab:extno}{extno}%
]}}
\end{f90format}
\aboutoptional

\begin{arguments}
{
\begin{tabular}{p{0.4\hsize} p{0.05\hsize} p{0.1\hsize} p{0.35\hsize}} \hline  
\textbf{name~\&~dimensionality} & \textbf{kind} & \textbf{in/out} & \textbf{description} \\ \hline
                   &   &   &                           \\ %%% for presentation
filename\mytarget{sub:write_asctab:filename}(LEN=\filenamelen) & CHR & IN & the FITS file to which the power spectrum is written. \\
lmax\mytarget{sub:write_asctab:lmax} & I4B & IN & Maximum $\ell$ value to be written. \\
ncl\mytarget{sub:write_asctab:ncl} & I4B & IN & 1 for temperature coeffecients only, 4 for polarisation. \\
clout\mytarget{sub:write_asctab:clout}(0:lmax,1:ncl) & SP/ DP & IN & the powerspectrum to be saved in the file.\\
nlheader\mytarget{sub:write_asctab:nlheader} & I4B & IN & number of header lines to write to the file. \\
header\mytarget{sub:write_asctab:header}(LEN=80) (1:nlheader) & CHR & IN & the header to the FITS-file. \\ 
\optional{extno\mytarget{sub:write_asctab:extno}}	& I4B & IN & extension number in which to write the data (0
                   based). \default {0}
\end{tabular}
}
\end{arguments}

\begin{example}
{
use \htmlref{healpix\_modules}{sub:healpix_modules}\\
%implicit none \\
real(\mylink{sub:healpix_types:sp}{SP}), allocatable, dimension(:,:) :: cl \\
character(len=80), dimension(1:100) :: header \\
allocate(cl(0:64,1:1))\\
call \htmlref{write\_minimal\_header}{sub:write_minimal_header}(header,'cl',nlmax=64)\\
call write\_asctab (cl,64,1,header,100,'cl.fits')  \\
}
{
Writes a power spectrum in the array cl(0:64,1:1) to a FITS-file called `cl.fits'. The cl array contains the temperature power spectrum $C_\ell^T$ up to an $\ell$ value of 64. 100 header lines are written to the file from the array header(1:100) which was previously filled the minimal required information for a power spectrum file.
}
\end{example}

% \newpage
\begin{modules}
  \begin{sulist}{} %%%% NOTE the ``extra'' brace here %%%%
  \item[\textbf{fitstools}] module, containing:
  \item[printerror] routine for printing FITS error messages.
  \item[\textbf{cfitsio}] library for FITS file handling.		
  \end{sulist}
\end{modules}

\begin{related}
  \begin{sulist}{} %%%% NOTE the ``extra'' brace here %%%%
  \item[\htmlref{alm2cl}{sub:alm2cl}] Routine computing the power spectrum from
  spherical harmonics coefficients $a_{\ell m}$
  \item[\htmlref{fits2cl}{sub:fits2cl}] Routine to read a FITS file created by write\_asctab.
  \item[\htmlref{write\_minimal\_header}{sub:write_minimal_header}] routine to write minimal FITS header
  \end{sulist}
\end{related}

\rule{\hsize}{2mm}

\newpage


\sloppy


\title{\healpix Fortran Subroutines Overview}
\docid{write\_bintab*} \section[write\_bintab*]{ }
\label{sub:write_bintab}
\docrv{Version 1.2}
\author{Frode K.~Hansen, Eric Hivon}
\abstract{This document describes the \healpix Fortran90 subroutine WRITE\_BINTAB.}

\begin{facility}
{This routine creates a binary FITS-file from a \healpix map. The routine can save a temperature map or both temperature and polarisation maps (T,Q,U) to the file.}
{\modFitstools}
\end{facility}

\begin{f90format}
{\mylink{sub:write_bintab:map}{map}%
, \mylink{sub:write_bintab:npix}{npix}%
, \mylink{sub:write_bintab:nmap}{nmap}%
, \mylink{sub:write_bintab:header}{header}%
, \mylink{sub:write_bintab:nlheader}{nlheader}%
, \mylink{sub:write_bintab:filename}{filename}%
 \optional{[, \mylink{sub:write_bintab:extno}{extno}%
]}}
\end{f90format}
\aboutoptional

\begin{arguments}
{
\begin{tabular}{p{0.30\hsize} p{0.05\hsize} p{0.08\hsize} p{0.49\hsize}} \hline  
\textbf{name~\&~dimensionality} & \textbf{kind} & \textbf{in/out} & \textbf{description} \\ \hline
                   &   &   &                           \\ %%% for presentation
map\mytarget{sub:write_bintab:map}(0:npix-1,1:nmap) & SP/ DP & IN & the map to write to the FITS-file.\\
npix\mytarget{sub:write_bintab:npix} & I4B/ I8B & IN & Number of pixels in the map.\\
nmap\mytarget{sub:write_bintab:nmap} & I4B & IN & number of maps to be written, 1 for temperature only, and 3 for (T,Q,U). \\
header\mytarget{sub:write_bintab:header}(LEN=80) (1:nlheader) & CHR & IN & The header for the FITS-file. \\
nlheader\mytarget{sub:write_bintab:nlheader} & I4B & IN & number of header lines to write to the file. \\
filename\mytarget{sub:write_bintab:filename}(LEN=*) & CHR & IN & the map(s) is (are) written to a FITS-file with this filename. \\
\optional{extno\mytarget{sub:write_bintab:extno}}	& I4B & IN & extension number in which to write the data (0
                   based). \default {0}
\end{tabular}
}
\end{arguments}

\begin{example}
{
call write\_bintab (map,12*32**2,3,header,120,'map.fits')  \\
}
{
Makes a binary FITS-file called `map.fits' from the \healpix maps (T,Q,U) in the array map(0:12*32**2-1,1:3). The number of pixels 12*32**2 corresponds to the number of pixels in a $N_{side}=32$ \healpix map. The header for the FITS-file is given in the string array header and the number of lines in the header is 120. 
}
\end{example}
\begin{modules}
  \begin{sulist}{} %%%% NOTE the ``extra'' brace here %%%%
  \item[\textbf{fitstools}] module, containing:
  \item[printerror] routine for printing FITS error messages.
  \item[\textbf{cfitsio}] library for FITS file handling.		
  \end{sulist}
\end{modules}

\begin{related}
  \begin{sulist}{} %%%% NOTE the ``extra'' brace here %%%%
  \item[\htmlref{input\_map}{sub:input_map}, \htmlref{read\_bintab}{sub:read_bintab}] routines which read a file created by \thedocid. 
  \item[\htmlref{map2alm}{sub:map2alm}] subroutine which analyse a map and returns the $a_{lm}$ coeffecients.
  \item[\htmlref{output\_map}{sub:output_map}] subroutine which calls \thedocid
  \item[\htmlref{write\_bintabh}{sub:write_bintabh}] subroutine to write a large
array into a FITS file piece by piece
  \item[\htmlref{input\_tod*}{sub:input_tod}] subroutine to read an arbitrary subsection of
  a large binary table
  \item[\htmlref{write\_minimal\_header}{sub:write_minimal_header}] routine to write minimal FITS header\end{sulist}
\end{related}

\rule{\hsize}{2mm}

\newpage


\sloppy


\title{\healpix Fortran Subroutines Overview}
\docid{write\_bintabh} \section[write\_bintabh*]{ }
\label{sub:write_bintabh}
\docrv{Version 1.2}
\author{Eric Hivon, Frode K.~Hansen}
\abstract{This document describes the \healpix Fortran90 subroutine WRITE\_BINTABH.}

\begin{facility}
{This routine is designed to write large (or huge) arrays into a binary table
extension of a FITS file. The user can
choose to write the array piece by piece. This is designed to deal with Time
Ordered Data set (tod).}
{\modFitstools}
\end{facility}

\begin{f90format}
{tod, npix, ntod, header, nlheader, filename, \optional{[extno, firstpix, repeat]}}
\end{f90format}
\aboutoptional

\begin{arguments}
{
\begin{tabular}{p{0.35\hsize} p{0.05\hsize} p{0.08\hsize} p{0.45\hsize}} \hline  
\textbf{name~\&~dimensionality} & \textbf{kind} & \textbf{in/out} & \textbf{description} \\ \hline
                   &   &   &                           \\ %%% for presentation
tod(0:npix-1,1:ntod) & SP/ DP & IN & The map or tod
  to write to the FITS file. It will be written in the file at the location
                   corresponding to pixels (or time samples)
                   {\tt firstpix} to {\tt firtpix + npix} -1.\\
npix & I8B & IN & Number of pixels or time samples in the map or TOD. See Note below.\\
ntod & I4B & IN & Number of maps or tods to be written. Each of them will be in a different column of the FITS binary table.\\
header(LEN=80) (1:nlheader) & CHR & IN & The header for the FITS file. \\
nlheader & I4B & IN & number of header lines to write to the file. \\
filename(LEN=\filenamelen) & CHR & IN & The array is written into a FITS file with this filename. \\
\optional{extno} & I4B & IN & extension number in which to write the data (0
                   based).  \default 0 \\
\optional{firstpix} & I8B & IN & 0 Location in the FITS file of the first
                   pixel (or time sample) to be written (0 based). \default 
                   0. See Note below.
                   \\
\optional{repeat} & I4B & IN & \parbox[t]{0.99\hsize}{ 
		Length of the element vector used in the binary
                   table. \default{1024 if {\tt npix} $\propto 1024$; 12000 if
                   {\tt npix} $> 12000$ and 1 otherwise}. \\
 		Choosing a large {\tt
                   repeat} for multi-column tables ({\tt ntod} $>1$) generally
                   speeds up the I/O. It also helps bringing the number of rows
                   of the table under $2^{31}$, which is a hard limit of
                   cfitsio. \\
		   If the number of samples or pixels of each map or TOD is not a multiple of 
		{\tt repeat}, then the last element vector will be padded with sentinel values 
\mylink{sub:healpix_types:hpx_sbadval}{\tt HPX\_SBADVAL} or
\mylink{sub:healpix_types:hpx_dbadval}{\tt HPX\_DBADVAL}.}
\end{tabular}
{\bf Note :} Indices and number of data elements larger than
                   $2^{31}$ are only accessible in FITS files on computers with 64 bit
                   enabled compilers and with some specific compilation options of
                   cfitsio (see cfitsio documentation).
}
\end{arguments}

\begin{example}
{
use \htmlref{healpix\_types}{sub:healpix_types} \\
use fitstools, only : \thedocid \\
character(len=80), dimension(1:128) :: hdr \\
real(SP), dimension(0:59,1) :: tod \\
character(len=\mylink{sub:healpix_types:filenamelen}{FILENAMELEN}) :: fname='tod.fits' \\
hdr(:) = ' ' \\
tod(:,1) = 1. \\
call \thedocid (tod, 50\_i8b, 1, hdr, 128, fname, firstpix=0\_i8b, repeat=10)  \\
tod = tod * 3. \\
call \thedocid (tod, 20\_i8b, 1, hdr, 128, fname, firstpix=40\_i8b)  
}
{
Writes into the FITS file `tod.fits' a 1 column binary table, where the first 40
data samples have the value $1.$ and the next 20 have the value $3.$ (Note that
in this example the
second call to \thedocid \ overwrites some of the pixels written by the first call). The samples will be
written in element vectors of length 10. The header for the FITS file is given in the
string array {\tt hdr} and its number of lines is 128. 
}
\end{example}

\begin{modules}
  \begin{sulist}{} %%%% NOTE the ``extra'' brace here %%%%
  \item[\textbf{fitstools}] module, containing:
  \item[printerror] routine for printing FITS error messages.
  \item[\textbf{cfitsio}] library for FITS file handling.		
  \end{sulist}
\end{modules}

\begin{related}
  \begin{sulist}{} %%%% NOTE the ``extra'' brace here %%%%
  \item[\htmlref{input\_tod*}{sub:input_tod}] routine that reads a file created by \thedocid. 
  \item[\htmlref{input\_map}{sub:input_map},
  \htmlref{read\_bintab}{sub:read_bintab}] routines to read \healpix sky map,
  \item[\htmlref{write\_minimal\_header}{sub:write_minimal_header}] routine to write minimal FITS header
  \end{sulist}
\end{related}

\rule{\hsize}{2mm}

\newpage


\sloppy


%%%\title{\healpix Fortran Subroutines Overview}
\docid{write\_dbintab} \section[write\_dbintab]{ }
\label{sub:write_dbintab}
\docrv{Version 1.1}
\author{Frode K.~Hansen}
\abstract{This document describes the \healpix Fortran90 subroutine WRITE\_DBINTAB.}

\begin{facility}
{This routine is obsolete. \\
To write $P_{lm}$ polynoms into a FITS file,
use 
\htmlref{write\_plm}{sub:write_plm} 
instead. \\
To write a Healpix map into a FITS file,
use 
\htmlref{write\_bintab}{sub:write_bintab} 
or 
\htmlref{output\_map}{sub:output_map}.}
{\modFitstools}
\end{facility}


\rule{\hsize}{2mm}

\newpage


\sloppy


\title{\healpix Fortran Subroutines Overview}
\docid{write\_fits\_cut4} \section[write\_fits\_cut4]{ }
\label{sub:write_fits_cut4}
\docrv{Version 1.3}
\author{Eric Hivon \& Frode K.~Hansen}
\abstract{This document describes the \healpix Fortran90 subroutine WRITE\_FITS\_CUT4.}

\begin{facility}
{This routine writes a cut sky \healpix map into a FITS file. The format used for the
FITS file follows the one used for Boomerang98 and is adapted from COBE/DMR. 
This routine can be used to store polarized maps, where the
information relative to the Stokes parameters I, Q and U are placed in extension
0, 1 and 2 respectively by successive invocation of the routine.}
{\modFitstools}
\end{facility}

\begin{f90format}
{\mylink{sub:write_fits_cut4:filename}{filename}%
, \mylink{sub:write_fits_cut4:np}{np}%
, \mylink{sub:write_fits_cut4:pixel}{pixel}%
, \mylink{sub:write_fits_cut4:signal}{signal}%
, \mylink{sub:write_fits_cut4:n_obs}{n\_obs}%
, \mylink{sub:write_fits_cut4:serror}{serror}%
 \optional{[, \mylink{sub:write_fits_cut4:header}{header}%
, \mylink{sub:write_fits_cut4:coord}{coord}%
, \mylink{sub:write_fits_cut4:nside}{nside}%
, \mylink{sub:write_fits_cut4:order}{order}%
,
\mylink{sub:write_fits_cut4:units}{units}%
, \mylink{sub:write_fits_cut4:extno}{extno}%
, \mylink{sub:write_fits_cut4:polarisation}{polarisation}%
]}}
\end{f90format}
\aboutoptional

\begin{arguments}
{
\begin{tabular}{p{0.3\hsize} p{0.05\hsize} p{0.05\hsize} p{0.5\hsize}} \hline  
\textbf{name\&dimensionality} & \textbf{kind} & \textbf{in/out} & \textbf{description} \\ \hline
                   &   &   &                           \\ %%% for presentation
filename\mytarget{sub:write_fits_cut4:filename}(LEN=\filenamelen) & CHR & IN & FITS file into which the cut sky map will be written\\
np\mytarget{sub:write_fits_cut4:np}           & I4B & IN & number of pixels to be written in the file \\
pixel\mytarget{sub:write_fits_cut4:pixel}(0:np-1)    & I4B & IN & index of observed (or valid) pixels \\
signal\mytarget{sub:write_fits_cut4:signal}(0:np-1)    & SP & IN & value of signal in each observed pixel\\
n\_obs\mytarget{sub:write_fits_cut4:n_obs}(0:np-1)   & I4B & IN & number of observation per pixel \\
serror\mytarget{sub:write_fits_cut4:serror}(0:np-1)   & SP  & IN & {\em rms} of signal in pixel, for white noise,
                   this is $\propto 1/\sqrt{{\rm n\_obs}}$. \\
%%%%%If Serror is present N\_Obs should also be present \\
\optional{header\mytarget{sub:write_fits_cut4:header}}(LEN=80)(1:) \ (OPTIONAL)    & CHR & IN &   FITS extension header to be included in the FITS file\\
\optional{coord\mytarget{sub:write_fits_cut4:coord}}(LEN=1)       & CHR & IN &   astrophysical coordinates ('C' or 'Q'
                   Celestial/eQuatorial, 'G' for Galactic, 'E' for Ecliptic)\\
\optional{nside\mytarget{sub:write_fits_cut4:nside}}    & I4B & IN &   \healpix resolution parameter of data set \\
\optional{order\mytarget{sub:write_fits_cut4:order}}     & I4B & IN &   \healpix ordering scheme, 1: RING, 2: NESTED \\
%\optional{header}(LEN=80)    & CHR & IN &   FITS header to be included in the FITS file\\
\optional{units\mytarget{sub:write_fits_cut4:units}}(LEN=20) & CHR & IN &  maps units (applies only to Signal and
                   Serror)\\
\optional{extno}\mytarget{sub:write_fits_cut4:extno}     & I4B & IN & (0 based) extension number in which to write data. \default{0}.
	  If set to 0 (or not set) {\em a new file is written from scratch}.
	  If set to a value
		  larger than 1, the corresponding extension is added or
		  updated, as long as all previous extensions already exist.
		  All extensions of the same file should use the same Nside,
Order and Coord \\
\optional{polarisaton}\mytarget{sub:write_fits_cut4:polarisation} & I4B & IN & if set to a non zero value, specifies that file will contain the I, Q and U polarisation
           Stokes parameter in extensions 0, 1 and 2 respectively, and sets the
FITS header keywords accordingly. If not set, the keywords found in \mylink{sub:write_fits_cut4:header}{\tt
header} will prevail.\\
\  & \ & \ & Note: the information relative to Nside, Order and Coord {\em has} to be
                   given, either thru these keyword or via the FITS Header. \\
\end{tabular}
}
\end{arguments}

% \begin{example}
% {
% npix= write\_fits\_cut4('map.fits', nmaps=nmaps, ordering=ordering,obs\_npix=obs\_npix, nside=nside, mlpol=mlpol, type=type, polarisation=polarisation)  \\
% }
% {
% Returns 1 or 3 in nmaps, dependent on wether 'map.fits' contain only
% temperature or both temperature and polarisation maps. The pixel ordering number is found by reading the keyword ORDERING in the FITS file. If this keyword does not exist, 0 is returned.
% }
% \end{example}
\newpage
\begin{modules}
  \begin{sulist}{} %%%% NOTE the ``extra'' brace here %%%%
  \item[\textbf{fitstools}] module, containing:
  \item[printerror] routine for printing FITS error messages.
  \item[\textbf{cfitsio}] library for FITS file handling.		
  \end{sulist}
\end{modules}

\begin{related}
  \begin{sulist}{} %%%% NOTE the ``extra'' brace here %%%%
  \item[anafast] executable that reads a \healpix map and analyses it. 
  \item[synfast] executable that generate full sky \healpix maps
  \item[\htmlref{getsize\_fits}{sub:getsize_fits}] routine to know the size of a FITS file and its type (eg, full sky vs cut sky)
  \item[\htmlref{input\_map}{sub:input_map}] all purpose routine to input a map of any kind from a FITS file
  \item[\htmlref{output\_map}{sub:output_map}] subroutine to write a FITS file from a \healpix map
  \item[\htmlref{read\_fits\_cut4}{sub:read_fits_cut4}] subroutine to read a \healpix cut sky map from a FITS file
  \item[\htmlref{write\_minimal\_header}{sub:write_minimal_header}] routine to write minimal FITS header
  \end{sulist}
\end{related}

\rule{\hsize}{2mm}

\newpage


\sloppy

%%%\title{\healpix Fortran Subroutines Overview}
\docid{write\_minimal\_header} \section[write\_minimal\_header]{ }
\label{sub:write_minimal_header}
\docrv{Version 1.2}
\author{Eric Hivon}
\abstract{This document describes the \healpix Fortran90 subroutine
WRITE\_MINIMAL\_HEADER.}

\begin{facility}
{This routine writes the baseline FITS header 
for the most common \healpix data sets: (cut sky or full sky) map, $C(\ell)$ power spectra and $a_{\ell m}$
coefficients.}
{\modHeadFits}
\end{facility}

\begin{f90format}
{\mylink{sub:write_minimal_header:header}{header}%
, \mylink{sub:write_minimal_header:dtype}{dtype}%
, \optional{[\mylink{sub:write_minimal_header:append}{append}%
, \mylink{sub:write_minimal_header:nside}{nside}%
, \mylink{sub:write_minimal_header:order}{order}%
, \mylink{sub:write_minimal_header:ordering}{ordering}%
, \mylink{sub:write_minimal_header:coordsys}{coordsys}%
, \mylink{sub:write_minimal_header:creator}{creator}%
, \mylink{sub:write_minimal_header:version}{version}%
, \mylink{sub:write_minimal_header:randseed}{randseed}%
, \mylink{sub:write_minimal_header:beam_leg}{beam\_leg}%
, \mylink{sub:write_minimal_header:fwhm_degree}{fwhm\_degree}%
, \mylink{sub:write_minimal_header:units}{units}%
, \mylink{sub:write_minimal_header:nlmax}{nlmax}%
, \mylink{sub:write_minimal_header:polar}{polar}%
, \mylink{sub:write_minimal_header:nmmax}{nmmax}%
, \mylink{sub:write_minimal_header:bcross}{bcross}%
, \mylink{sub:write_minimal_header:deriv}{deriv}%
, \mylink{sub:write_minimal_header:asym_cl}{asym\_cl}%
]} }
\end{f90format}
\aboutoptional

\begin{arguments}
{
\begin{tabular}{p{0.30\hsize} p{0.05\hsize} p{0.08\hsize} p{0.49\hsize}} \hline  
\textbf{name~\&~dimensionality} & \textbf{kind} & \textbf{in/out} & \textbf{description} \\ \hline
                   &   &   &                           \\ %%% for presentation
header\mytarget{sub:write_minimal_header:header}(LEN=80) \hskip 5cm DIMENSION(:) & CHR & INOUT & The FITS header to fill in. \\
%
dtype\mytarget{sub:write_minimal_header:dtype}(LEN=*)     & CHR & IN    & data to be put in the FITS file, must be
one of 'ALM', 'CL', 'MAP', 'CUTMAP' (case un-sensitive). \\
%
\end{tabular}
%---------
\begin{tabular}{p{0.30\hsize} p{0.05\hsize} p{0.08\hsize} p{0.49\hsize}} \hline  
\textbf{name~\&~dimensionality} & \textbf{kind} & \textbf{in/out} & \textbf{description} \\ \hline
                   &   &   &                           \\ %%% for presentation
\optional{append\mytarget{sub:write_minimal_header:append}} & LGT & IN &
if set to TRUE, the keywords will be appended to the content of {\tt header}
instead of written from scrath \\
%
\optional{nside\mytarget{sub:write_minimal_header:nside}} & I4B & IN & 
map resolution parameter;
required for dtype='MAP' and dtype='CUTMAP' \\
%
\optional{order\mytarget{sub:write_minimal_header:order}  } & I4B & IN & 
map ordering, either 1 (=ring) or 2
(=nested); see {\tt ordering}\\
%
\optional{ordering(LEN=*)\mytarget{sub:write_minimal_header:ordering}} & CHR & IN & 
map ordering, either 'RING' or 
'NESTED' (case un-sensitive); 
either {\tt order} or {\tt ordering} is required for dtype='MAP' and dtype='CUTMAP'\\
%
\optional{coordsys\mytarget{sub:write_minimal_header:coordsys}(LEN=*)} & CHR & IN & 
map coordinate system;
Valid choices are 'G' = Galactic, 'E' = Ecliptic,  'C'/'Q' = Celestial =
eQuatorial \\
%
\optional{creator\mytarget{sub:write_minimal_header:creator}(LEN=*)} & CHR & IN & 
name of software generating the
data set\\
%
\optional{version\mytarget{sub:write_minimal_header:version}(LEN=*)} & CHR & IN & 
version of {\tt creator} software\\
%
\optional{randseed\mytarget{sub:write_minimal_header:randseed}} & I4B & IN & 
random number generator seed used to generate the data\\
%
\optional{beam\_leg\mytarget{sub:write_minimal_header:beam_leg}(LEN=*)} & CHR & IN & 
File containing Legendre transform of symmetric beam\\
%
\optional{fwhm\_degree\mytarget{sub:write_minimal_header:fwhm_degree}} & DP & IN & 
FWHM in degrees of gaussian symmetric beam (FITS keyword: {\tt FWHM})\\
%
\optional{units\mytarget{sub:write_minimal_header:units}(LEN=*)} & CHR & IN & 
physical units of the data set (FITS keyword: {\tt TUNIT*}) \\
%
\optional{nlmax\mytarget{sub:write_minimal_header:nlmax}     } & I4B & IN & 
maximum multipole order $l$ of the data set (FITS keyword: {\tt MAX-LPOL})\\
%
\optional{polar\mytarget{sub:write_minimal_header:polar}     } & LGT & IN &
if set to {\tt .TRUE.}, the file to be written contains polarized data \\
%
\optional{nmmax\mytarget{sub:write_minimal_header:nmmax}     } & I4B & IN & 
maximum degree $m$ of data set (FITS keyword: {\tt MAX-MPOL}) \\
%
\optional{bcross\mytarget{sub:write_minimal_header:bcross}    } & LGT & IN &
if set to {\tt .TRUE.}, the magnetic cross terms power spectra (TB and EB) are
included;
only applies to dtype='CL' \\
%
\optional{deriv\mytarget{sub:write_minimal_header:deriv}     } & I4B & IN & 
order of derivatives to included in FITS file (0, 1 or 2); 
only applies to dtype='MAP' \\
%
\optional{asym\_cl\mytarget{sub:write_minimal_header:asym_cl}    } & LGT & IN &
if set to {\tt .TRUE.}, the asymmetric power spectra (ET, BT and BE on top of TE, TB and EB)
are included;
only applies to dtype='CL' 
\end{tabular}
}
\end{arguments}

\begin{example}
{
use healpix\_types \\
use head\_fits \\
character(len=80), dimension(1:60) :: header \\
call \thedocid(header, 'MAP', nside=256, ordering='Nested')  \\
call add\_card(header, 'HISTORY', 'Dummy map')\\
}
{
Writes in {\tt header} a \healpix compliant FITS header for a $\nside=256$ map with NESTED
ordering. Further HISTORY information is added with \htmlref{add\_card}{sub:add_card}
}
\end{example}

\begin{modules}
  \begin{sulist}{} %%%% NOTE the ``extra'' brace here %%%%
  \item[write\_hl] more general routine for adding a keyword to a header.
  \item[\textbf{cfitsio}] library for FITS file handling.		
  \end{sulist}
\end{modules}

\begin{related}
  \begin{sulist}{} %%%% NOTE the ``extra'' brace here %%%%
  \item[\htmlref{add\_card}{sub:add_card}] general purpose routine to write/edit an arbitrary
keyword into a FITS file header.
  \item[\htmlref{get\_card}{sub:get_card}] general purpose routine to read any keywords from a header in a FITS file.
  \item[\htmlref{del\_card}{sub:del_card}] routine to discard a keyword from a FITS header
  \item[\htmlref{read\_par}{sub:read_par}, \htmlref{number\_of\_alms}{sub:number_of_alms}] routines to read specific keywords from a
  header in a FITS file.
  \item[\htmlref{getsize\_fits}{sub:getsize_fits}] function returning the size of the data set in a fits
  file and reading some other useful FITS keywords
  \item[\htmlref{merge\_headers}{sub:merge_headers}] routine to merge two FITS headers
  \end{sulist}
\end{related}

\rule{\hsize}{2mm}

\newpage


\sloppy


\title{\healpix Fortran Subroutines Overview}
\docid{write\_plm} \section[write\_plm]{ }
\label{sub:write_plm}
\docrv{Version 1.2}
\author{Frode K.~Hansen, Eric Hivon}
\abstract{This document describes the \healpix Fortran90 subroutine WRITE\_PLM.}

\begin{facility}
{This routine creates a double precision binary FITS-file from a given array. The routine is used by the \healpix facility plmgen to store precomputed $P_{\ell m}(\theta)$.}
{\modFitstools}
\end{facility}

\begin{f90format}
{\mylink{sub:write_plm:plm}{plm}%
, \mylink{sub:write_plm:nplm}{nplm}%
, \mylink{sub:write_plm:nhar}{nhar}%
, \mylink{sub:write_plm:header}{header}%
, \mylink{sub:write_plm:nlheader}{nlheader}%
, \mylink{sub:write_plm:filename}{filename}%
, \mylink{sub:write_plm:nsmax}{nsmax}%
, \mylink{sub:write_plm:nlmax}{nlmax}%
}
\end{f90format}

\begin{arguments}
{
\begin{tabular}{p{0.4\hsize} p{0.05\hsize} p{0.05\hsize} p{0.40\hsize}} \hline  
\textbf{name\&dimensionality} & \textbf{kind} & \textbf{in/out} & \textbf{description} \\ \hline
                   &   &   &                           \\ %%% for presentation
plm\mytarget{sub:write_plm:plm}(0:nplm-1,1:nhar) & DP & IN & the array with the precomputed $P_{\ell m}(\theta)$ values.\\
nplm\mytarget{sub:write_plm:nplm} & I4B & IN & Number of $P_{\ell m}$ values to store.\\
nhar\mytarget{sub:write_plm:nhar} & I4B & IN & 1 for scalar $P_{\ell m}$ only and 3 for tensor harmonics. \\
header\mytarget{sub:write_plm:header}(LEN=80) (1:nlheader) & CHR & IN & The header for the FITS-file. \\
nlheader\mytarget{sub:write_plm:nlheader} & I4B & IN & number of header lines to write to the file. \\
filename\mytarget{sub:write_plm:filename}(LEN=\filenamelen) & CHR & IN & the precomputed $P_{\ell m}(\theta)$ values are written to this file. \\
nsmax\mytarget{sub:write_plm:nsmax} & I4B & IN & $\nside$ for the precomputed $P_{\ell m}\!\!$ s. \\
nlmax\mytarget{sub:write_plm:nlmax} & I4B & IN & maximum $\ell$  value for the precomputed $P_{\ell m}\!\!$ s. \\
\end{tabular}
}
\end{arguments}
\newpage
\begin{example}
{
call write\_plm (plm, 65*66*32, 1, header, 120, `plm\_32.fits', 32, 64)  \\
}
{
Makes a double precision binary FITS-file called `plm\_32.fits' from the precomputed $P_{\ell m}(\theta)$ in the array plm(0:65*66*32-1,1:1). The number 65*66*32 corresponds to the number of precomputed $P_{\ell m}\!\!$ s needed for a $\nside=32$ \healpix map synthesis/analysis. The header for the FITS-file is given in the string array header and the number of lines in the header is 120. 
}
\end{example}

\begin{modules}
  \begin{sulist}{} %%%% NOTE the ``extra'' brace here %%%%
  \item[\textbf{fitstools}] module, containing:
  \item[printerror] routine for printing FITS error messages.
  \item[\textbf{cfitsio}] library for FITS file handling.		
  \end{sulist}
\end{modules}

\begin{related}
  \begin{sulist}{} %%%% NOTE the ``extra'' brace here %%%%
  \item[\htmlref{read\_dbintab}{sub:read_dbintab}, \htmlref{read\_bintab}{sub:read_bintab}] routines which reads a file created by write\_plm. 
  \item[\htmlref{map2alm}{sub:map2alm}, \htmlref{alm2map}{sub:alm2map}] routines using precomputed $P_{\ell m}(\theta)$.
  \end{sulist}
\end{related}

\rule{\hsize}{2mm}

\newpage

\sloppy


\title{\healpix Fortran Subroutines Overview}
\docid{xcc\_v\_convert} \section[xcc\_v\_convert]{ }
\label{sub:xcc_v_convert}
\docrv{Version 2.0}
\author{Eric Hivon}
\abstract{This document describes the \healpix Fortran90 subroutine XCC\_V\_CONVERT.}

\begin{facility}
{This routine rotates a 3D coordinate vector from one astronomical coordinate
system to another.}
{\modCoordVConvert}
\end{facility}

\begin{f90format}
{\mylink{sub:xcc_v_convert:ivector}{ivector}%
, \mylink{sub:xcc_v_convert:iepoch}{iepoch}%
, \mylink{sub:xcc_v_convert:oepoch}{oepoch}%
, \mylink{sub:xcc_v_convert:isys}{isys}%
, \mylink{sub:xcc_v_convert:osys}{osys}%
, \mylink{sub:xcc_v_convert:ovector}{ovector}%
}
\end{f90format}

\begin{arguments}
{
\begin{tabular}{p{0.26\hsize} p{0.05\hsize} p{0.09\hsize} p{0.50\hsize}} \hline  
\textbf{name~\&~dimensionality} & \textbf{kind} & \textbf{in/out} & \textbf{description} \\ \hline
                   &   &   &                           \\ %%% for presentation
ivector\mytarget{sub:xcc_v_convert:ivector}(1:3) & DP & IN & 3D coordinate vector of one astronomical object, 
 in the input coordinate system.\\
iepoch\mytarget{sub:xcc_v_convert:iepoch} & DP & IN & epoch of the input astronomical coordinate system.\\
oepoch\mytarget{sub:xcc_v_convert:oepoch} & DP & IN & epoch of the output astronomical coordinate system.\\
isys\mytarget{sub:xcc_v_convert:isys}(len=*) & CHR & IN & input coordinate system, should be one of 'E'=Ecliptic, 'G'=Galactic, 'C'/'Q'=Celestial/eQuatorial.\\
osys\mytarget{sub:xcc_v_convert:osys}(len=*) & CHR & IN & output coordinate system, same choice as above.\\
ovector\mytarget{sub:xcc_v_convert:ovector}(1:3) & DP & IN & 3D coordinate vector of the same object, 
 in the output coordinate system.\\
\end{tabular}
}
\end{arguments}

\begin{example}
{
use healpix\_types \\
use coord\_v\_convert, only: xcc\_v\_convert \\
real(dp) :: vecin(1:3), vecout(1:3) \\
vecin = (/ 0\_dp, 0\_dp, 1\_dp /) \\
call xcc\_v\_convert(vecin, 2000.0\_dp, 2000.0\_dp, 'g', 'c', vecout)  \\
}
{Will produce in {\tt vecout} the location in Celestial coordinates (2000 epoch) of
the North Galactic Pole (defined in {\tt vecin})
}
\end{example}

% \begin{modules}
%   \begin{sulist}{} %%%% NOTE the ``extra'' brace here %%%%
%   \item[\textbf{alm\_tools}] module, containing:
% 	\item[\htmlref{generate\_beam}{sub:generate_beam}] routine to generate beam window function
% 	\item[\htmlref{pixel\_window}{sub:pixel_window}] routine to generate pixel window function
%   \end{sulist}
% \end{modules}

\begin{related}
  \begin{sulist}{} %%%% NOTE the ``extra'' brace here %%%%
   \item[\htmlref{coordsys2euler\_zyz}{sub:coordsys2euler_zyz}] produces the
Euler angles
 $\psi, \theta, \varphi$ in (Z,Y,Z) convention for rotation between standard astronomical coordinate systems.
  \item[\htmlref{ang2vec}{sub:ang2vec}, \htmlref{vec2ang}{sub:vec2ang}] Routine to convert spherical coordinates
  (co-latitude and longitude) into 3D vector coordinates and vice-versa.
  \end{sulist}
\end{related}

\rule{\hsize}{2mm}

\newpage


\end{document}
