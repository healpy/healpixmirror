
\sloppy


\title{\healpix Fortran Facility User Guidelines}
\docid{synfast} \section[synfast]{\nosectionname}
\label{fac:synfast}
\docrv{Version 1.2}
\author{Frode K.~Hansen, Benjamin D.~Wandelt, Eric Hivon}
\abstract{This document describes the \healpix facility SYNFAST.}

\begin{facility}
{This program can be used to create  \healpix maps (temperature only
or temperature and polarisation)  computed as realisations 
of random Gaussian
fields on a sphere characterized by the user provided 
theoretical power spectra,
or as constrained realisations of such fields characterised by the user
provided $a_{\ell m}$ coefficients and/or power spectra.
Total operation count scales as
 ${\cal {O}}(\npix^{3/2})$ with a prefactor dependent on the limiting spherical harmonics
order $\lmax$ of the actual problem. 
The map resolution, Gaussian beam FWHM,  
and random seed for the simulation can be selected by the user.
Spherical harmonics are either generated using the recurrence relations
during the execution of spectral synthesis, or  precomputed and read in
before the synthesis is executed. The latter is no longer recommended since
it provides no acceleration since the introduction of optimized algorithms
in \healpix v2.20. }
{src/f90/synfast/synfast.f90}
\end{facility}

\begin{f90facility}
{synfast [options] [parameter\_file]}
\end{f90facility}

\begin{options}
  \begin{optionlistwide}{} %%%% NOTE the ''extra'' brace here %%%%
    \item[{\tt -d}]
    \item[{\tt -}{\tt -}{\tt double}] double precision mode (see 
\htmlref{Notes on double/single precision modes}{fac:subsec:ioprec}
\latexhtml{ on page~\pageref{page:ioprec}}{})
    \item[{\tt -s}]
    \item[{\tt -}{\tt -}{\tt single}] single precision mode (default)
  \end{optionlistwide}
\end{options}

\begin{qualifiers}
  \begin{qulistwide}{} %%%% NOTE the ''extra'' brace here %%%%
    \item[{infile = }]\mytarget{fac:synfast:infile}%
 Defines the input power spectrum file,
	(default= cl.fits). Note that {\tt infile} is now optional :
    \thedocid\ can run even if only {\tt almsfile} is provided.
    \item[{outfile = }]\mytarget{fac:synfast:outfile}%
 Defines the output (RING ordered) map file,
(default= map.fits). Note that {\tt outfile} is now optional: if it set to 
      `' (empty string),  mo map is synthesized but the $a_{\ell m}$ generated can be output.
    \item[{outfile\_alms = }]\mytarget{fac:synfast:outfile_alms}%
 Defines the FITS file in which to output $a_{\ell m}$ used
      for the simulation (default= `')
     \item[{simul\_type = }]\mytarget{fac:synfast:simul_type}%
 Defines the simulation type, 1=temperature only (1 field),
       2=temperature+polarisation (3 fields), 3=temperature and its first
spatial derivatives (3 fields),
       4=temperature and its first and second spatial derivatives (6 fields), 5=temperature
       and polarisation, and their first derivatives (9 fields), 6=same as 5
       plus the second derivatives of (T,Q,U) (18 fields).
(default= 1).
    \item[{nsmax = }]\mytarget{fac:synfast:nsmax}%
 Defines the resolution of the map.
(default= 32)
     \item[{nlmax = }]\mytarget{fac:synfast:nlmax}%
 Defines the maximum $\ell$ value 
to be used in the simulation. WARNING: $\lmax$ can not exceed
the value $4\cdot$ {\tt nsmax}, because the coefficients of the  average Fourier 
pixel window functions
are precomputed and provided up to this limit.
(default= 64)
      \item[{iseed = }]\mytarget{fac:synfast:iseed}%
 Defines the random seed to be used 
for the generation of $a_{\ell m}$s from the power spectrum.
(default= -1)
    \item[{fwhm\_arcmin = }]\mytarget{fac:synfast:fwhm_arcmin}%
 Defines the FWHM size in arcminutes 
of the simulated Gaussian beam.
(default= 420.0)
%
    \item[{beam\_file = }]\mytarget{fac:synfast:beam_file}%
 Defines the FITS file (see \htmlref{''Beam window function files'' in introduction}{fac:subsec:beamfiles}) describing the
    Legendre window
    function of the circular beam to be used for the
    simulation. If set to an existing file name, it will override the
    {\tt fhwm\_arcmin} given above. (default=`')
%
     \item[{almsfile = }]\mytarget{fac:synfast:almsfile}%
 Defines the input filename for a file
    containing $a_{\ell m}$s for constrained realisations. 
(default= `'). If {\tt apply\_windows} is {\em false} 
those $a_{\ell m}$s are used as they are, without being multiplied
by the beam or pixel window function (with the assumption that they already have the
    correct window functions). If {\tt apply\_windows} is {\em true}, the beam and
    pixel window functions chosen above are applied to the constraining $a_{\ell m}$ (with the
    assumption that those are free of beam and pixel window function). The code
    does not check the validity of these asumptions; if none is true, use the
    \htmlref{alteralm}{fac:alteralm} facility to modify or remove
    the window functions contained in the constraining $a_{\ell m}$.
%
     \item[{apply\_windows = }]\mytarget{fac:synfast:apply_windows}%
  Determines how the constraining $a_{\ell m}$  read from
     {\tt almsfile} are
     treated with respect to window functions; see above for details. 
     y, yes, t, true, .true. and 1 are considered as {\em true}, while n, no, f,
     false, .false. and 0 are considered as {\em false}, (default = .false.).
%
     \item[{plmfile = }]\mytarget{fac:synfast:plmfile}%
 Defines the input  filename for a file
    containing  precomputed Legendre polynomials $P_{\ell m}$.
(default= `')
%
     \item[{windowfile = }]\mytarget{fac:synfast:windowfile}%
 Defines the input filename  for the pixel
    smoothing windows 
(default= pixel\_window\_n????.fits, see below).
%
     \item[{winfiledir = }]\mytarget{fac:synfast:winfiledir}%
  Defines the directory in which windowfile
    is located (default: see 
\htmlref{Notes on default files and directories}{fac:subsec:defdir}%
\latexhtml{ on page~\pageref{page:defdir}}{}).
  \end{qulistwide}
\end{qualifiers}

\begin{codedescription}
{%
Synfast reads the power spectrum from a file in  ascii FITS
format. This can contain either just the temperature power spectrum $C^T_{\ell}$s or 
temperature and polarisation power spectra: $C^T_{\ell}$, $C^E_{\ell}$, $C^B_{\ell}$
and $C^{T\times E}_{\ell}$ (see \mylink{fac:synfast:note1}{Note 1, below}). If {\tt simul\_type = 2} \thedocid\ generates 
Q and U maps as well as the temperature map. The output map(s)
is (are) saved in a FITS file. 
The $C_{\ell}$s are used up to the specified 
$\lmax$, which can not exceed $4\times$ nsmax. If {\tt simul\_type = 3} or
{\tt 4} the first derivatives of the temperature field or the first and second derivatives respectively
are output as well as the temperature itself: $T(p)$, $\left({\partial T}/{\partial \theta}, {\partial T}/{\partial \phi}/\sin\theta \right)\myhtmlimage{}
$, $\left({\partial^2 T}/{\partial \theta^2}, {\partial^2 T}/{\partial
  \theta\partial\phi}/\sin\theta,\right.\myhtmlimage{}$ 
$\left.{\partial^2 T}/{\partial \phi^2}/\sin^2\theta \right)\myhtmlimage{}$.
If {\tt simul\_type = 5} or
{\tt 6} the first derivatives of the (T,Q,U) fields or the first and second derivatives respectively
are output as well as the field themself: $T(p)$,  $Q(p)$,  $U(p)$,
$\left({\partial T}/{\partial \theta}, {\partial Q}/{\partial \theta}, {\partial
  U}/{\partial \theta}; {\partial T}/{\partial \phi}/\sin\theta, \ldots \right)\myhtmlimage{}
$, $\left({\partial^2 T}/{\partial \theta^2},\ldots; {\partial^2 T}/{\partial
  \theta\partial\phi}/\sin\theta,\ldots ;\right.\myhtmlimage{}$ 
$\left.{\partial^2 T}/{\partial \phi^2}/\sin^2\theta \ldots \right)\myhtmlimage{}$

The random sequence seed for generation of $a_{\ell m}$ from the
power spectrum should be non-zero integer. If 0 is provided, a seed is generated
randomly by the code, based on the current date and time.
The map can be convolved with a gaussian beam for which a beamsize can
be specified, or for an arbitrary {\em circular} beam for which the
Legendre transform is provided. The map is automatically convolved with a pixel window
function. These are stored in FITS files  in
the {\tt healpix/data} directory. If \thedocid\ is not run in a directory
which has these files, or from a directory which can reach these files
by a `{\tt ../data/}' or `{\tt ./data/}' specification, the system
variable {\tt HEALPIX} is used to locate the main \healpix directory
and its {\tt data} subdirectory is scanned. Failing this, the location of these
files must be specified (using winfiledir). In the interactive mode this is
requested only when necessary (see 
\htmlref{Notes on default directories}{fac:subsec:defdir}
\latexhtml{ on page~\pageref{page:defdir}}{}).

If some of the $a_{\ell m}$ in the simulations are constrained eg. from observations, a FITS file
with these $a_{\ell m}$ can be read. This FITS file contains 
the $a_{\ell m}$ for certain $\ell$ and $m$ values
and also the standard deviation for these $a_{\ell m}$. The sky
realisation which \thedocid\ produces will be statistically consistent
with the constraining $a_{\ell m}$.

The code can also be used
to generate a set of $a_{\ell m}$ matching the input power spectra, beam size and
pixel size with or without actually synthesizing the map. Those $a_{\ell m}$ can be
used as an input (constraining $a_{\ell m}$) to another \thedocid\ run.
%---------------------------------
\latexhtml{{\\ {\bf ...}}}{}
}
\end{codedescription}
%
\begin{cd_contd}
{%
%---------------------------------
Spherical harmonics values in the synthesis are obtained from a
recurrence on associated Legendre polynomials $P_{\ell m}(\theta)$. 
This recurrence consumed most of the CPU time used by \thedocid\ up to version
2.15. We have therefore included an option to load precomputed values for the
$P_{\ell m}(\theta)$ from a file generated by the \healpix facility
\htmlref{plmgen}{fac:plmgen}. Since the introduction of accelerated spherical
harmonic transforms in \healpix v2.20, this feature is obsolete and should no
longer be used. 

\thedocid\ will issue a warning if the input FITS file for the power spectrum does
not contain the keyword {\tt POLNORM}. This keyword indicates that the convention
used for polarization is consistent with CMBFAST (and consistent with \healpix
1.2). See the 
\latexhtml{\htmladdnormallink{\healpix Primer}{intro.pdf}}{%
           \htmladdnormallink{\healpix Primer}{intro.htm}}
for details on the
polarization convention and the interface with CMBFAST. If the
keyword is not found, {\em no attempt will be made} to renormalize the power
spectrum. 
If the keyword is present, it will be inherited by the simulated map.

\mytarget{fac:synfast:note1}{\bf Note 1:} to allow the generation of maps (and $a_{\ell m}$) with $C^{T\times B}_{\ell} \ne 0$ and/or $C^{E\times B}_{\ell} \ne 0$,
see the subroutine \htmlref{{\tt create\_alm}}{sub:create_alm}.
}
\end{cd_contd}


\begin{datasets}
{
\begin{tabular}{p{0.3\hsize} p{0.35\hsize}} \hline  
  \textbf{Dataset} & \textbf{Description} \\ \hline
                   &                      \\ %%% for presentation
  /data/pixel\_window\_nxxxx.fits & Files containing pixel windows for
                   various nsmax.\\ 
                   &                      \\ \hline %%% for presentation
\end{tabular}
} 
\end{datasets}

\begin{support}
%   \begin{sulistwide}{} %%%% NOTE the ''extra'' brace here %%%%
  \begin{sulist}{} %%%% NOTE the ''extra'' brace here %%%%
  \item[\htmlref{generate\_beam}{sub:generate_beam}] This \healpix Fortran
subroutine generates or reads the $B(\ell)$ window function used in \thedocid
  \item[\htmlref{map2gif}{fac:map2gif}] This \healpix Fortran facility can be used to visualise the
  output map of \thedocid.
  \item[\htmlref{mollview}{idl:mollview}] This \healpix IDL facility can be used to visualise the
  output map of \thedocid.
  \item[\htmlref{alteralm}{fac:alteralm}] This \healpix Fortran facility can be
  used to implement the beam and pixel window functions on the constraining
  $a_{\ell m}$s ({\tt almsfile} file).
  \item[\htmlref{anafast}{fac:anafast}] This \healpix Fortran facility can analyse a \healpix map and 
     	       save the $a_{\ell m}$ and $C_{\ell}$s to be read by \thedocid.
  \item[\htmlref{plmgen}{fac:plmgen}] This \healpix Fortran facility can be used to generate precomputed Legendre polynomials.
		
  \end{sulist}
%  \end{sulistwide}
\end{support}

\begin{examples}{1}
{
\begin{tabular}{ll} %%%% use this tabular format %%%%
synfast  \\
\end{tabular}
}
{
Synfast runs in interactive mode, self-explanatory.
}
\end{examples}

\vfill\newpage

\begin{examples}{2}
{
\begin{tabular}{ll} %%%% use this tabular format %%%%
synfast  filename \\
\end{tabular}
}
{When 'filename' is present, \thedocid\ enters the non-interactive mode and parses
its inputs from the file 'filename'. This has the following
structure: the first entry is a qualifier which announces to the parser
which input immediately follows. If this input is omitted in the
input file, the parser assumes the default value.
If the equality sign is omitted, then the parser ignores the entry.
In this way comments may also be included in the file.
In this example, the file contains the following qualifiers:\hfill\newline
\fileparam{\mylink{fac:synfast:simul_type}{simul\_type}%
 = 1}
\fileparam{\mylink{fac:synfast:nsmax}{nsmax}%
 = 32}
\fileparam{\mylink{fac:synfast:nlmax}{nlmax}%
 = 64}
\fileparam{\mylink{fac:synfast:iseed}{iseed}%
 = -1}
\fileparam{\mylink{fac:synfast:fwhm_arcmin}{fwhm\_arcmin}%
 = 420.0}
\fileparam{\mylink{fac:synfast:infile}{infile}%
 = cl.fits}
\fileparam{\mylink{fac:synfast:outfile}{outfile}%
 = map.fits}

Synfast reads in the $C_{\ell}$ power spectrum in 'cl.fits' up to $\ell=64$, and
produces the (RING ordered) map
'map.fits' which has $\nside=32$.
The map is convolved with a beam of FWHM 420.0 arcminutes. The $\mylink{fac:synfast:iseed}{iseed}%
=-1$ sets
the random seed for the realisation. A different $\mylink{fac:synfast:iseed}{iseed}%
$ would have given a different 
realisation from the same power spectrum.

Since \hfill\newline
\fileparam{\mylink{fac:synfast:outfile_alms}{outfile\_alms}%
 }
\fileparam{\mylink{fac:synfast:almsfile}{almsfile}%
 }
\fileparam{\mylink{fac:synfast:apply_windows}{apply\_windows}%
 }
\fileparam{\mylink{fac:synfast:plmfile}{plmfile}%
 }
\fileparam{\mylink{fac:synfast:beam_file}{beam\_file}%
 }
\fileparam{\mylink{fac:synfast:windowfile}{windowfile}%
 }
were omitted, they take their default values (empty strings). 
This means that no file for constrained realisation or precomputed
Legendre polynomials are read, the $a_{\ell m}$ generated in the process are not
output, and \thedocid\ attempts to find the pixel
window files in the default directories (see 
\htmlref{Notes on default files and directories}{fac:subsec:defdir}%
\latexhtml{ on page~\pageref{page:defdir}}).
}
\end{examples}

\begin{release}
  \begin{relist}
    \item Initial release (\healpix 0.90)
    \item Optional non-interactive operation. Proper FITS file
    support. Improved reccurence algorithm for $P_{\ell m}(\theta)$ which can compute to higher $\ell$ values. Improved pixel windows averaged over
    actual HEALPix pixels. New functionality: constrained realisations, precomputed
    $P_{\ell m}$. (\healpix 1.00)
    \item New functionality: constrained realisations and pixel
    windows are now available for polarization as well. Arbitrary
    circular beams can be used. New parser (\healpix 1.20)
    \item New functionnality: the generated $a_{\ell m}$ can be output, and the map
    synthesis itself can be skipped. First and second derivatives of the
    temperature field can be produced on demand.
    \item New functionnality: First and second derivatives of the
    $Q$ and $U$ Stokes field can be produced on demand.
    \item Bug correction: corrected numerical errors on derivatives 
$\partial X/\partial\theta$, 
$\partial^2 X/(\partial\theta\partial\phi\sin\theta)$, 
$\partial^2 X/\partial \theta^2$, 
for $X=Q,U$. See \htmlref{this appendix}{fac:sec:bug_synder} for details.
  (\healpix 2.14)
  \end{relist}
\end{release}

\begin{messages}
{
\begin{tabular}{p{0.25\hsize} p{0.1\hsize} p{0.35\hsize}} \hline  
  \textbf{Message} & \textbf{Severity} & \textbf{Text} \\ \hline
                   &                   &   \\ %%% for presentation
can not allocate memory for array xxx &  Fatal & You do not have
                   sufficient system resources to run this
                   facility at the map resolution you required. 
  Try a lower map resolution.  \\ 
                   &                   &   \\ \hline %%% for presentation

this is not a binary table & & the fitsfile you have specified is not 
of the proper format \\
                   &                   &   \\ \hline %%% for presentation
there are undefined values in the table! & & the fitsfile you have specified is not 
of the proper format \\
                  &                   &   \\ \hline %%% for presentation
the header in xxx is too long & & the fitsfile you have specified is not 
of the proper format \\
                  &                   &   \\ \hline %%% for presentation
XXX-keyword not found & & the fitsfile you have specified is not 
of the proper format \\
                  &                   &   \\ \hline %%% for presentation
found xxx in the file, expected:yyyy & & the specified fitsfile does not
contain the proper amount of data. \\
                   &                   &   \\ \hline %%% for presentation



\end{tabular}
} 
\end{messages}

\rule{\hsize}{2mm}

\newpage
