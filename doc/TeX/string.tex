

\sloppy

\title{\healpix Fortran Subroutines Overview}
\docid{string, strlowcase, strupcase} \section[string, strlowcase, strupcase]{ }
\label{sub:string}
\docrv{Version 2.1}
\author{Eric Hivon}
\abstract{This document describes the \healpix Fortran90 functions in module MISC\_UTILS.}

\begin{facility}
{The Fortran90 module misc\_utils contains three functions to create or
  manipulate character strings.}
{\modMiscUtils}
\end{facility}

\begin{arguments}
{
\begin{tabular}{p{0.28\hsize} p{0.05\hsize} p{0.10\hsize} p{0.47\hsize}} \hline  
\textbf{name~\&~dimensionality} & \textbf{kind} & \textbf{in/out} & \textbf{description} \\ \hline
                   &   &   &                           \\ %%% for presentation
number & LGT/ I4B/ SP/ DP & IN & number or boolean flag to be turned into a character string. \\
instring & CHR & IN & arbitrary character string. \\
outstring & CHR & --- & output character string. \\
format \hskip 3cm OPTIONAL & CHR & IN & character string describing Fortran
                   format of output. %% \\
%% upstring & CHR & --- & uppercase character string. \\
%% lowstring & CHR & --- & lowercase character string. 
\end{tabular}
}
\end{arguments}

\rule{\hsize}{0.7mm}
\textsc{\large{\textbf{FUNCTIONS: }}}\hfill\newline
{\tt outstring = string(number [,format])} 

 \begin{tabular}{@{}p{0.3\hsize}@{\hspace{1ex}}p{0.7\hsize}@{}}
                         & returns in {\tt outstring} its argument {\tt number} converted to a
                                         character string. If {\tt format} is provided it is used to
                                         format the output, if not, the fortran default format
                                         matching {\tt number}'s type is used. \\
     \end{tabular}\\\\

{\tt outstring = strlowcase(instring)} 

 \begin{tabular}{@{}p{0.3\hsize}@{\hspace{1ex}}p{0.7\hsize}@{}}
                         & returns in {\tt outstring} its argument {\tt instring}
                                         converted to lowercase. ASCII characters in the [A-Z] range
                                         are mapped to [a-z], while all others remain unchanged.\\
     \end{tabular}\\\\

{\tt outstring = strupcase(instring)} 

 \begin{tabular}{@{}p{0.3\hsize}@{\hspace{1ex}}p{0.7\hsize}@{}}
                         & returns in {\tt outstring} its argument {\tt instring}
                                         converted to uppercase. ASCII characters in the [a-z] range
                                         are mapped to [A-Z], while all others remain unchanged.\\
     \end{tabular}\\\\


\begin{example}
{
use misc\_utils \\
character(len=24) :: s1 \\
s1 = string(123,'(i5.5)') \\
print*, trim(s1) \\
print*,trim(strupcase('*aBcD-123')) \\
print*,trim(strlowcase('*aBcD-123')) \\
}
{ Will printout {\tt 00123}, {\tt *ABCD-123} and {\tt *abcd-123}.
}
\end{example}

%% \begin{modules}
%%   \begin{sulist}{} %%%% NOTE the ``extra'' brace here %%%%
%%  \item[] 
%%   \end{sulist}
%% \end{modules}

%% \begin{related}
%%   \begin{sulist}{} %%%% NOTE the ``extra'' brace here %%%%
%%   \item[]
%%   \end{sulist}
%% \end{related}

\rule{\hsize}{2mm}

\newpage
