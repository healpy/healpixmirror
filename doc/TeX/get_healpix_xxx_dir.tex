

\sloppy

%%%\title{\healpix Fortran Subroutines Overview}
\docid{get\_healpix\_main\_dir,~$\ldots$} \section[get\_healpix\_data\_dir, get\_healpix\_main\_dir, get\_healpix\_test\_dir]{ }
\label{sub:get_healpix_xxx_dir}
\docrv{Version 2.0}
\author{Eric Hivon}
\abstract{This document describes the \healpix Fortran90 functions in module paramfile\_io.}

\begin{facility}
{A few functions are available to return the full path to \healpix main directory
and its {\tt data} and {\tt test} subdirectories. This allow those paths to be
controlled by preprocessing macros or environment variables in case of
non-standard installation of the \healpix directory structure.}
{\modParamfileIo}
\end{facility}

%-------------------------------

\rule{\hsize}{0.7mm}
\textsc{\large{\textbf{FUNCTIONS: }}}\hfill\newline
{\tt hmd = get\_healpix\_main\_dir()} \mytarget{sub:get_healpix_xxx_dir:ghmd}

 \begin{tabular}{@{}p{0.3\hsize}@{\hspace{1ex}}p{0.7\hsize}@{}}
                         & returns the full path to the main
			\healpix directory. It will be determined, in this
			order, from the value of the
			preprocessing macros {\tt HEALPIX} and {\tt HEALPIXDIR}
			if they are defined or the
			environment variable {\tt \$HEALPIX} otherwise.\\
     \end{tabular}\\\\

{\tt hdd = get\_healpix\_data\_dir()} \mytarget{sub:get_healpix_xxx_dir:ghdd}

 \begin{tabular}{@{}p{0.3\hsize}@{\hspace{1ex}}p{0.7\hsize}@{}}
                         & returns the full path to
			\healpix {\tt data} subdirectory. It will be determined
			from the preprocessing macro {\tt HEALPIXDATA} or the environment variable {\tt
			\$HEALPIXDATA}. If both fail, it will return the list of directories \{{\tt
			. ../data ./data .. \$HEALPIX \$HEALPIX/data \$HEALPIX/../data
			\$HEALPIX$\backslash$data}\} separated by LineFeed.
\\
     \end{tabular}\\\\


{\tt htd = get\_healpix\_test\_dir()} \mytarget{sub:get_healpix_xxx_dir:ghtd}

 \begin{tabular}{@{}p{0.3\hsize}@{\hspace{1ex}}p{0.7\hsize}@{}}
                         & returns the full path to
			\healpix {\tt test} subdirectory. It will be determined,
			in this order, from the preprocessing macro {\tt HEALPIXTEST}, the environment
			variable {\tt \$HEALPIXTEST} or {\tt \$HEALPIX/test}.\\
     \end{tabular}\\\\

\vskip 3cm
%-------------------------------

% \begin{arguments}
% {
% \begin{tabular}{p{0.30\hsize} p{0.05\hsize} p{0.08\hsize} p{0.47\hsize}} \hline  
% \textbf{name~\&~dimensionality} & \textbf{kind} & \textbf{in/out} & \textbf{description} \\ \hline
%                    &   &   &                           \\ %%% for presentation
% test & LGT & IN & result of a logical test \\
% msg \hfill OPTIONAL & CHR & IN & character string describing nature of error \\
% errorcode \hfill OPTIONAL & I4B & IN & error status given to code interruption \\
% status & I4B & IN & value of the {\tt stat} flag returned by the F90 {\tt allocate} command \\
% code & CHR & IN & name of program or code in which allocation is made \\
% array & CHR & IN & name of array allocated \\
% directory & CHR & IN & directory name (contains a '/')\\
% filename & CHR & IN & file name \\
% \end{tabular}
% }
% \end{arguments}

%-------------------------------

% \begin{example}
% {
% program my\_code \\
% use misc\_utils \\
% real, allocatable, dimension(:) :: vector\\
% integer :: status \\
% real :: a = -1. \\
% \\
% allocate(vector(12345),stat=status) \\
% call assert\_alloc(status, 'my\_code', 'vector') \\
% \\
% call assert\_directory\_present('/home') \\
% \\
% call assert(a > 0., 'a is NEGATIVE !!!') \\
% \\
% end program my\_code\\
% }
% { Will issue a error message and stops the code if {\tt vector} can not be allocated, will stop the
%   code if '/home' is not found, and will stop the code and complain loudly about it 
% because {\tt a} is actually negative.
% }
% \end{example}

%% \begin{modules}
%%   \begin{sulist}{} %%%% NOTE the ``extra'' brace here %%%%
%%  \item[mk\_pix2xy, mk\_xy2pix] routines used in the conversion between pixel values and ``cartesian'' coordinates on the Healpix face.
%%   \end{sulist}
%% \end{modules}

%% \begin{related}
%%   \begin{sulist}{} %%%% NOTE the ``extra'' brace here %%%%
%%   \item[\htmlref{neighbours\_nest}{sub:neighbours_nest}] find neighbouring pixels.
%%   \item[\htmlref{ang2vec}{sub:ang2vec}] convert $(\theta,\phi)$ spherical coordinates into $(x,y,z)$ cartesian coordinates.
%%   \item[\htmlref{vec2ang}{sub:vec2ang}] convert $(x,y,z)$ cartesian coordinates into $(\theta,\phi)$ spherical coordinates.
%%   \end{sulist}
%% \end{related}

\rule{\hsize}{2mm}

\newpage
