
\sloppy


%\title{\healpix Fortran Subroutines Overview}
\docid{map2alm\_iterative} \section[map2alm\_iterative*]{ }
\label{sub:map2alm_iterative}
\docrv{Version 2.1}
\author{Eric Hivon}
\abstract{This document describes the \healpix Fortran90 subroutine MAP2ALM\_ITERATIVE*.}

\begin{facility}
{This routine covers and extends the functionalities of \htmlref{map2alm}{sub:map2alm}: it
analyzes a (polarised) \healpix {\em RING ordered} map and returns
its $a_{\ell m}$ coefficients for temperature (and polarisation) up to a specified
multipole, and use precomputed harmonics if those
are provided, but it also can also perform an iterative (Jacobi) determination of the $a_{\ell m}$, and
apply a pixel mask if one is provided.\\
\newcommand{\bA}{\textbf{A}}
\newcommand{\bS}{\textbf{S}}
\newcommand{\ba}{\textbf{a}}
\newcommand{\bm}{\textbf{m}}
\newcommand{\bw}{\textbf{w}}
Denoting $\bA$ and $\bS$ the 
analysis  (\htmlref{map2alm}{sub:map2alm}) and
synthesis (\htmlref{alm2map}{sub:alm2map})
operators and  $\ba, \bm$ and $\bw$, the $a_{\ell m}$, map and pixel mask vectors, the
Jacobi iterative process reads 
\begin{eqnarray}
	\label{eq:map2alm_it_a}
	\ba^{(n)} = \ba^{(n-1)} + \bA. \left( \bw.\bm - \bS .\ba^{(n-1)} \right),
\end{eqnarray}
with
\begin{eqnarray}
	\label{eq:map2alm_it_b}
	\ba^{(0)} = \bA.\bw.\bm.
\end{eqnarray}
%
During the processing, the standard deviation of the input map $\left(\bw.\bm\right)$ 
and the current residual map $\left(\bw.\bm - \bS .\ba^{(n-1)}\right)$ is printed out, with the latter expected
to get smaller and smaller as $n$ increases.\\
The standard deviation of map $x$ has the usual definition
$\sigma \equiv \sqrt{\sum_{p=1}^{N}\frac{(x(p)-\bar{x})^2}{N-1}}$, where the mean is
$\bar{x} \equiv  \sum_{p=1}^{N} \frac{x(p)}{N}$, and the index $p$ runs over all pixels.%
% Note that this iterative process is meaningful only if the 
% input map has (almost) no power outside the multipole range $[0, \lmax]$ of analysis.
% The application of a mask to the map will most likely break this assumption, 
% reducing the improvement to be expected from the iterations.
\\
In \htmlref{version 3.50}{sub:new3p50} a bug affecting previous versions of \thedocid{} has been fixed.
(It occured when 
\mylink{sub:map2alm_iterative:iter_order}{\texttt{iter\_order}}$>0$ 
was used in conjonction with a 
\mylink{sub:map2alm_iterative:mask}{\texttt{mask}} 
and/or a restrictive 
\mylink{sub:map2alm_iterative:zbounds}{\texttt{zbounds}}, 
with a magnitude that depended on each of those factors and was larger for non-boolean masks (ie, $\bw^2 \ne \bw$).
To assess the impact of this bug on previous results, the old implementation remains available in 
\texttt{map2alm\_iterative\_old}). 
The result was correct when the mask (if any) was applied to the map prior to the 
\thedocid{} calling, or when no iteration was requested.}
{\modAlmTools}
\end{facility}

\begin{f90format}
{\mylink{sub:map2alm_iterative:nsmax}{nsmax}%
, \mylink{sub:map2alm_iterative:nlmax}{nlmax}%
, \mylink{sub:map2alm_iterative:nmmax}{nmmax}%
, \mylink{sub:map2alm_iterative:iter_order}{iter\_order}%
, \mylink{sub:map2alm_iterative:map_TQU}{map\_TQU}%
, \mylink{sub:map2alm_iterative:alm_TGC}{alm\_TGC}%
 [, \mylink{sub:map2alm_iterative:zbounds}{zbounds}%
, \mylink{sub:map2alm_iterative:w8ring_TQU}{w8ring\_TQU}%
 ,
\mylink{sub:map2alm_iterative:plm}{plm}%
, \mylink{sub:map2alm_iterative:mask}{mask}%
]}
\end{f90format}

%\newpage
\begin{arguments}
{
\begin{tabular}{p{0.38\hsize} p{0.05\hsize} p{0.07\hsize} p{0.40\hsize}} \hline  
\textbf{name~\&~dimensionality} & \textbf{kind} & \textbf{in/out} & \textbf{description} \\ \hline
                   &   &   &                           \\ %%% for presentation
nsmax\mytarget{sub:map2alm_iterative:nsmax} & I4B & IN & the $\nside$ value of the map to analyse. \\
nlmax\mytarget{sub:map2alm_iterative:nlmax} & I4B & IN & the maximum $\ell$ value ($\lmax$) for the analysis. \\
nmmax\mytarget{sub:map2alm_iterative:nmmax} & I4B & IN & the maximum $m$ value for the analysis. \\
%
iter\_order\mytarget{sub:map2alm_iterative:iter_order} & I4B & IN & the order of Jacobi iteration. Increasing that order
improves the accuracy of the final $a_{\ell m}$ but increases the computation time $
T_{\mathrm{CPU}} \propto 1 + 2 \times $iter\_order. 
iter\_order  $=0$ is a straight analysis, while iter\_order $=3$ is usually a
good compromise. \\
%
map\_TQU\mytarget{sub:map2alm_iterative:map_TQU}(0:12*nsmax**2-1, 1:p) & SP/ DP & INOUT & input map. $p$ is 1 or 3
depending if temperature (T) only or temperature and polarisation (T, Q, U) are
to be analysed. It will be altered on output if a \mylink{sub:map2alm_iterative:mask}{mask} is provided and/or if \mylink{sub:map2alm_iterative:iter_order}{iter\_order}$>0$ and \mylink{sub:map2alm_iterative:zbounds}{zbounds} is provided.\\
%
alm\_TGC\mytarget{sub:map2alm_iterative:alm_TGC}(1:p, 0:nlmax, 0:nmmax) & SPC/ DPC & OUT & The $a_{\ell m}$ values output
from the analysis. 
$p$ is 1 or 3 depending on whether polarisation is included or not. In the former
case, the first index is (1,2,3) corresponding to (T,E,B). \\
%
zbounds\mytarget{sub:map2alm_iterative:zbounds}(1:2), \hskip 6cm OPTIONAL  & DP & IN & section of the map on which to perform the $a_{\ell m}$
                   analysis, expressed in terms of $z=\sin(\mathrm{latitude}) =
                   \cos(\theta).$ %zbounds_sub.tex:  for alm2map, map2alm, remove_dipole: describe processed area
%zbounds2_sub.tex: for apply_mask: describe pixels set to 0 (=unprocessed area)
If zbounds(1)$<$zbounds(2), it is
performed {\em on} the strip zbounds(1)$<z<$zbounds(2); if not,
it is performed {\em outside} the strip
%  zbounds(2)$<z<$zbounds(1). % OLD
zbounds(2)$\le z \le$zbounds(1). % NEW ??
% The whole sphere is treated if \texttt{zbounds=(/-1.0\_dp, 1.0\_dp/)} or if 
% it is absent.
If absent, the whole map is processed.
\\
%
\end{tabular}
%
\begin{tabular}{p{0.38\hsize} p{0.05\hsize} p{0.07\hsize} p{0.40\hsize}}   \hline  
w8ring\_TQU\mytarget{sub:map2alm_iterative:w8ring_TQU}(1:2*nsmax,1:p), \hskip 6cm OPTIONAL  & DP & IN & ring weights for
quadrature corrections. p is 1 for a temperature analysis and 3 for (T,Q,U). If absent, the
ring weights are all set to 1.\\
%
plm\mytarget{sub:map2alm_iterative:plm}(0:,1:p), \hskip 6cm OPTIONAL & DP & IN & If this
optional matrix is passed, precomputed scalar (and tensor) $P_{\ell m}(\theta)$ are
used instead of recursion. \\
%
mask\mytarget{sub:map2alm_iterative:mask}(0:12*nsmax**2-1,1:q), \hskip 6cm OPTIONAL & SP/ DP & IN & pixel mask,
assumed to have the same resolution (and RING ordering) as the map. The map {\tt
map\_TQU} is
multiplied by that mask before being analyzed, and will therefore be altered on
output. 
$q$ should be in $\{1,2,3\}$. If $p=q=3$, then each of
the 3 masks is applied to the respective map. If $p=3$ and $q=2$, the first mask
is applied to the first map, and the second mask to the second (Q) and third (U)
map. If $p=3$ and $q=1$, the same mask is applied to the 3 maps. Note: the output
$a_{\ell m}$ are computed directly on the masked map, and are {\em not} corrected for the
loss of power, correlation or leakage created by the mask.
\end{tabular}
}
\end{arguments}
%%\newpage

\begin{example}
{
use healpix\_types\\
use alm\_tools\\
use pix\_tools\\
integer(i4b) :: nside, lmax, npix, iter \\
real(sp), allocatable, dimension(:,:) :: map \\
real(sp), allocatable, dimension(:) :: mask \\
complex(spc), allocatable, dimension(:,:,:) :: alm \\
\\
nside = 256 \\
lmax = 512 \\
iter = 2\\
npix = nside2npix(nside) \\
allocate(map(0:npix-1,1:3)) \\
allocate(mask(0:npix-1)) \\
mask(0:) = 0. ! set unvalid pixels to 0\\
mask(0:10000-1) = 1. ! valid pixels \\
allocate(alm(1:3, 0:lmax, 0:lmax)\\
call map2alm\_iterative(nside, lmax, lmax, iter, map, alm, mask=mask)  \\
}
{
Analyses temperature and polarisation signals in the first 10000 pixels of {\tt map} (as
determined by {\tt mask}). The map has
an $\nside$ of 256, and the analysis is supposed to be performed up
to 512 in $\ell$ and $m$. The resulting $a_{\ell m}$ coefficients for
temperature and polarisation are returned in {\tt alm}. Uniform weights are
assumed. In order to improve the $a_{\ell m}$ accuracy, 2 Jacobi iterations are performed.
}
\end{example}

\begin{modules}
  \begin{sulist}{} %%%% NOTE the ``extra'' brace here %%%%
%  \item[ring\_analysis] Performs FFT for the ring analysis.
  \item[\htmlref{map2alm}{sub:map2alm}] Performs the alm analysis
  \item[\htmlref{alm2map}{sub:alm2map}] Performs the map synthesis
  \item[\textbf{misc\_util}] module, containing:
  \item[\htmlref{assert\_alloc}{sub:assert}] routine to print error message when an array is not
  properly allocated		
  \end{sulist}
\end{modules}

\begin{related}
  \begin{sulist}{} %%%% NOTE the ``extra'' brace here %%%%
  \item[\htmlref{anafast}{fac:anafast}] executable using \thedocid \ to analyse maps.
  \item[\htmlref{alm2map}{sub:alm2map}] routine performing the inverse transform of \thedocid.
  \item[\htmlref{alm2map\_spin}{sub:alm2map_spin}] synthesize spin weighted
maps.
  \item[\htmlref{dump\_alms}{sub:dump_alms}] write $a_{\ell m}$ coefficients
computed by \thedocid into a FITS file
  \item[\htmlref{map2alm\_spin}{sub:map2alm_spin}] analyze spin weighted maps.
  \end{sulist}
\end{related}

\rule{\hsize}{2mm}

\newpage
