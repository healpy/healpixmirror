
\sloppy


\title{\healpix Fortran Subroutines Overview}
\docid{pixel\_window} \section[pixel\_window]{ }
\label{sub:pixel_window}
\docrv{Version 2.0}
\author{Eric Hivon}
\abstract{This document describes the \healpix Fortran90 subroutine
PIXEL\_WINDOW.}

\newcommand{\wpix}{w_\textrm{pix}(\ell)}
\newcommand{\alm}{a_{\ell m}}
\newcommand{\almpix}{a_{\ell m}^\textrm{(pix)}}

\begin{facility}
{This routine returns the {\em averaged} $\ell$-space window function $\wpix$ (for temperature and
  polarisation) associated to \healpix\
  pixels of resolution parameter $\nside$. Because of the integration of the
signal over the
pixel area, the $\almpix$ coefficients of a pixelated map
are related to the unpixelated underlying $\alm$ by $\almpix = \alm \wpix$.\\
Unless specified otherwise, the $\wpix$ are read from the files
  \$HEALPIX/data/pixel\_window\_n????.fits.}
{\modAlmTools}
\end{facility}

\begin{f90format}
{\mylink{sub:pixel_window:pixlw}{pixlw}%
, \mylink{sub:pixel_window:nside}{nside}%
 [, \mylink{sub:pixel_window:windowfile}{windowfile}%
]}
\end{f90format}

\begin{arguments}
{
\begin{tabular}{p{0.30\hsize} p{0.05\hsize} p{0.05\hsize} p{0.50\hsize}} \hline  
\textbf{name~\&~dimensionality} & \textbf{kind} & \textbf{in/out} & \textbf{description} \\ \hline
                   &   &   &                           \\ %%% for presentation
pixlw\mytarget{sub:pixel_window:pixlw}(0:lmax,1:p) & DP & OUT & pixel window function(s) $\wpix$ generated. The first index
                   must be $\ell_\textrm{max}\leq 4\nside$. The second index runs from 1:1 for
                   temperature only, and 1:3 for polarisation. In the latter
                   case, 1=T, 2=E, 3=B.\\
nside\mytarget{sub:pixel_window:nside} & I4B & IN & \healpix\ $\nside$ resolution parameter. Unless {\tt
                   windowfile} is set, the file associated
                   with $\nside$ and shipped with the package is read by
                   default. If {\tt nside} = 0, the pixel is assumed infinitely
                   small and {\tt pixlw} is returned with value 1.\\
windowfile\mytarget{sub:pixel_window:windowfile} \hskip 2cm
(OPTIONAL) & CHR & IN & FITS file containing the pixel window to be read instead
                   of the default.
\end{tabular}
}
\end{arguments}

\begin{example}
{
call pixel\_window(pixlw, 64)  \\
}
{
returns in pixlw the pixel window function for $\nside = 64$.
}
\end{example}

\begin{modules}
  \begin{sulist}{} %%%% NOTE the ``extra'' brace here %%%%
  \item[\textbf{misc\_utils}] module, containing:
      \item[\htmlref{assert, fatal\_error}{sub:assert}] interrupt code in case of error
  \item[\textbf{extension}] module, containing:
     \item[\htmlref{getEnvironment}{sub:getenvironment}] read environment variable
  \item[\textbf{fitstools}] module, containing:
     \item[\htmlref{read\_dbintab}{sub:read_dbintab}] reads double precision binary table
  \end{sulist}
\end{modules}

\begin{related}
  \begin{sulist}{} %%%% NOTE the ``extra'' brace here %%%%
  \item[\htmlref{gaussbeam}{sub:gaussbeam}] routine to generate a gaussian
beam window function
  \item[\htmlref{generate\_beam}{sub:generate_beam}] returns a beam window function
  \item[\htmlref{alter\_alm}{sub:alter_alm}, \htmlref{rotate\_alm}{sub:rotate_alm}] modifies $\alm$ to emulate effect
of real space filtering and coordinate rotation respectively
  \item[\htmlref{alm2map}{sub:alm2map}] synthetize a \healpix map from its $\alm$
(or $\almpix$).
  \item[\htmlref{alm2map\_der}{sub:alm2map_der}] synthetize a map and its
derivatives from its $\alm$ (or $\almpix$).
  \end{sulist}
\end{related}

\rule{\hsize}{2mm}

\newpage
