
\sloppy


%%%\title{\healpix Fortran Subroutines Overview}
\docid{alms2fits*} \section[alms2fits*]{ }
\label{sub:alms2fits}
\docrv{Version 2.0}
\author{Frode K.~Hansen, Eric Hivon}
\abstract{This document describes the \healpix Fortran90 subroutine ALMS2FITS.}

\begin{facility}
{This routine stores  $a_{\ell m}$  values in a binary FITS file. Each FITS file
  extension created will contain one integer column with
  $index=\ell^2+\ell+m+1$, and 2 or 4 single (or double) precision columns with real/imaginary  $a_{\ell m}$  values and real/imaginary   standard deviation. One can store temperature $a_{\ell m}$ or temperature and polarisation, $a^T_{\ell m}$, $a^E_{\ell m}$ and $a^B_{\ell m}$. If temperature is specified, a FITS file with one extension is created. If polarisation is specified, a FITS file with 3 extensions one for each set of $a_{\ell m}$, $a_{\ell m}^T$, $a_{\ell m}^E$ and $a_{\ell m}^B$ is created.}
{\modFitstools}
\end{facility}

\begin{f90format}
{\mylink{sub:alms2fits:filename}{filename}%
, \mylink{sub:alms2fits:nalms}{nalms}%
, \mylink{sub:alms2fits:alms}{alms}%
, \mylink{sub:alms2fits:ncl}{ncl}%
, \mylink{sub:alms2fits:header}{header}%
, \mylink{sub:alms2fits:nlheader}{nlheader}%
, \mylink{sub:alms2fits:next}{next}%
}
\end{f90format}

\begin{arguments}
{
\begin{tabular}{p{0.4\hsize} p{0.05\hsize} p{0.05\hsize} p{0.40\hsize}} \hline  
\textbf{name~\&~dimensionality} & \textbf{kind} & \textbf{in/out} & \textbf{description} \\ \hline
                   &   &   &                           \\ %%% for presentation
filename\mytarget{sub:alms2fits:filename}(LEN=\filenamelen) & CHR & IN & filename for the FITS file to store the $a_{\ell m}$ in. \\
nalms\mytarget{sub:alms2fits:nalms} & I4B & IN & number of  $a_{\ell m}$  to store. \\
ncl\mytarget{sub:alms2fits:ncl} & I4B & IN & number of columns in the FITS file. If an standard deviation is given, this number is 5, otherwise it is 3. \\
next\mytarget{sub:alms2fits:next} & I4B & IN & the number of extensions. 1 for temperature only, 3
                   for temperature and polarisation. \\
\end{tabular}
\begin{tabular}{p{0.4\hsize} p{0.05\hsize} p{0.05\hsize} p{0.40\hsize}} \hline  
\textbf{name~\&~dimensionality} & \textbf{kind} & \textbf{in/out} & \textbf{description} \\ \hline
                   &   &   &                           \\ %%% for presentation
alms\mytarget{sub:alms2fits:alms}(1:nalms,1:ncl+1,1:next) & SP/ DP & IN & the $a_{\ell m}$ to write to the
                   file. alms(i,1,j) and alms(i,2,j) contain the $\ell$ and $m$
                   values for the ith  $a_{\ell m}$  (j=1,2,3 for
                   (T,E,B)). alms(i,3,j) and alms(i,4,j) contain the real and
                   imaginary value of the ith  $a_{\ell m}$. Finally, the standard
                   deviation for the ith  $a_{\ell m}$  is contained in alms(i,5,j)
                   (real) and alms(i,6,j) (imaginary). \\ 
nlheader\mytarget{sub:alms2fits:nlheader} & I4B & IN & number of header lines to write to the file. \\
header\mytarget{sub:alms2fits:header}(LEN=80) (1:nlheader, 1:next) & CHR & IN & the header to the FITS file. \\ 
\end{tabular}
}
\end{arguments}

\begin{example}
{
call alms2fits ('alms.fits', 65*66/2, alms, 3, header, 80, 3)  \\
}
{
Creates a FITS file with the $a_{\ell m}^T$, $a_{\ell m}^E$ and $a_{\ell m}^B$ values given in alms(1:65*66/2,1:4,1:3). The last index specifies (T,E,B). The second index gives l, m, real( $a_{\ell m}$ ), imaginary( $a_{\ell m}$ ) for each of the $a_{\ell m}$. The number 65*66/2 is the number of  $a_{\ell m}$  values up to an $\ell$ value of 64. 80 lines from header(1:80,1:3) is written to each extension.
}
\end{example}

\begin{modules}
  \begin{sulist}{} %%%% NOTE the ``extra'' brace here %%%%
  \item[write\_alms] routine called by \thedocid\ for each extension.
  \item[\textbf{fitstools}] module, containing:
  \item[printerror] routine for printing FITS error messages.
  \item[\textbf{cfitsio}] library for FITS file handling.		
  \end{sulist}
\end{modules}
\newpage
\begin{related}
  \begin{sulist}{} %%%% NOTE the ``extra'' brace here %%%%
  \item[\htmlref{fits2alms}{sub:fits2alms},
  \htmlref{read\_conbintab}{sub:read_conbintab}] routines to read $a_{\ell m}$ from
  a FITS file 
  \item[\htmlref{dump\_alms}{sub:dump_alms}] has the same function as \thedocid\ but with parameters passed differently.
  \end{sulist}
\end{related}

\rule{\hsize}{2mm}

\newpage
