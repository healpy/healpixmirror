
\sloppy


%%%\title{\healpix Fortran Subroutines Overview}
\docid{write\_fits\_partial} \section[write\_fits\_partial]{ }
\label{sub:write_fits_partial}
\docrv{Version 1.3}
\author{Eric Hivon \& Frode K.~Hansen}
\abstract{This document describes the \healpix Fortran90 subroutine WRITE\_FITS\_PARTIAL.}

\begin{facility}
{This routine writes unpolarised or polarised partial sky \healpix map into a FITS file.\\
For more information on the FITS file format supported in \healpix, 
including the one implemented in \thedocid,
see \url{\hpxfitsdoc}.}
{\modFitstools}
\end{facility}

\begin{f90format}
{\mylink{sub:write_fits_partial:filename}{filename}%
, \mylink{sub:write_fits_partial:pixel}{pixel}%
, \mylink{sub:write_fits_partial:cutmap}{cutmap}%
 \optional{[, \mylink{sub:write_fits_partial:header}{header}%
, \mylink{sub:write_fits_partial:coord}{coord}%
, \mylink{sub:write_fits_partial:nside}{nside}%
, \mylink{sub:write_fits_partial:order}{order}%
, \mylink{sub:write_fits_partial:units}{units}%
, \mylink{sub:write_fits_partial:extno}{extno}%
]}}
\end{f90format}
\aboutoptional

\begin{arguments}
{
% \begin{tabular}{p{0.3\hsize} p{0.05\hsize} p{0.05\hsize} p{0.5\hsize}} \hline  
% \textbf{name\&dimensionality} & \textbf{kind} & \textbf{in/out} & \textbf{description} \\ \hline
\begin{tabular}{p{0.30\hsize} p{0.05\hsize} p{0.08\hsize} p{0.49\hsize}} \hline  
\textbf{name~\&~dimensionality} & \textbf{kind} & \textbf{in/out} & \textbf{description} \\ \hline
                   &   &   &                           \\ %%% for presentation
filename\mytarget{sub:write_fits_partial:filename}(LEN=\mylink{sub:healpix_types:filenamelen}{\filenamelen}) & CHR & IN & FITS file in which the partial sky map will be written\\
pixel\mytarget{sub:write_fits_partial:pixel}(0:np-1)    & I4B/ I8B & IN & index of observed (or valid) pixels \\
cutmap\mytarget{sub:write_fits_partial:cutmap}(0:np-1,1:nc)    & SP/ DP & IN & value of polarised (if nc$=3$) or unpolarised (if nc$=1$) map value in each observed pixel\\
%
\optional{header\mytarget{sub:write_fits_partial:header}}(LEN=80)(1:) \ (OPTIONAL)    & CHR & IN &   FITS extension header to be included in the FITS file\\
\optional{coord\mytarget{sub:write_fits_partial:coord}}(LEN=1)       & CHR & IN &   astrophysical coordinates ('C' or 'Q'
                   Celestial/eQuatorial, 'G' for Galactic, 'E' for Ecliptic)\\
\optional{nside\mytarget{sub:write_fits_partial:nside}}    & I4B & IN &   \healpix resolution parameter of data set \\
\optional{order\mytarget{sub:write_fits_partial:order}}     & I4B & IN &   \healpix ordering scheme, 1: RING, 2: NESTED \\
%\optional{header}(LEN=80)    & CHR & IN &   FITS header to be included in the FITS file\\
\optional{units\mytarget{sub:write_fits_partial:units}}(LEN=20) & CHR & IN &  maps physical units (applies to all columns except PIXEL)\\
\optional{extno}\mytarget{sub:write_fits_partial:extno}     & I4B & IN & (0 based) extension number in which to write data. \default{0}.
	  If set to 0 (or not set) {\em a new file is written from scratch}.
	  If set to a value
		  larger than 1, the corresponding extension is added or
		  updated, as long as all previous extensions already exist.
		  All extensions of the same file should use the same Nside,
Order and Coord \\
% \optional{polarisaton}\mytarget{sub:write_fits_partial:polarisation} & I4B & IN & if set to a non zero value, specifies that file will contain the I, Q and U polarisation
%            Stokes parameter in extensions 0, 1 and 2 respectively, and sets the
% FITS header keywords accordingly. If not set, the keywords found in \mylink{sub:write_fits_partial:header}{\tt
% header} will prevail.\\
\  & \ & \ & Note: the information relative to Nside, Order and Coord {\em has} to be
                   given, either thru these keyword or via the FITS Header. \\
\end{tabular}
}
\end{arguments}

% \begin{example}
% {
% npix= write\_fits\_partial('map.fits', nmaps=nmaps, ordering=ordering,obs\_npix=obs\_npix, nside=nside, mlpol=mlpol, type=type, polarisation=polarisation)  \\
% }
% {
% Returns 1 or 3 in nmaps, dependent on wether 'map.fits' contain only
% temperature or both temperature and polarisation maps. The pixel ordering number is found by reading the keyword ORDERING in the FITS file. If this keyword does not exist, 0 is returned.
% }
% \end{example}
\newpage
\begin{modules}
  \begin{sulist}{} %%%% NOTE the ``extra'' brace here %%%%
  \item[\textbf{fitstools}] module, containing:
  \item[printerror] routine for printing FITS error messages.
  \item[\textbf{cfitsio}] library for FITS file handling.		
  \end{sulist}
\end{modules}

\begin{related}
  \begin{sulist}{} %%%% NOTE the ``extra'' brace here %%%%
  \item[anafast] executable that reads a \healpix map and analyses it. 
  \item[synfast] executable that generate full sky \healpix maps
  \item[\htmlref{getsize\_fits}{sub:getsize_fits}] routine to know the size of a FITS file and its type (eg, full sky vs cut sky)
  \item[\htmlref{input\_map}{sub:input_map}] all purpose routine to input a map of any kind from a FITS file
  \item[\htmlref{output\_map}{sub:output_map}] subroutine to write a FITS file from a \healpix map
  \item[\htmlref{read\_fits\_partial}{sub:read_fits_partial}] subroutine to read a \healpix partial sky map from a FITS file
  \item[\htmlref{write\_minimal\_header}{sub:write_minimal_header}] routine to write minimal FITS header
  \end{sulist}
\end{related}

\rule{\hsize}{2mm}

\newpage
