
\sloppy


\title{\healpix Fortran Facility User Guidelines}
\docid{hotspot} \section[hotspot]{\nosectionname}
\label{fac:hotspot}
\docrv{Version 1.1}
\author{Benjamin D.~Wandelt}
\abstract{This document describes the \healpix facility HOTSPOT.}

\begin{facility}
{This Fortran facility provides a means to find local extrema of
a map in \healpix format. It also serves to illustrate the use of the
following parts of the \healpix toolkit:
fast neighbour and extrema finding in the nested scheme,
in-place conversion between RING and NESTED pixel schemes } 
{src/f90/hotspot/HotSpots.F90}
\end{facility}

\begin{f90facility}
{hotspot}
\end{f90facility}

\begin{qualifiers}
  \begin{qulist}{} %%%% NOTE the ``extra'' brace here %%%%
    \item[{infile = }]\mytarget{fac:hotspot:infile}%
 Defines the input map file.
    \item[{extrema\_outfile = }]\mytarget{fac:hotspot:extrema_outfile}%
 Defines the output map file.
    \item[{maxima\_outfile = }]\mytarget{fac:hotspot:maxima_outfile}%
 Defines the output ascii list of maxima.
    \item[{minima\_outfile=  }] Defines the output ascii list of minima.
  \end{qulist}
\end{qualifiers}

\begin{codedescription}
{
 hotspot reads a healpix map in FITS format and generates the following outputs: 
1) a \healpix map in FITS format which is zero everywhere, except at
 pixels which contain local extrema. These pixels have the same
 values as in the input map.
2) an ASCII file which lists the pixel numbers and values of
 maxima, and
3) an ASCII file which lists the pixel numbers and values of
 minima.

The facility can be used in both an interactive
mode and a command mode, where command qualifiers
are fed to the facility using an input file.

Note the following limitations:
hotspot (and the toolkit neighbour finder which it uses) will
only work on maps with $N_{side}\ge 2$.
}
\end{codedescription}

\begin{datasets}
{
\begin{tabular}{p{0.3\hsize} p{0.35\hsize}} \hline  
  \textbf{Dataset} & \textbf{Description} \\ \hline
                   &                      \\ %%% for presentation
  None required & \\ 
                   &                      \\ \hline %%% for presentation
\end{tabular}
} 
\end{datasets}

\begin{support}
  \begin{sulist}{} %%%% NOTE the ``extra'' brace here %%%%
  \item[\htmlref{synfast}{fac:synfast}] This \healpix facility can generate a FITS format 
            sky map to be input to hotspot.
  \item[\htmlref{map2gif}{fac:map2gif}] This \healpix Fortran facility can be used to visualise the
  output map.
  \item[\htmlref{mollview}{idl:mollview}] This \healpix IDL facility can be used to visualise the
  output map.
  \end{sulist}
\end{support}

\begin{examples}{1}
{
\begin{tabular}{ll} %%%% use this tabular format %%%%
hotspot  \\
\end{tabular}
}
{
hotspot runs in interactive mode.
}
\end{examples}

\vfill\newpage

\begin{examples}{2}
{
\begin{tabular}{ll} %%%% use this tabular format %%%%
hotspot  filename \\
\end{tabular}
}
{When `filename' is present, hotspot enters the non-interactive mode and parses
its inputs from the file `filename'. This has the following
structure: the first entry is a qualifier which announces to the parser
which input immediately follows. If this input is omitted in the
input file, the parser assumes the default value shown below.
If the equality sign is omitted, then the parser ignores the entry.
In this way comments may also be included in the file.
In this example, the file contains the following qualifiers:\hfill\newline
\fileparam{\mylink{fac:hotspot:infile}{infile}%
 = map.fits}
\fileparam{\mylink{fac:hotspot:extrema_outfile}{extrema\_outfile}%
 = pixlminmax.fits}
\fileparam{\mylink{fac:hotspot:maxima_outfile}{maxima\_outfile}%
 = maxima.dat}
\fileparam{\mylink{fac:hotspot:minima_outfile}{minima\_outfile}%
 = minima.dat}
hotspot reads in the map `map.fits' and generates
an output map with name `pixlminmax.fits', and two ASCII files,
`maxima.dat' and
`minima.dat'.
}
\end{examples}

\begin{release}
  \begin{relist}
    \item Initial release (\healpix 0.90)
    \item Optional non-interactive operation. Proper FITS file support
    for input and output maps. (\healpix 1.00)
  \end{relist}
\end{release}
\newpage
\begin{messages}
{
\begin{tabular}{p{0.25\hsize} p{0.1\hsize} p{0.35\hsize}} \hline  
  \textbf{Message} & \textbf{Severity} & \textbf{Text} \\ \hline
                   &                   &   \\ %%% for presentation
can not allocate memory for array xxx &  Fatal & You do not have
                   sufficient system resources to run this
                   facility at the map resolution you required. 
  Try a lower map resolution.  \\ 
                   &                   &   \\ \hline %%% for presentation
\end{tabular}
} 
\end{messages}

\rule{\hsize}{2mm}

\newpage
