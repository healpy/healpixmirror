
\sloppy


\title{\healpix Fortran Facility User Guidelines}
\docid{smoothing} \section[smoothing]{\nosectionname}
\label{fac:smoothing}
\docrv{Version 1.1}
\author{Frode K.~Hansen}
\abstract{This document describes the \healpix facility SMOOTHING.}

\begin{facility}
{This program can be used to convolve a map with a gaussian beam. 
The input map can be given in RING or NESTED scheme and the smoothed map 
is written 
to a FITS file in the RING scheme.%
\\%
NOTE: This automated facility is susceptible to problems with non-commutativity
of discrete spherical harmonics transforms, described in the Recommendations
for Users of the \htmlref{anafast}{fac:anafast} facility. 
If very high accuracy of the results is
required in the spectral regime of $\ell > 2\cdot nsmax$, it is recommended
to choose an iterative computation of the $a_{\ell m}$ coefficients.
} 
{src/f90/smoothing/smoothing.f90}
\end{facility}

\begin{f90facility}
{smoothing [options] [parameter\_file]}
\end{f90facility}

%% \begin{options}
%%   \begin{optionlistwide}{} %%%% NOTE the ``extra'' brace here %%%%
%%     \item[{\tt -d}]
%%     \item[{\tt --double}] If present, all internal variables and arrays will be in double precision, and the output data will be written on disk at double precision (see Notes on I/O precision on page~\pageref{page:ioprec})
%%     \item[{\tt -s}]
%%     \item[{\tt --single}] If present, most internal variables and arrays will be in single precision, except for those involved in accuracy critical calculations such as spherical harmonics recursion, and the output data will be written on disk at single precision (default)
%%   \end{optionlistwide}
%% \end{options}

\begin{options}
  \begin{optionlistwide}{} %%%% NOTE the ``extra'' brace here %%%%
    \item[{\tt -d}]
    \item[{\tt -}{\tt -}{\tt double}] double precision mode (see Notes on double/single precision modes on page~\pageref{page:ioprec})
    \item[{\tt -s}]
    \item[{\tt -}{\tt -}{\tt single}] single precision mode (default)
  \end{optionlistwide}
\end{options}

\begin{qualifiers}
  \begin{qulist}{} %%%% NOTE the ``extra'' brace here %%%%
     \item[{simul\_type = }]\mytarget{fac:smoothing:simul_type}%
 Defines which map(s) to analyse, 1=temperature only, 2=temperature AND polarisation.
(default= 1)
    \item[{infile = }]\mytarget{fac:smoothing:infile}%
 Defines the filename for the FITS file containing the map to be smoothed. 
	(default= 'map.fits')
    \item[{nlmax = }]\mytarget{fac:smoothing:nlmax}%
 Defines the $\ell_{max}$ value for the application.
(default= 64)
    \item[{iter\_order = }]\mytarget{fac:smoothing:iter_order}%
 Defines the maximum order of quadrature 
      iteration to be used. (default=0, no iteration)
 \item[{fwhm\_arcmin = }]\mytarget{fac:smoothing:fwhm_arcmin}%
 Defines the FWHM in arcminutes of the gaussian 
beam for the convolution. (default=10)
    \item[{beam\_file = }] Defines the FITS file describing the
    Legendre window
    function of the circular beam to be used for the
    simulation. If set to an existing file name, it will override the
    {\tt fhwm\_arcmin} given above. default=`'
\item[{outfile = }]\mytarget{fac:smoothing:outfile}%
 Defines the filename for the file that will contain 
the smoothed map. (default='map\_smoothed.fits')
     \item[{plmfile = }] Defines the name for an input file
    containing  precomputed Legendre polynomials $P_{lm}$.
(default= no entry --- \thedocid executes the recursive evaluation 
of $P_{lm}$s)
\item[{w8file = }]\mytarget{fac:smoothing:w8file}%
 Defines name for an input file containing ring
  weights in the improved quadrature mode (default= no entry ---
the name is assumed to be 'weight\_ring\_n0xxxx.fits' where xxxx is nsmax)
\item[{w8filedir = }]\mytarget{fac:smoothing:w8filedir}%
 Gives the directory where the weight files are
to be found (default= no entry --- smoothing searches in the default
directories, see introduction)
\item[{won = }]\mytarget{fac:smoothing:won}%
 Set this to 1 if weight files are to be used,
otherwise set it to 0 (or 2). (default= 0)
  \end{qulist}
\end{qualifiers}

\begin{codedescription}
{
A FITS file containing a \healpix map in RING or NESTED scheme is read in.
The map is analysed and smoothed in fourier space with a  gaussian beam 
of a given FHWM. 
A new map is then synthesized using the smoothed $a_{\ell m}$ coeffecients. 
For a more accurate application, an iteration of arbitrary order can be applied. 
The output map is stored in {\em the same scheme} as the input map.
}
\end{codedescription}

\begin{datasets}
{
\begin{tabular}{p{0.3\hsize} p{0.35\hsize}} \hline  
  \textbf{Dataset} & \textbf{Description} \\ \hline
                   &                      \\ %%% for presentation
  data/weight\_ring\_n0xxxx.fits & Files containing ring weights
                   for the smoothing improved quadrature mode.\\ 
                   &                      \\ \hline %%% for presentation
\end{tabular}
} 
\end{datasets}

\begin{support}
  \begin{sulist}{} %%%% NOTE the ``extra'' brace here %%%%
  \item[\htmlref{generate\_beam}{sub:generate_beam}] This \healpix Fortran
subroutine generates or reads the $B(\ell)$ window function used in \thedocid
  \item[\htmlref{map2gif}{fac:map2gif}] This \healpix Fortran facility can be used to visualise the
  input and output maps of \thedocid.
  \item[\htmlref{mollview}{idl:mollview}] This \healpix IDL facility can be used to visualise the
  input and output maps of \thedocid.
  \item[\htmlref{synfast}{fac:synfast}] This \healpix facility can generate a map and also do the smoothing.
  \item[\htmlref{anafast}{fac:anafast}] This \healpix facility can analyse a smoothed map.		
  \end{sulist}
\end{support}

\begin{examples}{1}
{
\begin{tabular}{ll} %%%% use this tabular format %%%%
smoothing  \\
\end{tabular}
}
{
Smoothing runs in interactive mode, self-explanatory. 
}
\end{examples}

\vfill\newpage

\begin{examples}{2}
{
\begin{tabular}{ll} %%%% use this tabular format %%%%
smoothing  filename \\
\end{tabular}
}
{When `filename' is present, smoothing enters the non-interactive mode and parses
its inputs from the file `filename'. This has the following
structure: the first entry is a qualifier which announces to the parser
which input immediately follows. If this input is omitted in the
input file, the parser assumes the default value.
If the equality sign is omitted, then the parser ignores the entry.
In this way comments may also be included in the file.
In this example, the file contains the following
qualifiers:\hfill\newline
\fileparam{\mylink{fac:smoothing:simul_type}{simul\_type}%
 = 1}
\fileparam{\mylink{fac:smoothing:nlmax}{nlmax}%
 = 64}
\fileparam{\mylink{fac:smoothing:infile}{infile}%
 = map.fits}
\fileparam{\mylink{fac:smoothing:outfile}{outfile}%
 = map\_smoothed.fits}
\fileparam{\mylink{fac:smoothing:fwhm_arcmin}{fwhm\_arcmin}%
 = 10.}
\fileparam{\mylink{fac:smoothing:iter_order}{iter\_order}%
 = 1}
smoothes the \healpix temperature map contained in `map.fits' with 
a 10 arcmin FWHM beam. The resulting map is saved 
in `map\_smoothed.fits'. The map analysis/synthesis was carried 
out using fourier coeffecients up to an $\ell$ value of 64. A first
order iteration of the quadrature was performed.}

\end{examples}

\begin{release}
  \begin{relist}
    \item Initial release (\healpix 0.90)
    \item Extension to polarization and arbitrary {\em circular} beams (\healpix 1.20)
  \end{relist}
\end{release}

\newpage
\begin{messages}
{
\begin{tabular}{p{0.25\hsize} p{0.1\hsize} p{0.35\hsize}} \hline  
  \textbf{Message} & \textbf{Severity} & \textbf{Text} \\ \hline
                   &                   &   \\ %%% for presentation
can not allocate memory for array xxx &  Fatal & You do not have
                   sufficient system resources to run this
                   facility at the map resolution you required. 
  Try a lower map resolution.  \\ 
                   &                   &   \\ \hline %%% for presentation
\end{tabular}
} 
\end{messages}

\rule{\hsize}{2mm}

\newpage
