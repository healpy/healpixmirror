% -*- LaTeX -*-

% PLEASE USE THIS FILE AS A TEMPLATE FOR THE DOCUMENTATION OF YOUR OWN
% FACILITIES: IN PARTICULAR, IT IS IMPORTANT TO NOTE COMMENTS MADE IN
% THE TEXT AND TO FOLLOW THIS ORDERING. THE FORMAT FOLLOWS ONE USED BY
% THE COBE-DMR PROJECT.	
% A.J. Banday, April 1999.




\renewcommand{\facname}{{read\_fits\_s }}
\renewcommand{\FACNAME}{{READ\_FITS\_S }}
\sloppy



\title{\healpix IDL Facility User Guidelines}
\docid{\facname} \section[\facname]{ }
\label{idl:read_fits_s}
\docrv{Version 1.1}
\author{Eric Hivon}
\abstract{This document describes the \healpix facility \facname.}




\begin{facility}
{This IDL facility reads a FITS file into an IDL structure.}
{src/idl/fits/read\_fits\_s.pro}
\end{facility}

\begin{IDLformat}
{\FACNAME, \mylink{idl:read_fits_s:File}{File}%
, \mylink{idl:read_fits_s:Prim_stc}{Prim\_stc}%
, [\mylink{idl:read_fits_s:Xten_stc}{Xten\_stc}%
, \mylink{idl:read_fits_s:COLUMNS}{COLUMNS=}%
, \mylink{idl:read_fits_s:EXTENSION}{EXTENSION=}%
, \mylink{idl:read_fits_s:HELP}{/HELP}%
, \mylink{idl:read_fits_s:MERGE}{/MERGE}%
]
}
\end{IDLformat}

\begin{qualifiers}
  \begin{qulist}{} %%%% NOTE the ``extra'' brace here %%%%
 	\item[{File}]  \mytarget{idl:read_fits_s:File}
          name of a FITS file containing 
               the healpix map(s) in an extension or in the image field 

 	\item[{Prim\_stc}]   \mytarget{idl:read_fits_s:Prim_stc}
	variable containing on output an IDL structure with the following fields: \\
		- primary header (tag : 0, tag name : HDR) \\
		- primary image (if any, tag : 1, tag name : IMG)

       \item[{Xten\_stc}]  \mytarget{idl:read_fits_s:Xten_stc}
		  (optional), \\
	variable containing on output an IDL structure with the following fields: \\
		- extension header (tag : 0, tag name : HDR) \\
		- data column 1 (if any, tag : 1, tag name given by TTYPE1 (with all
                    spaces removed and only letters, digits and underscore) \\
		- data column 2 (if any, tag : 2, tag name given by TTYPE2)  \\
		...

      \item[{COLUMNS=}]  \mytarget{idl:read_fits_s:COLUMNS}
		(optional), \\
        list of columns to be read from a binary table 
        can be a list of integer (1 based) indexing the columns positions
        or a list of names matching the TTYPE* of the columns
        by default, all columns are read

       \item[{EXTENSION=}]  \mytarget{idl:read_fits_s:EXTENSION}
		(optional), \\
	extension unit to be read from FITS file: 
 either its 0-based ID number (ie, 0 for first extension {\em after} primary array) 
 or the case-insensitive value of its EXTNAME keyword.
	\default 0
  \end{qulist}
\end{qualifiers}

\begin{keywords}
  \begin{kwlist}{} %%% extra brace
	\item[{/HELP}]   \mytarget{idl:read_fits_s:HELP} if set, an extensive help is displayed and no
	file is read
	\item[{/MERGE}]  \mytarget{idl:read_fits_s:MERGE}
	if set {\tt Prim\_stc} contains : \\
		- the concatenated primary and extension header (tag name : HDR) \\
		- primary image (if any, tag name : IMG) \\
		- data column 1 ... \\
	and {\tt Exten\_stc} is set to 0 \\
	\default : not set (or set to 0)
   \end{kwlist}
\end{keywords}

\begin{codedescription}
{\thedocid{} reads in any type of FITS file (Image, Binary table or Ascii table) 
and outputs the data in IDL structures}
\end{codedescription}



\begin{related}
  \begin{sulist}{} %%%% NOTE the ``extra'' brace here %%%%
  \item[idl] version \idlversion or more is necessary to run \thedocid
  \item[synfast] This \healpix facility will generate the FITS format 
            sky map that can be read by \thedocid.
  \item[%
\htmlref{read\_fits\_cut4}{idl:read_fits_cut4},
\htmlref{read\_fits\_map}{idl:read_fits_map}]
  \item[%
\htmlref{read\_tqu}{idl:read_tqu},
\htmlref{read\_fits\_s}{idl:read_fits_s}]
\healpix IDL routines to read cut-sky maps, full-sky maps, polarized full-sky maps and
arbitrary data sets from FITS files

  \item[\htmlref{write\_fits\_sb}{idl:write_fits_sb}] This \healpix IDL facility can be used to generate FITS format 
            sky maps readable by \thedocid.
  \end{sulist}
\end{related}


\begin{example}
{
\begin{tabular}{l} %%%% use this tabular format %%%%
\thedocid,  'dmr\_skymap\_90a\_4yr.fits', pdata, xdata \\
\end{tabular}
}
{\thedocid{} reads in the file 'dmr\_skymap\_90a\_4yr.fits'. On output, pdata
contains the primary header and xdata is a structure whose first field is the
extension header, and the other fields are vectors with respective tag names 
PIXEL, SIGNAL, N\_OBS, SERROR, ... (see {\tt help,/struc,xdata})}
\end{example}

