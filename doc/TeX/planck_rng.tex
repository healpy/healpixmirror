
\sloppy


%%%\title{\healpix Fortran Subroutines Overview}
\docid{planck\_rng} \section[planck\_rng derived type]{ }
\label{sub:planck_rng}
\docrv{Version 2.0}
\author{Eric Hivon}
\abstract{This document describes the \healpix Fortran90 type \thedocid.}

\begin{facility}
{The derived type \thedocid\ is used by the Random Number Generation routines 
\htmlref{rand\_init}{sub:rand_init},
\htmlref{rand\_uni}{sub:rand_uni}, 
\htmlref{rand\_gauss}{sub:rand_gauss} to describe fully the current RNG
sequence.\\
Most users do not need to know about the \thedocid\ definition. It may be
useful for those wanting to take a snapshot of the RNG sequence they are using (by eg,
dumping the latest values of \thedocid\ structure on disk) so that the same sequence can be resumed
later on from that same point.}
{\modRngmod}
\end{facility}


% %---------------------
% \newenvironment{mytable}[1]{%
% \begin{minipage}[b]{\linewidth}{%
% % \renewcommand{\thefootnote}{\fnsymbol{footnote}}
% % \renewcommand{\footnoterule}{}
% {#1}
% }%
% \end{minipage}
% }
%---------------------

The type \thedocid\ is a structure defined as

\begin{mytable}{%
\begin{tabularx}{\linewidth}{lcX}
name & type  & definition \\
\hline
x, y, z, w & I4B & internal variables of uniform RNG\\
gset & DP & internal variable for Gaussian RNG\\
empty & LGT & flag used by Gaussian RNG\\
\hline
\end{tabularx}
}%
\end{mytable}



% \begin{example}
% {
% use planck\_rng \\
% real(kind=DP) :: dx \\
% print*,' pi = ',PI
% }
% {
% The value of {\tt PI}, as well as all other \thedocid\ parameters are made known
% to the code
% }
% \end{example}

\begin{related}
  \begin{sulist}{} %%%% NOTE the ``extra'' brace here %%%%
  \item[\htmlref{rand\_gauss}{sub:rand_gauss}] function which returns a  random normal deviate.
  \item[\htmlref{rand\_uni}{sub:rand_uni}] function which returns a random uniform deviate.
   \item[\htmlref{rand\_init}{sub:rand_init}] subroutine to initiate a random number sequence. 
  \end{sulist}
\end{related}

\rule{\hsize}{2mm}

\newpage
