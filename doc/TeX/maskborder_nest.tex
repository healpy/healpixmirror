\sloppy
\docid{maskborder\_nest}\section[maskborder\_nest]{ }
\label{sub:maskborder_nest}
\docrv{Version 1.0}
\author{Eric Hivon}
\abstract{This document describes the \healpix Fortran90 subroutine MASKBORDER\_NEST.}

\begin{facility}
{For a input binary mask in NESTED ordering, \thedocid\ identifies the pixels
located on the inner boundary of {\em invalid} regions
}
{\modMaskTools}
\end{facility}

\begin{f90format}
{\mylink{sub:maskborder_nest:nside}{nside}%
, \mylink{sub:maskborder_nest:mask_in}{mask\_in}%
, \mylink{sub:maskborder_nest:mask_out}{mask\_out}%
, \mylink{sub:maskborder_nest:nbordpix}{nbordpix}%
, \optional{[ \mylink{sub:maskborder_nest:border_value}{border\_value}%
]}}
\end{f90format}
\aboutoptional

\begin{arguments}
{
\begin{tabular}{p{0.35\hsize} p{0.05\hsize} p{0.1\hsize} p{0.40\hsize}} \hline  
\textbf{name~\&~dimensionality} & \textbf{kind} & \textbf{in/out} & \textbf{description} \\ \hline
                   &   &   &                           \\ %%% for presentation
nside\mytarget{sub:maskborder_nest:nside} & I4B & IN & The $\nside$ value of the input mask. \\
mask\_in\mytarget{sub:maskborder_nest:mask_in}(0:Npix-1) & I4B & IN & Input binary NESTED-ordered mask. Npix = 12*nside*nside\\
mask\_out\mytarget{sub:maskborder_nest:mask_out}(0:Npix-1) &I4B & OUT & Output NESTED-ordered mask, in which inner border
pixels (defined as 0-valued and adjacent to 1-valued pixels) take the value 2
(or {\tt border\_value}). Can be the same
array as mask\_in.\\
nbordpix\mytarget{sub:maskborder_nest:nbordpix} & I4B & OUT & Number of border pixels found\\
\optional{border\_value\mytarget{sub:maskborder_nest:border_value}} & I4B & IN & value to be given to border pixels in
output mask. \default{2}.
\end{tabular}
}
\end{arguments}

\begin{example}
{
use healpix\_types \\
use healpix\_modules \\
% use pix\_tools, only : nside2npix \\
% use alm\_tools, only : maskborder\_nest \\
% integer(I4B) :: nside, lmax, mmax, npix, spin\\
% real(SP), dimension(:,:), allocatable :: map \\
% complex(SPC), dimension(:,:,:), allocatable :: alm \\
% \ldots \\
% nside=256 ; lmax=512 ; mmax=lmax ; spin=4\\
% npix=nside2npix(nside)\\
% allocate(alm(1:2,0:lmax,0:mmax))\\
% allocate(map(0:npix-1,1:2))\\
\ldots \\
call \thedocid(nside, mask\_in, mask\_in, nbordpix)  \\
}
{For a binary input mask {\tt mask\_in}, it will look for border pixels and output
their number in {\tt nborpix}. In this example the {\tt mask\_in} will be
modified so that border pixels take value 2 on output.
}
\end{example}

\begin{modules}
  \begin{sulist}{} %%%% NOTE the ``extra'' brace here %%%%
  \item[\textbf{mask\_tools}] mask processing module (see related routines below)
  \end{sulist}
\end{modules}

\begin{related}
  \begin{sulist}{} %%%% NOTE the ``extra'' brace here %%%%
	\maskToolsRelated
  \end{sulist}
\end{related}

\rule{\hsize}{2mm}

\newpage
