% -*- LaTeX -*-


\renewcommand{\facname}{{query\_triangle }}
\renewcommand{\FACNAME}{{QUERY\_TRIANGLE }}
\sloppy



\title{\healpix IDL Facility User Guidelines}
\docid{\facname} \section[\facname]{ }
\label{idl:query_triangle}
\docrv{Version 1.2}
\author{Eric Hivon}
\abstract{This document describes the \healpix IDL facility \facname.}




\begin{facility}
{This IDL facility provides a means to find the index of all pixels belonging to
a sperical triangle defined by its vertices
}
{src/idl/toolkit/query\_triangle.pro}
\end{facility}

\begin{IDLformat}
{\facname, Nside, Vector1, Vector2, Vector3, Listpix, [Nlist, NESTED=, INCLUSIVE=]}
\end{IDLformat}

\begin{qualifiers}
  \begin{qulist}{} %%%% NOTE the ``extra'' brace here %%%%
    \item[Nside] \healpix resolution parameter used to index the pixel list (scalar integer)
    \item[Vector1] 3D cartesian position vector of the triangle first  vertex
    \item[Vector2] 3D cartesian position vector of the triangle second vertex
    \item[Vector3] 3D cartesian position vector of the triangle third  vertex
          NB : the norm of Vector* does not have to be one, what is
          considered is the intersection of the sphere with the line of
          direction Vector*.
    \item[Listpix] on output: list of ordered index for the pixels found 
    in the triangle. The RING numbering scheme is used unless the keyword {\tt NESTED} is set.
     (=-1 if the triangle is too small and no pixel is found)
    \item[Nlist] on output: number of pixels in Listpix (=0 if no pixel is found).
  \end{qulist}
\end{qualifiers}

\begin{keywords}
  \begin{kwlist}{} %%% extra brace
    \item[NESTED =] if set, the output list uses the NESTED numbering scheme
    instead of the default RING
    \item[INCLUSIVE = ] if set, all the pixels overlapping (even partially)
                   with the triangle are listed, otherwise only those whose
                   center lies within the triangle are listed
  \end{kwlist}
\end{keywords}  

\begin{codedescription}
{\facname finds the pixels within the given triangle in a selective way WITHOUT
scanning all the sky pixels. The numbering scheme of the output list and the
inclusiveness of the triangle can be changed}
\end{codedescription}



\begin{related}
  \begin{sulist}{} %%%% NOTE the ``extra'' brace here %%%%
    \item[idl] version \idlversion or more is necessary to run \facname.
    \item[ang2pix, pix2ang] conversion between angles and pixel index
    \item[vec2pix, pix2vec] conversion between vector and pixel index
    \item[\htmlref{query\_disc}{idl:query_disc}, \htmlref{query\_polygon}{idl:query_polygon},]
    \item[\htmlref{query\_strip}{idl:query_strip}, \htmlref{query\_triangle}{idl:query_triangle}] render the list of pixels enclosed
  respectively in a given disc, polygon, latitude strip and triangle
  \end{sulist}
\end{related}

\begin{example}
{
\begin{tabular}{l} %%%% use this tabular format %%%%
\facname, 256L, [1,0,0],[0,1,0],[0,0,1], listpix, nlist
\end{tabular}
}
{
On return listpix contains the index of the (98560) pixels lying in the octant
$(x>0,y>0,y>0)$.
The pixel indices correspond to the RING scheme with resolution 256.
}
\end{example}


