\sloppy

%%%\title{\healpix Fortran Subroutines Overview}
\docid{getEnvironment} \section[getEnvironment]{ }
\label{sub:getenvironment}
\docrv{Version 1.2}
\author{Eric Hivon}
\abstract{This document describes the \healpix Fortran90 subroutine
getEnvironment.}

\begin{facility}
{This subroutine emulates the C routine {\tt getenv}, which returns the value of
an environment variable.\\
Starting with release 3.60, it calls the F2003 extension subroutine \texttt{get\_environment\_variable}.}
{\modExtension}
\end{facility}

\begin{f90format}
{\mylink{sub:getenvironment:name}{name}%
, \mylink{sub:getenvironment:value}{value}%
}
\end{f90format}

\begin{arguments}
{
\begin{tabular}{p{0.3\hsize} p{0.05\hsize} p{0.1\hsize} p{0.45\hsize}} \hline  
\textbf{name~\&~dimensionality} & \textbf{kind} & \textbf{in/out} & \textbf{description} \\ \hline
                   &   &   &                           \\ %%% for presentation
name\mytarget{sub:getenvironment:name} & CHR & IN & name of the environment variable \\
value\mytarget{sub:getenvironment:value} & CHR & OUT & value of the environment variable 
\end{tabular}}
\end{arguments}

\begin{example}
{
use extension \\
character(len=128) :: healpixdir \\
call \thedocid('HEALPIX', healpixdir) \\
print*,healpixdir
}
{
Will return the value of the {\tt \$HEALPIX} system variable (if it is defined)
}
\end{example}

\begin{related}
  \begin{sulist}{} %%%% NOTE the ``extra'' brace here %%%%
  \item[\htmlref{getArgument}{sub:getargument}] returns list of command line arguments
  \item[\htmlref{nArguments}{sub:narguments}] returns number of command line arguments
  \end{sulist}
\end{related}

\rule{\hsize}{2mm}

\newpage
