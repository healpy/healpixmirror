
\sloppy

\title{\healpix Fortran Subroutines Overview}
\docid{merge\_headers} \section[merge\_headers]{ }
\label{sub:merge_headers}
\docrv{Version 1.2}
\author{Frode K.~Hansen, Eric Hivon}
\abstract{This document describes the \healpix Fortran90 subroutine MERGE\_HEADERS.}

\begin{facility}
{This routine merges two FITS headers.}
{\modHeadFits}
\end{facility}

\begin{f90format}
{header1, header2}
\end{f90format}

\begin{arguments}
{
\begin{tabular}{p{0.4\hsize} p{0.05\hsize} p{0.1\hsize} p{0.35\hsize}} \hline  
\textbf{name\&dimensionality} & \textbf{kind} & \textbf{in/out} & \textbf{description} \\ \hline
                   &   &   &                           \\ %%% for presentation
header1(LEN=80) DIMENSION(:) & CHR & IN & First header. \\
header2(LEN=80) DIMENSION(:) & CHR & INOUT & Second header. On output,
                   will contain the concatenation of (in that order) header2 and
                   header1. If header2 is too short to allow the
                   merging the output will be truncated\\
\end{tabular}
}
\end{arguments}

\begin{example}
{
call merge\_headers(header1, header2)  \\
}
{
On output header2 will contain the original header2, followed by the
content of header1
}
\end{example}

\begin{modules}
  \begin{sulist}{} %%%% NOTE the ``extra'' brace here %%%%
  \item[write\_hl] more general routine for adding a keyword to a header.
  \item[\textbf{cfitsio}] library for FITS file handling.		
  \end{sulist}
\end{modules}

\begin{related}
  \begin{sulist}{} %%%% NOTE the ``extra'' brace here %%%%
  \item[\htmlref{add\_card}{sub:add_card}] general purpose routine to write any keywords into a FITS
  file header
  \item[\htmlref{get\_card}{sub:get_card}] general purpose routine to read any keywords from a header in a FITS file.
  \item[\htmlref{del\_card}{sub:del_card}] routine to discard a keyword from a FITS header
  \item[\htmlref{read\_par}{sub:read_par}, \htmlref{number\_of\_alms}{sub:number_of_alms}] routines to read specific keywords from a
  header in a FITS file.
  \item[\htmlref{getsize\_fits}{sub:getsize_fits}] function returning the size of the data set in a fits
  file and reading some other useful FITS keywords
  \end{sulist}
\end{related}

\rule{\hsize}{2mm}

\newpage
