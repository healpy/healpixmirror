
\sloppy


\title{\healpix Fortran Subroutines Overview}
\docid{getsize\_fits} \section[getsize\_fits]{ }
\label{sub:getsize_fits}
\docrv{Version 2.0}
\author{Eric Hivon \& Frode K.~Hansen}
\abstract{This document describes the \healpix Fortran90 subroutine GETSIZE\_FITS.}

\begin{facility}
{This routine reads the number of maps and/or the pixel ordering of a FITS file containing a \healpix map.}
{src/f90/mod/fitstools.F90}
\end{facility}

\begin{f90function}
{filename \optional{[, nmaps, ordering, obs\_npix, nside, mlpol, type, polarisation,
    fwhm\_arcmin, beam\_leg, coordsys, polcconv, extno]}}
\end{f90function}
\aboutoptional

\begin{arguments}
{
\begin{tabular}{p{0.3\hsize} p{0.05\hsize} p{0.05\hsize} p{0.5\hsize}} \hline  
\textbf{name~\&~dimensionality} & \textbf{kind} & \textbf{in/out} & \textbf{description} \\ \hline
                   &   &   &                           \\ %%% for presentation
var & I8B & OUT & number of pixels or time samples in the chosen extension of
                   the FITS file \\
filename(LEN=*) & CHR & IN & filename of the FITS-file containing \healpix map(s). \\
\optional{nmaps} (OPTIONAL) & I4B & OUT & number of maps in the extension. \\
\optional{ordering} (OPTIONAL) & I4B & OUT & pixel ordering, 0=unknown, 1=RING, 2=NESTED \\
\optional{obs\_npix} (OPTIONAL) & I4B & OUT & number of non blanck pixels. It is set to -1 if it can not be determined from header
information alone\\
\optional{nside} (OPTIONAL)  & I4B & OUT & Healpix resolution parameter Nside. Returns a negative value if not found.  \\
\optional{mlpol} (OPTIONAL)  & I4B & OUT & maximum multipole used to generate the map
                   (for simulated map). Returns a negative value if not found.\\
\optional{type} (OPTIONAL)  & I4B & OUT & 
             \parbox[t]{\hsize}{Healpix/FITS file type\\
             $<$0 : file not found, or not valid\\
             0  : image only fits file, deprecated Healpix format
                   (var = 12 * nside * nside) \\
             1  : ascii table, generally used for C(l) storage \\
             2  : binary table : with implicit pixel indexing (full sky)
                   (var = 12 * nside * nside) \\
             3  : binary table : with explicit pixel indexing (generally cut sky)
                   (var $\le$ 12 * nside * nside) \\
           999  : unable to determine the type }\\
\optional{polarisation} (OPTIONAL)  & I4B & OUT & 
		\parbox[t]{\hsize}{presence of polarisation data in the file\\
             $<$0 : can not find out\\
              0 : no polarisation\\
              1 : contains polarisation (Q,U or G,C)} \\
\optional{fwhm\_arcmin} (OPTIONAL) & DP & OUT & returns the beam FWHM read from FITS header, 
                            translated from Deg (hopefully) to arcmin.
                         Returns a negative value if not found. \\
\optional{beam\_leg}(LEN=*) (OPTIONAL) & CHR & OUT & filename of beam or
             filtering window function applied to data
	     (FITS keyword BEAM\_LEG). Returns a empty string if not found. \\
\optional{coordsys}(LEN=20) (OPTIONAL) & CHR & OUT & string describing the pixelisation
                   astrophysical coordinates. 
		'G' = Galactic, 'E' = ecliptic, 'C' = celestial = equatorial. 
		Returns a empty string if not found. \\
\optional{polcconv} (OPTIONAL) & I4B & OUT & polarisation coordinate convention (see
             Healpix primer for details) 0=unknown, 1=COSMO, 2=IAU \\
\optional{extno} (OPTIONAL)  & I4B & IN & extension number (0 based) for which information
             is provided. Default = 0 (first extension). 
\end{tabular}
}
\end{arguments}

\newpage
\begin{example}
{
npix= getsize\_fits('map.fits', nmaps=nmaps, ordering=ordering, obs\_npix=obs\_npix, nside=nside, mlpol=mlpol, type=type, polarisation=polarisation)  \\
}
{
Returns 1 or 3 in nmaps, dependent on wether 'map.fits' contain only
temperature or both temperature and polarisation maps. The pixel ordering number is found by reading the keyword ORDERING in the FITS file. If this keyword does not exist, 0 is returned.
}
\end{example}
\begin{modules}
  \begin{sulist}{} %%%% NOTE the ``extra'' brace here %%%%
  \item[\textbf{fitstools}] module, containing:
  \item[printerror] routine for printing FITS error messages.
  \item[\textbf{cfitsio}] library for FITS file handling.		
  \end{sulist}
\end{modules}

\begin{related}
  \begin{sulist}{} %%%% NOTE the ``extra'' brace here %%%%
  \item[\htmlref{getnumext\_fits}{sub:getnumext_fits}] routine returning the number of extension in a FITS
  file
  \item[\htmlref{input\_map}{sub:input_map}] routine to read a \healpix FITS file
%  \item[anafast] executable that reads a \healpix map and analyses it. 
  \end{sulist}
\end{related}

\rule{\hsize}{2mm}

\newpage
