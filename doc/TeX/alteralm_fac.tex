
\sloppy


\title{\healpix Fortran Facility User Guidelines}
\docid{alteralm} \section[alteralm]{\nosectionname}
\label{fac:alteralm}
\docrv{Version 1.2}
\author{Eric Hivon}
\abstract{This document describes the \healpix facility ALTERALM.}

\begin{facility}
{This program can be used to modify a set of $a_{\ell m}$ spherical harmonics
  coefficients, as those extracted by \htmlref{anafast}{fac:anafast} or 
  simulated by \htmlref{synfast}{fac:synfast}, before
  they are used as constraints on a synfast run. Currently the alterations
  possible are %
\begin{itemize}
    \item rotation (using Wigner matrices) of the $a_{\ell m}$ from the input
    coordinate system to any other standard astrophysical coordinate system. The
    resulting $a_{\ell m}$ can be used with e.g. synfast to generate a map in the
    new coordinate system.
    \item removal of the pixel and beam window functions of the input
  $a_{\ell m}$ (corresponding to the pixel size and beam shape of the map from which
  they were extracted) and implementation of an arbitrary pixel and beam window
  function.
 \begin{equation} a_{\ell m}^{\rm OUT} = a_{\ell m}^{\rm IN} \frac{B^{\rm OUT}(\ell) P^{\rm 
 OUT}(\ell)}{B^{\rm IN}(\ell) P^{\rm IN}(\ell)}, \label{eq:alteralm} \end{equation}
where $P(\ell)$ is the pixel window function, and $B(\ell)$ is the beam window
 function (assuming a circular beam) or any other $\ell$ space filter (eg,
 Wiener filter). For an infinitely small pixel (or beam) one would have $P(\ell) =
 1$ (resp. $B(\ell) = 1$) for any $\ell$.
\end{itemize}%
}%
{src/f90/alteralm/alteralm.f90}
\end{facility}

\begin{f90facility}
{alteralm [options] [parameter\_file]}
\end{f90facility}

\begin{options}
  \begin{optionlistwide}{} %%%% NOTE the ''extra'' brace here %%%%
    \item[{\tt -d}]
    \item[{\tt -}{\tt -}{\tt double}] double precision mode (see 
  \htmlref{Notes on double/single precision modes}{fac:subsec:ioprec}%
\latexhtml{ on page~\pageref{page:ioprec}}{})
    \item[{\tt -s}]
    \item[{\tt -}{\tt -}{\tt single}] single precision mode (default)
  \end{optionlistwide}
\end{options}

\begin{qualifiers}
  \begin{qulist}{} %%%% NOTE the ''extra'' brace here %%%%
    \item[{infile\_alms = }]\mytarget{fac:alteralm:infile_alms}%
 Defines the FITS file from which to read the input
	$a_{\ell m}$.
    \item[{outfile\_alms = }]\mytarget{fac:alteralm:outfile_alms}%
 Defines the FITS file in which to write the altered
	$a_{\ell m}$.
    \item[{fwhm\_arcmin\_in = }]\mytarget{fac:alteralm:fwhm_arcmin_in}%
 Defines the FWHM size in arcminutes 
      of the Gaussian beam present in the input $a_{\ell m}$. The output $a_{\ell m}$ will be
      corrected from it, see Eq.~(\ref{eq:alteralm}). (default= value of FWHM keyword in {\tt infile\_alms}).
    \item[{beam\_file\_in = }]\mytarget{fac:alteralm:beam_file_in}%
 Defines the FITS file (see \htmlref{''Beam window function files'' in introduction}{fac:subsec:beamfiles}) describing the
      Legendre window function of the circular beam present in the input $a_{\ell m}$. The output $a_{\ell m}$ will be
      corrected from it, see Eq.~(\ref{eq:alteralm}). If set to an existing file name, it will override the
    {\tt fhwm\_arcmin\_in} given above. (default= value of the BEAM\_LEG keyword in {\tt infile\_alms})
     \item[{nlmax\_out = }]\mytarget{fac:alteralm:nlmax_out}%
 Defines the maximum $\ell$ value 
       to be used for the output $a_{\ell m}$s. (default= maximum $\ell$ of input
       $a_{\ell m}$ = value of MAX-LPOL keyword in {\tt infile\_alms}).
     \item[{nsmax\_in = }]\mytarget{fac:alteralm:nsmax_in}%
 If it can not be determined from the input file {\tt infile\_alms}, asks
       for the \healpix resolution parameter $\nside$ whose
       window function is applied to the input $a_{\ell m}$
     \item[{nsmax\_out = }]\mytarget{fac:alteralm:nsmax_out}%
 Defines the \healpix resolution parameter $\nside$ whose
       window function will be applied to the output $a_{\ell m}$. Could be set
       to 0 for infinitely small pixels, ie no pixel window function (default= same as input's $\nside$).
    \item[{fwhm\_arcmin\_out = }]\mytarget{fac:alteralm:fwhm_arcmin_out}%
 Defines the FWHM size in arcminutes 
      of the Gaussian beam to be applied to $a_{\ell m}$, see
      Eq.~(\ref{eq:alteralm}). (default= {\tt fwhm\_arcmin\_in}).
    \item[{beam\_file\_out = }] Defines the FITS file 
(see \htmlref{''Beam window function files'' in introduction}{fac:subsec:beamfiles}) describing the
      Legendre window function of the circular beam to be applied $a_{\ell m}$. If
      set to an existing file name, it will override the 
    {\tt fhwm\_arcmin\_out} given above. (default= '' '')
    \item[{coord\_in = }]\mytarget{fac:alteralm:coord_in}%
 Defines astrophysical coordinates used to compute the
    input $a_{\ell m}$. Valid choices are 'G' = Galactic, 'E' = Ecliptic, 
    'C'/'Q' = Celestial = eQuatorial. (default = value of COORDSYS keyword read
    from input FITS file)
    \item[{epoch\_in = }]\mytarget{fac:alteralm:epoch_in}%
 Defines astronomical epoch of input coordinate system (default=2000)
    \item[{coord\_out = }]\mytarget{fac:alteralm:coord_out}%
 Defines astrophysical coordinates into which to rotate
    the $a_{\ell m}$ (default = {\tt coord\_in})
    \item[{epoch\_out = }]\mytarget{fac:alteralm:epoch_out}%
 Defines astronomical epoch of output coordinate system
    (default={\tt epoch\_in})
     \item[{windowfile\_in = }]\mytarget{fac:alteralm:windowfile_in}%
	Defines the input filename from which to read the pixel window function parameterized by {\tt nsmax\_in}
(default= pixel\_window\_n????.fits, see 
\htmlref{Notes on default files and directories}{fac:subsec:defdir}%
\latexhtml{ on page~\pageref{page:defdir}}{})
     \item[{winfiledir\_in = }]\mytarget{fac:alteralm:winfiledir_in}%
	Defines the directory in which {\tt windowfile\_in}
    is located (default : see above).
     \item[{windowfile\_out = }]\mytarget{fac:alteralm:windowfile_out}%
	Defines the input filename from which to read the pixel window function parameterized by {\tt nsmax\_out}
(default= pixel\_window\_n????.fits, see above)
     \item[{winfiledir\_out = }]\mytarget{fac:alteralm:winfiledir_out}%
	Defines the directory in which {\tt windowfile\_out}
    is located (default : see above).

  \end{qulist}
\end{qualifiers}

\begin{codedescription}
{%
Alteralm can modify temperature as well as polarisation $a_{\ell m}$. It will also
modify the error on the $a_{\ell m}$ if those are provided. It works best if the
input FITS file contains the relevant information on the beam size and shape,
maximum multipoles, ...
}
\end{codedescription}

\begin{datasets}
{
\begin{tabular}{p{0.3\hsize} p{0.35\hsize}} \hline  
  \textbf{Dataset} & \textbf{Description} \\ \hline
                   &                      \\ %%% for presentation
  /data/pixel\_window\_nxxxx.fits & Files containing pixel windows for
                   various nsmax.\\ 
                   &                      \\ \hline %%% for presentation
\end{tabular}
} 
\end{datasets}

\begin{support}
  \begin{sulist}{} %%%% NOTE the ''extra'' brace here %%%%
  \item[\htmlref{generate\_beam}{sub:generate_beam}] This \healpix Fortran
subroutine generates or reads the $B(\ell)$ window function(s) used in \thedocid
  \item[\htmlref{anafast}{fac:anafast}] This \healpix Fortran facility can
     	       analyse a \healpix map to extract the $a_{\ell m}$ that can be
     	       altered by \thedocid.
  \item[\htmlref{synfast}{fac:synfast}] This \healpix facility can generate a
  \healpix map from a power spectrum $C_\ell$, with the possibility of including
  constraining $a_{\ell m}$ as those obtained with \thedocid.
		
  \end{sulist}
\end{support}

\begin{examples}{1}
{
\begin{tabular}{ll} %%%% use this tabular format %%%%
alteralm  \\
\end{tabular}
}
{
Alteralm runs in interactive mode, self-explanatory.
}
\end{examples}

%% \vfill\newpage

\begin{examples}{2}
{
\begin{tabular}{ll} %%%% use this tabular format %%%%
alteralm  filename \\
\end{tabular}
}
{When 'filename' is present, alteralm enters the non-interactive mode and parses
its inputs from the file 'filename'. This has the following
structure: the first entry is a qualifier which announces to the parser
which input immediately follows. If this input is omitted in the
input file, the parser assumes the default value.
If the equality sign is omitted, then the parser ignores the entry.
In this way comments may also be included in the file.
In this example, the file contains the following qualifiers:\hfill\newline
\fileparam{\mylink{fac:alteralm:infile_alms}{infile\_alms}%
 = alm.fits}
\fileparam{\mylink{fac:alteralm:nlmax_out}{nlmax\_out}%
 = 512}
\fileparam{\mylink{fac:alteralm:fwhm_arcmin_out}{fwhm\_arcmin\_out}%
 = 20.0}
\fileparam{\mylink{fac:alteralm:coord_out}{coord\_out}%
 = G}
\fileparam{\mylink{fac:alteralm:outfile_alms}{outfile\_alms}%
 = newalm.fits}

Alteralm reads the $a_{\ell m}$ from 'alm.fits'. Since \hfill\newline
\fileparam{\mylink{fac:alteralm:nsmax_in}{nsmax\_in}%
 }
\fileparam{\mylink{fac:alteralm:nsmax_out}{nsmax\_out}%
 }
\fileparam{\mylink{fac:alteralm:fwhm_arcmin_in}{fwhm\_arcmin\_in}%
 }
\fileparam{\mylink{fac:alteralm:beam_file_in}{beam\_file\_in}%
 } 
\fileparam{\mylink{fac:alteralm:coord_in}{coord\_in}%
 }
\fileparam{\mylink{fac:alteralm:epoch_in}{epoch\_in}%
 }
\fileparam{\mylink{fac:alteralm:epoch_out}{epoch\_out}%
 }
\fileparam{\mylink{fac:alteralm:windowfile_in}{windowfile\_in}%
 }
\fileparam{\mylink{fac:alteralm:winfiledir_in}{winfiledir\_in}%
 }
\fileparam{\mylink{fac:alteralm:windowfile_out}{windowfile\_out}%
 }
\fileparam{\mylink{fac:alteralm:winfiledir_out}{winfiledir\_out}%
 }
have their default values, the pixel size will remain the same, the $a_{\ell m}$ will be corrected
from its input beam (whatever it was, assuming the relevant information can be
found), and a gaussian beam of 20.0 arcmin will be applied
instead, the $a_{\ell m}$ will also be rotated from their original coordinate system
(whatever it was, assuming the relevant information can be found)
into Galactic coordinates, assuming a year 2000 epoch for both,
 and only the multipoles up to 512 will be written in
'newalm.fits'.
}
\end{examples}

\begin{release}
  \begin{relist}
    \item Initial release (\healpix 2.00)
  \end{relist}
\end{release}

\begin{messages}
{
\begin{tabular}{p{0.25\hsize} p{0.1\hsize} p{0.35\hsize}} \hline  
  \textbf{Message} & \textbf{Severity} & \textbf{Text} \\ \hline
                   &                   &   \\ %%% for presentation
can not allocate memory for array xxx &  Fatal & You do not have
                   sufficient system resources to run this
                   facility at the map resolution you required. 
  Try a lower map resolution.  \\ 
                   &                   &   \\ \hline %%% for presentation

this is not a binary table & & the fitsfile you have specified is not 
of the proper format \\
                   &                   &   \\ \hline %%% for presentation
there are undefined values in the table! & & the fitsfile you have specified is not 
of the proper format \\
                  &                   &   \\ \hline %%% for presentation
the header in xxx is too long & & the fitsfile you have specified is not 
of the proper format \\
                  &                   &   \\ \hline %%% for presentation
XXX-keyword not found & & the fitsfile you have specified is not 
of the proper format \\
                  &                   &   \\ \hline %%% for presentation
found xxx in the file, expected:yyyy & & the specified fitsfile does not
contain the proper amount of data. \\
                   &                   &   \\ \hline %%% for presentation

alteralm$>$ no information found on input alms beam & Fatal & no information on
the input beam was found, neither from parsing the FITS file header, nor from
what the user provided.

\end{tabular}
} 
\end{messages}

\rule{\hsize}{2mm}

\newpage
