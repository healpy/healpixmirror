\sloppy

\title{\healpix Fortran Subroutines Overview}
\docid{getArgument} \section[getArgument]{ }
\label{sub:getargument}
\docrv{Version 1.1}
\author{Eric Hivon}
\abstract{This document describes the \healpix Fortran90 subroutine
getArgument.}

\begin{facility}
{This subroutine emulates the C routine {\tt getarg}, which returns the value of
a given command line argument.}
{\modExtension}
\end{facility}

\begin{f90format}
{index, value}
\end{f90format}

\begin{arguments}
{
\begin{tabular}{p{0.3\hsize} p{0.05\hsize} p{0.1\hsize} p{0.45\hsize}} \hline  
\textbf{name~\&~dimensionality} & \textbf{kind} & \textbf{in/out} & \textbf{description} \\ \hline
                   &   &   &                           \\ %%% for presentation
index & I4B & IN & index of the command line argument (where the first argument
                   has index 1) \\
value & CHR & OUT & value of the argument 
\end{tabular}}
\end{arguments}

% \begin{example}
% {
% use extension \\
% character(len=128) :: healpixdir \\
% call getargument('HEALPIX', healpixdir) \\
% print*,healpixdir
% }
% {
% Will return the value of the {\tt \$HEALPIX} system variable (if it is defined)
% }
% \end{example}

\begin{related}
  \begin{sulist}{} %%%% NOTE the ``extra'' brace here %%%%
  \item[\htmlref{getEnvironment}{sub:getenvironment}] returns value of
  environment variable
%   \item[\htmlref{getArgument}{sub:getargument}] returns list of command line arguments
  \item[\htmlref{nArguments}{sub:narguments}] returns number of command line arguments
  \end{sulist}
\end{related}

\rule{\hsize}{2mm}

\newpage
