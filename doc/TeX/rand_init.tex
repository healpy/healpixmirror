
\sloppy


\title{\healpix Fortran Subroutines Overview}
\docid{rand\_init} \section[rand\_init]{ }
\label{sub:rand_init}
\docrv{Version 2.0}
\author{Eric Hivon}
\abstract{This document describes the \healpix Fortran90 subroutine RAND\_INIT.}

\begin{facility}
{This routine initializes, with up to 4 seeds, a randomn number sequence. 
 The generator being primed is an F90 port of an xorshift generator described
  in Marsaglia, Journal of Statistical Software 2003, vol 8.
  It has a theoretical period of $2^{128} - 1 \approx 3.4 10^{38}$.
Please refer to the ``Comment on Random Number Generator''
  in the Fortran90 facilities guidelines.
}
{src/f90/mod/rngmod.f90}
\end{facility}

\begin{f90format}
{rng\_handle, [seed1, seed2, seed3, seed4]}
\end{f90format}

\begin{arguments}
{
\begin{tabular}{p{0.3\hsize} p{0.15\hsize} p{0.1\hsize} p{0.35\hsize}} \hline  
\textbf{name~\&~dimensionality} & \textbf{kind} & \textbf{in/out} & \textbf{description} \\ \hline
                   &   &   &                           \\ %%% for presentation
rng\_handle & planck\_rng & OUT & structure of type {\tt planck\_rng}
                   containing on output all information necessary to continue same random sequence. \\ 
seed1 (OPTIONAL)& I4B & IN & first seed of the random sequence. Can be of arbitray
                   sign. If set to
                   zero or not provided will be replaced internally by a non-zero hard coded value.   \\
seed2 (OPTIONAL)& I4B & IN & second seed. Same properties as above  \\
seed3 (OPTIONAL)& I4B & IN & third seed. Same as above.  \\
seed4 (OPTIONAL)& I4B & IN & fourth seed. Same as above.  \\
\end{tabular}
}
\end{arguments}

\begin{example}
{
use rngmod \\
type(planck\_rng) :: rng\_handle \\
call rand\_init(rng\_handle, 12345, 6789012)  \\
}
{
initiates a random sequence with the pair of seeds (12345, 6789012).
}
\end{example}

\begin{related}
  \begin{sulist}{} %%%% NOTE the ``extra'' brace here %%%%
  \item[\htmlref{planck\_rng}{sub:planck_rng}] derived type describing RNG state
  \item[\htmlref{rand\_gauss}{sub:rand_gauss}] function which returns a  random normal deviate.
  \item[\htmlref{rand\_uni}{sub:rand_uni}] function which returns a random uniform deviate.
%%   \item[\htmlref{rand\_init}{sub:rand_init}] subroutine to initiate a random number sequence. 
  \end{sulist}
\end{related}

\rule{\hsize}{2mm}

\newpage
