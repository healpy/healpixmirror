% -*- LaTeX -*-


\renewcommand{\facname}{{nside2npweights }}
\renewcommand{\FACNAME}{{NSIDE2NPWEIGHTS }}

\sloppy



\title{\healpix IDL Facility User Guidelines}
\docid{nside2npweights} \section[nside2npweights]{ }
\label{idl:nside2npweights}
\docrv{Version 1.0}
\author{Eric Hivon}
\abstract{This document describes the \healpix IDL facility \facname.}


\begin{facility}
{This IDL facility provides the number pixel-based quadrature weights (in compact non-redundant form) 
for a given resolution parameter Nside. Because of the \healpix layout symmetries,
$N_w  \simeq \npix/16$, allowing economical storage on disc.%
}
{src/idl/toolkit/nside2npweights.pro}
\end{facility}

\begin{IDLformat}
{\mylink{idl:nside2npweights:Npweights}{Npweights=}%
\FACNAME(\mylink{idl:nside2npweights:Nside}{Nside}
 [,\mylink{idl:nside2npweights:ERROR}{ERROR=}%
, \mylink{idl:nside2npweights:HELP}{/HELP}%
])}
\end{IDLformat}

\begin{qualifiers}
  \begin{qulist}{} %%%% NOTE the ``extra'' brace here %%%%
    \item[Nside\mytarget{idl:nside2npweights:Nside}%
] \healpix resolution parameter (scalar integer),
         should be a valid Nside (power of 2 in $\{1,\ldots,2^{29}\}$)
    \item[Npweights\mytarget{idl:nside2npweights:Npweights}%
] number of non-redundant weights
  \end{qulist}
\end{qualifiers}

\begin{keywords}
  \begin{kwlist}{} %%% extra brace
    \item[ERROR\mytarget{idl:nside2npweights:ERROR}=] error flag, set to 1 on output if Nside is NOT valid, or
    stays to 0 otherwise.
    \item[/HELP\mytarget{idl:nside2npweights:HELP}] if set on input, the documentation header 
  is printed out and the routine exits
  \end{kwlist}
\end{keywords}  

\begin{codedescription}
{\facname outputs the number of different pixel-based weights
  $$N_w=\frac{(\nside+1)(3\nside+1)}{4}.$$ If the argument $\nside$ is not
  valid, a  warning is issued and the error flag is raised.}
\end{codedescription}



\begin{related}
  \begin{sulist}{} %%%% NOTE the ``extra'' brace here %%%%
    \item[idl] version \idlversion or more is necessary to run \facname.	
  \item[\htmlref{unfold\_weights}{idl:unfold_weights}] generates a full sky map of pixel-based or ring-based quadrature weights
  \end{sulist}
\end{related}

\begin{example}
{
\begin{tabular}{ll} %%%% use this tabular format %%%%
Npweights = nside2npweights(256, ERROR=error)
\end{tabular}
}
{
Npweights will be 49408 the number of pixel-based weights for the \healpix
resolution parameter 256 and error will be 0}
\end{example}

