

\sloppy

\title{\healpix Fortran Subroutines Overview}
\docid{assert,assert\_alloc, assert\_directory\_present,~$\ldots$} \section[assert, assert\_alloc, assert\_directory\_present, assert\_present, fatal\_error]{ }
\label{sub:assert}
\docrv{Version 2.0}
\author{Eric Hivon}
\abstract{This document describes the \healpix Fortran90 functions in module MISC\_UTILS.}

\begin{facility}
{The Fortran90 module misc\_utils contains a few routines to test an assertion and return an error
  message if it is false.}
{\modMiscUtils}
\end{facility}

%-------------------------------

\rule{\hsize}{0.7mm}
\textsc{\large{\textbf{SUBROUTINES: }}}\hfill\newline
{\tt call assert(test [, msg, errcode])} 

 \begin{tabular}{@{}p{0.3\hsize}@{\hspace{1ex}}
                        p{0.7\hsize}@{}} & if {\tt test} is true, proceeds with normal code execution. If
                        {\tt test} is false, issues a standard error message
                        (unless {\tt msg} is provided) and stops the code execution with the status
                        {\tt errcode} (or 1 by default). \\
     \end{tabular}\\

{\tt call assert\_alloc(status, code, array)} 

 \begin{tabular}{@{}p{0.3\hsize}@{\hspace{1ex}}
                        p{0.7\hsize}@{}} & if {\tt status} is 0, proceeds with normal code execution. If
                        not, issues an error message indicating a problem during memory allocation
                        of 
                        {\tt array} in program {\tt code}, and stops the code execution.\\
     \end{tabular}\\


{\tt call assert\_directory\_present(directory)} 

 \begin{tabular}{@{}p{0.3\hsize}@{\hspace{1ex}}
                        p{0.7\hsize}@{}} & issues an error message and stops the code execution if
                        the directory named {\tt directory} can not be found\\
     \end{tabular}\\


{\tt call assert\_not\_present(filename)} 

 \begin{tabular}{@{}p{0.3\hsize}@{\hspace{1ex}}
                        p{0.7\hsize}@{}} & issues an error message and stops the code execution if
                        a file with name {\tt filename} already exists.\\
     \end{tabular}\\

{\tt call assert\_present(filename)} 

 \begin{tabular}{@{}p{0.3\hsize}@{\hspace{1ex}}
                        p{0.7\hsize}@{}} & issues an error message and stops the code execution if
                        the file named {\tt filename} can not be found.\\
     \end{tabular}\\

{\tt call fatal\_error([msg])} 

{\tt call fatal\_error} 

 \begin{tabular}{@{}p{0.3\hsize}@{\hspace{1ex}}
                        p{0.7\hsize}@{}} & issue an (optional user defined) error message and stop the code execution.\\
     \end{tabular}\\


\vskip 3cm
%-------------------------------

\begin{arguments}
{
\begin{tabular}{p{0.30\hsize} p{0.05\hsize} p{0.08\hsize} p{0.47\hsize}} \hline  
\textbf{name~\&~dimensionality} & \textbf{kind} & \textbf{in/out} & \textbf{description} \\ \hline
                   &   &   &                           \\ %%% for presentation
test & LGT & IN & result of a logical test \\
msg \hfill OPTIONAL & CHR & IN & character string describing nature of error \\
errorcode \hfill OPTIONAL & I4B & IN & error status given to code interruption \\
status & I4B & IN & value of the {\tt stat} flag returned by the F90 {\tt allocate} command \\
code & CHR & IN & name of program or code in which allocation is made \\
array & CHR & IN & name of array allocated \\
directory & CHR & IN & directory name (contains a '/')\\
filename & CHR & IN & file name \\
\end{tabular}
}
\end{arguments}

%-------------------------------

\begin{example}
{
program my\_code \\
use misc\_utils \\
real, allocatable, dimension(:) :: vector\\
integer :: status \\
real :: a = -1. \\
\\
allocate(vector(12345),stat=status) \\
call assert\_alloc(status, 'my\_code', 'vector') \\
\\
call assert\_directory\_present('/home') \\
\\
call assert(a > 0., 'a is NEGATIVE !!!') \\
\\
end program my\_code\\
}
{ Will issue a error message and stops the code if {\tt vector} can not be allocated, will stop the
  code if '/home' is not found, and will stop the code and complain loudly about it 
because {\tt a} is actually negative.
}
\end{example}

%% \begin{modules}
%%   \begin{sulist}{} %%%% NOTE the ``extra'' brace here %%%%
%%  \item[mk\_pix2xy, mk\_xy2pix] routines used in the conversion between pixel values and ``cartesian'' coordinates on the Healpix face.
%%   \end{sulist}
%% \end{modules}

%% \begin{related}
%%   \begin{sulist}{} %%%% NOTE the ``extra'' brace here %%%%
%%   \item[\htmlref{neighbours\_nest}{sub:neighbours_nest}] find neighbouring pixels.
%%   \item[\htmlref{ang2vec}{sub:ang2vec}] convert $(\theta,\phi)$ spherical coordinates into $(x,y,z)$ cartesian coordinates.
%%   \item[\htmlref{vec2ang}{sub:vec2ang}] convert $(x,y,z)$ cartesian coordinates into $(\theta,\phi)$ spherical coordinates.
%%   \end{sulist}
%% \end{related}

\rule{\hsize}{2mm}

\newpage
