
\sloppy


\title{\healpix IDL Facility User Guidelines}
\docid{same\_shape\_pixels\_XXXX} \section[same\_shape\_pixels\_XXXX]{ }
\label{idl:same_shape_pixels_xxx}
\docrv{Version 1.0}
\author{E. Hivon}
\abstract{This document describes the \healpix facilities
  SAME\_SHAPE\_PIXELS\_RING and SAME\_SHAPE\_PIXELS\_NEST.}

\begin{facility}
{These IDL facilities provide the ordered list of all \healpix pixels having the same shape
  as a given template, for a resolution parameter $\nside$.
%% Any pixel can be {\em matched in shape}
%%   to a single of these templates by a combination of  a rotation around the polar axis with 
%%   reflexion(s) around a meridian and/or the equator. 
}
{src/idl/toolkit/same\_shape\_pixels\_nest.pro, src/idl/toolkit/same\_shape\_pixels\_ring.pro}
\end{facility}

\begin{IDLformat}
{same\_shape\_pixels\_nest, Nside, Template, List\_Pixels\_Nest [, Reflexion, NREPLICATIONS=]}
\end{IDLformat}
\begin{IDLformat}
{same\_shape\_pixels\_ring, Nside, Template, List\_Pixels\_Ring [, Reflexion, NREPLICATIONS=]}
\end{IDLformat}

\begin{qualifiers}
  \begin{qulist}{} %%%% NOTE the ``extra'' brace here %%%%

\item[{Nside}] (IN, scalar) the \healpix $\nside$ parameter. 
\item[{Template}] (IN, scalar) identification number of the
                   template (this number is independent of the numbering scheme considered).
\item[{List\_Pixel\_Nest}] (OUT, vector) ordered list of NESTED scheme identification numbers
  for all pixels having the same shape as the template provided
\item[{List\_Pixel\_Ring}] (OUT, vector) ordered list of RING scheme identification numbers
  for all pixels having the same shape as the template provided
\item[{Reflexion}] (OUT, OPTIONAL, vector) in \{0, 3\} encodes the transformation(s) to
                   apply to each of the returned pixels to match exactly in
                   shape and position the template provided. 0: rotation around the polar axis only,
                   1: rotation + East-West swap (ie, reflexion around meridian),
                   2: rotation + North-South swap (ie, reflexion around
                   Equator), 3: rotation + East-West and North-South swaps
  \end{qulist}
\end{qualifiers}

\begin{keywords}
 \begin{kwlist}{}
\item[{NREPLICATIONS}] (OUT, OPTIONAL, scalar) number of pixels having the same shape as
  the template. It is also the length of the vectors {\tt List\_Pixel\_Nest},
  {\tt List\_Pixel\_Ring} and {\tt Reflexion}. It is either 8, 16, 4$\nside$ or
  8$\nside$.
 \end{kwlist}
\end{keywords}



\begin{codedescription}
{\thedocid\ provide the ordered list of all \healpix pixels having the same shape
  as a given template, for a resolution parameter $\nside$. Depending on the
  template considered the number of such pixels is either 8, 16, 4$\nside$ or
  8$\nside$. The template pixels are all located in the Northern Hemisphere, or on the
 Equator.
They are chosen to have their center located at
\begin{eqnarray}
     z=\cos(\theta)\ge 2/3,  &\ &    0< \phi \leq \pi/2,   \nonumber\\            %[Nside*(Nside+2)/4]
     2/3 > z \geq 0,  &\ & \phi=0, \quad{\rm or}\quad  \phi=\frac{\pi}{4\nside}.  \nonumber %[Nside]
\myhtmlimage{}
\end{eqnarray}
 They are numbered continuously from 0, starting at the North Pole, with the index
 increasing in $\phi$, and then increasing for decreasing $z$.
}
\end{codedescription}


\begin{example}
{
same\_shape\_pixels\_ring, 256, 1234, list\_pixels, reflexion, nrep=np  \\
}
{
Returns in {\tt list\_pixels} the RING-scheme index of the all the pixels having
the same shape as the template \#1234 for $\nside=256$. Upon return {\tt reflexion} will
contain the reflexions to apply to each pixel returned to match the template,
and {\tt np} will contain the number of pixels having that same shape (16 in that case).
}
\end{example}
\begin{related}
  \begin{sulist}{} %%%% NOTE the ``extra'' brace here %%%%
  \item[\htmlref{nside2templates}{idl:nside2ntemplates}] returns the
  number of template pixel shapes available for a given $\nside$.
  \item[\htmlref{template\_pixel\_ring}{idl:template_pixel_xxx}] 
  \item[\htmlref{template\_pixel\_nest}{idl:template_pixel_xxx}] 
  return
  the template shape matching the pixel provided
  \end{sulist}
\end{related}

\rule{\hsize}{2mm}

