
\sloppy


%%%\title{\healpix Fortran Subroutines Overview}
\docid{rand\_uni} \section[rand\_uni]{ }
\label{sub:rand_uni}
\docrv{Version 2.0}
\author{Eric Hivon}
\abstract{This document describes the \healpix Fortran90 subroutine RAND\_UNI.}

\begin{facility}
{This routine returns a number out of a pseudo-random uniform deviate (ie, its
  probability distribution is uniform in the range ]0,1[).
}
{\modRngmod}
\end{facility}

\begin{f90function}
{\mylink{sub:rand_uni:rng_handle}{rng\_handle}%
}
\end{f90function}

\begin{arguments}
{
\begin{tabular}{p{0.3\hsize} p{0.15\hsize} p{0.1\hsize} p{0.35\hsize}} \hline  
\textbf{name~\&~dimensionality} & \textbf{kind} & \textbf{in/out} & \textbf{description} \\ \hline
                   &   &   &                           \\ %%% for presentation
rng\_handle\mytarget{sub:rand_uni:rng_handle} & planck\_rng & INOUT & structure of type {\tt planck\_rng}
                   containing on all information necessary to continue same
                   random sequence. \\ 
var & DP & OUT & number belonging to a pseudo-random uniform deviate.
\end{tabular}
}
\end{arguments}

\begin{example}
{
use healpix\_types \\
use rngmod \\
type(planck\_rng) :: rng\_handle \\
real(dp) :: uni \\
\\
call rand\_init(rng\_handle, 12345, 6789012)  \\
uni = rand\_uni(rng\_handle)
}
{
initiates a random sequence with the pair of seeds (12345, 6789012), and
generates one number out of the uniform deviate.
}
\end{example}

\begin{related}
  \begin{sulist}{} %%%% NOTE the ``extra'' brace here %%%%
  \item[\htmlref{planck\_rng}{sub:planck_rng}] derived type describing RNG state
  \item[\htmlref{rand\_gauss}{sub:rand_gauss}] function which returns a  random normal deviate.
%%  \item[\htmlref{rand\_uni}{sub:rand_uni}] function which returns a random uniform deviate.
  \item[\htmlref{rand\_init}{sub:rand_init}] subroutine to initiate a random number sequence. 
  \end{sulist}
\end{related}

\rule{\hsize}{2mm}

\newpage
