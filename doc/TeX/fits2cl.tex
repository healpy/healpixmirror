
\sloppy


\title{\healpix Fortran Subroutines Overview}
\docid{fits2cl*} \section[fits2cl*]{ }
\label{sub:fits2cl}
\docrv{Version 2.0}
\author{Eric Hivon, Frode K.~Hansen}
\abstract{This document describes the \healpix Fortran90 subroutine FITS2CL.}

\begin{facility}
{This routine reads a power spectrum or beam window function from a FITS ASCII
or binary table. 
The routine can read temperature coefficients $C_l^T$ or both temperature and 
polarisation coefficients $C_l^T$, $C_l^E$, $C_l^B$, $C_l^{T\times E}$. If the 
keyword PDMTYPE is found in the header, fits2cl assumes the table to be in the 
special format used by {\em Planck} and will ignore the first data column. 
If the input FITS file contains several
extensions or HDUs, the one to be read can be specified thanks to the CFITSIO 
\htmladdnormallink{Extended File Name Syntax}{http://heasarc.gsfc.nasa.gov/docs/software/fitsio/c/c_user/node81.html}, using its number (eg, {\tt file.fits[2]} or {\tt file.fits+2}) or its
{\tt EXTNAME} value (eg. {\tt file.fits[beam\_100x100]}). By default, only the first valid
extension will be read.}
{\modFitstools}
\end{facility}

\begin{f90format}
{\mylink{sub:fits2cl:filename}{filename}%
, \mylink{sub:fits2cl:clin}{clin}%
, \mylink{sub:fits2cl:lmax}{lmax}%
, \mylink{sub:fits2cl:ncl}{ncl}%
, \mylink{sub:fits2cl:header}{header}%
, [\mylink{sub:fits2cl:units}{units}%
]}
\end{f90format}

\begin{arguments}
{
\begin{tabular}{p{0.4\hsize} p{0.05\hsize} p{0.1\hsize} p{0.35\hsize}} \hline  
\textbf{name~\&~dimensionality} & \textbf{kind} & \textbf{in/out} & \textbf{description} \\ \hline
                   &   &   &                           \\ %%% for presentation
filename\mytarget{sub:fits2cl:filename}(LEN=\filenamelen) & CHR & IN & the FITS file containing the power spectrum. \\
lmax\mytarget{sub:fits2cl:lmax} & I4B & IN & Maximum $\ell$ value to be read. \\
ncl\mytarget{sub:fits2cl:ncl} & I4B & IN & 1 for temperature coeffecients only, 4 for polarisation. \\
clin\mytarget{sub:fits2cl:clin}(0:lmax,1:ncl) & SP/ DP & OUT & the power spectrum read from the file.\\
header\mytarget{sub:fits2cl:header}(LEN=80) (1:) & CHR & OUT & the header read from the FITS-file. \\ 
units\mytarget{sub:fits2cl:units}(LEN=80) (1:) & CHR & OUT & the column units read from the FITS-file. \\ 
\end{tabular}
}
\end{arguments}

%\newpage
\begin{example}
{
call fits2cl ('cl.fits',cl,64,4,header,units)  \\
}
{
Reads a power spectrum from the FITS file `cl.fits' and stores the result in cl(0:64,1:4) which are the $C_l$ coeffecients up to $l=64$ for ($T$, $E$, $B$, $T\times E$). The FITS header is returned in header, the column units in units.
}
\end{example}
\begin{modules}
  \begin{sulist}{} %%%% NOTE the ``extra'' brace here %%%%
  \item[\textbf{fitstools}] module, containing:
  \item[printerror] routine for printing FITS error messages.
  \item[\textbf{cfitsio}] library for FITS file handling.		
  \end{sulist}
\end{modules}

\begin{related}
  \begin{sulist}{} %%%% NOTE the ``extra'' brace here %%%%
  \item[\htmlref{create\_alm}{sub:create_alm}] Routine to create $a_{\ell m}$ values
  from an input power spectrum.
  \item[\htmlref{write\_asctab}{sub:write_asctab}] Routine to create an ascii
  FITS file containing a power spectrum.
  \end{sulist}
\end{related}

\rule{\hsize}{2mm}

\newpage
