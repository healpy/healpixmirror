
\sloppy


%%%\title{\healpix Fortran Subroutines Overview}
\docid{nside2npweights} \section[nside2npweights]{ }
\label{sub:nside2npweights}
\docrv{Version 1.1}
\author{E. Hivon}
\abstract{This document describes the \healpix Fortran90 subroutine NSIDE2NPWEIGHTS.}

\begin{facility}
{Function to return the number of pixel-based weights (in compact form) for a given Nside:
$$N_w=\frac{(\nside+1)(3\nside+1)}{4} \simeq \frac{\npix}{16}$$ 
}
{\modPixTools}
\end{facility}

\begin{f90function}
{\mylink{sub:nside2npweights:nside}{nside}%
}
\end{f90function}

\begin{arguments}
{
\begin{tabular}{p{0.3\hsize} p{0.05\hsize} p{0.1\hsize} p{0.45\hsize}} \hline  
\textbf{name~\&~dimensionality} & \textbf{kind} & \textbf{in/out} & \textbf{description} \\ \hline
                   &   &   &                           \\ %%% for presentation
nside\mytarget{sub:nside2npweights:nside} & I4B & IN & the $\nside$ parameter. \\
var & I8B & OUT & the number of template pixels $N_w$.\\
\end{tabular}
}
\end{arguments}

\begin{example}
{
use \htmlref{healpix\_modules}{sub:healpix_modules} \\
integer(\htmlref{I8B}{sub:healpix_types}) :: nw8 \\
nw8 = nside2npweights(256)  \\
}
{
Returns in {\tt nw8} the number of non-redundant \healpix pixel-based weights (49408) for the resolution
parameter 256.
}
\end{example}
\begin{related}
  \begin{sulist}{} %%%% NOTE the ``extra'' brace here %%%%
  \item[\htmlref{unfold\_weightsfile}{sub:unfold_weightsfile}] reads of FITS file containing a list ring-based or pixel-based weights into a full sky map
  \end{sulist}
\end{related}

\rule{\hsize}{2mm}

