
\sloppy


\title{\healpix Fortran Subroutines Overview}
\docid{rotate\_alm*} \section[rotate\_alm*]{ }
\label{sub:rotate_alm}
\docrv{Version 2.0}
\author{Eric Hivon}
\abstract{This document describes the \healpix Fortran90 subroutine ROTATE\_ALM.}

\begin{facility}
{This routine transform the scalar (and tensor) $a_{\ell m}$ coefficients to
emulate the effect of an arbitrary rotation of the underlying map. The rotation is done
directly on the $a_{\ell m}$ using the Wigner rotation matrices, computed by
recursion.
To rotate the $a_{\ell m}$ for $\ell \leq \ell_\textrm{max}$ the number of
operations scales like $\ell^3_\textrm{max}$.}
{\modAlmTools}
\end{facility}

\begin{f90format}
{\mylink{sub:rotate_alm:nlmax}{nlmax}%
, \mylink{sub:rotate_alm:alm_TGC}{alm\_TGC}%
, \mylink{sub:rotate_alm:psi}{psi}%
, \mylink{sub:rotate_alm:theta}{theta}%
, \mylink{sub:rotate_alm:phi}{phi}%
}
\end{f90format}

\begin{arguments}
{
\begin{tabular}{p{0.36\hsize} p{0.05\hsize} p{0.09\hsize} p{0.40\hsize}} \hline  
\textbf{name~\&~dimensionality} & \textbf{kind} & \textbf{in/out} & \textbf{description} \\ \hline
                   &   &   &                           \\ %%% for presentation
nlmax\mytarget{sub:rotate_alm:nlmax} & I4B & IN & maximum $\ell$ value for the $a_{\ell m}$.\\
alm\_TGC\mytarget{sub:rotate_alm:alm_TGC}(1:p,0:nlmax,0:nlmax) & SPC/ DPC & INOUT & complex $a_{\ell m}$ values
                   before and after rotation of the coordinate system.  
	The first index here runs from 1:1 for
                   temperature only, and 1:3 for polarisation. In the latter
                   case,  1=T, 2=E, 3=B. \\
% \end{tabular}
% \begin{tabular}{p{0.36\hsize} p{0.05\hsize} p{0.09\hsize} p{0.40\hsize}}
%                    \hline  
psi\mytarget{sub:rotate_alm:psi}	& DP & IN & first rotation: angle $\psi$ about the z-axis.
All angles are in radians and should lie in [-2$\pi$,2$\pi$], the rotations are
active and the referential system is assumed to be right handed, the routine
\htmlref{coordsys2euler\_zyz}{sub:coordsys2euler_zyz} can be used to generate
the Euler angles
$\psi, \theta, \varphi$ for rotation between standard astronomical coordinate
systems; \\
theta\mytarget{sub:rotate_alm:theta}	& DP & IN & second rotation: angle $\theta$ about the original
(unrotated) y-axis; \\
phi\mytarget{sub:rotate_alm:phi}	& DP & IN & third rotation: angle $\varphi$ about the original (unrotated) z-axis;
\end{tabular}
}
\end{arguments}

\begin{example}
{
use alm\_tools, only: rotate\_alm \\
...\\
call rotate\_alm(64, alm\_TGC, PI/3., 0.5\_dp, 0.0\_dp)  \\
}
{
Transforms scalar and tensor $a_{\ell m}$ for $\ell_{\rm
  max}=m_\textrm{max} = 64$ to emulate a rotation of the underlying map by
($\psi=\pi/3, \theta=0.5, \varphi=0\myhtmlimage{}$).
}
\end{example}

\begin{example}
{
use coord\_v\_convert, only: coordsys2euler\_zyz \\
use alm\_tools, only: rotate\_alm \\
...\\
call coordsys2euler\_zyz(2000.0\_dp, 2000.0\_dp, 'E', 'G', psi, theta, phi) \\
call rotate\_alm(64, alm\_TGC, psi, theta, phi)  \\
}
{
Rotate the $a_{\ell m}$ from Ecliptic to Galactic coordinates.
}
\end{example}

% \begin{modules}
%   \begin{sulist}{} %%%% NOTE the ``extra'' brace here %%%%
%   \item[\textbf{alm\_tools}] module, containing:
% 	\item[\htmlref{generate\_beam}{sub:generate_beam}] routine to generate beam window function
% 	\item[\htmlref{pixel\_window}{sub:pixel_window}] routine to generate pixel window function
%   \end{sulist}
% \end{modules}

\begin{related}
  \begin{sulist}{} %%%% NOTE the ``extra'' brace here %%%%
  \item[\htmlref{coordsys2euler\_zyz}{sub:coordsys2euler_zyz}] can be used to generate
the Euler angles $\psi, \theta, \varphi \myhtmlimage{}$ for rotation between standard astronomical coordinate systems
  \item[\htmlref{create\_alm}{sub:create_alm}] Routine to create $a_{\ell m}$ coefficients.
  \item[\htmlref{alter\_alm}{sub:alter_alm}] Routine to modify $a_{\ell m}$
  coefficients to apply or remove the effect of an instrumental beam.
  \item[\htmlref{map2alm}{sub:map2alm}]  Routines to analyze a \healpix sky map into its $a_{\ell m}$
  coefficients.
  \item[\htmlref{alm2map}{sub:alm2map}] Routines to synthetize a \healpix sky map from its $a_{\ell m}$
  coefficients.
  \item[\htmlref{alms2fits}{sub:alms2fits}, \htmlref{dump\_alms}{sub:dump_alms}]
  Routines to save a set of $a_{\ell m}$ in a FITS file.  
  \item[\htmlref{xcc\_v\_convert}{sub:xcc_v_convert}] rotates a 3D coordinate
vector from one astronomical coordinate system to another.
  \end{sulist}
\end{related}

\rule{\hsize}{2mm}

\newpage
