

\sloppy

\title{\healpix Fortran Subroutines Overview}
\docid{long\_count,~long\_size} \section[long\_count,~long\_size]{ }
\label{sub:long_intrinsic}
\docrv{Version 2.0}
\author{Eric Hivon}
\abstract{This document describes the \healpix Fortran90 functions in module LONG\_INTRINSIC.}

\begin{facility}
{The Fortran90 module {\tt long\_intrinsic} contains a subset of  
intrinsic functions (currently {\tt count} and {\tt size}) compiled so that they return \htmlref{I8B}{sub:healpix_types} variables
instead of the default integer (generally \htmlref{I4B}{sub:healpix_types}),
therefore allowing the handling of arrays with more than $2^{31}-1$
elements.}
{\modLongIntrinsic}
\end{facility}

%-------------------------------

\rule{\hsize}{0.7mm}
\textsc{\large{\textbf{FUNCTIONS: }}}\hfill\newline
{\tt cnt = long\_count(mask1)} 

 \begin{tabular}{@{}p{0.3\hsize}@{\hspace{1ex}}
                        p{0.7\hsize}@{}} & returns the I8B integer value that is
the number of elements of the logical array {\tt mask1} that have the value {\tt
true}. 
     \end{tabular}\\

{\tt sz = long\_size(array1 [,dim])} \\
{\tt sz = long\_size(array2 [,dim])} 

 \begin{tabular}{@{}p{0.3\hsize}@{\hspace{1ex}}
                        p{0.7\hsize}@{}} & returns the I8B integer value that is
the size of the 1D array {\tt array1} or 2D array {\tt array2} or their
extent along the dimension {\tt dim} if the scalar integer {\tt dim} is provided.
     \end{tabular}\\

%\vskip 0.1cm
%-------------------------------

\begin{arguments}
{
\begin{tabular}{p{0.30\hsize} p{0.05\hsize} p{0.08\hsize} p{0.47\hsize}} \hline  
\textbf{name~\&~dimensionality} & \textbf{kind} & \textbf{in/out} & \textbf{description} \\ \hline
                   &   &   &                           \\ %%% for presentation
cnt & I8B & OUT & number of elements with value {\tt true} \\
sz & I8B & OUT & size or extent of array \\
mask1(:) & LGT & IN & 1D logical array \\
array1(:) & I4B/ I8B/ SP/ DP & IN & 1D integer or real array \\
array2(:,:) & I4B/ I8B/ SP/ DP & IN & 2D integer or real array \\
dim\ \ \ (OPTIONAL) & I4B & IN & dimension (starting at 1) along which the array
extent is measured.
\end{tabular}
}
\end{arguments}

%-------------------------------

\begin{example}
{
 use \htmlref{healpix\_modules}{sub:healpix_modules} \\
 real(SP), dimension(:,:), allocatable :: bigarray \\
 allocate(bigarray(2\_i8b**31+5, 3)) \\
 print*,       size(bigarray),       size(bigarray,1),       size(bigarray,dim=2) \\
 print*, long\_size(bigarray), long\_size(bigarray,1), long\_size(bigarray,dim=2) \\
 deallocate(bigarray)
}
{Will return (with default compilation options)
\begin{tabbing}
     -2147483633 \= -2147483643  \= 3 \\
     6442450959  \> 2147483653   \> 3
\end{tabbing}
meaning that {\tt long\_size} handles correctly this large array while by default
{\tt size} does not.}
\end{example}


%% \begin{modules}
%%   \begin{sulist}{} %%%% NOTE the ``extra'' brace here %%%%
%%  \item[mk\_pix2xy, mk\_xy2pix] routines used in the conversion between pixel values and ``cartesian'' coordinates on the Healpix face.
%%   \end{sulist}
%% \end{modules}

%% \begin{related}
%%   \begin{sulist}{} %%%% NOTE the ``extra'' brace here %%%%
%%   \item[\htmlref{neighbours\_nest}{sub:neighbours_nest}] find neighbouring pixels.
%%   \item[\htmlref{ang2vec}{sub:ang2vec}] convert $(\theta,\phi)$ spherical coordinates into $(x,y,z)$ cartesian coordinates.
%%   \item[\htmlref{vec2ang}{sub:vec2ang}] convert $(x,y,z)$ cartesian coordinates into $(\theta,\phi)$ spherical coordinates.
%%   \end{sulist}
%% \end{related}

\rule{\hsize}{2mm}

\newpage
