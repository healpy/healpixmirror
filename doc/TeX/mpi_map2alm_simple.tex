
\sloppy


\title{\healpix Fortran Subroutines Overview}
\docid{mpi\_map2alm\_simple*} \section[mpi\_map2alm\_simple*]{ }
\label{sub:mpi_map2alm_simple}
\docrv{Version 1.0}
\author{Hans K. Eriksen}
\abstract{This document describes the \healpix Fortran 90 subroutine MPI\_MAP2ALM\_SIMPLE*.}

\begin{facility}
{This subroutine provides a simplified (one-line) interface to the MPI version of
map2alm. It supports both temperature and polarization inputs in both
single and double precision. It must only be run by all processors in 
the MPI communicator.
}
{\modMpiAlmTools}
\end{facility}

\begin{f90format}
{\mylink{sub:mpi_map2alm_simple:comm}{comm}%
, \mylink{sub:mpi_map2alm_simple:map}{map}%
, \mylink{sub:mpi_map2alm_simple:alms}{alms}%
, [\mylink{sub:mpi_map2alm_simple:zbounds}{zbounds}%
], [\mylink{sub:mpi_map2alm_simple:w8ring}{w8ring}%
]}
\end{f90format}

\begin{arguments}
{
\begin{tabular}{p{0.4\hsize} p{0.05\hsize} p{0.05\hsize} p{0.40\hsize}} \hline  
\textbf{name~\&~dimensionality} & \textbf{kind} & \textbf{in/out} & \textbf{description} \\ \hline
                   &   &   &                           \\ %%% for presentation
comm\mytarget{sub:mpi_map2alm_simple:comm} & I4B & IN & MPI communicator. \\
map\mytarget{sub:mpi_map2alm_simple:map}(0:npix-1,1:nmaps) & SP or DP & IN & input map. If
nmaps=1, only temperature information is included; if nmaps=3,
polarization information is included\\
alms\mytarget{sub:mpi_map2alm_simple:alms}(1:nmaps,0:lmax,0:nmax) & SPC or DPC & IN & output alms. 
nmaps must
equal that of the input map\\
zbounds\mytarget{sub:mpi_map2alm_simple:zbounds}(1:2) & DP & IN & section of the map on which to perform the $a_{lm}$
                   analysis, expressed in terms of $z=\sin({\rm latitude}) =
                   \cos(\theta)$. %zbounds_sub.tex:  for alm2map, map2alm, remove_dipole: describe processed area
%zbounds2_sub.tex: for apply_mask: describe pixels set to 0 (=unprocessed area)
If zbounds(1)$<$zbounds(2), it is
performed {\em on} the strip zbounds(1)$<z<$zbounds(2); if not,
it is performed {\em outside} the strip
%  zbounds(2)$<z<$zbounds(1). % OLD
zbounds(2)$\le z \le$zbounds(1). % NEW ??
% The whole sphere is treated if \texttt{zbounds=(/-1.0\_dp, 1.0\_dp/)} or if 
% it is absent.
If absent, the whole map is processed.
 (OPTIONAL) \\
\end{tabular}
\begin{tabular}{p{0.4\hsize} p{0.05\hsize} p{0.05\hsize} p{0.40\hsize}} \hline  
w8ring\mytarget{sub:mpi_map2alm_simple:w8ring}\_TQU(1:2*nsmax, 1:p) & DP & IN & ring weights for quadrature corrections. If ring weights are not used, this array should be 1 everywhere. p is 1 for a temperature analysis and 3 for (T,Q,U). (OPTIONAL)\\
\end{tabular}
}
\end{arguments}
%%\newpage

\begin{example}
{
\hspace*{1cm}call mpi\_map2alm\_simple(comm, map, alms)\\
}
{
This example executes a parallel map2alm operation through the
one-line interface. Although all processors must supply allocated
arrays to the routine, only the root processor's information will be
used as input, and only the root processor's alms will be complete
after execution. 
}
\end{example}

\begin{modules}
  \begin{sulist}{} %%%% NOTE the ``extra'' brace here %%%%
  \item[\textbf{alm\_tools}] module
  \end{sulist}
\end{modules}

\begin{related}
  \begin{sulist}{} %%%% NOTE the ``extra'' brace here %%%%
   \item[\htmlref{mpi\_cleanup\_alm\_tools}{sub:mpi_cleanup_alm_tools}] Frees memory that is allocated by the current routine. 
   \item[\htmlref{mpi\_initialize\_alm\_tools}{sub:mpi_initialize_alm_tools}] Allocates memory and defines variables for the mpi\_alm\_tools module. 
  \item[\htmlref{mpi\_alm2map}{sub:mpi_alm2map}] Routine for executing a parallel inverse spherical harmonics transform (root processor interface)
  \item[\htmlref{mpi\_alm2map\_slave}{sub:mpi_alm2map_slave}] Routine for executing a parallel inverse spherical harmonics transform (slave processor interface)
  \item[\htmlref{mpi\_map2alm}{sub:mpi_map2alm}] Routine for executing a parallel spherical harmonics transform (root processor interface)
  \item[\htmlref{mpi\_map2alm\_slave}{sub:mpi_map2alm_slave}] Routine for executing a parallel spherical harmonics transform (slave processor interface)
  \item[\htmlref{mpi\_alm2map\_simple}{sub:mpi_alm2map_simple}] One-line interface to the parallel inverse spherical harmonics transform 
  \end{sulist}
\end{related}


\rule{\hsize}{2mm}

\newpage
