% -*- LaTeX -*-


\sloppy

\title{\healpix IDL Facility User Guidelines}
\docid{help\_st} \section[help\_st]{ }
\label{idl:help_st}
\docrv{Version 1.0}
\author{Eric Hivon}
\abstract{This document describes the \healpix IDL facility help\_st.}

\begin{facility}
{This IDL facility provides some HELP-like information on any IDL variable,
and especially on sub-structures.}
{src/idl/misc/help\_st.pro}
\end{facility}

\begin{IDLformat}
{\thedocid,
\mylink{idl:help_st:var}{Var}}
\end{IDLformat}

\begin{qualifiers}
  \begin{qulist}{} %%%% NOTE the ``extra'' brace here %%%%
    \item[Var] \mytarget{idl:help_st:var}%
	IDL variable, of any kind
  \end{qulist}
\end{qualifiers}

% \begin{keywords}
%   \begin{kwlist}{} %%% extra brace
%   \end{kwlist}
% \end{keywords}  

\begin{codedescription}
{%
If \mylink{idl:help_st:var}{\texttt{Var}} is an IDL structure, \thedocid{} does a recursive \texttt{HELP,/STRUCTURES} on \texttt{Var} and each of its substructure, otherwise it does the equivalent of \texttt{HELP, Var} (see respectively 
Examples 
\mylink{idl:help_st:example1}{\#1} and 
\mylink{idl:help_st:example2}{\#2} below)}
\end{codedescription}



\begin{related}
  \begin{sulist}{} %%%% NOTE the ``extra'' brace here %%%%
    \item[idl] version \idlversion or more is necessary to run \thedocid.
  \end{sulist}
\end{related}

\begin{examples}
{1}
{
\mytarget{idl:help_st:example1}
\begin{tabular}{l} %%%% use this tabular format %%%%
\htmlref{init\_healpix}{idl:init_healpix} ; make sure that \texttt{!healpix} is defined\\
help, /structure, !healpix\\
\thedocid, !healpix
\end{tabular}
}
{
%\parbox[t]{\hsize}{
\begin{minipage}[t]{\hsize}
the example above compares the output of \texttt{help,/structures} which only describes the top structure: 
\\
\\
{\scriptsize{\texttt{ % put font size before font type to get line spacing right
** Structure <151cef8>, 7 tags, length=528, data length=524, refs=2:    \\
\begin{tabular}{lll}
   VERSION      &   STRING  &  '3.40'   \\
   DATE         &   STRING  &  '2018-01-01'   \\
   DIRECTORY    &   STRING  &  '/home/user/Healpix'   \\
   PATH         &   STRUCT  &  -> <Anonymous> Array[1]   \\
   NSIDE        &   LONG    &  Array[30]   \\
   BAD\_VALUE   &   FLOAT   &   -1.63750e+30   \\
   COMMENT      &   STRING  &  Array[15]   
\end{tabular}
}}}
\\
\par
and \texttt{\thedocid}, which describes each sub-structure:
\\   
{\scriptsize{\texttt{ % put font size before font type to get line spacing right   
** Structure <151cef8>, 7 tags, length=528, data length=524, refs=2:   \\
\begin{tabular}{lll}
  .VERSION           &     STRING  &   '3.40'   \\
  .DATE              &     STRING  &   '2018-01-01'   \\
  .DIRECTORY         &     STRING  &   '/home/user/Healpix'   \\
  .PATH.BIN.CXX      &     STRING  &   '/home/user/Healpix/src/cxx/generic\_gcc/bin/'\\
  .PATH.BIN.F90      &     STRING  &   '/home/user/Healpix/bin/'   \\
  .PATH.DATA         &     STRING  &   '/home/user/Healpix/data/'   \\
  .PATH.DOC.HTML     &     STRING  &   '/home/user/Healpix/doc/html/'   \\
  .PATH.DOC.PDF      &     STRING  &   '/home/user/Healpix/doc/pdf/'   \\
  .PATH.SRC          &     STRING  &   '/home/user/Healpix/src/'   \\
  .PATH.TEST         &     STRING  &   '/home/user/Healpix/test/'   \\
  .NSIDE             &     LONG    &   Array[30]   \\
  .BAD\_VALUE        &     FLOAT   &    -1.63750e+30   \\
  .COMMENT           &     STRING  &   Array[15]   
\end{tabular}
}}}
\end{minipage}
%}
}
\end{examples}



\begin{examples}
{2}
{
\mytarget{idl:help_st:example2}
\begin{tabular}{l} %%%% use this tabular format %%%%
a=0\\
help,a+1\\
\thedocid, a+1
\end{tabular}
}
{\parbox[t]{\hsize}{
will print out
\\
{\scriptsize{\texttt{ 
<Expression>    \hspace{3em} INT  \hspace{3em}     =   \hspace{3em}     1 \\
A+1             \hspace{3em} INT   \hspace{3em}    =   \hspace{3em}     1
}}}
}}
\end{examples}



