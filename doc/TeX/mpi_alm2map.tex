
\sloppy


%%%\title{\healpix Fortran Subroutines Overview}
\docid{mpi\_alm2map*} \section[mpi\_alm2map*]{ }
\label{sub:mpi_alm2map}
\docrv{Version 1.0}
\author{Hans K. Eriksen}
\abstract{This document describes the \healpix Fortran 90 subroutine MPI\_ALM2MAP*.}

\begin{facility}
{This subroutine implements MPI parallelization of the serial alm2map
routine. It supports both temperature and polarization inputs in both
single and double precision. It must only be run by the root node of
the MPI communicator.
}
{\modMpiAlmTools}
\end{facility}

\begin{f90format}
{\mylink{sub:mpi_alm2map:alms}{alms}%
, \mylink{sub:mpi_alm2map:map}{map}%
}
\end{f90format}

\begin{arguments}
{
\begin{tabular}{p{0.4\hsize} p{0.05\hsize} p{0.05\hsize} p{0.40\hsize}} \hline  
\textbf{name~\&~dimensionality} & \textbf{kind} & \textbf{in/out} & \textbf{description} \\ \hline
                   &   &   &                           \\ %%% for presentation
alms\mytarget{sub:mpi_alm2map:alms}(1:nmaps,0:lmax,0:nmax) & SPC or DPC & IN & Input alms. If
nmaps=1, only temperature information is included; if nmaps=3,
polarization information is included \\ 
map\mytarget{sub:mpi_alm2map:map}(0:npix,1:nmaps) & SP or DP & OUT & Output map. nmaps must match 
that of the input alms array.\\
\end{tabular}
}
\end{arguments}
%%\newpage

\begin{example}
{
call mpi\_comm\_rank(comm, myid, ierr)\\
if (myid == root) then\\
\hspace*{1cm}call mpi\_initialize\_alm\_tools(comm, nsmax, nlmax, nmmax, \\
\hspace*{3cm}zbounds,polarization, precompute\_plms)\\
\hspace*{1cm}call mpi\_alm2map(alms, map)\\
else \\
\hspace*{1cm}call mpi\_initialize\_alm\_tools(comm)\\
\hspace*{1cm}call mpi\_alm2map\_slave\\
end\\
call mpi\_cleanup\_alm\_tools\\
}
{
This example 1) initializes the mpi\_alm\_tools module (i.e.,
allocates internal arrays and defines required parameters), 2)
executes a parallel alm2map operation, and 3) frees the previously
allocated memory.
}
\end{example}

\begin{modules}
  \begin{sulist}{} %%%% NOTE the ``extra'' brace here %%%%
  \item[\textbf{alm\_tools}] module
  \end{sulist}
\end{modules}

\begin{related}
  \begin{sulist}{} %%%% NOTE the ``extra'' brace here %%%%
   \item[\htmlref{mpi\_cleanup\_alm\_tools}{sub:mpi_cleanup_alm_tools}] Frees memory that is allocated by the current routine. 
   \item[\htmlref{mpi\_initialize\_alm\_tools}{sub:mpi_initialize_alm_tools}] Allocates memory and defines variables for the mpi\_alm\_tools module. 
  \item[\htmlref{mpi\_alm2map\_slave}{sub:mpi_alm2map_slave}] Routine for executing a parallel inverse spherical harmonics transform (slave processor interface)
  \item[\htmlref{mpi\_map2alm}{sub:mpi_map2alm}] Routine for executing a parallel spherical harmonics transform (root processor interface)
  \item[\htmlref{mpi\_map2alm\_slave}{sub:mpi_map2alm_slave}] Routine for executing a parallel spherical harmonics transform (slave processor interface)
  \item[\htmlref{mpi\_alm2map\_simple}{sub:mpi_alm2map_simple}] One-line interface to the parallel inverse spherical harmonics transform 
  \item[\htmlref{mpi\_map2alm\_simple}{sub:mpi_map2alm_simple}] One-line interface to the parallel spherical harmonics transform 
  \end{sulist}
\end{related}

\rule{\hsize}{2mm}

\newpage
