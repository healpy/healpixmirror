
\sloppy

\title{\healpix Fortran Subroutines Overview}
\docid{add\_card} \section[add\_card]{ }
\label{sub:add_card}
\docrv{Version 1.2}
\author{Frode K.~Hansen, Eric Hivon}
\abstract{This document describes the \healpix Fortran90 subroutine ADD\_CARD.}

\begin{facility}
{This routine writes a keyword of any kind into a FITS header. It is a wrapper to other routines that write keywords of different kinds.}
{src/f90/mod/head\_fits.F90}
\end{facility}

\begin{f90format}
{header, kwd, value \optional{[, comment, update]} }
\end{f90format}
\aboutoptional

\begin{arguments}
{
\begin{tabular}{p{0.4\hsize} p{0.05\hsize} p{0.1\hsize} p{0.35\hsize}} \hline  
\textbf{name~\&~dimensionality} & \textbf{kind} & \textbf{in/out} & \textbf{description} \\ \hline
                   &   &   &                           \\ %%% for presentation
header(LEN=80) DIMENSION(:) & CHR & INOUT & The header to write the keyword to. \\
kwd(LEN=*) & CHR & IN & the FITS keyword to write. Should be shorter
                   or equal to 8 characters.\\
value & any & IN & the value to give to the keyword. \\
\optional{comment(LEN=*)} & CHR & IN & comment to the keyword. \\ 
\optional{update} & LGT & IN & if set to {\tt .true.}, the first occurence of the keyword {\tt
kwd} in {\tt header} will be updated (and all other occurences removed); otherwise, the keyword will be appended at
the end (and any previous occurence removed). If the keyword is either 'HISTORY'
or 'COMMENT', {\tt update} is ignored and the keyword is peacefully appended at the end of the header.\\ 
\end{tabular}
}
\end{arguments}

\begin{example}
{
character(len=80), dimension(1:120) :: header \\
header = '' ! very important \\
call add\_card(header,'NSIDE',256,'the nside of the map')  \\
}
{
Gives the keyword `NSIDE' the value 256 in the given header-string. It is
important to make sure that the {\tt header} string array is empty before attempting
to write
anything in it.
}
\end{example}

\begin{modules}
  \begin{sulist}{} %%%% NOTE the ``extra'' brace here %%%%
  \item[write\_hl] more general routine for adding a keyword to a header.
  \item[\textbf{cfitsio}] library for FITS file handling.		
  \end{sulist}
\end{modules}

\begin{related}
  \begin{sulist}{} %%%% NOTE the ``extra'' brace here %%%%
  \item[\htmlref{write\_minimal\_header}{sub:write_minimal_header}] routine to
write \healpix compliant baseline FITS header
  \item[\htmlref{get\_card}{sub:get_card}] general purpose routine to read any keywords from a header in a FITS file.
  \item[\htmlref{del\_card}{sub:del_card}] routine to discard a keyword from a FITS header
  \item[\htmlref{read\_par}{sub:read_par}, \htmlref{number\_of\_alms}{sub:number_of_alms}] routines to read specific keywords from a
  header in a FITS file.
  \item[\htmlref{getsize\_fits}{sub:getsize_fits}] function returning the size of the data set in a fits
  file and reading some other useful FITS keywords
  \item[\htmlref{merge\_headers}{sub:merge_headers}] routine to merge two FITS headers
  \end{sulist}
\end{related}

\rule{\hsize}{2mm}

\newpage
