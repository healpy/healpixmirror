\sloppy

\title{\healpix Fortran Subroutines Overview}
\docid{complex\_fft} \section[complex\_fft]{ }
\label{sub:complex_fft}
\docrv{Version 1.1}
\author{Martin Reinecke}
\abstract{This document describes the \healpix Fortran90 subroutine
complex\_fft.}

\begin{facility}
{This routine performs a forward or backward Fast Fourier Transformation
on its argument {\tt data}.}
{\modHealpixFft}
\end{facility}

\begin{f90format}
{data, backward}
\end{f90format}

\begin{arguments}
{
\begin{tabular}{p{0.3\hsize} p{0.05\hsize} p{0.1\hsize} p{0.45\hsize}} \hline  
\textbf{name\&dimensionality} & \textbf{kind} & \textbf{in/out} & \textbf{description} \\ \hline
                   &   &   &                           \\ %%% for presentation
data(:) & XXX & INOUT &
  array containing the input and output data. It can be of type
  real(sp), real(dp), complex(spc) or complex(dpc). If it is of type real,
  it is interpreted as an array of size(data)/2 complex variables.  \\
backward & LGT & IN & Optional argument. If present and true, perform backward transformation, else forward \\
\end{tabular}}
\end{arguments}

\begin{example}
{
use healpix\_fft \\
call complex\_fft (data, backward=.true.)
}
{
Performs a backward FFT on data.
}
\end{example}

\begin{related}
  \begin{sulist}{} %%%% NOTE the ``extra'' brace here %%%%
  \item[\htmlref{real\_fft}{sub:real_fft}] routine for FFT of real data
  \end{sulist}
\end{related}

\rule{\hsize}{2mm}

\newpage
