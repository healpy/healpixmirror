
\sloppy


%%%\title{\healpix Fortran Subroutines Overview}
\docid{read\_fits\_partial} \section[read\_fits\_partial]{ }
\label{sub:read_fits_partial}
\docrv{Version 1.3}
\author{Eric Hivon \& Frode K.~Hansen}
\abstract{This document describes the \healpix Fortran90 subroutine READ\_FITS\_PARTIAL.}

\begin{facility}
{This routine reads unpolarised or polarised partial sky \healpix map from a FITS file.\\
For more information on the FITS file format supported in \healpixns, 
including the one implemented in \thedocid,
see \url{\hpxfitsdoc}.}
{\modFitstools}
\end{facility}

\begin{f90format}
{\mylink{sub:read_fits_partial:filename}{filename}%
, \mylink{sub:read_fits_partial:pixel}{pixel}%
, \mylink{sub:read_fits_partial:cutmap}{cutmap}%
, \optional{[\mylink{sub:read_fits_partial:header}{header}%
%, \mylink{sub:read_fits_partial:units}{units}%
, \mylink{sub:read_fits_partial:extno}{extno}%
]}}
\end{f90format}
\aboutoptional

\begin{arguments}
{
\begin{tabular}{p{0.30\hsize} p{0.05\hsize} p{0.08\hsize} p{0.49\hsize}} \hline  
\textbf{name~\&~dimensionality} & \textbf{kind} & \textbf{in/out} & \textbf{description} \\ \hline
                   &   &   &                           \\ %%% for presentation
filename\mytarget{sub:read_fits_partial:filename}(LEN=\mylink{sub:healpix_types:filenamelen}{\filenamelen}) & CHR & IN & FITS file to be read from,
                   containing a partial sky map \\
pixel\mytarget{sub:read_fits_partial:pixel}(0:np-1)    & I4B/ I8B & OUT & index of observed (or valid) pixels \\
cutmap\mytarget{sub:read_fits_partial:cutmap}(0:np-1,1:nc)     & SP/ DP & OUT & value of unpolarised or polarised map for each observed pixel\\
\optional{header\mytarget{sub:read_fits_partial:header}}(LEN=80)(1:)    & CHR & OUT &   FITS extension header \\
% \optional{units\mytarget{sub:read_fits_partial:units}}(LEN=20)       & CHR & OUT &  maps units (applies only to
%                    Signal and Serror, which are assumed to have the same units) \\
\optional{extno\mytarget{sub:read_fits_partial:extno}}  & I4B & IN & extension number (0 based) for which map
             is read. Default = 0 (first extension). 
\end{tabular}
}
\end{arguments}

\begin{example}
{%
%use healpix\_types\\
use \htmlref{healpix\_modules}{sub:healpix_modules}\\
% \\
% implicit none\\
character(len=FILENAMELEN) :: file\\
integer(i4b) :: nmaps, polarisation, npix, nside\\
integer(i4b), allocatable, dimension(:)   :: pixel\\
real(SP),     allocatable, dimension(:,:) :: data\\
character(len=80), dimension(1:100)       :: header=``''\\
\\
file=``\url{\hpxfitsexamples/partial\_TQU.fits}''\\
npix = \htmlref{getsize\_fits}{sub:getsize_fits}(file, nmaps=nmaps, polarisation=polarisation)\\
print*, npix, nmaps, polarisation\\
allocate(pixel(0:npix-1))\\
allocate(data(0:npix-1,1:3))\\
call \thedocid(file, pixel, data, header=header)\\
print*,pixel(0), data(0,1:3)\\
}{
reads a remote partial sky FITS file and prints the index and IQU values of the first pixel its contains.
}
% {
% npix= read\_fits\_partial('map.fits', nmaps=nmaps, ordering=ordering,obs\_npix=obs\_npix, nside=nside, mlpol=mlpol, type=type, polarisation=polarisation)  \\
% }
% {
% Returns 1 or 3 in nmaps, dependent on wether 'map.fits' contain only
% temperature or both temperature and polarisation maps. The pixel ordering number is found by reading the keyword ORDERING in the FITS file. If this keyword does not exist, 0 is returned.
% }
\end{example}
\begin{modules}
  \begin{sulist}{} %%%% NOTE the ``extra'' brace here %%%%
  \item[\textbf{fitstools}] module, containing:
  \item[printerror] routine for printing FITS error messages.
  \item[\textbf{cfitsio}] library for FITS file handling.		
  \end{sulist}
\end{modules}

\begin{related}
  \begin{sulist}{} %%%% NOTE the ``extra'' brace here %%%%
  \item[anafast] executable that reads a \healpix map and analyses it. 
  \item[synfast] executable that generate full sky \healpix maps
  \item[\htmlref{getsize\_fits}{sub:getsize_fits}] routine to know the size of a FITS file and its type (eg, full sky vs cut sky)
  \item[\htmlref{input\_map}{sub:input_map}] all purpose routine to input a map of any kind from a FITS file
  \item[\htmlref{output\_map}{sub:output_map}] subroutine to write a FITS file from a full sky \healpix map
  \item[\htmlref{write\_fits\_partial}{sub:write_fits_partial}] subroutine to write a partial map into a FITS file which can be read by \thedocid{} and/or \htmlref{input\_map}{sub:input_map}.
  \end{sulist}
\end{related}

\rule{\hsize}{2mm}

\newpage
