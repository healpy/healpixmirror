% -*- LaTeX -*-


\renewcommand{\facname}{{ud\_grade}}
\renewcommand{\FACNAME}{{UD\_GRADE}}

\sloppy



\title{\healpix IDL Facility User Guidelines}
\docid{ud\_grade} \section[ud\_grade]{ }
\label{idl:ud_grade}
\docrv{Version 1.0}
\author{Eric Hivon}
\abstract{This document describes the \healpix IDL facility \facname.}




\begin{facility}
{This IDL facility provides a means to upgrade/degrade or reorder a full
sky or cut-sky \healpix map contained in a FITS file or loaded in memory.
}
{src/idl/toolkit/ud\_grade.pro}
\end{facility}

\begin{IDLformat}
{\FACNAME, 
\mylink{idl:ud_grade:map_in}{Map\_in}, 
\mylink{idl:ud_grade:map_out}{Map\_out} [, 
\mylink{idl:ud_grade:bad_data}{BAD\_DATA}=, 
\mylink{idl:ud_grade:help}{HELP}=, 
\mylink{idl:ud_grade:nside_out}{NSIDE\_OUT}=, 
\mylink{idl:ud_grade:order_in}{ORDER\_IN}=, 
\mylink{idl:ud_grade:order_out}{ORDER\_OUT}=, 
\mylink{idl:ud_grade:pessimistic}{/PESSIMISTIC}]}
\end{IDLformat}

\begin{qualifiers}
  \begin{qulist}{} %%%% NOTE the ``extra'' brace here %%%%
    \item[Map\_in] \mytarget{idl:ud_grade:map_in} input map: either a character
string with the name of a FITS file containing a full-sky or cut-sky Healpix
data set, 
      or a memory vector (real, integer, ...) containing a {\em full sky} data
      set.
    \item[Map\_out] \mytarget{idl:ud_grade:map_out} reordered map: if map\_in was a filename, map\_out should be
    a filename, otherwise map\_out should point to a memory array
  \end{qulist}
\end{qualifiers}

\begin{keywords}
  \begin{kwlist}{} %%% extra brace
    \item[BAD\_DATA =] \mytarget{idl:ud_grade:bad_data}%
	flag value of missing pixels.
          \default{{\tt !healpix.bad\_value} $\equiv -1.6375\ 10^{30}$}.
    \item[/HELP] \mytarget{idl:ud_grade:help}%
	 if set, the documentation header is printed out and the code exits
    \item[NSIDE\_OUT =] \mytarget{idl:ud_grade:nside_out}%
	 output resolution parameter, can be
    larger or smaller than the input one (scalar integer).
	 \default{same as input: map unchanged or simply reordered}
    \item[ORDER\_IN =] \mytarget{idl:ud_grade:order_in}%
	input map ordering (either 'RING' or 'NESTED')
	\default{same as the input FITS keyword ORDERING if applicable}.
    \item[ORDER\_OUT =] \mytarget{idl:ud_grade:order_out}%
	output map ordering (either 'RING' or 'NESTED')
	\default{same as ORDER\_IN}.
    \item[/PESSIMISTIC] \mytarget{idl:ud_grade:pessimistic}%
	\parbox[t]{0.5\hsize}{if set, during {\bf degradation} each big pixel containing one
    bad or missing small pixel is also considered as bad, \\
        if not set, each big pixel containing at least one good pixel
    is considered as good (optimistic)
       default = 0 (:not set)}
  \end{kwlist}
\end{keywords}  

\begin{codedescription}
{\facname{} can upgrade/degrade a \healpix map using the hierarchical
properties of \healpix. It can also reorder a sky map (from NEST to RING and
vice-versa). It operates on FITS files as well as on memory variables. Cut-sky
operations are only accessible via FITS files.
The degradation/upgradation is done assuming an
intensive quantity (like temperature) that does not scale with surface area. 
In case of degradation a big pixel that contains at least one bad small pixel is
considered as bad itself. When operating on FITS files, the header information
from the input file that is not directly related the ordering/resolution is
copied unchanged into the output file.}
\end{codedescription}



\begin{related}
  \begin{sulist}{} %%%% NOTE the ``extra'' brace here %%%%
    \item[idl] version \idlversion or more is necessary to run \facname.
    \item[\htmlref{reorder}{idl:reorder}] reorder a full sky Healpix map.
  \end{sulist}
\end{related}

\begin{examples}{1}
{
\begin{tabular}{l} %%%% use this tabular format %%%%
\facname,  'map\_512.fits', 'map\_256.fits', nside\_out = 256
\end{tabular}
}
{
\facname{} reads the FITS file map\_512.fits (that allegedly contains a map with
NSIDE=512), and write in the FITS file map\_256.fits a map degraded to resolution 256, with
the same ordering.
}
\end{examples}
\begin{examples}{2}
{
\begin{tabular}{l} %%%% use this tabular format %%%%
\facname,  'map\_512.fits', 'map\_Nest256.fits', nside\_out = 256, \$ \\
\          \ order\_out = 'NESTED'
\end{tabular}
}
{
\facname{} reads the FITS file map\_512.fits (that allegedly contains a map with
NSIDE=512), 
and writes in the FITS file map\_Nest256.fits a map degraded to resolution 256,
with NESTED ordering.
}
\end{examples}
\begin{examples}{3}
{
\begin{tabular}{l} %%%% use this tabular format %%%%
\htmlref{read\_fits\_map}{idl:read_fits_map}, 'map\_Nest256.fits', mymap\\
\facname,  mymap, mymap2, nside\_out = 1024, order\_in='NESTED', order\_out='RING'
\end{tabular}
}
{
mymap is IDL variable containing a \healpix NESTED-ordered map with resolution nside=256.
\facname{} upgrades this map to a resolution of 1024, reorder it to RING and write
it in the IDL vector mymap2.
}
\end{examples}


