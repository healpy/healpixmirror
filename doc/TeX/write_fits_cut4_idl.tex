% -*- LaTeX -*-


\renewcommand{\facname}{{write\_fits\_cut4 }}
\renewcommand{\FACNAME}{{WRITE\_FITS\_CUT4 }}
\sloppy



\title{\healpix IDL Facility User Guidelines}
\docid{\facname} \section[\facname]{ }
\label{idl:write_fits_cut4}
\docrv{Version 1.2}
\author{Eric Hivon}
\abstract{This document describes the \healpix facility \facname.}




\begin{facility}
{This IDL facility writes out a cut sky \healpix map into a FITS file according to
the \healpix convention. The format used for the
FITS file follows the one used for Boomerang98 and is adapted from
COBE/DMR. This routine can be used to store polarized maps, where the
information relative to the Stokes parameters I, Q and U are placed in extension
0, 1 and 2 respectively by successive invocation of the routine.}
{src/idl/fits/write\_fits\_cut4.pro}
\end{facility}

\begin{IDLformat}
{\FACNAME, \mylink{idl:write_fits_cut4:File}{File}%
, \mylink{idl:write_fits_cut4:Pixel}{Pixel}%
, \mylink{idl:write_fits_cut4:Signal}{Signal}%
 [, \mylink{idl:write_fits_cut4:N_Obs}{N\_Obs}%
, \mylink{idl:write_fits_cut4:Serror}{Serror}%
, 
\mylink{idl:write_fits_cut4:COORDSYS}{COORDSYS=}%
, 
\mylink{idl:write_fits_cut4:EXTENSION}{EXTENSION=}%
, 
\mylink{idl:write_fits_cut4:HDR}{HDR=}%
,
\mylink{idl:write_fits_cut4:NESTED}{/NESTED}%
, 
\mylink{idl:write_fits_cut4:NSIDE}{NSIDE=}%
, 
\mylink{idl:write_fits_cut4:ORDERING}{ORDERING=}%
, 
\mylink{idl:write_fits_cut4:POLARISATION}{/POLARISATION}%
,
\mylink{idl:write_fits_cut4:RING}{/RING}%
,
\mylink{idl:write_fits_cut4:UNITS}{UNITS=}%
,
\mylink{idl:write_fits_cut4:XHDR}{XHDR=}%
,
\mylink{idl:write_fits_cut4:HELP}{HELP=}%
]
}
\end{IDLformat}

\begin{qualifiers}
  \begin{qulist}{} %%%% NOTE the ``extra'' brace here %%%%
 	\item[{File}] \mytarget{idl:write_fits_cut4:File}
          name of a FITS file in which the map is to be written

 	\item[{Pixel}] \mytarget{idl:write_fits_cut4:Pixel}
	 (LONG or LONG64 vector), \\ index of observed (or valid) pixels

 	\item[{Signal}]  \mytarget{idl:write_fits_cut4:Signal}
	 (FLOAT or DOUBLE vector, same size as Pixel), \\ value of signal in each observed pixel

 	\item[{N\_Obs}]  \mytarget{idl:write_fits_cut4:N_Obs}
	 (LONG or INT or LONG64 vector, Optional, same size as Pixel), \\ number of
	 observation per pixel. \\
         If absent, the field {\tt N\_OBS} will take a value of 1 in the output file.
         If set to a scalar constant, {\tt N\_OBS} will take this value in the
	 output file

 	\item[{Serror}]  \mytarget{idl:write_fits_cut4:Serror}
	 (FLOAT or DOUBLE vector, Optional, same size as Pixel)\\ {\em rms} of signal in pixel, for white noise,
                   this is $\propto 1/\sqrt{{\rm n\_obs}}$ \\
         If absent, the field {\tt SERROR} will take a value of 0.0 in the output file.
         If set to a scalar constant, {\tt SERROR} will take this value in the
	 output file

  \end{qulist}
\end{qualifiers}

\begin{keywords}
  \begin{kwlist}{} %%% extra brace
       \item[{COORDSYS=}] \mytarget{idl:write_fits_cut4:COORDSYS}
		  (optional), \\
		if set to either 'C', 'E' or 'G',  specifies that the
		Healpix coordinate system is respectively Celestial=equatorial,
		  Ecliptic or Galactic.
		(The relevant keyword is then added/updated in the extension
		  header, but the map is NOT rotated)

	\item[{EXTENSION=}] \mytarget{idl:write_fits_cut4:EXTENSION}
		  (optional), \\
	  (0 based) extension number in which to write data. \default{0}.
	  If set to 0 (or not set) {\em a new file is written from scratch}.
	  If set to a value
		  larger than 1, the corresponding extension is added or
		  updated, as long as all previous extensions already exist.
		  All extensions of the same file should use the same ORDERING,
		  NSIDE and COORDSYS.

    	\item[HDR=] \mytarget{idl:write_fits_cut4:HDR}%	
		(optional), \\
		String array containing the information to be put in
		the primary header. 

	\item[{/NESTED}]\mytarget{idl:write_fits_cut4:NESTED}%
	(optional)
         if set, specifies that the map is in the NESTED ordering
	scheme\\
	\seealso Ordering and Ring 

	\item[{NSIDE=}]  \mytarget{idl:write_fits_cut4:NSIDE}
		(optional), \\
		scalar integer, \healpix resolution parameter of the
		data set. The resolution parameter should be made
		available to the FITS file, either thru this
		qualifier, or via the header (see XHDR).

	\item[{ORDERING=}] \mytarget{idl:write_fits_cut4:ORDERING}
		  (optional), \\
		if set to either 'ring' or 'nested' (case un-sensitive),
		  specifies that the map is respectively in RING or NESTED
		  ordering scheme\\
		\seealso Nested and Ring \\
	The ordering information should be made
		available to the FITS file, either thru a combination
		  of Ordering/Ring/Nested, or via the header (see XHDR).

	\item[{/POLARISATION}] \mytarget{idl:write_fits_cut4:POLARISATION}
	  specifies that file will contain the I, Q and U polarisation
           Stokes parameter in extensions 0, 1 and 2 respectively, and sets the
FITS header keywords accordingly

	\item[{/RING}]  \mytarget{idl:write_fits_cut4:RING} if set, specifies that the map is in the RING ordering
	scheme\\
	\seealso Ordering and Nested

	\item[{UNITS=}] \mytarget{idl:write_fits_cut4:UNITS}
		(optional), \\
		string describing the physical units of the data set (only applies
		to Signal and Serror)

    	\item[XHDR=] \mytarget{idl:write_fits_cut4:XHDR}%
		(optional), \\
		String array containing the information to be put in
		the extension header. 

    	\item[HELP=] \mytarget{idl:write_fits_cut4:HELP}%
		(optional), \\
		if set, an extensive help is displayed, and no file is written

   \end{kwlist}
\end{keywords}

\begin{codedescription}
{For more information on the FITS file format supported in \healpixns, 
including the one implemented in \facname,
see \url{\hpxfitsdoc}.}
% {\parbox[t]{\hsize}{\facname writes out the information contained in {\tt Prim\_stc} and {\tt
% Exten\_stc} in the primary unit and extension of the FITS file
% {\tt File} respectively . Coordinate systems can also be specified by {\tt Coordsys}. Specifying the
% ordering scheme is compulsary and can be done either in {\tt Header} or by setting {\tt
% Ordering} or {\tt Nested} or {\tt Ring} to the correct value. If {\tt
% Ordering} or {\tt Nested} or {\tt Ring} is set, its value overrides what is
% given in {\tt Header}. \\

% The data is assumed to represent a full sky data set with 
% the number of data points npix = 12*Nside*Nside
% unless   
% \\Partial is set OR the input fits header contains OBJECT =
%                'PARTIAL' \\
%        AND \\
%          the Nside qualifier is given a valid value OR the FITS header contains
%                  a NSIDE}}
\end{codedescription}



\begin{related}
  \begin{sulist}{} %%%% NOTE the ``extra'' brace here %%%%
  \item[idl] version \idlversion or more is necessary to run \thedocid
  \item[\htmlref{read\_fits\_cut4}{idl:read_fits_cut4}] This \healpix IDL facility can be used to read in maps
  written by \thedocid.
  \item[%
\htmlref{write\_fits\_cut4}{idl:write_fits_cut4},
\htmlref{write\_fits\_partial}{idl:write_fits_partial},
\htmlref{write\_fits\_map}{idl:write_fits_map}]
  \item[%
\htmlref{write\_tqu}{idl:write_tqu},
\htmlref{write\_fits\_sb}{idl:write_fits_sb}]
\healpix IDL routines to write cut-sky and partial maps, full-sky maps, polarized full-sky maps and
arbitrary data sets into FITS files

  \item[sxaddpar] This IDL routine (included in \healpix package) can be used to update
  or add FITS keywords to the header in {\tt HDR} and {\tt XHDR}
  \end{sulist}
\end{related}


\begin{examples}{1}
{
\begin{tabular}{ll} %%%% use this tabular format %%%%
\thedocid, 'map\_cut.fits', pixel, temperature, /ring,
nside=32, /pol \\
\end{tabular}
}
{writes in 'map\_cut.fits' a FITS file containing the temperature measured in a
  set of \healpix pixel.
}
\end{examples}
\begin{examples}{2}
{
\begin{tabular}{l} %%%% use this tabular format %%%%
\thedocid,  'tqu\_cut.fits', pixel, temperature, n\_t, s\_t, \$ \\
\hspace{1em}		    /ring, nside=32, /pol \\
\thedocid,  'tqu\_cut.fits', pixel, qstokes, n\_q, s\_q, \$ \\
\hspace{1em}		    /ring, nside=32, /pol, ext=1\\
\thedocid,  'tqu\_cut.fits', pixel, ustokes, n\_u, s\_u, \$ \\
\hspace{1em}		    /ring, nside=32, /pol, ext=2\\
\end{tabular}
}
{writes in 'tqu\_cut.fits' a FITS file with three extensions, each of them containing
information on the observed pixel, the measured signal, the number of
observations and noise per pixel, for the three Stokes parameters I, Q and U
respectively. The \healpix ring ordered scheme and the resolution $\nside=32$ is assumed.
}
\end{examples}


