% -*- LaTeX -*-

\sloppy

\title{\healpix IDL Facility User Guidelines}
\docid{hpx2gs} \section[hpx2gs]{ }
\label{idl:\thedocid}
\docrv{Version 1.0}
\author{Eric Hivon}
\abstract{This document describes the \healpix facility \thedocid.}

% q customized for hpx2gs IDL routines adapted from mollview qualifiers
  \newenvironment{qualifiers_hpx2gs}
    {\rule{\hsize}{0.7mm}
     \textsc{\Large{\textbf{QUALIFIERS}}}\hfill\newline%
	\renewcommand{\arraystretch}{1.5}%
	}

% \newcommand{\sizeoneg}{0.19\hsize}
%\newcommand{\sizetwog}{0.08\hsize}
%\newcommand{\sizethrg}{0.70\hsize}
% \newcommand{\sizeoneg}{0.12\hsize} % see hpx2dm_idl.tex
% \newcommand{\sizethrg}{0.85\hsize}

\begin{facility}
{This IDL facility provides a means to turn a \healpix  map into a image that
can be visualized with 
\htmladdnormallink{Google Earth}{http://earth.google.com/}
 or 
\htmladdnormallink{Google Sky}{http://earth.google.com/sky/skyedu.html}%
.}
{src/idl/visu/hpx2gs.pro}
\end{facility}

\begin{IDLformat}
{\thedocid, 
\mylink{idl:hpx2gs:file}{File}, 
[ \mylink{idl:hpx2gs:select}{Select}, ]
[ \mylink{idl:hpx2gs:coord_in}{COORD\_IN=}, 
\mylink{idl:hpx2gs:help}{/HELP}, 
\mylink{idl:hpx2gs:kml}{KML=}, 
\mylink{idl:hpx2gs:png}{PNG=}, 
\mylink{idl:hpx2gs:reso_arcmin}{RESO\_ARCMIN=}, 
\mylink{idl:hpx2gs:subtitle}{SUBTITLE=}, 
\mylink{idl:hpx2gs:titleplot}{TITLEPLOT=},%
  + %
\mylink{idl:hpx2gs:other_keywords}{most of cartview keywords}%
$\ldots$
]
}
\end{IDLformat}

\newpage
\begin{qualifiers_hpx2gs}
\begin{tabular}{p{\sizeoneg}%
% p{\sizetwog} 
p{\sizethrg}}
%% \begin{tabular}{p{0.14\hsize} p{0.08\hsize} p{0.75\hsize}}
\hline  
%\textbf{name}  & \textbf{description} \\ \hline
%                    &   &                            \\ %%% for presentation
File \mytarget{idl:hpx2gs:file}    & \parbox[t]{0.95\hsize}{Required\\
                    name of a FITS file containing 
               the \healpix\ map in an extension or in the image field, \\
          {\em or}\  \  name of an {\em online} variable (either array or
structure) containing the \healpix\ map (See note below);\\
          if Save is set   :    name of an IDL saveset file containing
               the \healpix\ map stored under the variable  {\tt data} \\
	\nodefault}\\ 

Select \mytarget{idl:hpx2gs:select}    & \parbox[t]{0.95\hsize}{Optional\\
		  column of the BIN FITS table to be plotted, can be either  \\
                -- a name : value given in TTYPEi of the FITS file \\
                        NOT case sensitive and can be truncated, \\
                        (only letters, digits and underscore are valid) \\
               -- an integer        : number i of the column
                            containing the data, starting with 1 (also valid if
		  {\tt File} is an online array) \\
                   \default {1 for full sky maps, 'SIGNAL' column for FITS files
		  containing cut sky maps}}\\
\end{tabular}
\end{qualifiers_hpx2gs}

\begin{keywords}
  \begin{kwlist}{} %%%% NOTE the ``extra'' brace here %%%%
\item [COORD\_IN= \mytarget{idl:hpx2gs:coord_in}] 1-character scalar, describing the input data coordinate system:\\
                either 'C' or 'Q' : Celestial2000 = eQuatorial,\\
                       'E'        : Ecliptic,\\
                       'G'        : Galactic.\\
             If set, it will over-ride the coordinates read from the FITS file header (when
             applicable). In absence of information, the input coordinates is
assumed to be celestial.\\
             The data will be rotated so that the output coordinates are Celestial, as expected by Google Sky

\item [{/HELP} \mytarget{idl:hpx2gs:help}]  Prints out the documentation header 

\item [{KML=} \mytarget{idl:hpx2gs:kml}] Name of the KML file to be created (if the {\tt .kml} suffix is missing,
     it will be added automatically)
     \default {{\tt 'hpx2googlesky.kml'}}

\item [{PNG=} \mytarget{idl:hpx2gs:png}] Name of the PNG overlay file to be created. Only to be used if you want the
     filename to be different from the default 
	(\default{ same as KML file, with a {\tt .png} suffix instead
     of {\tt .kml}})

\item [{RESO\_ARCMIN=} \mytarget{idl:hpx2gs:reso_arcmin}] Pixel angular size in arcmin (at the equator) of the cartesian
     map generated \default{30}

\item [{SUBTITLE=} \mytarget{idl:hpx2gs:subtitle}] information on the data, 
will appear in KML file {\tt GroundOverlay
     description} field

\item [{TITLEPLOT=} \mytarget{idl:hpx2gs:titleplot}] information on the data, 
will appear in KML file {\tt GroundOverlay
     name} field

\item [\mytarget{idl:hpx2gs:other_keywords}
\mylink{idl:mollview:asinh}{/ASINH}, ]
\item [\mylink{idl:mollview:colt}{COLT}=,
\mylink{idl:mollview:factor}{FACTOR}=,
\mylink{idl:mollview:flip}{/FLIP},
\mylink{idl:mollview:glsize}{GLSIZE}=,
\mylink{idl:mollview:graticule}{GRATICULE}=,
\mylink{idl:mollview:hbound}{HBOUND}=,]
\item [\mylink{idl:mollview:hist_equal}{/HIST\_EQUAL},
\mylink{idl:mollview:iglsize}{IGLSIZE}=,
\mylink{idl:mollview:igraticule}{IGRATICULE}=,
\mylink{idl:mollview:log}{/LOG},
\mylink{idl:mollview:max}{MAX}=,
\mylink{idl:mollview:min}{MIN}=, ]
\item [\mylink{idl:mollview:nested}{/NESTED},
% \mylink{idl:mollview:no_dipole}{/NO\_DIPOLE},
% \mylink{idl:mollview:no_monopole}{/NO\_MONOPLE},
\mylink{idl:mollview:offset}{OFFSET}=, ]
\item [\mylink{idl:mollview:outline}{OUTLINE}=,
\mylink{idl:mollview:polarization}{POLARIZATION}=,
\mylink{idl:mollview:preview}{/PREVIEW},]
\item [\mylink{idl:mollview:quadcube}{/QUADCUBE},
\mylink{idl:mollview:save}{SAVE}=,
\mylink{idl:mollview:silent}{/SILENT}, ]
\item [\mylink{idl:mollview:truecolors}{TRUECOLORS}=]
 those keywords have the same meaning as in
\htmlref{cartview}{idl:cartview} and 
\htmlref{mollview}{idl:mollview}


  \end{kwlist}
\end{keywords}
%***************************************************************


\begin{codedescription}
{\thedocid\ reads in a \healpix sky map in FITS format or from a memory array and generates a
cartesian projection of it in a PNG file, as well as a Google Sky compatible 
\htmladdnormallink{KML}{http://earth.google.com/kml}
file. Missing or unobserved pixels in the input data will be
   totally 'transparent' in the output file.
}
\end{codedescription}

%\include{mollview_idl_related}
\begin{related}
  \begin{sulist}{} %%%% NOTE the ``extra'' brace here %%%%
  \item[{\ }] see \htmlref{cartview}{idl:cartview}
  \item[\htmlref{hpx2dm}{idl:hpx2dm}] turns Healpix maps into DomeMaster images
  \end{sulist}
\end{related}


\begin{example}
{
\begin{tabular}{l} %%%% use this tabular format %%%%

map  = findgen(48) \\
\thedocid, map, kml='my\_map.kml',title='my map in Google'\\
\end{tabular}
}
{produces in {\tt my\_map.kml} and in {\tt my\_map.png} an image of the input map that can be seen with
Google Sky.
To do so, start GoogleEarth or GoogleSky and open {\tt my\_map.kml}. Under
Mac\-OSX, simply type {\tt open my\_map.kml} on the command line.
}
\end{example}

\newpage
