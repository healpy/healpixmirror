% -*- LaTeX -*-

% PLEASE USE THIS FILE AS A TEMPLATE FOR THE DOCUMENTATION OF YOUR OWN
% FACILITIES: IN PARTICULAR, IT IS IMPORTANT TO NOTE COMMENTS MADE IN
% THE TEXT AND TO FOLLOW THIS ORDERING. THE FORMAT FOLLOWS ONE USED BY
% THE COBE-DMR PROJECT.	
% A.J. Banday, April 1999.




\sloppy



\title{\healpix IDL Facility User Guidelines}
\docid{gnomcursor} \section[gnomcursor]{ }
\label{idl:\thedocid}
\docrv{Version 1.2}
\author{Eric Hivon}
\abstract{This document describes the \healpix facility gnomcursor.}


\begin{facility}
{This IDL facility provides a point-and-click interface for finding
the astronomical location, value and pixel index of the pixels nearest 
to the pointed position on a gnomonic projection of a \healpix map.}
{src/idl/visu/gnomcursor.pro}
\end{facility}

\begin{IDLformat}
{GNOMCURSOR, [cursor\_type=, file\_out=]}
\end{IDLformat}

\begin{qualifiers}
%  \begin{qulist}{} %%%% NOTE the ``extra'' brace here %%%%
%  	\item[{cursor\_type=}] cursor type to be used \\
% 	\default {34}
%  \end{qulist}
\hbox{\hspace{5cm}		see \htmlref{mollcursor}{idl:mollcursor}}
\end{qualifiers}

%\begin{keywords}
%  \begin{kwlist}{} %%%% NOTE the ``extra'' brace here %%%%
%  \end{kwlist}
%\end{keywords}

\begin{codedescription}
{gnomcursor should be called immediately after gnomview. It gives the longitude,
latitude, map value and pixel number
corresponding to the cursor position in the window containing the map generated
by gnomview. For more details, or in case
of problems under {\bf Mac OS X}, see \htmlref{mollcursor}{idl:mollcursor}.}
\end{codedescription}



\begin{related}
\hbox{\hspace{5cm}	see \htmlref{mollcursor}{idl:mollcursor}}
%   \begin{sulist}{} %%%% NOTE the ``extra'' brace here %%%%
%   \item[idl] version \idlversion or more is necessary to run gnomcursor
%   \item[gnomview] This IDL \healpix facility will generate the gnomonic map
% 		on which gnomcursor can be run.
%   \end{sulist}
\end{related}


\begin{example}
{
\begin{tabular}{ll} %%%% use this tabular format %%%%
gnomcursor & \ 
\end{tabular}
}
{After gnomview has read in a map and generated
its gnomonic projection, gnomcursor is run to determine the
position and flux of bright synchrotron sources, for example.}
\end{example}


