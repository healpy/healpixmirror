
\sloppy


\title{\healpix Fortran Subroutines Overview}
\docid{alter\_alm*} \section[alter\_alm*]{ }
\label{sub:alter_alm}
\docrv{Version 2.0}
\author{Eric Hivon}
\abstract{This document describes the \healpix Fortran90 subroutine ALTER\_ALM.}

\begin{facility}
{This routine modifies scalar (and tensor) $a_{\ell m}$ by multiplying them by a beam window
  function described by a FWHM (in the case of a gaussian beam) or read from an external
  file (in the more general case of a circular beam)  $a_{\ell m}
  \longrightarrow a_{\ell m} b(\ell)\myhtmlimage{}$ . It can also be used to
  multiply the  $a_{\ell m}$ by an arbitray function of $\ell$.}
{\modAlmTools}
\end{facility}

\begin{f90format}
{\mylink{sub:alter_alm:nsmax}{nsmax}%
, \mylink{sub:alter_alm:nlmax}{nlmax}%
, \mylink{sub:alter_alm:nmmax}{nmmax}%
, \mylink{sub:alter_alm:fwhm_arcmin}{fwhm\_arcmin}%
, \mylink{sub:alter_alm:alm_TGC}{alm\_TGC}%
 [, \mylink{sub:alter_alm:beam_file}{beam\_file}%
, \mylink{sub:alter_alm:window}{window}%
]}
\end{f90format}

\begin{arguments}
{
\begin{tabular}{p{0.36\hsize} p{0.05\hsize} p{0.09\hsize} p{0.40\hsize}} \hline  
\textbf{name~\&~dimensionality} & \textbf{kind} & \textbf{in/out} & \textbf{description} \\ \hline
                   &   &   &                           \\ %%% for presentation
nsmax\mytarget{sub:alter_alm:nsmax} & I4B & IN & $\nside$ resolution parameter of the map associated with the $a_{lm}$
                   considered. Currently has no effect on the routine. \\ 
nlmax\mytarget{sub:alter_alm:nlmax} & I4B & IN & maximum $\ell$ value for the $a_{\ell m}$.   \\
nmmax\mytarget{sub:alter_alm:nmmax} & I4B & IN & maximum $m$ value for the $a_{\ell m}$.   \\
fwhm\_arcmin\mytarget{sub:alter_alm:fwhm_arcmin} & SP/ DP & IN & fwhm size of the gaussian beam in arcminutes. \\
alm\_TGC\mytarget{sub:alter_alm:alm_TGC}(1:p,0:nlmax,0:nmmax) & SPC/ DPC & INOUT & complex $a_{\ell m}$ values
                   to be altered.  The first index here runs from 1:1 for
                   temperature only, and 1:3 for polarisation. In the latter
                   case,  1=T, 2=E, 3=B. \\
\end{tabular}
\begin{tabular}{p{0.36\hsize} p{0.05\hsize} p{0.09\hsize} p{0.40\hsize}} \hline  
beam\_file\mytarget{sub:alter_alm:beam_file}(LEN=\filenamelen) \hskip 2cm (OPTIONAL)& CHR & IN & name of the file
                   containing the (non necessarily gaussian) window function
                   $B_\ell$  of a circular beam. If present, it will override
                   the argument {\tt fwhm\_arcmin}.  \\
window\mytarget{sub:alter_alm:window}(0:nlw,1:d) \hskip 5cm (OPTIONAL)& SP/ DP & IN & arbitrary window by which to multiply the
                   $a_{\ell m}$. If present, it overrides both {\tt fwhm\_arcmin}
                   and {\tt beam\_file}. If nlw $<$ nlmax, the $a_{\ell m}$ with
                   $\ell \in \{$nlw+1,nlmax$\}$ are set to 0, and a warning is issued. If $d<p$ the
                   window for temperature is replicated for polarisation.
\end{tabular}
}
\end{arguments}

\begin{example}
{
call alter\_alm(64, 128, 128, 1, 5.0, alm\_TGC)  \\
}
{
Alters scalar and tensor $a_{lm}$ of a map with $\nside=64$, 
 $\ell_\textrm{max}=m_\textrm{max} = 128$ by multiplying them by the beam window function of a
gaussian beam with FWHM = 5 arcmin.
}
\end{example}

\begin{modules}
  \begin{sulist}{} %%%% NOTE the ``extra'' brace here %%%%
  \item[\textbf{alm\_tools}] module, containing:
	\item[\htmlref{generate\_beam}{sub:generate_beam}] routine to generate beam window function
	\item[\htmlref{pixel\_window}{sub:pixel_window}] routine to generate pixel window function
  \end{sulist}
\end{modules}

\begin{related}
  \begin{sulist}{} %%%% NOTE the ``extra'' brace here %%%%
  \item[\htmlref{create\_alm}{sub:create_alm}] Routine to create $a_{\ell m}$ coefficients.
  \item[\htmlref{rotate\_alm}{sub:rotate_alm}] Routine to rotate $a_{\ell m}$
  coefficients between 2 different arbitrary coordinate systems.
  \item[\htmlref{map2alm}{sub:map2alm}]  Routines to analyze a \healpix sky map into its $a_{\ell m}$
  coefficients.
  \item[\htmlref{alm2map}{sub:alm2map}] Routines to synthetize a \healpix sky map from its $a_{\ell m}$
  coefficients.
  \item[\htmlref{alms2fits}{sub:alms2fits}, \htmlref{dump\_alms}{sub:dump_alms}]
  Routines to save a set of $a_{lm}$ in a FITS file.  
  \end{sulist}
\end{related}

\rule{\hsize}{2mm}

\newpage
