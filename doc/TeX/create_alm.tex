
\sloppy


\title{\healpix Fortran Subroutines Overview}
\docid{create\_alm*} \section[create\_alm*]{ }
\label{sub:create_alm}
\docrv{Version 2.0}
\author{Frode K.~Hansen, Eric Hivon}
\abstract{This document describes the \healpix Fortran90 subroutine CREATE\_ALM.}

\begin{facility}
{This routine generates scalar (and tensor) $a_{\ell m}$ for a temperature (and
  polarisation) power spectrum read from an input FITS
file. The $a_{\ell m}$ are gaussian distributed with a zero mean, and their
  amplitude is multiplied with the $\ell$-space window function of a gaussian
  beam characterized by its FWHM or an arbitrary circular beam
and a pixel window read from an external file.}
{\modAlmTools}
\end{facility}

\begin{f90format}
{\mylink{sub:create_alm:nsmax}{nsmax}%
, \mylink{sub:create_alm:nlmax}{nlmax}%
, \mylink{sub:create_alm:nmmax}{nmmax}%
, \mylink{sub:create_alm:polar}{polar}%
, \mylink{sub:create_alm:filename}{filename}%
, \mylink{sub:create_alm:rng_handle}{rng\_handle}%
, \mylink{sub:create_alm:fwhm_arcmin}{fwhm\_arcmin}%
, \mylink{sub:create_alm:alm_TGC}{alm\_TGC}%
, \mylink{sub:create_alm:header}{header}%
 \optional{[,
\mylink{sub:create_alm:windowfile}{windowfile}%
, \mylink{sub:create_alm:units}{units}%
, \mylink{sub:create_alm:beam_file}{beam\_file}%
]}}
\end{f90format}
\aboutoptional

\begin{arguments}
{
\begin{tabular}{p{0.4\hsize} p{0.05\hsize} p{0.1\hsize} p{0.35\hsize}} \hline  
\textbf{name~\&~dimensionality} & \textbf{kind} & \textbf{in/out} & \textbf{description} \\ \hline
                   &   &   &                           \\ %%% for presentation
nsmax\mytarget{sub:create_alm:nsmax} & I4B & IN & $\nside$ of the map to be synthetized from the $a_{\ell m}$
                   created by this routine. \\ 
nlmax\mytarget{sub:create_alm:nlmax} & I4B & IN & maximum $\ell$ value to be considered (MAX=$4\nside$
if {\tt windowfile} is provided).   \\
nmmax\mytarget{sub:create_alm:nmmax} & I4B & IN & maximum $m$ value for the $a_{\ell m}$.   \\
%
polar\mytarget{sub:create_alm:polar} & I4B & IN & if set to 0, only Temperature (scalar) $a_{\ell m}$ are
generated. If set to 1, 'conventional' polarization is added. If set to 2, and if
the relevant information is in {\tt filename}, polarization is generated
assuming non-zero correlation of Curl (B) modes with Temperature (T) and Gradient
(E) modes. Note that the \htmlref{\tt synfast}{fac:synfast} facility calls \thedocid\ with {\tt polar}=0 {\em or} {\tt polar}=1\\
%
filename\mytarget{sub:create_alm:filename}(LEN=\filenamelen) & CHR & IN & name of FITS file containing power
spectra in the order TT, [EE, BB, TE, [TB, EB]] (terms in brackets are optional) \\
%% iseed & I4B & INOUT & seed for generation of gaussian distributed random numbers
%%                    to the $a_{\ell m}$. 
%%                    Set to a negative integer to (re-)initialise the
%%                    random number generator. On exit iseed will be positive
%%                    definite. If set to a positive value, the information contained in {\tt
%%                    rng\_handle} will be used to carry on an existing random sequence.\\
rng\_handle\mytarget{sub:create_alm:rng_handle} & \htmlref{planck\_rng}{sub:planck_rng} & \hskip 2cm INOUT & structure containing
information necessary to continue a random sequence
initiated {\em previously} with the 
subroutine \htmlref{{\tt rand\_init}}{sub:rand_init}. Consecutive calls to \thedocid {} can be made after a
single invocation to \htmlref{{\tt rand\_init}}{sub:rand_init}.\\
%
fwhm\_arcmin\mytarget{sub:create_alm:fwhm_arcmin} & SP/ DP & IN & FWHM size of the gaussian beam in arcminutes. \\
%
alm\_TGC\mytarget{sub:create_alm:alm_TGC}(1:p,0:nlmax,0:nmmax) & SPC/ DPC & OUT & complex $a_{\ell m}$ values
generated from the powerspectrum in the FITS-file. The first index here runs
form 1:1 for temperature only, and 1:3 for polarisation. In the latter case,
1=T, 2=E, 3=B. \\
%
%--------------------------------
\end{tabular} 
\begin{tabular}{p{0.4\hsize} p{0.05\hsize} p{0.1\hsize} p{0.35\hsize}} \hline  
\textbf{name~\&~dimensionality} & \textbf{kind} & \textbf{in/out} & \textbf{description} \\ \hline
                   &   &   &                           \\ %%% for presentation
%--------------------------------
header\mytarget{sub:create_alm:header}(LEN=80),dimension(60) & CHR & OUT & part of header  which
will be included in the FITS-file containing the
map  synthesised from the $a_{\ell m}$  which create\_alm generates. \\
%
\optional{windowfile\mytarget{sub:create_alm:windowfile}}(LEN=\filenamelen) & CHR & IN & full filename specification
of the FITS file with the pixel window function (defined for $\ell\le4 \nside$) \\
%
\optional{units\mytarget{sub:create_alm:units}}(LEN=80),dimension(1:) & CHR & OUT & physical units of the created
$a_{\ell m}$ (square-root of the input power spectrum units). \\
\optional{beam\_file\mytarget{sub:create_alm:beam_file}}(LEN=\filenamelen) & CHR & IN & name of the file containing
the (non necessarily gaussian) window function $B_\ell$ of a circular beam. If present, it will override
the argument {\tt fwhm\_arcmin}. \\
\end{tabular}
}
\end{arguments}

\begin{example}
{
use alm\_tools, only: create\_alm \\
use rngmod, only: rand\_init, planck\_rng \\
type(planck\_rng) :: rng\_handle \\
\\
call rand\_init(rng\_handle, -1) \\
call create\_alm(64, 128, 128, 1, 'cl.fits', rng\_handle, 5.0, alm\_TGC, \&\\
 \&  header, 'data/pixel\_window\_n0064.fits')  \\
}
{
Creates scalar and tensor $a_{\ell m}$ from the power spectrum given in the file
`cl.fits'. The map to be created from these $a_{\ell m}$ is assumed to have
$N_{side}=64$. $C_l$s from the power spectrum are used up to an $\ell$ value of
128. 
Corresponding $a_{\ell m}$ values up to l=128 and m=128 are created as gaussian distributed
complex numbers. Their are drawn from a sequence of pseudo-random numbers
initiated with a seed of -1. 
The produced $a_{\ell m}$ are convolved with a gaussian beam of FWHM 5 arcminutes
and a pixel window read from 'data/pixel\_window\_n0064.fits'. It is assumed that after the return
from this routine, a map is generated from the created
$a_{\ell m}$. For this purpose, {\tt header} is updated with FITS format information
describing the origin and history of these $a_{\ell m}$.
}
\end{example}

\begin{modules}
  \begin{sulist}{} %%%% NOTE the ``extra'' brace here %%%%
  \item[\textbf{alm\_tools}] \underline{module}, containing:
	\item[pow2alm\_units] routine to convert from power spectrum units to
  $a_{\ell m}$ units
	\item[\htmlref{generate\_beam}{sub:generate_beam}] routine to generate beam window function
	\item[\htmlref{pixel\_window}{sub:pixel_window}] routine to read in pixel window function
  \item[\textbf{utilities}] \underline{module}, containing:
        \item[die\_alloc] routine that prints an error message if there is not enough space for allocation of variables.
  \item[\textbf{fitstools}] \underline{module}, containing:
        \item[\htmlref{fits2cl}{sub:fits2cl}] routine to read a FITS
  file containing a power spectrum.
        \item[\htmlref{read\_dbintab}{sub:read_dbintab}] routine to read a FITS-binary file containing the pixel window functions.
  \item[\textbf{head\_fits}] \underline{module}, containing:
       \item[\htmlref{add\_card}{sub:add_card}] routine to add a keyword to a FITS header.
       \item[\htmlref{get\_card}{sub:get_card}] routine to read a keyword value from
  FITS header.
       \item[\htmlref{merge\_headers}{sub:merge_headers}] routine to merge two FITS headers.
%%   \item[\textbf{ran\_tools}] module, containing:
%%   \item[randgauss\_boxmuller] function which returns a gaussian distributed random
%%   number. Please refer to the ``Comment on Random Number Generator''
%%   in the Fortran90 facilities guidelines.
  \item[\textbf{rngmod}] \underline{module}, containing:
       \item[\htmlref{rand\_gauss}{sub:rand_gauss}] function which returns a gaussian distributed random
  number. 
  \end{sulist}
\end{modules}

\begin{related}
  \begin{sulist}{} %%%% NOTE the ``extra'' brace here %%%%
  \item[\htmlref{rand\_init}{sub:rand_init}] subroutine to initiate a random number sequence. 
  \item[synfast] executable using \thedocid\ to synthesize CMB maps from a given
  power spectrum.
  \item[\htmlref{alm2map}{sub:alm2map}] Routine to transform a set of $a_{\ell m}$ created by \thedocid\ to a \healpix map.
  \item[\htmlref{alms2fits}{sub:alms2fits}, \htmlref{dump\_alms}{sub:dump_alms}]
  Routines to save a set of $a_{\ell m}$ in a FITS file.  
  \end{sulist}
\end{related}

\rule{\hsize}{2mm}

\newpage
