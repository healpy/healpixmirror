
\sloppy


\title{\healpix IDL Facility User Guidelines}
\docid{template\_pixel\_xxxx} \section[template\_pixel\_xxxx]{ }
\label{idl:template_pixel_xxx}
\docrv{Version 1.0}
\author{E. Hivon}
\abstract{This document describes the \healpix facilities
  TEMPLATE\_PIXEL\_RING and TEMPLATE\_PIXEL\_NEST.}

\begin{facility}
{These IDL facilities provide the index of the template pixel associated with a given
  \healpix pixel, for a resolution parameter $\nside$. 
%Any pixel can be {\em matched in shape}
%  to a single of these templates by a combination of  a rotation around the polar axis with 
%  reflexion(s) around a meridian and/or the equator. 

%The template pixels are all located in the Northern Hemisphere, or on the
% Equator.
%They are chosen to have their center located at
%\begin{eqnarray}
%     z=\cos(\theta)\ge 2/3,  &\ &    0< \phi \leq \pi/2,   \nonumber\\            %[Nside*(Nside+2)/4]
%     2/3 > z \geq 0,  &\ & \phi=0, \quad{\rm or}\quad  \phi=\frac{\pi}{4\nside}.  \nonumber %[Nside]
%\myhtmlimage{}
%\end{eqnarray}
% They are numbered continuously from 0, starting at the North Pole, with the index
% increasing in $\phi$, and then increasing for decreasing $z$.
}
{src/idl/toolkit/template\_pixel\_nest.pro, src/idl/toolkit/template\_pixel\_ring.pro}
\end{facility}

\begin{IDLformat}
{template\_pixel\_nest, Nside, Pixel\_Nest, Template, Reflexion}
\end{IDLformat}
\begin{IDLformat}
{template\_pixel\_ring, Nside, Pixel\_Ring, Template, Reflexion}
\end{IDLformat}

%\ mylink: to avoid automatic processing by make_internal_links.sh

\begin{qualifiers}
  \begin{qulist}{} %%%% NOTE the ``extra'' brace here %%%%

\item[{Nside}] (IN, scalar) the \healpix $\nside$ parameter. 
\item[{Pixel\_Nest}] (IN, scalar or vector) NESTED scheme pixel identification number(s) over the range \{0,$12\nside^2-1$\}.
\item[{Pixel\_Ring}] (IN, scalar or vector) RING scheme pixel identification number(s) over the
                   range \{0,$12\nside^2-1$\}.
\item[{Template}] (OUT, scalar or vector) identification number(s) of the
                   template matching in shape the pixel(s) provided (the numbering
                   scheme of the pixel templates is the same for both routines). 
\item[{Reflexion}] (OUT, scalar or vector) in \{0, 3\} encodes the transformation(s) to
                   apply to each pixel provided to match exactly in
                   shape and position its respective template. 0: rotation around the polar axis only,
                   1: rotation + East-West swap (ie, reflexion around meridian),
                   2: rotation + North-South swap (ie, reflexion around
                   Equator), 3: rotation + East-West and North-South swaps
  \end{qulist}
\end{qualifiers}

\begin{codedescription}
{\thedocid\ provide the index of the template pixel associated with a given
  \healpix pixel, for a resolution parameter $\nside$. 

Any pixel can be {\em matched in shape}
  to a single of these templates by a combination of  a rotation around the polar axis with 
  reflexion(s) around a meridian and/or the equator. 

The template pixels are all located in the Northern Hemisphere, or on the
 Equator.
They are chosen to have their center located at
\begin{eqnarray}
     z=\cos(\theta)\ge 2/3,  &\ &    0< \phi \leq \pi/2,   \nonumber\\            %[Nside*(Nside+2)/4]
     2/3 > z \geq 0,  &\ & \phi=0, \quad{\rm or}\quad  \phi=\frac{\pi}{4\nside}.  \nonumber %[Nside]
\myhtmlimage{}
\end{eqnarray}
 They are numbered continuously from 0, starting at the North Pole, with the index
 increasing in $\phi$, and then increasing for decreasing $z$.
}
\end{codedescription}


\begin{example}
{
template\_pixel\_ring, 256, 500000, template, reflexion  \\
}
{
Returns in {\tt template} the index of the template pixel (16663) whose shape matches
that of the pixel \#500000 for $\nside=256$. Upon return {\tt reflexion} will
contain 2, meaning that the template must be reflected around a meridian and
around the equator (and then rotated around the polar axis) in order to match
the pixel.
}
\end{example}
\begin{related}
  \begin{sulist}{} %%%% NOTE the ``extra'' brace here %%%%
  \item[\htmlref{nside2templates}{idl:nside2ntemplates}] returns the
  number of template pixel shapes available for a given $\nside$.
  \item[\htmlref{same\_shape\_pixels\_ring}{idl:same_shape_pixels_xxx}] 
  \item[\htmlref{same\_shape\_pixels\_nest}{idl:same_shape_pixels_xxx}] 
  return
  the ordered list of pixels having the same shape as a given pixel template
  \end{sulist}
\end{related}

\rule{\hsize}{2mm}

